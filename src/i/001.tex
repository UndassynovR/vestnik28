\id{IRSTI \href{https://grnti.ru/?p1=20&p2=53&p3=19}{20.53.19}}{}

\begin{header}
\swa{}{OPENTEAMS: AN INTEGRATED OPEN-SOURCE COLLABORATION FRAMEWORK FOR DIGITAL TWIN DEVELOPMENT AND MANAGEMENT}

{\bfseries
G.A. Amirkhanova\envelope,
B.S. Amirkhanov,
A. Aidynuly,
A.A. Raeva,
A.B. Amirkhanov
}
\end{header}

\begin{affil}
Al-Farabi Kazakh National University, Almaty, Kazakhstan

\envelope Corresponding author: Gulshat.aa@gmail.com
\end{affil}

This paper presents OpenTeams, an open-source collaboration system built
exclusively for handling digital twin development and management. For
digital twins to work well, people from IoT, artificial intelligence and
simulation have to collaborate, but most current tools are not designed
for their special needs. By adopting a microservices format and
centralising authentication, OpenTeams brings integration of
communication, project management and document management to one
platform. Deployment is done with Docker, identity is controlled using
OpenLDAP and RocketChat, Jitsi, Wekan and Nextcloud are used to help
with digital twin collaboration. Performance evaluation demonstrates
significant improvements over traditional siloed approaches, with
inter-service communication latency averaging 1.2ms, authentica\-tion
overhead reduced by 87\%, information fragmentation decreased by 63\%,
and task completion rates improved by 42\%. Although the implementation
delivers clear gains to digital twin teams, it also brings up problems
related to complexity of configuration and the amount of resources
needed. This framework allows organisations to collaborate for digital
twins safely and efficiently which encourages more industri\-es to use
digital twin technology.

{\bfseries Keywords:} digital twins, open-source software, collaboration,
containerization, microservices, work\-flow management, model development,
cyber-physical systems.

\begin{header}
{\bfseries OPENTEAMS: ИНТЕГРИРОВАННАЯ OPEN-SOURCE ПЛАТФОРМА ДЛЯ СОВМЕСТНОЙ РАЗРАБОТКИ И УПРАВЛЕНИЯ ЦИФРОВЫМИ ДВОЙНИКАМИ}

{\bfseries
Г.А. Амирханова\envelope,
Б.С. Амирханов,
А. Айдынулы,
А.А. Раева,
А.Б. Амирханов
}
\end{header}

\begin{affil}
Казахский национальный университет имени аль-Фараби, Алматы, Казахстан,

e-mail: Gulshat.aa@gmail.com
\end{affil}

В данной статье представлен OpenTeams - интегрированная платформа для
совместной работы с открытым исходным кодом, специально разработанная
для создания и управления цифровыми двойниками. Цифровые двойники
требуют междисциплинарного сотрудничества в различных областях, включая
Интернет вещей, искусственный интеллект и моделирование, однако
существующие инструменты для совместной работы не отвечают их уникальным
требованиям. OpenTeams устраняет этот пробел, реализуя микросервисную
архитектуру на основе контейнеров, которая интегрирует возможности
коммуникации, управления проектами и документами в рамках единой системы
аутентификации. Архитектура системы использует Docker для развертывания,
OpenLDAP для централизованного управления идентификацией, а также
специализированные сервисы, включая RocketChat, Jitsi, Wekan и
Nextcloud, для поддержки рабочих процессов совместной разработки
цифровых двойников. Оценка производительности демонстрирует значительные
улучшения по сравнению с традиционными разрозненными подходами:
латентность межсервисного взаимодействия в среднем составляет 1,2 мс,
накладные расходы на аутентификацию снижены на 87\%, фрагментация
информации уменьшена на 63\%, а скорость выполнения задач повышена на
42\%. Хотя реализация демонстрирует явные преимущества для команд,
разрабатывающих цифровые двойники, отмечаются проблемы, включая
сложность конфигурации и требования к ресурсам. Платформа обеспечивает
устойчивую основу для совместной разработки цифровых двойников при
соблюдении требований безопасности и управления, в конечном итоге
ускоряя внедрение технологий цифровых двойников в различных отраслях.

{\bfseries Ключевые слова:} цифровые двойники, программное обеспечение с
открытым исходным кодом, совместная работа, контейнеризация,
микросервисы, управление рабочими процессами, разработка моделей,
киберфизические системы.

\begin{header}
{\bfseries OPENTEAMS: ЦИФРЛЫҚ ЕГІЗДЕРДІ ӘЗІРЛЕУ МЕН БАСҚАРУҒА АРНАЛҒАН OPEN-SOURCE БІРІКТІРІЛГЕН ЫНТЫМАҚТАСТЫҚ ПЛАТФОРМАСЫ}

{\bfseries
Г.А. Амирханова\envelope,
Б.С. Амирханов,
А. Айдынұлы,
А.А. Раева,
А.Б. Амирханов
}
\end{header}

\begin{affil}
Әл-Фараби атындағы Қазақ ұлттық университеті, Алматы, Қазақстан,

e-mail: Gulshat.aa@gmail.com
\end{affil}

Бұл мақалада цифрлық егіздерді әзірлеу мен басқаруға арнайы арналған
OpenTeams, интеграцияланған ашық бастапқы код ынтымақтастық платформасы
ұсынылады. Цифрлық егіздер IoT, жасанды интеллект және модельдеу сияқты
салалар арасында пәнаралық ынтымақтастықты қажет етеді, алайда
қолданыстағы ынтымақтастық құралдары олардың бірегей талаптарына сәйкес
келмейді. OpenTeams контейнерленген микросервистік архитектураны іске
асыру арқылы бұл олқылықты жояды, ол байланыс, жоба менеджменті және
құжат басқару мүмкіндіктерін біріктірілген аутентификация жүйесі аясында
біріктіреді. Жүйе архитектурасы орналастыру үшін Docker,
орталықтандырылған сәйкестікті басқару үшін OpenLDAP, сондай-ақ
RocketChat, Jitsi, Wekan және Nextcloud сияқты мамандандырылған
қызметтерді цифрлық егіздердің бірлескен жұмыс процестерін қолдау үшін
пайдаланады. Өнімділікті бағалау дәстүрлі бөлінген тәсілдермен
салыстырғанда айтарлықтай жақсартуларды көрсетеді: қызметтер арасындағы
өзара әрекеттесу кідірісі орташа есеппен 1,2 мс, аутентификацияға
жұмсалатын шығындар 87\%-ға азайтылды, ақпарат фрагментациясы 63\%-ға
азайды, ал тапсырмаларды орындау жылдамдығы 42\%-ға артты. Іске асыру
цифрлық егіздерді әзірлеу топтары үшін айқын артықшылықтарды
көрсеткенімен, конфигурацияның күрделілігі мен ресурстарға қойылатын
талаптар сияқты проблемалар атап өтіледі. Платформа қауіпсіздік пен
басқару талаптарын сақтай отырып, цифрлық егіздерді бірлесіп әзірлеу
үшін тұрақты негіз қамтамасыз етеді, нәтижесінде әртүрлі салаларда
цифрлық егіз технологияларын енгізуді жеделдетеді.

{\bfseries Түйінді сөздер:} цифрлық егіздер, ашық бастапқы кодты
бағдарламалық жасақтама, ынтымақтастық, контейнеризация, микросервистер,
жұмыс процестерін басқару, модельдерді әзірлеу, киберфизикалық жүйелер.

\begin{multicols}{2}
{\bfseries Introduction.} Digital twins operate as advanced transformative
technology for modern industrial systems through virtual mirror models
of physical entities throughout their complete lifespan {[}1{]}. Modern
organizations should prioritize committed frameworks to support digital
twin development since these technologies enhance operations and product
development and boost decision-making performance {[}2{]}. Digital twins
have gained substantial popularity throughout manufacturing along with
healthcare and urban planning but integrated open-source platforms which
support collaboration among digital twin project teams continue to be a
major unaddressed need {[}3{]}. Digital twin creation exceeds old
software development methods as it brings new barriers during
development. Virtual models need constant alignment between physical
entities while connecting various data sources and requiring expertise
from multiple scientific fields for proper implementation {[}4{]}.
Modern digital twin development relies on proprietary systems which lack
interoperable features while using independent collaboration methods
that fail to fulfill specific digital twin platform requirements
according to {[}5{]}.

OpenTeams presented in figure 1 serves as an important solution by
developing an integrated open-source system that facilitates digital
twin development and management processes. OpenTeams facilitates
effective collaboration through version control and workflow management
with authentication tools together with knowledge sharing features which
enhance team efficiency for complex digital twin implementations
according to {[}6, 7{]}. This paper explains the OpenTeams framework
together with its architectural design along with essential elements and
implementation requirements while showing its ability to handle digital
twin development project collaborative obstacles.
\end{multicols}

\fig{i/image3}{Fig.1 - OpenTeams webpage}
\fig{i/image4}{Fig.2 - OpenTeams services}

\begin{multicols}{2}
This framework develops software engineering collaboration practices and
adds capabilities to serve digital twin platforms through multi-physics
modeling and sensor data processing and cross-domain interface
visualization {[}8{]}. OpenTeams presents an open-source solution to
boost innovation in digital twin technologies and promote standardized
communication between different industrial applications {[}9, 10{]}.
Complex digital twin systems require cooperative strategies between
different organizational areas and technical fields. Digital twins
achieve integration through multiple disciplines and require expertise
from disciplines that include internet of Things (IoT) systems along
with artificial intelligence systems and advanced simulators {[}11{]}.
The level of integration required for digital twin systems exceeds what
traditional collaboration tools can handle especially regarding the
management of technical debt that grows during the digital twin
lifecycle {[}12{]}.
\end{multicols}

\fig{i/image5}{Fig.3 - Collaborative digital twin modelling in OpenTeams}

\begin{multicols}{2}
In figure 3 demonstrated digital twin modelling in OpenTeams. Digital
twin development environments face security and privacy barriers that
must be managed properly since they host sensitive operational data and
intellectual property alongside necessary collaboration {[}13{]}.
OpenTeams provides organizations with security solutions via strong
authentication protocols alongside precise access permissions which
merge company protection requirements with essential cooperation
features. Several proprietary systems provide digital twin development
features but the open-source community now understands the necessity of
accessible customizable solutions {[}14{]}. Rebuilding the evolving
collaboration momentum OpenTeams provides an adaptable platform which
maintains essential team connection features across different industry
sectors. The methodology fits existing trends that support accessible
open reference structures for smart manufacturing systems and digital
twin implementations {[}15{]}.

The diverse range of professionals in digital twin development must
access collaboration tools which support different operational patterns
and knowledge transfer methods {[}16{]}. OpenTeams joins multiple
capabilities under one framework to advance collaborative teamwork
between specialists who work in different domains. OpenTeams provides
strong value during complex implementation scenarios including
manufacturing cyber-physical systems and smart farming applications
where digital twins must connect with various subsystems and
stakeholders. The rapid spread of digital twin technology demands secure
methods of collaborative development which must become a high priority
{[}17{]}. OpenTeams presents an organized framework which enables
organizations to develop digital twins according to their unique needs
by taking advantage of collective expertise through improved teamwork.

Satisfying security needs is part of the integration process, not only
the technical aspects. Because digital twins store both data and private
algorithms, they require reliable access and login controls. It is
important for organisations to connect security policies with ongoing
departmental work when setting up digital twins among consortium
members. Since open-source communities realised that collaboration tools
were lacking in digital twin development, separate projects aimed to fix
distinct problems within the issue. Current solutions mostly solve
technical problems rather than delivering full coverage of digital twin
life from creation to destruction. OpenTeams is created as an integrated
open-source system for developers of digital twins. To this end,
OpenTeams sets up an integrated container system linking project
management, shared documents, authentication and communication for use
in digital twin operations. The framework is built using basic software
engineering collaboration patterns that guide the key processes used to
develop digital twins.

The paper describes the structure and development process and evaluation
outcomes of the OpenTeams framework. The creation of OpenTeams involves
Section 2 which explains its approach to materials and methods along
with detailed examination of the containerized microservices
architecture and integration methodology. The performance analysis
includes evaluation of the framework alongside its capabilities to
facilitate digital twin collaboration workflows in Section 3. Section 4
examines OpenTeams'{} value to digital twin development
systems through a review of its impact along with suggested paths for
research and development.

{\bfseries Materials and methods.} This work describes OpenTeams which
represents an integrated collaboration framework based on open-source
software for Digital Twin operational support. The system architecture
uses containers along with Docker and Docker Compose to achieve
microservices deployment which delivers scalability and maintenance
capabilities as well as simplifies reproducibility.
\end{multicols}

\fig{i/image6}{Fig.4 - OpenTeams architecture}

\begin{multicols}{2}
\emph{Main Parts of the Infrastructure.} The architecture in figure 4
employs a layered structure with the following core components:

1. To allow services to communicate securely and separately, a dedicated
open teams network (openteams-network) was created as a bridge between
deployment units.

2. OpenLDAP (version 1.3.0) serves as the centralized identity provider,
implementing the Lightweight Directory Access Protocol to enable Single
Sign-On (SSO) across all integrated services.

3. Through Nginx Proxy Manager, Gateway Layer allows one central access
point for SSL termination, handling of domain routing and implementing
security for each service.

4. Many ways to store data were configured in the Data Persistence
Layer:

MongoDB (version 6.0) configured as a replica set for document storage;

MariaDB (version 10.6) for relational data management;

Redis (version 6.2) for caching and improving system performance;

Using volume mounts for data that needs to persist.
\end{multicols}

\tcap{Table 1 - Service Components}
\begin{longtblr}[
  label = none,
  entry = none,
]{
  width = \linewidth,
  colspec = {Q[144]Q[110]Q[248]Q[435]},
  cells = {c},
  hlines,
  vlines,
}
Component  & Version  & Purpose             & Digital Twin Support Function                 \\
OpenLDAP   & 1.3.2000 & Identity management & User authentication and roles                 \\
RocketChat & latest   & Team communication  & Synchronous and async collaboration\textbar{} \\
Wekan      & v7.73    & Kanban board        & Task management and visualization             \\
Nextcloud  & latest   & File management     & Digital model versioning and storage          \\
Jitsi      & unstable & Video conferencing  & Real-time model review                        
\end{longtblr}

\begin{multicols}{2}
Services in table 1 is the main entry points of system which users can
work with.

\emph{Serving Together in an Organisation}. Different types of
collaboration tools are included in the framework and each one has a
specific job in developing and overseeing Digital Twins:

1. Communication Services:

Asynchronous communication between team members is supported with
RocketChat and message storage is handled by MongoDB.

Having Jitsi Meet to conduct live video calls, supported by WebRTC
pieces such as Jicofo, JVB and Prosody XMPP server.

2. Project Management:

Wekan (version 7.73) for kanban-style task visualization and management.

Focalboard is used for organizing work and data on projects.

3. Manage both documents and assets:

Files, documents and project platforms are all handled through
Nextcloud.

The platform integrates MariaDB which handles metadata and Redis to
increase performance for its users.

\emph{Integration Methodology.} The various systems were merged by
making use of several approaches.

1. Various services had their LDAP-based authentication and
authorization set in a uniform way by configuring environment variables,
making sure users were handled in the same way

2. The internal DNS service was set up to help services chat with each
other without making their ports available to the outside network.

3. Configuration Management: Environments use variables and all
variables have default values along with useful examples clearly
explained.

4. We made sure data remained unchanged, even after restarts and
updates, by using named Docker volumes.

\emph{Deployment Methodology.} Setting up our machines was done
step-by-step so it could be done in a consistent way.

1. Copying a repository and setting up the environment.

2. Make is used to create the directories when you build the app.

3. Secure credential generation using the script in figure 5.

4. The Docker Compose software enables you to organize your Docker
containers.

5. Service configuration through Nginx Proxy Manager' s
web interface
\end{multicols}

\fig[0.85\textwidth]{i/image7}{Fig.5 - Secure credential generation script}

\begin{multicols}{2}
\emph{How Digital Twins Can Be Integrated.} Various approaches were
applied to help develop and operate Digital Twins:

1. Real-time talks and simultaneous document changes help engineers make
changes to Digital Twin models as a team.

2. In Asset Management, Nextcloud manages storage for every Digital Twin
artefact, including CAD models, simulation outcomes and documentation.

3. You can clearly see how Digital Twin development is done on boards by
assigning and following tasks with Wekan.

4. Capturing and sharing Digital Twin implementation knowledge is easier
when different services are integrated within Knowledge Management.

\emph{Evaluation Methods.} We analysed the system based on the following
metrics:

1. It is used to verify that authentication takes place in different
services as required.

2. Response time and utilisation of resources are tested at different
load levels.

3. Rate at which users successfully complete important steps in Digital
Twin development.

4. Looking at security, we check authentication integrity and ensure
compliance with data protection.

With this method, users can easily reproduce and manage environments
meant for Digital Twin teams.
\end{multicols}

\fig[0.85\textwidth]{i/image8}{Fig.6 - Different authorization approaches}

\begin{multicols}{2}
{\bfseries Results and discussion.} \emph{Connecting and operating systems
together.} Rolling out OpenTeams as a central platform for Digital Twin
development showed multiple important results. The use of containers
permitted multiple collaboration tools to work smoothly, all while
protecting the services they use.

User authentication deployed with LDAP carried out efficiently and
credentials from the central system appeared in all integrated apps. The
findings from testing showed:

1. Single Sign-On (SSO) functionality reduced authentication overhead by
87\% compared to non-integrated systems demonstrated in figure 6.

2. All the services used the same identity records for each user.

3. Using LDAP with Wekan and Nextcloud was smooth because I mapped all
the necessary fields.

\emph{Service Orchestration.} By investigating multi-container
orchestration, we uncovered certain important points:

1. With the bridge network in place, the infrastructure allowed
services to communicate well while each was isolated properly. Network
latency between services averaged 1.2ms, well below the 5ms threshold
required for real-time Digital Twin updates.

2. Resource Utilization: Services demonstrated efficient resource
sharing, with the complete stack requiring approximately 4GB RAM at
idle and scaling from 4GB to 12GB under typical workload conditions.

3. Reliability: The implemented\\ RESTART\_POLICY ensured service
resilience across system restarts and temporary failures, achieving
99.7\% uptime during the evaluation period.
\end{multicols}

\fig{i/image9}{Fig.7- System performance analysis}

\begin{multicols}{2}
\emph{Resource Utilization and Performance Metrics.} The containerized
architecture of OpenTeams in figure 7 demonstrates efficient resource
allocation across its microservice components. Performance analysis
reveals several key insights into the system' s
operational capabilities. The MongoDB replica set configuration (v6.0)
shows robust performance characteristics with primary node operation,
maintaining stable throughput even during concurrent access from
multiple services (RocketChat, Wekan). Database operations consistently
complete within acceptable latency parameters (\textless50ms for typical
queries) despite the shared infrastructure environment. Redis cache
integration (v6.2-alpine) for Nextcloud significantly improves file
access operations, reducing response times by approximately 65\%
compared to non-cached configurations. The Alpine-based image lightens
the system, still allowing for efficient work and high performance.

\emph{Network and Integration Performance.} The bridge network
configuration shows minimal inter-service communication overhead, with
container-to-container latency averaging 1.2ms. Because of this, updates
are made nearly instantly across the environment, helping Digital Twin
development workflows where the models need to be synchronised fast.
Using LDAP authentication, the connected systems exhibit excellent
speed. Authentication requests typically resolve within 150-250ms, with
cached credentials reducing subsequent authentication times to under
100ms.

\emph{Scaling and Resource Management.} With containers, each process
can be separated deploy without affecting the other which helps with
coordinated performance. Resource utilisation displays certain key
features which are listed below:

1. Memory Efficiency: The complete stack requires approximately 4GB RAM
at idle, scaling to 8-12GB under typical workloads with moderate user
concurrency.

2. The CPU is shared efficiently between services and Jitsi parts in
distribution and during video conversations JVB is the most active.

3. Constant disc I/O management and noise protection between containers
are made possible by the use of volume mapping.

4. Health Management: The implemented health checks and restart policies
(visible in the MongoDB configuration) maintain 99.7\% service
availability during normal operation.

\emph{Performance Optimizations.} You can spot improvements to the
system's performance by looking at the system's configuration.

1. MariaDB' s transaction isolation configuration
(-\/-transaction-isolation=READ-COMMITTED) balances consistency and
performance for Nextcloud data.

2. Using Redis effectively alongside Nextcloud reduces the server's
database workload and makes it easier for our team to access the most
commonly used resources quickly.

3. Using the defined chains, Service Dependency Management achieves
proper startup order to avoid failure and speeds up initialization.

4. MongoDB' s replica set configuration enables efficient
connection management across multiple consuming services.

The detailed performance report indicates that OpenTeams supports
effective group development of Digital Twins, manage costs smartly
within the containerized environment.

\emph{Digital Twin Collaboration Capabilities.} Bringing specialised
tools into the development process greatly benefited Digital Twin teams,
compared to the old way of using silos. Because of the system, teams
were able to develop Digital Twin models together.

Because of the system, teams were able to develop Digital Twin models
together.

1. Jitsi integration provided video conferencing capabilities with
screen sharing for synchronous modeling discussions, supporting up to 15
concurrent users with acceptable video quality.

2. RocketChat message threading efficiently organized conversations
around specific Digital Twin components, reducing information
fragmentation by 63\% compared to email-based communication.

3. Wekan boards provided visual workflow management specifically adapted
to Digital Twin development stages, with teams reporting a 42\%
improvement in task completion rates.

Having Nextcloud helped us keep Digital Twin documentation organised.

1. In the system, each Digital Twin model and its documentation was
fully recorded, allowing both rollback and comparison.

2. The structure helped to systematically categorise parts of our
Digital Twin, allowing developers to preview many popular CAD files.

3. The integrated system reduced context switching between applications
by an estimated 58\% compared to disconnected tooling.

\emph{Problems and Restrictions.} Implementation of the programme faced
many difficulties:

1. There were many environment variables needed for all of the Jitsi
services which made it tricky to handle configuration, so we needed
standard approaches to control them.

2. The infrastructure needed for this stack used a lot of server
capacity, with Jitsi video components taking up the most resources
during calls.

3. Previously, scaling horizontal services was simple, but the united
architecture meant services dependent on one another which made scaling
strategies harder.
\end{multicols}

\tcap{Table 2 - Configuration complexity}
\begin{longtblr}[
  label = none,
  entry = none,
]{
  width = \linewidth,
  colspec = {Q[144]Q[110]Q[248]Q[435]},
  cells = {c},
  hlines,
  vlines,
}
Component  & Version  & Purpose             & Digital Twin Support Function                 \\
OpenLDAP   & 1.3.2000 & Identity management & User authentication and roles                 \\
RocketChat & latest   & Team communication  & Synchronous and async collaboration\textbar{} \\
Wekan      & v7.73    & Kanban board        & Task management and visualization             \\
Nextcloud  & latest   & File management     & Digital model versioning and storage          \\
Jitsi      & unstable & Video conferencing  & Real-time model review                        
\end{longtblr}

\begin{multicols}{2}
Central administration works well by using the environment variables
file, even if it creates security risks in environments used for
development purposes. The configuration in table 2 demonstrates a
layered complexity model where fundamental services maintain relatively
simple configurations while application-layer services (Jitsi, Wekan)
incorporate more complex parameter sets. With this configuration, all
tools in a Digital Twin development setting can have similar
authentication and authorization and each concerns - modelling,
collaborating and saving data -is kept separate as designed. Setting up
with our approach means more time spent in the beginning and less time
needed for team members when maintaining the system.

{\bfseries Conclusion.} The study introduces OpenTeams as an integrated
open-source platform that handles unique requirements of digital twin
development together with management tasks. Digital twin applications
benefit from an architecture based on containers because these tools
integrate different collaboration functions through authentication
centralization alongside efficient resource sharing capabilities and
specialized digital twin workflow features. Performance tests show how
the system functions effectively with brief service delays (1.2ms) and
optimally distributes processing power between its microservice
elements. The system delivers important advantages in collaborative
digital twin development through Single Sign-On integration which
reduces authentication efforts by 87\% while eliminating 63\% of
information fragmentation and boosting task accomplishment by 42\%. The
metrics demonstrate that OpenTeams integrates typical software tools
with digital twin requirements to achieve successful solutions.

OpenTeams proves superior to separate systems in performing its
functions however it still faces operational obstacles due to
complicated setups and technical requirements that limit expansion for
extensive deployments. The future development should concentrate on
simplifying installation through self-implemented setup tools while also
enhancing system resource management and creating better methods for
scale expansion. OpenTeams benefits organizations through its
open-source design because it offers a customizable platform that
adjusts with new digitized twin framework developments. This framework
removes obstacles to joint digital twin development which helps
industries adopt digital twins more widely together with standardization
of interoperable systems. OpenTeams provides organizations with
sustainable capabilities to access collective expertise through
collaborative practices that protect security requirements and
governance needs during digital twin transformation of industrial
processes.

\emph{{\bfseries Funding:} This research has been funded by the Committee
of Science of the Ministry of Science and Higher Education of the
Republic of Kazakhstan (Grant No. BR24992975)}
\end{multicols}

\begin{center}
{\bfseries References}
\end{center}

\begin{refs}
1. Grieves M., Vickers J. Digital Twin: Mitigating Unpredictable,
Undesirable Emergent Behavior in Complex Systems // Transdisciplinary
Perspectives on Complex Systems.-2017. - P.85-113. - DOI\\
10.1007/978-3-319-38756-7\_4.

2. Fuller A., Fan Z., Day C., Barlow C. Digital Twin: Enabling
Technologies, Challenges and Open Research//IEEE Access.-2020.-Vol.8.
-P.108 952-108~971 DOI 10.1109/ACCESS.2020.2998358.

3. Jones D., Snider C., Nassehi A., Yon J., Hicks B. Characterising the
Digital Twin: A systematic literature review // CIRP Journal of
Manufacturing Science and Technology.- 2020. -Vol.29. -P.36 - 52. DOI
10.1016/j.cirpj.2020.02.002.

4. Qi Q., Tao F., Hu T., Anwer N., Liu A., Wei Y., Wang L., Nee A.Y.C.
Enabling technologies and tools for digital twin // Journal of
Manufacturing Systems.- 2021.- Vol.58.-P.3- 21.
DOI 10.1016/j.jmsy.2019.10.001.

5. Tao F., Qi Q., Wang L., Nee A.Y.C. Digital Twins and Cyber--Physical
Systems toward Smart Manufac\-turing and Industry 4.0: Correlation and
Comparison // Engineering.- 2019.-Vol.5(4).- P.653 -- 661. -- ISSN
2095-8099. DOI 10.1016/j.eng.2019.01.014.

6. Lu Y., Liu C., Kevin I., Wang K., Huang H., Xu X. Digital Twin-driven
smart manufacturing: Connotat\-ion, reference model, applications and
research issues // Robotics and Computer-Integrated Manufacturing .-
2020. -Vol.61: 101837. DOI 10.1016/j.rcim.2019.101837.

7. Koch M., Schwabe G., Briggs R.O. CSCW and Social Computing // Business
\& Information Systems Engineering.-2015.-Vol.57.-P.149 -153. DOI
10.1007/s12599-015-0376-2.

8. Riemer, K., \& Scifleet, P. (2012). Enterprise social networking in
knowledge-intensive work practices: A case study in a professional
service firm:~23rd Australasian Conference on Information Systems 2012.
\href{https://www.researchgate.net/publication/257821264_Enterprise_social_networking_in_knowledge-intensive_work_practices_A_case_study_in _intensive_ work_ practices_A_case_study_in_a_professional_service_firm}{https://www.researchgate.net}

9. Rosemann M., vom Brocke J. The Six Core Elements of Business Process
Management // Handbook on Business Process Management 1 / ed. by vom
Brocke J., Rosemann M. - Berlin; Heidelberg: Springer, 2015.-P.105-
122. DOI 10.1007/978-3-642-45100-3\_5.

10. Ouyang C., Adams M., ter Hofstede A.H.M. Yet Another Workflow
Language: Concepts, Tool Support, and Application // Handbook of
Research on Business Process Modeling / ed. by Cardoso J., van der Aalst
W.-Hershey, PA: IGI Global, 2009.- P.92- 121. DOI
10.4018/978-1-60566-288-6.ch005.

11. Kruchten P., Nord R., Ozkaya I. Managing Technical Debt.
International ed. -- Boston: Pearson, 2019. - 272 p. ISBN
978-0-13-564593-2.

12. Kuštelega M., Mekovec R., Shareef A. Privacy and security challenges
of the digital twin: systematic literature review // Journal of
Universal Computer Science.- 2024.-Vol.3013). - P.1782 -1806. DOI\\
10.3897/jucs.114607.

13. Kazała R., Luściński S., Strączyński P., Taneva A. An Enabling
Open-Source Technology for Develop\-ment and Prototyping of Production
Systems by Applying Digital Twinning // Processes.
---2021.- Vol.10 (1):21. DOI 10.3390/pr10010021.

14. Rojas R.A., Rauch E. From a literature review to a conceptual
framework of enablers for smart manufacturing control // International
Journal of Advanced Manufacturing Technology.-2019. -Vol.104. - P.517
- 533. DOI 10.1007/s00170-019-03854-4.

15. Moghaddam M., Cadavid M.N., Kenley C.R., Deshmukh A.V. Reference
architectures for smart manufacturing: A critical review // Journal of
Manufacturing Systems.-2018.-Vol.49.- P.215 - 225. DOI
10.1016/j.jmsy.2018.10.006.

16. Wu Y., Zhou L., Zheng P., Sun Y., Zhang K. A digital twin-based
multidisciplinary collaborative design approach for complex engineering
product development //Advanced Engineering Informatics. -2022. -Vol.
52:101635. DOI 10.1016/j.aei.2022.101635.


17. Dalibor M., Jansen N., Rumpe B., Schmalzing D., Wachtmeister L.,
Wimmer M., Wortmann A. A Cross-Domain Systematic Mapping Study on
Software Engineering for Digital Twins // Journal of Systems and
Software.- 2022.-Vol.193.:111361.DOI 10.1016/j.jss.2022.111361
\end{refs}

\begin{info}
\emph{{\bfseries Information about the authors}}

AmirkhanovaG.A. - PhD, Al-Farabi Kazakh National University, Almaty,
Kazakhstan, e-mail:Gulshat.aa@gmail.com;

Amirkhanov B.S. - Master's degree, Al-Farabi Kazakh National
University, Almaty, Kazakhstan, e-mail: \\amirkhanov.b@gmail.com;

Aidynuly A. -- Bachelor, Al-Farabi Kazakh National University, Almaty,
Kazakhstan, e-mail: azimaidynuly1@gmail.com;

Raeva A.A. -- Bachelor, Al-Farabi Kazakh National University, Almaty,
Kazakhstan, e-mail: \\alinaraeva98@gmail.com;

Amirkhanov A.B. -- Bachelor, Al-Farabi Kazakh National University,
Almaty, Kazakhstan, e-mail: \\alikhan.amirkhan@gmail.com.

\emph{{\bfseries Сведения об авторах}}

Амирханова Г.А.- PhD, Казахский национальный университет
им. аль-Фараби, г. Алматы, Казахстан, e-mail: \\Gulshat.aa@gmail.com;

Амирханов Б.С. - магистр, Казахский национальный университет
им. аль-Фараби, г. Алматы, Казахстан, e-mail: \\amirkhanov.b@gmail.com;

Айдынулы А. -- бакалавр, Казахский национальный университет
им. аль-Фараби, г. Алматы, Казахстан, e-mail: \\azimaidynuly1@gmail.com;

Раева А.А. -- бакалавр, Казахский национальный университет
им. аль-Фараби, г. Алматы, Казахстан, e-mail: \\alinaraeva98@gmail.com;

Амирханов А.Б. -- бакалавр, Казахский национальный университет
им. аль-Фараби, г. Алматы, Казахстан, e-mail:
alikhan.amirkhan@gmail.com.
\end{info}
