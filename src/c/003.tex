\id{IRSTI 61.31.57}{}

{\bfseries HYDROGEN STORAGE IN POROUS CARBON MATERIALS OBTAINED BASED ON
SHUBARKOL COAL}

{\bfseries \tsp{1,2,3}M.K.
Kazankapova}{\bfseries \envelope ,
\tsp{1,2,3}B.T.
Yermagambet}{\bfseries ,
\tsp{1,2,3}Zh.M.
Kassenova}{\bfseries ,\tsp{2}B.A.
Kapsalyamov},

{\bfseries \tsp{1,2}A.B.
Malgazhdarova}{\bfseries ,
\tsp{1,3}Zh.T.
Dauletzhanova}{\bfseries ,\tsp{1,2}U.M.
Kozhamuratova},

{\bfseries \tsp{1,2}G. K.
Mendaliyev}{\bfseries ,
\tsp{1}A.S.
Akshekina}

\emph{\tsp{1}«Institute of Coal Chemistry and Technology»
LLP, Astana, Kazakhstan,}

\emph{\tsp{2}L.N. Gumilyov Eurasian National
University, Astana, Kazakhstan,}

\emph{\tsp{3}Kazakh university of technology and business
named after K. Kulazhanov, Astana, Kazakhstan}

\envelope Corresponding author е-mail:
coaltech@bk.ru,
maira\_1986@mail.ru

The aim of the research work was to study the hydrogen sorption
properties of carbonized adsorbents obtained from coal from the
Shubarkol deposit in Kazakhstan. The adsorbents were prepared by heat
treatment with K\tsb{2}CO\tsb{3}. The morphological
features of the obtained adsorbents were determined by the SEM method.
The structure of the materials was confirmed using transmission electron
microscopy (TEM). The results of the studies showed the largest specific
surface area (653.024 m²/g), which determines its effective hydrogen
sorption capacity. The adsorption-desorption isotherm of mesopores of
the powdered carbonized adsorbent
"Shubarkol:K\tsb{2}CO\tsb{3}" 1:0.5, 900ºC refers to
a type I microporous material. The adsorption-desorption isotherm of the
powdered carbonized adsorbent
"Shubarkol:K\tsb{2}CO\tsb{3}" 1:1, 900ºC indicates
that it has a macroporous or uneven solid surface of type II. In this
regard, the powder "Shubarkol:K\tsb{2}CO\tsb{3}" 1:1
is recognized as the optimal material for storing hydrogen due to its
high specific surface, developed microporous structure and ability to
adsorb hydrogen.

{\bfseries Keywords:} coal, adsorbent, porosity, carbonization, adsorption,
hydrogen.

{\bfseries ШҰБАРКӨЛ КӨМІРІ НЕГІЗІНДЕ АЛЫНҒАН КЕУЕКТІ КӨМІРТЕКТІ
МАТЕРИАЛДАРДА СУТЕКТІ САҚТАУ}

{\bfseries \tsp{1,2,3}М.Қ. Қазанқапова\envelope ,
\tsp{1,2,3}Б.Т.Ермағамбет, \tsp{1,2,3}Ж.M.
Касенова, \tsp{2}Б.А.Капсалямов,}

{\bfseries \tsp{1,2}А.Б. Малғаждарова,
\tsp{1,3}Ж.Т. Даулетжанова, \tsp{1,2}Ұ.М.
Қожамұратова,}

{\bfseries \tsp{1,2}Г.К. Мендалиев, \tsp{1}Ә.С.
Акшекина}

\emph{\tsp{1}«Көмір химиясы және технология институты» ЖШС,
Астана, Қазақстан,}

\emph{\tsp{2}Л.Н. Гумилев атындағы Еуразия ұлттық
университеті, Астана, Қазақстан,}

\emph{\tsp{3}Қ.Құлажанов атындағы Қазақ технология және
бизнес университеті, Астана, Қазақстан,}

\emph{е-mail: coaltech@bk.ru,
maira\_1986@mail.ru}

Зерттеу жұмысы Қазақстанның «Шұбаркөл» кен орнының көмірінен алынған
карбонизацияланған адсорбенттерінің сутегіні сорбциялау қасиеттерін
зерттеуге арналған. Адсорбенттерді
K\tsb{2}CO\tsb{3}-мен өңдеп, термиялық процесс
арқылы дайындалды. Алынған адсорбенттердің морфологиялық ерекшеліктері
СЭМ әдісі арқылы анықталып. Өткізгіш электрондық микроскоп (ӨЭМ) әдісі
арқылы материалдардың құрылымы дәлелденді. Зерттеу нәтижелері бойынша ең
үлкен меншікті бетінің ауданы (653.024 м²/г) көрсетті, бұл оның сутекті
тиімді адсорбциялау қабілетін айқындайды.
«Шұбаркөл:K\tsb{2}СО\tsb{3}» 1:0,5, 900ºC
ұнтақталған карбонизацияланған адсорбентінің мезо кеуектерінің
адсорбция-десорбция изотермасым І типке микрокеуекті материалға жатады.
«Шұбаркөл:K\tsb{2}СО\tsb{3}» 1:1, 900ºC ұнтақталған
карбонизацияланған адсорбентінің адсорбция-десорбция изотермасым ІІ
типке макрокеуекті немесе тегіс емес қатты бетке ие екенін көрсетеді.
Осыған байланысты «Шұбаркөл:K\tsb{2}СО\tsb{3}» 1:1
ұнтағы жоғары меншікті бетінің ауданына, дамыған микрокеуекті құрылымына
және сутекті сіңіру қабілетіне байланысты сутегін сақтау үшін оңтайлы
материал болып танылды.

{\bfseries Түйін сөздер:} көмір, адсорбент, кеуектілік, карбонизация,
адсорбция, сутегі.

{\bfseries ХРАНЕНИЕ ВОДОРОДА В ПОРИСТЫХ УГЛЕРОДНЫХ МАТЕРИАЛАХ ПОЛУЧЕННЫХ НА
ОСНОВЕ ШУБАРКУЛЬСКОГО УГЛЯ}

{\bfseries \tsp{1,2,3}М.К. Казанкапова,
\tsp{1,2,3}Б.Т. Ермағамбет, \tsp{1,2,3}Ж.M.
Касенова, \tsp{2}Б.А.Капсалямов,}

{\bfseries \tsp{1,2}А.Б. Малғаждарова,
\tsp{1,3}Ж.Т. Даулетжанова, \tsp{1,2}Ұ.М.
Қожамұратова,}

{\bfseries \tsp{1,2}Г.К. Мендалиев, \tsp{1}Ә.С.
Акшекина}

\emph{\tsp{1}ТОО «Институт химии угля и технологии», Астана,
Казахстан,}

\emph{\tsp{2}Евразийский национальный университет им. Л.Н.
Гумилева, Астана, Казахстан,}

\emph{\tsp{3}Казахский университет технологии и бизнеса
имени К. Кулажанова, Астана, Казахстан,}

\emph{е-mail: coaltech@bk.ru,
maira\_1986@mail.ru}

Целью научно-исследовательской работы было изучение водородосорбционных
свойств карбонизированных адсорбентов, полученных из угля месторождения
«Шубарколь» Казахстана. Адсорбенты были приготовлены путем термической
обработки с K\tsb{2}CO\tsb{3}. Методом СЭМ
определены морфологические особенности полученных адсорбентов. Структура
материалов была подтверждена с помощью просвечивающей электронной
микроскопии (ПЭМ). По результатам исследований показана наибольшая
удельная поверхность (653,024 м²/г), что определяет его эффективную
водородосорбционную способность. Изотерма адсорбции-десорбции мезопор
порошкообразного карбонизированного адсорбента
«Шубарколь:K\tsb{2}CO\tsb{3}» 1:0,5, 900ºC относится
к микропористому материалу I типа. Изотерма адсорбции-десорбции
порошкообразного карбонизированного адсорбента
«Шубарколь:K\tsb{2}CO\tsb{3}» 1:1, 900ºC
свидетельствует о том, что он имеет макропористую или неровную твердую
поверхность II типа. В связи с этим порошок
«Шубаркол:K\tsb{2}CO\tsb{3}» 1:1 признан оптимальным
материалом для хранения водорода благодаря высокой удельной поверхности,
развитой микропористой структуре и способности поглощать водород.

{\bfseries Ключевые слова:} уголь, адсорбент, пористость, карбонизация,
адсорбция, водород.

{\bfseries Introduction.} There is a growing interest in renewable energy
sources as a global strategy towards net-zero carbon emissions {[}1{]}.
This is evident from a massive utilization of renewable energy sources
to produce electricity in some parts of Europe {[}2{]}. More recently,
the use of hydrogen as a clean alternate fuel has seen remarkable
interest from the relevant stakeholders and the scientific community
{[}3-5{]}.

There are main four ways to store of hydrogen: liquefaction, compressed
gas, metal hydrides, and adsorption. So far researches on hydrogen
storage shows that pressurized tank (gaseous H\tsb{2}), liquid
hydrogen, metal hydride (solid hydrogen) and cryoadsorption (condensed
phase of H\tsb{2}) are the most promising alternatives. The
hydrogen adsorption in porous solids and specially activated carbons, is
still considered one interesting and safe alternative. Initially,
carried out researches were based on the use of cryogenic systems, which
are not useful from an economic point of view. Recently, research has
focused on the search for the ideal adsorbent that can be used at room
temperature as well as allows the storage of interesting amounts of
H\tsb{2} such as activated carbons {[}6{]}.

Nowadays, the hydrogen industry in Kazakhstan is in its infancy.
Nevertheless, Kazakhstan is making significant strides in the field of
hydrogen technol-ogies and production. In January 2023, President
Kassym-Jomart Tokayev of Kazakhstan described green hydrogen as a
"promising direction" {[}7{]}.

Hydrogen storage in carbon-based materials is considered a viable
solution to increase hydrogen density, which is currently a major
problem hindering the hydrogen supply chain, which becomes denser when
stored in a porous solid as opposed to pure compressed gas.{[}8{]} The
hydrogen density stored in nanoporous carbon materials can be
approximated to the density of liquid hydrogen, i.e., 70
kg/m\tsp{3}. To achieve such a value, which is four orders
of magnitude higher than the density of gaseous hydrogen at room
temperature and atmospheric pressure (0.089 kg/m\tsp{3}),
the system temperature needs to be significantly reduced. For example,
at 77 K, hydrogen adsorption improves due to enhanced interaction with
the carbon surface via van der Waals forces. Thus, adsorption is the
most effective storage method at 298 K and up to 20 MPa. This means that
under these conditions, the amount of hydrogen stored is always higher
in the presence of the adsorbent than in the case of hydrogen storage by
simple compression in an empty vessel of the same volume. Therefore,
based on carbon, storage will allow the use of a smaller volume hydrogen
tank. Indeed, if 5 kg of hydrogen can be stored at 20 MPa and 298 K in
an empty high-pressure tank of 340 L, the required volume decreases to
263 L in the presence of a carbon adsorbent for storing the same amount
of hydrogen under the same conditions. Such advantages are provided by
the unique texture of porous carbon {[}9{]}.

Activated carbons are one of the widespread used porous carbon materials
which usually have specific surface areas up to 3000
m\tsp{2}/g\tsp{-1} and wide pore sizes ranging
from micropore to mesopore and even macropore, but with abundance of
micropore size less than 1 nm. The highly developed porosity makes
activated carbon a promising candidate for adsorption, separation,
purification and gas storage, especially for hydrogen storage. In
general, the hydrogen storage capacity of activated carbon has appeared
to be proportional to the surface area and micropore volume.
Experimental findings showed hydrogen uptake values in the range of
0.2--5.5 wt\%, with a clear dependency on the activated carbons
evaluated. Moreover, the majority of experimental results have
demonstrated that H\tsb{2} uptakes of about 2.5 wt\% at low
pressures (1--10 bar) and 5.5 wt\% at high pressures (up to 60 bar) and
77 K were achievable. Again, the highest H\tsb{2} adsorption
measured at 100 bar and room temperature is below 1 wt\%, even with
highly developed pore structures and specific surface areas up to 2800
m\tsp{2}/g\tsp{-1} {[}10-13{]}.

The hydrogen storage methods motivated us to study hydrogen adsorption
on several porous carbon adsorbents of various origins that differed in
the porous structure and chemical state of the surface at the near
ambient temperatures and not-extreme high pressures. Therefore, this
study was aimed to determine the factors affecting the efficiency of
hydrogen adsorption-based storage with AC as an adsorbent, namely the
influence of thermodynamic conditions and structural and energy
characteristics.

{\bfseries Materials and methods.} There are various approaches to
producing carbon materials, which include the preparation and
modification of the initial coal, carbonization, and subsequent
activation using gas or chemical agents. One of the promising methods
for obtaining porous carbon materials from carbonaceous feedstock is the
use of activating agents during thermal processing. When in contact with
salt, the lignite structure begins to rearrange even at room
temperature, and upon heating, the salt promotes the development of
specific surface area, as well as an increase in total pore volume and
micropore volume. K₂CO₃ is considered a more effective activating agent
compared to Na₂CO₃. The enhanced efficiency of K₂CO₃ is attributed to
the larger ionic radius of potassium (0.267 nm) in comparison to that of
sodium (0.190 nm). The activation atmosphere (N₂, CO₂, or H₂O) also
influences the structural properties of the activated carbon. It has
been found that, compared to CO₂ and steam, nitrogen serves as a good
alternative as an activation medium. Increasing the mass ratio of salt
to coal, the heating temperature, and the holding time leads to higher
porosity and a larger specific surface area of the resulting carbon.

The experimental setup and procedure (hydrogen adsorption) were as
follows: The hydrogen saturation unit for porous carbon material
consisted of several sections. The hydrogen source was a QL 500 series
hydrogen generator with a maximum output of 500
cm\tsp{3}/min, and the hydrogen generator used water from a
Thermo Scientific deionizer. The total flow rate was monitored using a
rotameter.

An autoclaved glass reactor with a working volume of up to 0.5 liters
was used as a laboratory reactor. The reactor vessel base has stainless
steel legs. The reactor vessel design is based on a glass cylinder,
which is fixed between the top cover and the metal bottom with a bottom
drain valve. Temperature control is achieved by circulating
thermostatically controlled water through a jacket at the bottom of the
reactor.

The pressure was measured every minute. The initial pressure was -0.75
bar, and a vacuum environment was created using a pump. The
H\tsb{2} absorption process was carried out at temperatures of
298 K, 313 K, 343 K. The mass of the porous carbon material was 97 g.
The process was continued until the pressure in the reactor reached a
constant value. The absorption capacity was calculated from the gas
balance (based on the measured amount of H\tsb{2} and the
known gas flow rate).

The aim of the work is to study the method for obtaining carbon sorbents
from oxidized coal "Shubarkol" with a developed structure and high
adsorption characteristics, as well as the use of the obtained
adsorbents for hydrogen storage.

Chemical analysis and surface morphology were studied by
energy-dispersive X-ray spectroscopy using a SEM (Quanta 3D 200i) with
an EDAX energy-dispersive analysis attachment. Specific surface area
analysis was performed using a 3Flex 3500 high-performance adsorption
analyzer (Micromeritics, USA) with a standard SmartVacprep programmable
degasser.

{\bfseries Results and discussion.} The results of the physicochemical
characteristics of the samples and the elemental composition are
presented in Tables 1 and 2.

{\bfseries Table 1 - Results of physicochemical analysis of
"Shubarkol/К\tsb{2}СО\tsb{3}" 1:0.5; 1:1 adsorbents}

%% \begin{longtable}[]{@{}
%%   >{\centering\arraybackslash}p{(\linewidth - 16\tabcolsep) * \real{0.2415}}
%%   >{\centering\arraybackslash}p{(\linewidth - 16\tabcolsep) * \real{0.0968}}
%%   >{\centering\arraybackslash}p{(\linewidth - 16\tabcolsep) * \real{0.0969}}
%%   >{\centering\arraybackslash}p{(\linewidth - 16\tabcolsep) * \real{0.0968}}
%%   >{\centering\arraybackslash}p{(\linewidth - 16\tabcolsep) * \real{0.0807}}
%%   >{\centering\arraybackslash}p{(\linewidth - 16\tabcolsep) * \real{0.0968}}
%%   >{\centering\arraybackslash}p{(\linewidth - 16\tabcolsep) * \real{0.0969}}
%%   >{\centering\arraybackslash}p{(\linewidth - 16\tabcolsep) * \real{0.0968}}
%%   >{\centering\arraybackslash}p{(\linewidth - 16\tabcolsep) * \real{0.0969}}@{}}
%% \toprule\noalign{}
%% \begin{minipage}[b]{\linewidth}\centering
%% {\bfseries Name of adsorbents}
%% \end{minipage} & \begin{minipage}[b]{\linewidth}\centering
%% {\bfseries \emph{W\tsp{r}\tsb{t}},\%}
%% \end{minipage} & \begin{minipage}[b]{\linewidth}\centering
%% {\bfseries \emph{A\tsp{r}},\%}
%% \end{minipage} & \begin{minipage}[b]{\linewidth}\centering
%% {\bfseries \emph{V\tsp{d}}, \%}
%% \end{minipage} & \begin{minipage}[b]{\linewidth}\centering
%% {\bfseries \emph{V}\tsb{Σ (water),} cm\tsp{3}/g}
%% \end{minipage} & \begin{minipage}[b]{\linewidth}\centering
%% {\bfseries ρ\tsb{bulk}, g/cm\tsp{3}}
%% \end{minipage} & \begin{minipage}[b]{\linewidth}\centering
%% {\bfseries рН\tsb{aqueous extract}}
%% \end{minipage} & \begin{minipage}[b]{\linewidth}\centering
%% {\bfseries \emph{А\tsb{m.o.}},}
%% 
%% {\bfseries mg/g}
%% \end{minipage} & \begin{minipage}[b]{\linewidth}\centering
%% {\bfseries \emph{А\tsb{m.b.}}, mg/g}
%% \end{minipage} \\
%% \midrule\noalign{}
%% \endhead
%% \bottomrule\noalign{}
%% \endlastfoot
%% "Shubarkol-K\tsb{2}CO\tsb{3}" (1:0.5, 900°C) powder
%% carbonized adsorbent & 12.73 & 8.83 & 52.95 & 1.63 & 0.76 & 7.41 & 33.23
%% & 34.31 \\
%% "Shubarkol-K\tsb{2}CO\tsb{3}" (1:1, 900°C) powder
%% carbonized adsorbent & 11.45 & 8.85 & 51.25 & 1.83 & 0.72 & 7.05 & 38.45
%% & 40.12 \\
%% \end{longtable}

{\bfseries Table 2 - Results of the elemental analysis of the samples}

%% \begin{longtable}[]{@{}
%%   >{\centering\arraybackslash}p{(\linewidth - 20\tabcolsep) * \real{0.2601}}
%%   >{\centering\arraybackslash}p{(\linewidth - 20\tabcolsep) * \real{0.0877}}
%%   >{\centering\arraybackslash}p{(\linewidth - 20\tabcolsep) * \real{0.0738}}
%%   >{\centering\arraybackslash}p{(\linewidth - 20\tabcolsep) * \real{0.0732}}
%%   >{\centering\arraybackslash}p{(\linewidth - 20\tabcolsep) * \real{0.0733}}
%%   >{\centering\arraybackslash}p{(\linewidth - 20\tabcolsep) * \real{0.0624}}
%%   >{\centering\arraybackslash}p{(\linewidth - 20\tabcolsep) * \real{0.0624}}
%%   >{\centering\arraybackslash}p{(\linewidth - 20\tabcolsep) * \real{0.0732}}
%%   >{\centering\arraybackslash}p{(\linewidth - 20\tabcolsep) * \real{0.0733}}
%%   >{\centering\arraybackslash}p{(\linewidth - 20\tabcolsep) * \real{0.0875}}
%%   >{\centering\arraybackslash}p{(\linewidth - 20\tabcolsep) * \real{0.0732}}@{}}
%% \toprule\noalign{}
%% \multirow{2}{=}{\begin{minipage}[b]{\linewidth}\centering
%% {\bfseries Name}
%% \end{minipage}} &
%% \multicolumn{10}{>{\centering\arraybackslash}p{(\linewidth - 20\tabcolsep) * \real{0.7399} + 18\tabcolsep}@{}}{%
%% \begin{minipage}[b]{\linewidth}\centering
%% {\bfseries Elemental content, mass \%}
%% \end{minipage}} \\
%% & \begin{minipage}[b]{\linewidth}\centering
%% C
%% \end{minipage} & \begin{minipage}[b]{\linewidth}\centering
%% O
%% \end{minipage} & \begin{minipage}[b]{\linewidth}\centering
%% Na
%% \end{minipage} & \begin{minipage}[b]{\linewidth}\centering
%% Mg
%% \end{minipage} & \begin{minipage}[b]{\linewidth}\centering
%% Al
%% \end{minipage} & \begin{minipage}[b]{\linewidth}\centering
%% Si
%% \end{minipage} & \begin{minipage}[b]{\linewidth}\centering
%% S
%% \end{minipage} & \begin{minipage}[b]{\linewidth}\centering
%% K
%% \end{minipage} & \begin{minipage}[b]{\linewidth}\centering
%% Ca
%% \end{minipage} & \begin{minipage}[b]{\linewidth}\centering
%% Fe
%% \end{minipage} \\
%% \midrule\noalign{}
%% \endhead
%% \bottomrule\noalign{}
%% \endlastfoot
%% "Shubarkol-K\tsb{2}CO\tsb{3}" (1:0.5, 900°C) powder
%% carbonized adsorbent & 53.01 & 18.69 & 0.27 & 0.35 & 1.81 & 1.07 & 0.44
%% & 11.56 & 6.70 & 6.10 \\
%% "Shubarkol-K\tsb{2}CO\tsb{3}" (1:1, 900°C) powder
%% carbonized adsorbent & 68.24 & 13.88 & 0.15 & 0.54 & 1.01 & 1.03 & 0.31
%% & 1.53 & 8.28 & 5.03 \\
%% \end{longtable}

The physicochemical properties of porous carbon materials obtained as a
result of the carbonization process were determined by the adsorption
capacity according to methylene blue and methyl orange indicators. As a
result, the adsorbent "Shubarkol-K\tsb{2}CO\tsb{3}"
in the ratio of 1:1 showed a relatively high adsorption capacity. In
terms of elemental composition, the adsorbent in the ratio of 1:1 had a
higher carbon content. The higher the carbon content, the more porous
the adsorbent obtained and the more suitable it is for further use as an
adsorbent.

The morphology of the obtained adsorbents was investigated at various
scales using scanning electron microscopy (SEM). Figures 1 and 2 show
the results of SEM analysis of the
"Shubarkol-K\tsb{2}CO\tsb{3}" 1:0.5 900°C powder
carbonized adsorbent and the
"Shubarkol-K\tsb{2}CO\tsb{3}" 1:1 900°C powder
carbonized adsorbent.

%% \begin{longtable}[]{@{}
%%   >{\centering\arraybackslash}p{(\linewidth - 2\tabcolsep) * \real{0.5009}}
%%   >{\centering\arraybackslash}p{(\linewidth - 2\tabcolsep) * \real{0.4991}}@{}}
%% \toprule\noalign{}
%% \begin{minipage}[b]{\linewidth}\raggedright
%% \fig{c/image7.tiff}{}
%% \end{minipage} & \begin{minipage}[b]{\linewidth}\centering
%% \fig{c/image8.tiff}{}
%% \end{minipage} \\
%% \midrule\noalign{}
%% \endhead
%% \bottomrule\noalign{}
%% \endlastfoot
%% a & b \\
%% \fig{c/image9.tiff}{} &
%% \fig{c/image10.tiff}{} \\
%% c & d \\
%% \end{longtable}

{\bfseries Fig.1 - SEM results of the powder carbonized adsorbent
"Shubarkol-K\tsb{2}CO\tsb{3}" 1:0.5 at 900°C:
a,b-х5000; c-х20000; d-х50000}

%% \begin{longtable}[]{@{}
%%   >{\centering\arraybackslash}p{(\linewidth - 2\tabcolsep) * \real{0.4999}}
%%   >{\centering\arraybackslash}p{(\linewidth - 2\tabcolsep) * \real{0.5001}}@{}}
%% \toprule\noalign{}
%% \begin{minipage}[b]{\linewidth}\raggedright
%% \fig{c/image11.tiff}{}
%% \end{minipage} & \begin{minipage}[b]{\linewidth}\centering
%% \fig{c/image12.tiff}{}
%% \end{minipage} \\
%% \midrule\noalign{}
%% \endhead
%% \bottomrule\noalign{}
%% \endlastfoot
%% a & b \\
%% \fig{c/image13.tiff}{} &
%% \fig{c/image14.tiff}{} \\
%% c & d \\
%% \end{longtable}

{\bfseries Fig.2 - SEM results of the powder carbonized adsorbent
"Shubarkol-K\tsb{2}CO\tsb{3}" 1:1 at 900°C: a-х200;
b-x1000; c-х10000; d-х20000}

As a result of the SEM study, porous structures consisting of large and
small particles are clearly visible in the samples carbonized at a
temperature of 900°C. It is known that the presence of developed
micropores on the surface layer contributes to an increase in the
adsorption capacity. Figures 3 and 4 show the results of TEM analysis of
the "Shubarkol-K\tsb{2}CO\tsb{3}" 1:0.5 900°C powder
carbonized adsorbent and the
"Shubarkol-K\tsb{2}CO\tsb{3}" 1:1 900°C powder
carbonized adsorbent.

%% \begin{longtable}[]{@{}
%%   >{\centering\arraybackslash}p{(\linewidth - 2\tabcolsep) * \real{0.4968}}
%%   >{\centering\arraybackslash}p{(\linewidth - 2\tabcolsep) * \real{0.5032}}@{}}
%% \toprule\noalign{}
%% \begin{minipage}[b]{\linewidth}\raggedright
%% \fig{c/image15}{}
%% \end{minipage} & \begin{minipage}[b]{\linewidth}\raggedright
%% \fig{c/image16}{}
%% \end{minipage} \\
%% \midrule\noalign{}
%% \endhead
%% \bottomrule\noalign{}
%% \endlastfoot
%% a & b \\
%% \fig{c/image17}{} &
%% \fig{c/image18}{} \\
%% c & d \\
%% \end{longtable}

{\bfseries Fig.3 - TEM results of the powder carbonized adsorbent
"Shubarkol-K\tsb{2}CO\tsb{3}" 1:0.5 at 900°C}

%% \begin{longtable}[]{@{}
%%   >{\centering\arraybackslash}p{(\linewidth - 2\tabcolsep) * \real{0.4968}}
%%   >{\centering\arraybackslash}p{(\linewidth - 2\tabcolsep) * \real{0.5032}}@{}}
%% \toprule\noalign{}
%% \begin{minipage}[b]{\linewidth}\raggedright
%% \fig{c/image19}{}
%% \end{minipage} & \begin{minipage}[b]{\linewidth}\raggedright
%% \fig{c/image20}{}
%% \end{minipage} \\
%% \midrule\noalign{}
%% \endhead
%% \bottomrule\noalign{}
%% \endlastfoot
%% a & b \\
%% \fig{c/image21}{} &
%% \fig{c/image22}{} \\
%% c & d \\
%% \end{longtable}

{\bfseries Fig.4 - TEM results of the powder carbonized adsorbent
"Shubarkol-K\tsb{2}CO\tsb{3}" 1:1 at 900°C}

The results of the TEM images provide clear images of the internal
structure of the carbon adsorbent. As shown in Figure 3, the particles
are assembled in the form of large agglomerates, and at their edges
there are clearly visible graphitized layers and curved structures.
These graphitized structures may increase the thermal stability of the
adsorbent. The images presented in Figure 4 show the formation of
nanostructured particles, as well as clearly visible flaky and layered
structures. This indicates that the carbonization process contributed to
the formation of nanoparticles with different morphologies within the
carbon matrix. In general, the obtained microscopic data indicate that
the carbon materials synthesized using Shubarkol coal and K₂CO₃ are
highly structured, multi-porous and nanostructured. Such properties
allow these adsorbents to be effectively used in gas or hydrogen
storage.

%% \begin{longtable}[]{@{}
%%   >{\centering\arraybackslash}p{(\linewidth - 0\tabcolsep) * \real{1.0000}}@{}}
%% \toprule\noalign{}
%% \begin{minipage}[b]{\linewidth}\centering
%% \fig{c/image23}{}
%% \end{minipage} \\
%% \midrule\noalign{}
%% \endhead
%% \bottomrule\noalign{}
%% \endlastfoot
%% a \\
%% \fig{c/image24}{} \\
%% b \\
%% \end{longtable}

{\bfseries Fig.5 - Adsorption curves of the powder carbonized adsorbent
"Shubarkol-K\tsb{2}CO\tsb{3}" a-1:0.5; b-1:1 (based
on N\tsb{2}, by the BET method)}

{\bfseries Table 3 - Adsorption characteristics of samples
(N\tsb{2})}

%% \begin{longtable}[]{@{}
%%   >{\centering\arraybackslash}p{(\linewidth - 14\tabcolsep) * \real{0.1812}}
%%   >{\centering\arraybackslash}p{(\linewidth - 14\tabcolsep) * \real{0.1058}}
%%   >{\centering\arraybackslash}p{(\linewidth - 14\tabcolsep) * \real{0.1208}}
%%   >{\centering\arraybackslash}p{(\linewidth - 14\tabcolsep) * \real{0.0937}}
%%   >{\centering\arraybackslash}p{(\linewidth - 14\tabcolsep) * \real{0.0876}}
%%   >{\centering\arraybackslash}p{(\linewidth - 14\tabcolsep) * \real{0.0906}}
%%   >{\centering\arraybackslash}p{(\linewidth - 14\tabcolsep) * \real{0.1661}}
%%   >{\centering\arraybackslash}p{(\linewidth - 14\tabcolsep) * \real{0.1542}}@{}}
%% \toprule\noalign{}
%% \begin{minipage}[b]{\linewidth}\centering
%% {\bfseries Name}
%% \end{minipage} & \begin{minipage}[b]{\linewidth}\centering
%% {\bfseries S\tsb{BET}, m\tsp{2}/g}
%% \end{minipage} & \begin{minipage}[b]{\linewidth}\centering
%% {\bfseries Total pore volume, cm³/g}
%% \end{minipage} & \begin{minipage}[b]{\linewidth}\centering
%% {\bfseries Micropores, cm³/g}
%% \end{minipage} & \begin{minipage}[b]{\linewidth}\centering
%% {\bfseries Mesopores, cm³/g}
%% \end{minipage} & \begin{minipage}[b]{\linewidth}\centering
%% {\bfseries Macropores, cm³/g}
%% \end{minipage} & \begin{minipage}[b]{\linewidth}\centering
%% {\bfseries SBJH, for pore width from 17,000 Å to 3,000,000 Å, m²/g}
%% \end{minipage} & \begin{minipage}[b]{\linewidth}\centering
%% {\bfseries Average width of adsorption pores BJH (4V/A), Å}
%% \end{minipage} \\
%% \midrule\noalign{}
%% \endhead
%% \bottomrule\noalign{}
%% \endlastfoot
%% "Shubarkol-K\tsb{2}CO\tsb{3}" (1:0.5, 900°C) powder
%% carbonized adsorbent & 676.889 & 0.318 & 0.276 & 0.036 & 0.006 & 42.2087
%% & 37.616 \\
%% "Shubarkol-K\tsb{2}CO\tsb{3}" (1:1, 900°C) powder
%% carbonized adsorbent & 653.024 & 0.435 & 0.205 & 0.183 & 0.048 &
%% 137.2048 & 62.144 \\
%% \end{longtable}

%% \begin{longtable}[]{@{}
%%   >{\centering\arraybackslash}p{(\linewidth - 0\tabcolsep) * \real{1.0000}}@{}}
%% \toprule\noalign{}
%% \begin{minipage}[b]{\linewidth}\centering
%% \fig{c/image25}{}
%% \end{minipage} \\
%% \midrule\noalign{}
%% \endhead
%% \bottomrule\noalign{}
%% \endlastfoot
%% a \\
%% \fig{c/image26}{} \\
%% b \\
%% \end{longtable}

{\bfseries Fig.6 - Adsorption curves of the powder carbonized adsorbent
"Shubarkol-K\tsb{2}CO\tsb{3}" a-1:0.5; b-1:1 (based
on CO\tsb{2})}

{\bfseries Table 4 -- Results of the study of sorption of PCM (by
CO\tsb{2})}

%% \begin{longtable}[]{@{}
%%   >{\centering\arraybackslash}p{(\linewidth - 14\tabcolsep) * \real{0.1328}}
%%   >{\centering\arraybackslash}p{(\linewidth - 14\tabcolsep) * \real{0.1244}}
%%   >{\centering\arraybackslash}p{(\linewidth - 14\tabcolsep) * \real{0.1213}}
%%   >{\raggedright\arraybackslash}p{(\linewidth - 14\tabcolsep) * \real{0.1365}}
%%   >{\raggedright\arraybackslash}p{(\linewidth - 14\tabcolsep) * \real{0.1061}}
%%   >{\raggedright\arraybackslash}p{(\linewidth - 14\tabcolsep) * \real{0.1213}}
%%   >{\raggedright\arraybackslash}p{(\linewidth - 14\tabcolsep) * \real{0.1365}}
%%   >{\raggedright\arraybackslash}p{(\linewidth - 14\tabcolsep) * \real{0.1213}}@{}}
%% \toprule\noalign{}
%% \multirow{2}{=}{\begin{minipage}[b]{\linewidth}\centering
%% {\bfseries Name}
%% \end{minipage}} &
%% \multicolumn{3}{>{\centering\arraybackslash}p{(\linewidth - 14\tabcolsep) * \real{0.3821} + 4\tabcolsep}}{%
%% \begin{minipage}[b]{\linewidth}\centering
%% {\bfseries Dubinin-Radushkevich method}
%% \end{minipage}} &
%% \multicolumn{4}{>{\centering\arraybackslash}p{(\linewidth - 14\tabcolsep) * \real{0.4851} + 6\tabcolsep}@{}}{%
%% \begin{minipage}[b]{\linewidth}\centering
%% {\bfseries Dubinin-Astakhov method}
%% \end{minipage}} \\
%% & \begin{minipage}[b]{\linewidth}\centering
%% {\bfseries Maximum capacity of micropores, mmol/g}
%% \end{minipage} & \begin{minipage}[b]{\linewidth}\centering
%% {\bfseries Maximum volume of micropores, cm³/g}
%% \end{minipage} & \begin{minipage}[b]{\linewidth}\centering
%% {\bfseries Equivalent surface area, m²/g}
%% \end{minipage} & \begin{minipage}[b]{\linewidth}\centering
%% {\bfseries Maximum capacity of micropores, mmol/g}
%% \end{minipage} & \begin{minipage}[b]{\linewidth}\centering
%% {\bfseries Micropore volume limitation, cm³/g}
%% \end{minipage} & \begin{minipage}[b]{\linewidth}\centering
%% {\bfseries Equivalent surface area, m²/g}
%% \end{minipage} & \begin{minipage}[b]{\linewidth}\centering
%% {\bfseries Average equivalent pore width, Å}
%% \end{minipage} \\
%% \midrule\noalign{}
%% \endhead
%% \bottomrule\noalign{}
%% \endlastfoot
%% "Shubarkol-K\tsb{2}CO\tsb{3}" (1:0.5, 900°C) powder
%% carbonized adsorbent & 7.9279 & 0.377916 & 811.635593 & 8.68047 &
%% 0.413789 & 1359.46542 & 10.93346 \\
%% "Shubarkol-K\tsb{2}CO\tsb{3}" (1:1, 900°C) powder
%% carbonized adsorbent & 4.3212 & 0.205985 & 442.386469 & 9.96985 &
%% 0.475252 & 1308.89576 & 14.52375 \\
%% \end{longtable}

{\bfseries Fig.7- shows the types of adsorption-desorption isotherms}

\fig{c/image27}{}

{\bfseries Fig.7 - Types of adsorption-desorption isotherms}

As a result, the types of adsorption-desorption isotherms of the
obtained adsorbents were determined and a conclusion was drawn. The
adsorption-desorption isotherm of the mesopores of the powdered
carbonized adsorbent ``Shubarkol:K\tsb{2}СО\tsb{3}''
1:0.5, 900ºC belongs to type I. The isotherm of the obtained sample
belongs to type I - this is a microporous material. The
adsorption-desorption isotherm of the powdered carbonized adsorbent
``Shubarkol:K\tsb{2}СО\tsb{3}'' 1:1, 900ºC belongs
to type II. The isotherm of the obtained sample belongs to type II -
this type indicates that the adsorbent has a macroporous or uneven solid
surface.

The time dependence of hydrogen adsorption was determined using a
laboratory device for powdered carbonized adsorbents
"Shubarkol:K\tsb{2}СО\tsb{3}" 1:0.5; 1:1 900 ºС.
During the study, the dynamics of the hydrogen adsorption process on
porous carbon materials at temperatures of 298 K, 313 K, 343 K were
studied. The time dependence of the amount of adsorbed hydrogen was
considered, which allows us to get an idea of
\hspace{0pt}\hspace{0pt}the speed and efficiency of the hydrogen
adsorption process. Analysis of the hydrogen adsorption rate at each
time interval showed that the adsorption rate increases at the beginning
of the process, reaching a maximum of 23 minutes at T=298 K, 18 minutes
at T=313 K and 12 minutes at T=343 K, and then gradually stabilizes.
This may indicate saturation of the adsorbent surface and a decrease in
active centers available for hydrogen adsorption. The results of the
study are shown in Table 5.

{\bfseries Table 5 - Results of the study of hydrogen sorption with PCM}

%% \begin{longtable}[]{@{}
%%   >{\raggedright\arraybackslash}p{(\linewidth - 4\tabcolsep) * \real{0.3337}}
%%   >{\raggedright\arraybackslash}p{(\linewidth - 4\tabcolsep) * \real{0.3331}}
%%   >{\raggedright\arraybackslash}p{(\linewidth - 4\tabcolsep) * \real{0.3332}}@{}}
%% \toprule\noalign{}
%% \multicolumn{3}{@{}>{\centering\arraybackslash}p{(\linewidth - 4\tabcolsep) * \real{1.0000} + 4\tabcolsep}@{}}{%
%% \begin{minipage}[b]{\linewidth}\centering
%% {\bfseries "Shubarkol-K\tsb{2}CO\tsb{3}" (1:0.5, 900°C)
%% powder carbonized adsorbent}
%% \end{minipage}} \\
%% \midrule\noalign{}
%% \endhead
%% \bottomrule\noalign{}
%% \endlastfoot
%% Temperature & Adsorption H\tsb{2} \% & Adsorption
%% H\tsb{2} cm\tsp{3}/kg \\
%% 25 ºС & 0.637538 & 0.714043 \\
%% 40 ºС & 0.595036 & 0.666440 \\
%% 70 ºС & 0.616287 & 0.690241 \\
%% \multicolumn{3}{@{}>{\centering\arraybackslash}p{(\linewidth - 4\tabcolsep) * \real{1.0000} + 4\tabcolsep}@{}}{%
%% "Shubarkol-K\tsb{2}CO\tsb{3}" (1:1, 900°C) powder
%% carbonized adsorbent} \\
%% Temperature & Adsorption H\tsb{2} \% & Adsorption
%% H\tsb{2} cm\tsp{3}/kg \\
%% 25 ºС & 0.727543 & 0.814849 \\
%% 40 ºС & 0.679041 & 0.760525 \\
%% 70 ºС & 0.703292 & 0.787687 \\
%% \end{longtable}

\fig{c/image28}{}

{\bfseries Fig.8 - Dynamics of hydrogen adsorption process on porous
carbon material "Shubarkol:K\tsb{2}CO\tsb{3}" 1:0.5
at temperatures of 298 K, \hspace{0pt}\hspace{0pt}313 K, 343 K}

{\bfseries Table 5 - Calculation of the hydrogen adsorption process on
porous carbon material "Shubarkol:K\tsb{2}CO\tsb{3}"
1:0.5, \%}

%% \begin{longtable}[]{@{}
%%   >{\raggedleft\arraybackslash}p{(\linewidth - 8\tabcolsep) * \real{0.2214}}
%%   >{\raggedleft\arraybackslash}p{(\linewidth - 8\tabcolsep) * \real{0.1828}}
%%   >{\raggedleft\arraybackslash}p{(\linewidth - 8\tabcolsep) * \real{0.2013}}
%%   >{\raggedleft\arraybackslash}p{(\linewidth - 8\tabcolsep) * \real{0.1821}}
%%   >{\raggedleft\arraybackslash}p{(\linewidth - 8\tabcolsep) * \real{0.2124}}@{}}
%% \toprule\noalign{}
%% \multirow{2}{=}{\begin{minipage}[b]{\linewidth}\raggedleft
%% {\bfseries Temperature, К}
%% \end{minipage}} &
%% \multicolumn{4}{>{\centering\arraybackslash}p{(\linewidth - 8\tabcolsep) * \real{0.7786} + 6\tabcolsep}@{}}{%
%% \begin{minipage}[b]{\linewidth}\centering
%% {\bfseries Time, min}
%% \end{minipage}} \\
%% & \begin{minipage}[b]{\linewidth}\raggedleft
%% {\bfseries 10}
%% \end{minipage} & \begin{minipage}[b]{\linewidth}\raggedleft
%% {\bfseries 12}
%% \end{minipage} & \begin{minipage}[b]{\linewidth}\raggedleft
%% {\bfseries 14}
%% \end{minipage} & \begin{minipage}[b]{\linewidth}\raggedleft
%% {\bfseries 16}
%% \end{minipage} \\
%% \midrule\noalign{}
%% \endhead
%% \bottomrule\noalign{}
%% \endlastfoot
%% 273 & 0.350 & 0.557 & 0.786 & 1.037 \\
%% 278 & 0.318 & 0.524 & 0.754 & 1.008 \\
%% 283 & 0.288 & 0.494 & 0.725 & 0.982 \\
%% 288 & 0.261 & 0.466 & 0.699 & 0.960 \\
%% 293 & 0.235 & 0.441 & 0.677 & 0.941 \\
%% 298 & 0.211 & 0.419 & 0.657 & 0.926 \\
%% 303 & 0.189 & 0.399 & 0.640 & 0.914 \\
%% 308 & 0.169 & 0.381 & 0.627 & 0.907 \\
%% 313 & 0.152 & 0.367 & 0.617 & 0.902 \\
%% 318 & 0.136 & 0.354 & 0.610 & 0.902 \\
%% 323 & 0.122 & 0.345 & 0.606 & 0.905 \\
%% 328 & 0.110 & 0.338 & 0.605 & 0.911 \\
%% 333 & 0.100 & 0.333 & 0.607 & 0.922 \\
%% 338 & 0.093 & 0.331 & 0.612 & 0.936 \\
%% 343 & 0.087 & 0.332 & 0.621 & 0.953 \\
%% 348 & 0.083 & 0.335 & 0.632 & 0.974 \\
%% 353 & 0.081 & 0.341 & 0.647 & 0.999 \\
%% 358 & 0.081 & 0.350 & 0.665 & - \\
%% 363 & 0.084 & 0.361 & 0.686 & - \\
%% 368 & 0.088 & 0.374 & 0.710 & - \\
%% 373 & 0.094 & 0.390 & 0.737 & - \\
%% 378 & 0.102 & 0.409 & 0.768 & - \\
%% \end{longtable}

\fig{c/image29}{}

{\bfseries Fig.9 - Dynamics of hydrogen adsorption process on porous
carbon material of "Shubarkol:K\tsb{2}CO\tsb{3}"
1:0.5 adsorbent}

This porous carbon from Shubarkol coal (chemical activation K₂CO₃ 1:0.5)
is capable of absorbing up to \textasciitilde0.8 wt\% hydrogen at room
temperature (\textasciitilde298 K) for \textasciitilde15 minutes,
indicating its suitability for rapid hydrogen sorption. Increasing the
temperature to 343 K reduces the adsorption threshold by about 10\%
relative to 298 K, confirming the negative effect of temperature on
hydrogen storage in carbonaceous materials. Thus, for maximum hydrogen
adsorption efficiency, low temperatures and sufficient retention times
(about 15 minutes to saturation) are suitable for this material. These
results may be useful when designing hydrogen storage systems,
indicating the need to cool the adsorbent to increase its capacity and
take into account the kinetic characteristics of pore filling.

\fig{c/image30}{}

{\bfseries Fig.10 - Dynamics of the process of hydrogen adsorption on
porous carbon material "Shubarkol:K\tsb{2}CO\tsb{3}
1:0.5"}

\fig{c/image31}{}

{\bfseries Fig.11 - Dynamics of hydrogen adsorption process at
temperatures of 298 K, 313 K, 343 K on a porous carbon material weighing
97 g}

\fig{c/image32}{}

{\bfseries Fig.12 - Dynamics of the process of hydrogen adsorption on
porous carbon material "Shubarkol:K\tsb{2}CO\tsb{3}
1:0.5" at temperatures of 298 K, \hspace{0pt}\hspace{0pt}313 K, 343 K}

If we calculate it in a similar way, the activation energy of hydrogen
adsorption on this carbon material is approximately -1.6 kJ/mol--1.6
kJ/mol. The negative value of Ea indicates that the adsorption rate
decreases with increasing temperature. This behavior is typical of
exothermic processes of physical adsorption: as the temperature
increases, it becomes more difficult for molecules to attach to the
surface, and the efficiency of adsorption by the adsorbate decreases. A
practically negative (close to zero) activation energy indicates the
absence of a significant energy barrier for adsorption - the process
proceeds practically without an activation limit, and the slight
decrease in the rate with increasing temperature is due to thermodynamic
factors (weakening of the binding to the adsorbate). Based on the
Arrhenius plot and linear regression, the calculated activation energy
of hydrogen adsorption is approximately 1.6 kJ/mol (with a minus sign).
This is a very low absolute value, close to zero, which is consistent
with the idea of \hspace{0pt}\hspace{0pt}physical adsorption of hydrogen
on porous carbon without a significant energy barrier.

\fig{c/image33}{}

{\bfseries Fig.13 - Dynamics of hydrogen adsorption process on
K\tsb{2}CO\tsb{3} 1:1 porous carbon material of
"Shubarkol" coal at temperatures of 298 K, 313 K, 343 K}

{\bfseries Table 6 -- Dynamics of the hydrogen adsorption process in porous
carbon material of coal "Shubarkol:K\tsb{2}CO\tsb{3}
1:1", \%}

%% \begin{longtable}[]{@{}
%%   >{\raggedleft\arraybackslash}p{(\linewidth - 20\tabcolsep) * \real{0.1539}}
%%   >{\raggedleft\arraybackslash}p{(\linewidth - 20\tabcolsep) * \real{0.0674}}
%%   >{\raggedleft\arraybackslash}p{(\linewidth - 20\tabcolsep) * \real{0.0828}}
%%   >{\raggedleft\arraybackslash}p{(\linewidth - 20\tabcolsep) * \real{0.0829}}
%%   >{\raggedleft\arraybackslash}p{(\linewidth - 20\tabcolsep) * \real{0.0966}}
%%   >{\raggedleft\arraybackslash}p{(\linewidth - 20\tabcolsep) * \real{0.0931}}
%%   >{\raggedleft\arraybackslash}p{(\linewidth - 20\tabcolsep) * \real{0.0846}}
%%   >{\raggedleft\arraybackslash}p{(\linewidth - 20\tabcolsep) * \real{0.0846}}
%%   >{\raggedleft\arraybackslash}p{(\linewidth - 20\tabcolsep) * \real{0.0846}}
%%   >{\raggedleft\arraybackslash}p{(\linewidth - 20\tabcolsep) * \real{0.0846}}
%%   >{\raggedleft\arraybackslash}p{(\linewidth - 20\tabcolsep) * \real{0.0850}}@{}}
%% \toprule\noalign{}
%% \multirow{2}{=}{\begin{minipage}[b]{\linewidth}\raggedleft
%% Temperature, К
%% \end{minipage}} &
%% \multicolumn{10}{>{\centering\arraybackslash}p{(\linewidth - 20\tabcolsep) * \real{0.8461} + 18\tabcolsep}@{}}{%
%% \begin{minipage}[b]{\linewidth}\centering
%% Time, min
%% \end{minipage}} \\
%% & \begin{minipage}[b]{\linewidth}\raggedleft
%% 0
%% \end{minipage} & \begin{minipage}[b]{\linewidth}\raggedleft
%% 1
%% \end{minipage} & \begin{minipage}[b]{\linewidth}\raggedleft
%% 3
%% \end{minipage} & \begin{minipage}[b]{\linewidth}\raggedleft
%% 7
%% \end{minipage} & \begin{minipage}[b]{\linewidth}\raggedleft
%% 10
%% \end{minipage} & \begin{minipage}[b]{\linewidth}\raggedleft
%% 15
%% \end{minipage} & \begin{minipage}[b]{\linewidth}\raggedleft
%% 19
%% \end{minipage} & \begin{minipage}[b]{\linewidth}\raggedleft
%% 22
%% \end{minipage} & \begin{minipage}[b]{\linewidth}\raggedleft
%% 25
%% \end{minipage} & \begin{minipage}[b]{\linewidth}\raggedleft
%% 30
%% \end{minipage} \\
%% \midrule\noalign{}
%% \endhead
%% \bottomrule\noalign{}
%% \endlastfoot
%% 273 & 0 & 0 & 0.104 & 0.302 & 0.432 & 0.615 & 0.731 & 0.799 & 0.853 &
%% 0.907 \\
%% 280 & 0 & 0.004 & 0.099 & 0.274 & 0.392 & 0.564 & 0.679 & 0.752 & 0.815
%% & 0.894 \\
%% 287 & 0 & 0.012 & 0.094 & 0.248 & 0.355 & 0.517 & 0.632 & 0.710 & 0.780
%% & 0.880 \\
%% 294 & 0 & 0.017 & 0.088 & 0.225 & 0.322 & 0.476 & 0.590 & 0.671 & 0.748
%% & 0.867 \\
%% 301 & 0 & 0.021 & 0.083 & 0.204 & 0.294 & 0.440 & 0.554 & 0.637 & 0.720
%% & 0.854 \\
%% 308 & 0 & 0.022 & 0.077 & 0.186 & 0.269 & 0.409 & 0.522 & 0.608 & 0.694
%% & 0.840 \\
%% 315 & 0 & 0.022 & 0.070 & 0.170 & 0.248 & 0.383 & 0.496 & 0.583 & 0.672
%% & 0.827 \\
%% 322 & 0 & 0.020 & 0.064 & 0.157 & 0.231 & 0.362 & 0.474 & 0.562 & 0.654
%% & 0.814 \\
%% 329 & 0 & 0.016 & 0.057 & 0.146 & 0.218 & 0.347 & 0.458 & 0.546 & 0.638
%% & 0.801 \\
%% 336 & 0 & 0.009 & 0.050 & 0.138 & 0.209 & 0.336 & 0.446 & 0.534 & 0.625
%% & 0.787 \\
%% 343 & 0 & 0.001 & 0.043 & 0.133 & 0.204 & 0.331 & 0.440 & 0.526 & 0.616
%% & 0.774 \\
%% \end{longtable}

\fig{c/image34}{}

{\bfseries Fig.14 - Dynamics of the hydrogen adsorption process on porous
carbon material of the adsorbent
"Shubarkol:K\tsb{2}CO\tsb{3} 1:1" 30 min}

\fig{c/image35}{}

{\bfseries Fig.15 - Hydrogen adsorption (\%) on porous carbon material
containing K₂CO₃ (1:1) from "Shubarkol" coal}

In the initial stages (up to 10 minutes), the adsorption intensity
increases especially rapidly. At low temperatures (273--294 K), maximum
adsorption values \hspace{0pt}\hspace{0pt}are achieved - above 0.9\%. As
the temperature increases, the sorption capacity decreases, which
indicates the physical nature of adsorption and possible desorption upon
heating.

\fig{c/image36}{}

{\bfseries Fig.16 - Dynamics of hydrogen adsorption process on porous
carbon material based on Shubarkol:K2CO3 1:1 at temperatures of 298 K,
313 K, 343 K}

{\bfseries Table 7 - Experimental results}

%% \begin{longtable}[]{@{}
%%   >{\centering\arraybackslash}p{(\linewidth - 6\tabcolsep) * \real{0.2269}}
%%   >{\centering\arraybackslash}p{(\linewidth - 6\tabcolsep) * \real{0.2255}}
%%   >{\centering\arraybackslash}p{(\linewidth - 6\tabcolsep) * \real{0.3635}}
%%   >{\centering\arraybackslash}p{(\linewidth - 6\tabcolsep) * \real{0.1842}}@{}}
%% \toprule\noalign{}
%% \begin{minipage}[b]{\linewidth}\centering
%% {\bfseries Temperature , К}
%% \end{minipage} & \begin{minipage}[b]{\linewidth}\centering
%% {\bfseries Time, min}
%% \end{minipage} & \begin{minipage}[b]{\linewidth}\centering
%% {\bfseries Adsorption, \%}
%% \end{minipage} & \begin{minipage}[b]{\linewidth}\centering
%% {\bfseries R²}
%% \end{minipage} \\
%% \midrule\noalign{}
%% \endhead
%% \bottomrule\noalign{}
%% \endlastfoot
%% 298 & х, min & y = 0.0455*x + 0.1087 & 0.9688 \\
%% 313 & х, min & y = 0.0411*x + 0.1273 & 0.9548 \\
%% 343 & х, min & y = 0.0415*x + 0.1345 & 0.9408 \\
%% \end{longtable}

{\bfseries Table 8 - Calculating activation energy}

%% \begin{longtable}[]{@{}
%%   >{\raggedright\arraybackslash}p{(\linewidth - 4\tabcolsep) * \real{0.2794}}
%%   >{\centering\arraybackslash}p{(\linewidth - 4\tabcolsep) * \real{0.3478}}
%%   >{\centering\arraybackslash}p{(\linewidth - 4\tabcolsep) * \real{0.3728}}@{}}
%% \toprule\noalign{}
%% \begin{minipage}[b]{\linewidth}\centering
%% {\bfseries Temperature , К}
%% \end{minipage} & \begin{minipage}[b]{\linewidth}\centering
%% {\bfseries Equation}
%% \end{minipage} & \begin{minipage}[b]{\linewidth}\centering
%% {\bfseries Slope (k), \%/min}
%% \end{minipage} \\
%% \midrule\noalign{}
%% \endhead
%% \bottomrule\noalign{}
%% \endlastfoot
%% 298 & y = 0.0455*x + 0.1087 & 0.0455 \\
%% 313 & y = 0.0411*x + 0.1273 & 0.0411 \\
%% 343 & y = 0.0415*x + 0.1345 & 0.0415 \\
%% \end{longtable}

{\bfseries Table 9 -- Data for linear approximation of the dependence of ln
on 1/T}

%% \begin{longtable}[]{@{}
%%   >{\raggedright\arraybackslash}p{(\linewidth - 6\tabcolsep) * \real{0.1837}}
%%   >{\centering\arraybackslash}p{(\linewidth - 6\tabcolsep) * \real{0.2691}}
%%   >{\centering\arraybackslash}p{(\linewidth - 6\tabcolsep) * \real{0.2964}}
%%   >{\centering\arraybackslash}p{(\linewidth - 6\tabcolsep) * \real{0.2508}}@{}}
%% \toprule\noalign{}
%% \begin{minipage}[b]{\linewidth}\centering
%% {\bfseries T (K)}
%% \end{minipage} & \begin{minipage}[b]{\linewidth}\centering
%% {\bfseries 1/T (1/K)}
%% \end{minipage} & \begin{minipage}[b]{\linewidth}\centering
%% {\bfseries k (\%/min)}
%% \end{minipage} & \begin{minipage}[b]{\linewidth}\centering
%% {\bfseries ln(k)}
%% \end{minipage} \\
%% \midrule\noalign{}
%% \endhead
%% \bottomrule\noalign{}
%% \endlastfoot
%% 298 & 0.003356 & 0.0455 & -3.09004 \\
%% 313 & 0.003194 & 0.0411 & -3.19175 \\
%% 343 & 0.002916 & 0.0415 & -3.18206 \\
%% \end{longtable}

Now let' s find the slope for the line using the least
squares method:
lnk=−E\tsb{a}/R\hspace{0pt}\hspace{0pt}⋅T\hspace{0pt}+lnA

k\tsb{1}=−Ea/R=Δlnk/Δ(1/T)=(−3,09004+3,11915)/(0,003356−0,003194)

k\tsb{1}=0,02911/0,000162≈179,69

Ea=−R⋅k\tsb{1}=−8,314⋅(179,69)≈ -1493,954   J/mol ≈-1,5 
kJ/mol

Due to the negative activation energy, the process is thermodynamically
unfavorable at high temperatures.

The highest adsorption is observed at low temperature (298 K). When the
temperature increases to 313 K, the adsorption decreases slightly, which
is typical of physical adsorption (which is exothermic). When further
increasing to 343 K, a slight recovery of adsorption occurs, which may
be due to additional activation of deeper pores or experimental error.

{\bfseries Table 10 - Comparative analysis}

%% \begin{longtable}[]{@{}
%%   >{\raggedright\arraybackslash}p{(\linewidth - 6\tabcolsep) * \real{0.3809}}
%%   >{\centering\arraybackslash}p{(\linewidth - 6\tabcolsep) * \real{0.1760}}
%%   >{\centering\arraybackslash}p{(\linewidth - 6\tabcolsep) * \real{0.2035}}
%%   >{\centering\arraybackslash}p{(\linewidth - 6\tabcolsep) * \real{0.2397}}@{}}
%% \toprule\noalign{}
%% \begin{minipage}[b]{\linewidth}\raggedright
%% {\bfseries Parameter}
%% \end{minipage} & \begin{minipage}[b]{\linewidth}\centering
%% {\bfseries K\tsb{2}CО\tsb{3} 1:1}
%% \end{minipage} & \begin{minipage}[b]{\linewidth}\centering
%% {\bfseries K\tsb{2}CО\tsb{3} 1:0,5}
%% \end{minipage} & \begin{minipage}[b]{\linewidth}\centering
%% {\bfseries Difference (\%)}
%% \end{minipage} \\
%% \midrule\noalign{}
%% \endhead
%% \bottomrule\noalign{}
%% \endlastfoot
%% 298 K Adsorption (см³/кг) & 814.80 & 714.00 & 14.00\% \\
%% 313 K Adsorption & 760.50 & 666.40 & 14.10\% \\
%% 343 K Adsorption & 787.70 & 690.20 & 14.10\% \\
%% \end{longtable}

{\bfseries Table 11 - Comparative analysis of adsorption, \%}

%% \begin{longtable}[]{@{}
%%   >{\raggedright\arraybackslash}p{(\linewidth - 6\tabcolsep) * \real{0.3475}}
%%   >{\centering\arraybackslash}p{(\linewidth - 6\tabcolsep) * \real{0.2103}}
%%   >{\centering\arraybackslash}p{(\linewidth - 6\tabcolsep) * \real{0.2105}}
%%   >{\centering\arraybackslash}p{(\linewidth - 6\tabcolsep) * \real{0.2317}}@{}}
%% \toprule\noalign{}
%% \multirow{2}{=}{\begin{minipage}[b]{\linewidth}\raggedright
%% {\bfseries Temperature, К}
%% \end{minipage}} &
%% \multicolumn{2}{>{\centering\arraybackslash}p{(\linewidth - 6\tabcolsep) * \real{0.4208} + 2\tabcolsep}}{%
%% \begin{minipage}[b]{\linewidth}\centering
%% {\bfseries Coal "Shubarkol"/ K\tsb{2}CO\tsb{3}}
%% \end{minipage}} & \begin{minipage}[b]{\linewidth}\centering
%% {\bfseries Difference, \%}
%% \end{minipage} \\
%% & \begin{minipage}[b]{\linewidth}\centering
%% 1:0,5
%% \end{minipage} & \begin{minipage}[b]{\linewidth}\centering
%% 1:1
%% \end{minipage} & \begin{minipage}[b]{\linewidth}\centering
%% \end{minipage} \\
%% \midrule\noalign{}
%% \endhead
%% \bottomrule\noalign{}
%% \endlastfoot
%% 298 & 0.637538 & 0.727543 & 12.37 \\
%% 313 & 0.595036 & 0.679041 & 12.37 \\
%% 343 & 0.616287 & 0.703292 & 12.37 \\
%% \end{longtable}

\fig{c/image37}{}

{\bfseries Fig.17 - Comparative analysis of hydrogen adsorption}

{\bfseries Table 12 - Hydrogen adsorption calculation, \%}

%% \begin{longtable}[]{@{}
%%   >{\raggedright\arraybackslash}p{(\linewidth - 18\tabcolsep) * \real{0.1726}}
%%   >{\raggedleft\arraybackslash}p{(\linewidth - 18\tabcolsep) * \real{0.0920}}
%%   >{\raggedleft\arraybackslash}p{(\linewidth - 18\tabcolsep) * \real{0.0919}}
%%   >{\raggedleft\arraybackslash}p{(\linewidth - 18\tabcolsep) * \real{0.0919}}
%%   >{\raggedleft\arraybackslash}p{(\linewidth - 18\tabcolsep) * \real{0.0919}}
%%   >{\raggedleft\arraybackslash}p{(\linewidth - 18\tabcolsep) * \real{0.0919}}
%%   >{\raggedleft\arraybackslash}p{(\linewidth - 18\tabcolsep) * \real{0.0919}}
%%   >{\raggedleft\arraybackslash}p{(\linewidth - 18\tabcolsep) * \real{0.0919}}
%%   >{\raggedleft\arraybackslash}p{(\linewidth - 18\tabcolsep) * \real{0.0919}}
%%   >{\raggedleft\arraybackslash}p{(\linewidth - 18\tabcolsep) * \real{0.0920}}@{}}
%% \toprule\noalign{}
%% \multirow{2}{=}{\begin{minipage}[b]{\linewidth}\raggedright
%% Coal:К\tsb{2}СО\tsb{3}
%% \end{minipage}} &
%% \multicolumn{9}{>{\centering\arraybackslash}p{(\linewidth - 18\tabcolsep) * \real{0.8274} + 16\tabcolsep}@{}}{%
%% \begin{minipage}[b]{\linewidth}\centering
%% {\bfseries Temperature, К}
%% \end{minipage}} \\
%% & \begin{minipage}[b]{\linewidth}\raggedleft
%% 273
%% \end{minipage} & \begin{minipage}[b]{\linewidth}\raggedleft
%% 283
%% \end{minipage} & \begin{minipage}[b]{\linewidth}\raggedleft
%% 293
%% \end{minipage} & \begin{minipage}[b]{\linewidth}\raggedleft
%% 303
%% \end{minipage} & \begin{minipage}[b]{\linewidth}\raggedleft
%% 313
%% \end{minipage} & \begin{minipage}[b]{\linewidth}\raggedleft
%% 323
%% \end{minipage} & \begin{minipage}[b]{\linewidth}\raggedleft
%% 333
%% \end{minipage} & \begin{minipage}[b]{\linewidth}\raggedleft
%% 343
%% \end{minipage} & \begin{minipage}[b]{\linewidth}\raggedleft
%% 353
%% \end{minipage} \\
%% \midrule\noalign{}
%% \endhead
%% \bottomrule\noalign{}
%% \endlastfoot
%% 1:1 & 0.916 & 0.835 & 0.772 & 0.728 & 0.701 & 0.693 & 0.702 & 0.729 &
%% 0.775 \\
%% 1:0.5 & 0.890 & 0.826 & 0.778 & 0.746 & 0.729 & 0.729 & 0.745 & 0.777 &
%% 0.825 \\
%% 1:0.33 & 0.865 & 0.817 & 0.783 & 0.763 & 0.758 & 0.766 & 0.788 & 0.824 &
%% 0.874 \\
%% 1:0.25 & 0.839 & 0.808 & 0.789 & 0.781 & 0.786 & 0.802 & 0.831 & 0.872 &
%% 0.924 \\
%% 1:0.2 & 0.814 & 0.799 & 0.794 & 0.799 & 0.814 & 0.839 & 0.874 & 0.919 &
%% 0.974 \\
%% 1:17 & 0.789 & 0.790 & 0.799 & 0.817 & 0.842 & 0.876 & 0.917 & 0.966 &
%% 1.024 \\
%% 1:0.14 & 0.763 & 0.781 & 0.805 & 0.835 & 0.870 & 0.912 & 0.960 & 1.014 &
%% 1.074 \\
%% 1:0.13 & 0.738 & 0.772 & 0.810 & 0.852 & 0.899 & 0.949 & 1.003 & 1.061 &
%% 1.123 \\
%% 1:0.11 & 0.712 & 0.763 & 0.816 & 0.870 & 0.927 & 0.985 & 1.046 & 1.109 &
%% 1.173 \\
%% 1:0.10 & 0.687 & 0.754 & 0.821 & 0.888 & 0.955 & 1.022 & 1.089 & 1.156 &
%% 1.223 \\
%% \end{longtable}

On average, increasing the mass fraction of K₂CO₃ from 1:0.5 to 1:1
leads to an increase in the adsorption capacity by \textasciitilde14\%
in volume. This confirms that intensive activation of K₂CO₃ improves the
sorption properties. Both materials show a characteristic physical
adsorption property: high efficiency at low temperatures. At K₂CO₃:1:1,
hydrogen adsorption is 10--15\% higher than at 1:0.5. The difference is
observed at all temperatures and is most noticeable at
\textasciitilde714 cm³/kg vs. \textasciitilde815 at 298 K. This confirms
that the high concentration of the activator (K₂CO₃) contributes to good
porosity and high sorption capacity.

{\bfseries Conclusion.} As a result of the conducted research, the
physicochemical properties, morphological features, surface area of
\hspace{0pt}\hspace{0pt}carbon adsorbents obtained using Shubarkol coal
and potassium carbonate (K₂CO₃) were studied, and the hydrogen storage
process was carried out in laboratory conditions. The results showed
that the adsorbent "Shubarkol-K\tsb{2}CO\tsb{3}"
(1:1, 900°C) showed the highest BET specific surface area (653.024
m²/g), which determines its ability to effectively adsorb hydrogen. TEM
images confirmed the nanostructured nature of the material. Graphitized
regions, layered structures, and agglomerated particles were detected.
The results of scanning electron microscopy (SEM) and elemental analysis
showed that carbon adsorbents have high porosity, which is an important
factor improving their hydrogen storage capabilities.

The results of the conducted research can serve as the basis for the
development of promising carbon adsorbents for hydrogen storage systems.

\emph{{\bfseries Acknowledgement.} This research has been funded by the
Science Committee of the Ministry of Science and Higher Education of the
Republic of Kazakhstan (Grant No. AP19577512 "Development of scientific
and technical bases for obtaining microporous carbon nanomaterials for
hydrogen separation and storage").}

{\bfseries References}

1. Tarkowski R., Czapowski G. Salt domes in Poland--Potential sites for
hydrogen storage in caverns// International Journal of Hydrogen
Energy.-2018. -Vol 43(46).-P.21414-21427.
\href{https://doi.org/10.1016/j.ijhydene.2018.09.212}{DOI
10.1016/j.ijhydene.2018.09.212}.

2. Sgobbi A., Nijs W., De Miglio R., Chiodi A., Gargiulo M., Thiel, C.
How far away is hydrogen? Its role in the medium and long-term
decarbonisation of the European energy system// International Journal of
Hydrogen Energy.-2016.-Vol 41(1).-P.19-35.
\href{https://doi.org/10.1016/j.ijhydene.2015.09.004}{DOI
10.1016/j.ijhydene.2015.09.004}.

3. Pan B., Yin X., Ju Y., Iglauer S. Underground hydrogen storage:
Influencing parameters and future outlook// Advances in Colloid and
Interface Science.- 2021.-Vol 294.- P.102473.

\href{https://doi.org/10.1016/j.cis.2021.102473}{DOI
10.1016/j.cis.2021.102473}.

4. Tarkowski R. Underground hydrogen storage: Characteristics and
prospects// Renewable and Sustainable Energy
Reviews.-2019.-Vol.105.-P.86-94.
\href{https://doi.org/10.1016/j.rser.2019.01.051}{DOI
10.1016/j.rser.2019.01.051}.

5. Carden P. O., Paterson L. Physical, chemical and energy aspects of
underground hydrogen storage// International Journal of Hydrogen
Energy.-1979.-Vol.4(6).-P.559-569.
\href{https://doi.org/10.1016/0360-3199(79)90083-1}{DOI
10.1016/0360-3199(79)90083-1}.

6. Musyoka N. M., Wdowin M., Rambau K. M., Franus W., Panek R., Madej
J., Czarna-Juszkiewicz D. Synthesis of activated carbon from high-carbon
coal fly ash and its hydrogen storage application// Renewable Energy.-
2020. -Vol.155. -P.1264-1271.
\href{https://doi.org/10.1016/j.renene.2020.04.003}{DOI
10.1016/j.renene.2020.04.003}.

7. Abdimomyn S., Malik S., Skakov M., Koyanbayev Y., Miniyazov A.,
Malchik F. Hydrogen storage materials: Promising materials for
Kazakhstan's hydrogen storage industry// Eurasian Chemico-Technological
Journal. - 2024. --Vol.26(3).- P.113-132.
\href{https://doi.org/10.18321/ectj1635}{DOI 10.18321/ectj1635}

8. Musyoka N. M., Wdowin M., Rambau K. M., Franus W., Panek R., Madej
J., Czarna-Juszkiewicz D. Synthesis of activated carbon from high-carbon
coal fly ash and its hydrogen storage application// Renewable Energy.
-2020.- Vo.l 155.- P.1264-1271.
\href{https://doi.org/10.1016/j.renene.2020.04.003}{DOI
10.1016/j.renene.2020.04.003}.

9. Kazankapova M. K., Yermagambet B. T., Kozhamuratova U. M.,
Dauletzhanova Z. T., Kapsalyamov B. A., Malgazhdarova A. B., Beisembaeva
K. A. Obtaining and investigating sorption capacity of carbon
nanomaterials derived from coal for hydrogen storage// ES Energy and
Environment. -2024.- Vol.25.- P.1234.
\href{https://dx.doi.org/10.30919/esee1234}{DOI 10.30919/esee1234}.

10. Bénard P., Chahine R. Storage of hydrogen by physisorption on carbon
and nanostructured materials// Scripta Materialia.-2007.-Vol.56(10).-
P.803-808.

\href{https://doi.org/10.1016/j.scriptamat.2007.01.008}{DOI
10.1016/j.scriptamat.2007.01.008}.

11. Kayiran S. B., Lamari F. D., Levesque D. Adsorption properties and
structural characterization of activated carbons and nanocarbons// The
Journal of Physical Chemistry. -2004.-Vol.108(39). - P.15211-15215.
\href{https://doi.org/10.1021/jp048169c}{DOI 10.1021/jp048169c}.

12. Anson A., Callejas M. A., Benito A. M., Maser W. K., Izquierdo M.
T., Rubio B., Martınez M. T. Hydrogen adsorption studies on single wall
carbon nanotubes// Carbon.- 2004.-Vol.42(7). - P.1243-1248.
\href{https://doi.org/10.1016/j.carbon.2004.01.038}{DOI
10.1016/j.carbon.2004.01.038}.

13. Xia Y., Yang Z., Zhu Y. Porous carbon-based materials for hydrogen
storage: advancement and challenges// Journal of Materials Chemistry.-
2013. - Vol 1(33).- P.9365-9381.

\href{https://doi.org/10.1039/C3TA10583K}{DOI 10.1039/C3TA10583K}.

\emph{{\bfseries Information about the authors}}

Kazankapova M.K. - PhD in Philosophy, Associate Professor, Corresponding
Member of KazNANS, Leading Researcher, Head of Laboratory of LLP
"Institute of Coal Chemistry and Technology", Astana, Kazakhstan,
e-mail: \href{mailto:maira_1986@mail.ru}{};

Yermagambet B.T. - Doctor of Chemical Sciences, Professor, Academician
of KazNANS, Project Manager, Chief Researcher, Director of LLP
"Institute of Coal Chemistry and Technology", Astana, Kazakhstan,
e-mail: \href{mailto:bake.yer@mail.ru}{};

Kassenova Zh.M. - Candidate of Chemical Sciences (PhD), Member of
KazNANS, Leading Researcher, Deputy Director of LLP "Institute of Coal
Chemistry and Technology", Astana, Kazakhstan, e-mail:
zhanar\_k\_68@mail.ru;

Kapsalyamov B.A. - Doctor of Chemical Sciences, Professor, Eurasian
National University of L.N. Gumilyov, Astana, Kazakhstan, e-mail:
ba.kapsalyamov@gmail.com;

Malgazhdarova A.B. - Junior Researcher «Institute of Coal Chemistry and
Technology», master student Eurasian National University of L.N.
Gumilyov, Astana, Kazakhstan, e-mail:
\href{mailto:malgazhdarova.ab@mail.ru}{};

Dauletzhanova Zh. T. - PhD Doctor, Technology, Kazakh University of
Technology and Business named after K. Kulazhanov, Leading Researcher of
LLP "Institute of Coal Chemistry and Technology" Astana, Kazakhstan,
e-mail:
\href{mailto:kaliyeva_zhanna@mail.ru}{};

Kozhamuratova U.M. -Junior Researcher «Institute of Coal Chemistry and
Technology», master student Eurasian National University of L.N.
Gumilyov, Astana, Kazakhstan, e-mail:
\href{mailto:kozhamuratova.u@mail.ru}{};

Mendaliyev G. K. -- Junior Researcher «Institute of Coal Chemistry and
Technology», master student Eurasian National University of L.N.
Gumilyov, Astana, Kazakhstan, e-mail:
\href{mailto:ganimen02@mail.ru}{};

Akshekina A.S. -- Senior Lab Assistant «Institute of Coal Chemistry and
Technology», Astana, Kazakhstan, e-mail:
\href{mailto:akshekina11@gmail.com}{}.

\emph{{\bfseries Сведения об авторах}}

Казанкапова М.К. - PhD, асс. профессор, чл.-корр. КазНАЕН, ведущий
научный сотрудник, заведующий лабораторией ТОО «Институт химии и
технологии угля», Астана, Казахстан, e-mail:
\href{mailto:maira_1986@mail.ru}{};

Ермагамбет Б.Т.- доктор химических наук, профессор, академик КазНАЕН,
руководитель проекта, главный научный сотрудник, директор ТОО «Институт
химии и технологии угля», Астана, Казахстан, e-mail:


Касенова Ж.М. - кандидат химических наук (PhD), член КазНАЕН, ведущий
научный сотрудник, заместитель директора ТОО «Институт углехимии и
технологии», Астана, Казахстан, e-mail: zhanar\_k\_68@mail.ru;

Капсалямов Б.А.- доктор химических наук, профессор, Евразийский
национальный университет им. Л.Н. Гумилева, Астана, Казахстан, e-mail:
ba.kapsalyamov@gmail.com;

Малғаждарова А.Б. - младший научный сотрудник ТОО «Институт химии угля и
технологии», магистрант Евразийского национального университета им.
Л.Н.Гумилева, Астана, Казахстан, e-mail:


Даулетжанова Ж.Т. - доктор PhD, доцент Казахского университета
технологии и бизнеса им. К. Кулажанова, Астана, ведущий научный
сотрудник ТОО «Институт Химии угля и технологии», Астана, Казакстан
e-mail:
\href{mailto:kaliyeva_zhanna@mail.ru}{};

Қожамұратова Ұ.М. - младший научный сотрудник ТОО «Институт химии угля и
технологии», магистрант Евразийского национального университета им.
Л.Н.Гумилева, Астана, Казахстан, e-mail:
\href{mailto:kozhamuratova.u@mail.ru}{};

Мендалиев Г.К. - младший научный сотрудник ТОО «Институт химии угля и
технологии», магистрант Евразийского национального университета им.
Л.Н.Гумилева, Астана, Казахстан, e-mail:
\href{mailto:ganimen02@mail.ru}{};

Акшекина Ә.С. - старший лаборант ТОО «Институт химии угля и технологии»,
Астана, Казахстан, e-mail:
\href{mailto:akshekina11@gmail.com}{}.\
