\id{IRSTI 61.51.21}{}

\begin{header}
\swa{}{PROCESS SIMULATION AND EVALUATION OF METHANOL RECOVERY UNIT USING ASPEN HYSYS}

{\bfseries
\tsp{1,2}K.G. Satenov,
\tsp{1-4}Ye.M. Suleimen\envelope 
}
\end{header}

\begin{affil}
\tsp{1}LLP KMG Engineering, Astana, Kazakhstan,

\tsp{2}Kazakh University of Technology and Business named after K.Kulazhanov, Astana, Kazakhstan,

\tsp{3}Sh.Ualikhanov Kokshetau University, Laboratory of NMR Spectroscopy, Kazakhstan,

\tsp{4}LLP Institute of Applied Chemistry, Astana, Kazakhstan

\envelope Corresponding author: syerlan75@yandex.kz
\end{affil}

Natural gas is one of the most economical and environmentally friendly
energy sources, playing a crucial role in Kazakhstan's gas industry. The
optimization of gas processing technologies, particularly methanol
regeneration, is essential for improving efficiency and reducing
operational costs. Methanol (CH\tsb{3}OH) is widely used in
the oil and gas industry as a hydrate inhibitor, but its high
consumption necessitates regeneration to minimize costs and
environmental impact.

This study focuses on the technical and economic feasibility of methanol
regeneration from methanol-water solutions (MWS) at the ``X'' gas
processing plant in western Kazakhstan. The Aspen HYSYS software was
used to simulate the methanol recovery process, employing the
Peng-Robinson equation of state. The study explored different process
configurations, considering separation, purification, and rectification
technologies.

{\bfseries Keywords}: Methanol-Water Solutions (MWS), Complex gas treatment
plant (CGTP), The Peng-Robin\-son equation of state with the Stryjek and
Vera modification (PRSV), Air Cooler Unit (ACU).

\begin{header}
{\bfseries ASPEN HYSYS КӨМЕГІМЕН МЕТАНОЛДЫ ҚАЛПЫНА КЕЛТІРУ ҚОНДЫРҒЫСЫН
ТЕХНОЛОГИЯЛЫҚ МОДЕЛЬДЕУ ЖӘНЕ БАҒАЛАУ}

{\bfseries
\tsp{1,2}Қ.Г. Сатенов,
\tsp{1-4}Е.М. Сүлеймен\envelope
}
\end{header}

\begin{affil}
\tsp{1}ҚМГ Инжиниринг ЖШС, Астана қ., Қазақстан,

\tsp{2}Қ. Құлажанов атындағы Қазақ технология және бизнес университеті, Астана, Қазақстан,

\tsp{3}Ш.Уәлиханов атындағы Көкшетау Университеті, ЯМР Спектроскопия зертханасы, Қазақстан,

\tsp{4}Қолданбалы Химия Институты ЖШС, Астана, Қазақстан,

e-mail:syerlan75@yandex.kz
\end{affil}

Табиғи газ - Қазақстанның газ өнеркәсібінде маңызды рөл атқаратын, ең
үнемді әрі экологиялық тұрғыдан тиімді энергия көздерінің бірі. Газ
өңдеу технологияларын, әсіресе метанолды регенерациялау үдерісін
оңтайландыру - тиімділікті арттырып, операциялық шығындарды азайту үшін
маңызды. Метанол (CH₃OH) мұнай-газ саласында гидрат ингибиторы ретінде
кеңінен қолданылады, алайда оның жоғары тұтыну мөлшері шығындар мен
экологиялық әсерді азайту үшін оны регенерациялауды қажет етеді.

Бұл зерттеу Батыс Қазақстандағы ``X'' газ өңдеу зауытында метанол-сулы
ерітінділерден (МСЕ) метанолды қалпына келтірудің техникалық және
экономикалық орындылығына бағытталған. Метанолды қалпына келтіру үдерісі
Aspen HYSYS бағдарламасы арқылы модельденіп, Peng--Robinson күй теңдеуі
қолданылды. Зерттеу барысында бөлу, тазарту және ректификациялау
технологияларын қамтитын түрлі үдеріс конфигурациялары қарастырылды.

{\bfseries Түйін сөздер:} Метанол-сулы ерітінділер (МСЕ), Кешенді газ өңдеу
зауыты (КГӨЗ), Стрижек және Вера модификациясымен Пенг-Робинсон күй
теңдеуі (PRSV), Ауа салқындатқыш құрылғысы (АСҚ).

\begin{header}
{\bfseries МОДЕЛИРОВАНИЕ ПРОЦЕССА И ОЦЕНКА РАБОТЫ УСТАНОВКИ ИЗВЛЕЧЕНИЯ
МЕТАНОЛА С ИСПОЛЬЗОВАНИЕМ ASPEN HYSYS}

{\bfseries
\tsp{1,2}К.Г. Сатенов,
\tsp{1-4}Е.М. Сулеймен\envelope
}
\end{header}

\begin{affil}
\tsp{1}ТОО КМГ Инжиниринг, Астана, Казахстан,

\tsp{2}Казахский университет технологии и бизнеса им. К. Кулажанова, Астана, Казахстан,

\tsp{3}Кокшетауский университет им. Ш.Уалиханова, лаборатория ЯМР-спектроскопии, Казахстан,

\tsp{4}ТОО "Институт прикладной химии", Астана, Казахстан,

e-mail:syerlan75@yandex.kz
\end{affil}

Природный газ - один из самых экономичных и экологически безопасных
источников энергии, играющий важную роль в газовой промышленности
Казахстана. Оптимизация технологий переработки газа, в частности
регенерации метанола, имеет ключевое значение для повышения
эффективности и снижения эксплуатационных затрат. Метанол (CH₃OH) широко
применяется в нефтегазовой отрасли в качестве ингибитора
гидратообразования, однако его высокий расход требует регенерации для
минимизации затрат и снижения экологического воздействия.

В данном исследовании рассматривается техническая и экономическая
целесообразность регенерации метанола из метанол-водных растворов (МВР)
на газоперерабатывающем заводе ``X'' в Западном Казахстане. Для
моделирования процесса восстановления метанола использовалось
программное обеспечение Aspen HYSYS с применением уравнения состояния
Пенга-Робинсона. В исследовании были рассмотрены различные конфигурации
процесса, включая технологии разделения, очистки и ректификации.

{\bfseries Ключевые слова:} Водометанольный раствор (ВМР), Установка
комплексной подготовки газа (УКПГ), Уравнение состояния Пенга--Робинсона
с модификацией Стрижека и Веры (PRSV), Воздушный охладитель (ВО).

\begin{multicols}{2}
{\bfseries Introduction.} Natural gas is one of the most economical sources
of fuel and energy resources, occupying a special place in the world raw
material base. The reason is its high consumer characteristics, low
production and transportation costs, as well as widespread use in many
areas of human activity {[}1-3{]}.

Great hopes are placed on natural gas as the cheapest, highly
environmentally friendly fuel in preparation for the transition to the
wider use of alternative non-traditional types of electricity (solar,
wind, water, geothermal energy).

The gas industry of Kazakhstan is an important source of income and
occupies an important place in the socio-economic development of the
Republic of Kazakhstan, therefore significant funds are allocated for
its development and modernization {[}4-7{]}. The development of
resource-saving technologies is an important area for the oil and gas
industry.

Complex gas treatment plants are a set of direct and indirect equipment
designed for collecting and processing natural gas and gas condensate.

They include separation, purification, drying and cooling units, as well
as booster compressor stations. When developing oil and gas fields in a
sharply continental climate, it is necessary to consider the possibility
of hydrate formation {[}8-11{]}.

In technologies for production, treatment and transportation of oil and
gas, gas hydrates because serious problems associated with disruption of
these operating processes {[}12-13{]}.

Gas hydrates are solid crystalline compounds that are formed under
certain thermobaric conditions from an aqueous solution, ice, water
vapor and low molecular weight gases. They have an appearance similar to
ice or snow. Methanol (CH\tsb{3}OH) is used in the oil and gas
industry as a hydrate inhibitor.

In order to optimize operating costs, by reducing the volume of methanol
purchases and its delivery to the point of use, methanol regeneration
units are integrated into the gas treatment process, while the economic
efficiency and feasibility of methanol regeneration is determined for
each individual case, considering the impact various conditions and
factors in the conditions present at a particular enterprise.

{\bfseries Materials and methods}. The most optimal technology for methanol
regeneration is the MWS rectification process using additional equipment
for purification from mechanical impurities, salts, and hydrogen
sulfide~{[}11{]}.

The purpose of studying the methanol regeneration process is to
determine the technical feasibility and economic feasibility of
regenerating methanol from MWS released during the preparation of well
fluid at the processing units of the ``X'' field, located in the western
region of Kazakhstan.

In order to determine the technical feasibility of methanol
regeneration, it is necessary to determine the quality of the base MWS,
as well as the volume of purified methanol yield. Assess capital costs
for the construction of a new MRU and cost-effectiveness analysis.

The methanol recovery plant was modeled to produce methanol by treating
industrial wastewater from a CGTP, which is then suitable for injection
into the reservoir, using the Aspen HYSYS software package, designed to
study the processes of preparing oil and gas feedstocks~{[}14-15{]}.

The thermodynamic package used was the Peng-Robinson equation of state
modified by Stryjek and Vera (PRSV), which more accurately predicts the
phase behavior of hydrocarbon systems, especially systems consisting of
dissimilar components.

The PRSV equation of state performs rigorous three-phase instantaneous
calculations for aqueous systems containing H\tsb{2}O,
CH\tsb{3}OH or glycols, as well as systems containing other
hydrocarbons or non-hydrocarbon compounds in the second liquid phase
{[}16-17{]}.

{\bfseries Results and discussion.} The goal of this process simulations
was to produce the final methanol product at the ``X'' field gas
processing plant located in the western region of Kazakhstan, with the
following requirements:

- methanol (CH\tsb{3}OH) - not less than 95 \% wt.;

- hydrogen sulfide (H\tsb{2}S) - no more than 1 ppm;

- water (H\tsb{2}O) - no more than 5 \% wt.

Table 1 presents the indicators for methanol concentration (\% wt.) and
MWS consumption (m\tsp{3}/day), adopted for two options,
designated as Base Option - (B1) and Alternative Option - (B2).
\end{multicols}

\tcap{Table 1 - Indicators for methanol concentration (\% wt) and MWS consumption}
\begin{longtblr}[
  label = none,
  entry = none,
]{
  cells = {c},
  hlines,
  vlines,
}
\textbf{Options}          & \textbf{Methanol concentration, \% wt} & \textbf{MWS consumption, m}\textsuperscript{\textbf{3}}\textbf{/day} \\
Basic option - (B1)       & 1.5                                    & 50                                                                   \\
Alternative option - (B2) & 1.31                                   & 8                                                                    
\end{longtblr}

To characterize the MWS at the inlet to the new MRU in the Aspen HYSYS
program, the compositions of HC, total suspended particles (TSP) and
salts for both options were specified as follows (Table 2-4):

\tcap{Table 2 - Composition and properties of hydrocarbons in MWS}
\begin{longtblr}[
  label = none,
  entry = none,
]{
  cells = {c},
  hlines,
  vlines,
}
\textbf{HC composition} & \textbf{Mole fractions} & \textbf{Mass fractions} \\
CH$_4$                  & 0.0420                  & 0.0058                  \\
C$_2$H$_6$              & 0.0420                  & 0.0108                  \\
C$_3$H$_8$              & 0.0699                  & 0.0264                  \\
\textit{iso}-C$_4$H$_{10}$ & 0.0420               & 0.0208                  \\
\textit{n}-C$_4$H$_{10}$   & 0.0420               & 0.0208                  \\
\textit{iso}-C$_5$H$_{12}$ & 0.0559               & 0.0345                  \\
\textit{n}-C$_5$H$_{12}$   & 0.0559               & 0.0345                  \\
\textit{n}-C$_6$H$_{14}$   & 0.0629               & 0.0464                  \\
\textit{n}-C$_7$H$_{16}$   & 0.0559               & 0.0479                  \\
\textit{n}-C$_8$H$_{18}$   & 0.0559               & 0.0546                  \\
\textit{n}-C$_9$H$_{20}$   & 0.0559               & 0.0613                  \\
\textit{n}-C$_{10}$H$_{22}$& 0.0699               & 0.0851                  \\
\textit{n}-C$_{11}$        & 0.0699               & 0.0934                  \\
\textit{n}-C$_{12}$        & 0.0699               & 0.1018                  \\
\textit{n}-C$_{13}$        & 0.0699               & 0.1102                  \\
\textit{n}-C$_{14}$        & 0.0699               & 0.1186                  \\
\textit{n}-C$_{15}$        & 0.0699               & 0.1270                  \\
\textbf{Total}             & \textbf{1.0000}      & \textbf{1.0000}         \\
\end{longtblr}

\begin{multicols}{2}
To model the properties of TSP in MWS in Aspen HYSYS, the hypothetical
components (pseudo-components) tab was used.

The Aspen HYSYS library allows you to create hypothetical components to
describe hydrocarbon feedstocks. In the calculations, kaolin was used as
a hypothetical component, having a molar mass of 258.0 and a density of
2600.0 kg/m\tsp{3}, as the most suitable component in terms
of characteristics (Table 3).
\end{multicols}

\tcap{Table 3- Composition and properties of TSP in MWS}
\begin{longtblr}[
  label = none,
  entry = none,
  note{} = {\emph{*hypothetical component - kaolin}}
]{
  colspec = {c c c},
  hlines,
  vlines,
}
\textbf{Composition}              & \textbf{Mole fractions} & \textbf{Mass fractions} \\
TSP                               & 1.0000                  & 1.0000                  \\

\SetCell[c=3]{c} \textbf{Properties} \\

Molecular weight*                 & \SetCell[c=2]{c} 258.0  \\
Density (kg/m\textsuperscript{3}) & \SetCell[c=2]{c} 2600.0 \\
Diameter (mm)                     & \SetCell[c=2]{c} 1.000  \\
\end{longtblr}


\tcap{Table 4- Composition of salts in MWS}
\begin{longtblr}[
  label = none,
  entry = none,
]{
  cells = {c},
  hlines,
  vlines,
}
\textbf{Composition} & \textbf{Mole fractions} & \textbf{Mass fractions} \\
NaCl                 & 0.7085                  & 0.5000                  \\
Na$_2$SO$_4$               & 0.2915                  & 0.5000                  
\end{longtblr}

\begin{multicols}{2}
When modeling the new MRU, the need to remove such undesirable
components as hydrocarbons, salts, TSP and H\tsb{2}S from the
input flow of the MWS was considered. The design of the new MRU was
supplemented with the following equipment: a three-phase separator with
a filter-coalescer for removing hydrocarbons, a hydrocyclone for
removing TSP, an evaporator for removing salts, a stripping column for
removing H\tsb{2}S.

The final process diagrams of the new unit for the two options are
identical, in Fig.1 shows a diagram of the new MRU only for the basic
version (B1). The operating parameters of the stripping column for two
options are shown in Fig.2-3.
\end{multicols}

\fig[0.9\textwidth]{c/image4}{Fig.1 - Process diagram of the new MRU for option B1}

\fig{c/image5}{Fig.2 - Parameters of the stripping column for
H\tsb{2}S removal of the new MRU, according to B1 option}

\fig{c/image6}{Fig.3 - Parameters of the stripping column for
H\tsb{2}S removal of the new MRU, according to B2 option}

\begin{multicols}{2}
Saturated MWS with a pressure of about 2.7 bar after the reducing valve
and a temperature of about 31.5 °C enters the input heat
exchanger-heater MWS HEx-1 ``steam-liquid'' (1), in which it is heated
to approximately 94.50 °C so that improve the separation of MWS from
hydrocarbons carried away from the process lines of the gas treatment
plant. After heating in HEx-1, the MWS enters the separator V-102 (2),
which is also equipped with a coalescer filter for more complete removal
of hydrocarbons from the MWS, since even a small number of dissolved
hydrocarbons can form stable emulsions with the aqueous phase of the
MWS. The pressure in the separator V-102 is maintained at 2.5-2.55 bar
abs. At the same time, the separator V-102 serves to remove light
hydrocarbon gases if they were captured by the MWS flow. It is assumed
that the captured hydrocarbons, under their own pressure, are
periodically discharged into the closed drainage system of the gas
treatment plant, and the MWS enters the hydrocyclone (3) to remove the
suspended solids, which are discharged from the bottom of the
hydrocyclone. The MWS after the hydrocyclone passes through the HEx-2
heat exchanger (4), where it is heated to 105 °C before entering the
evaporator (5). The evaporator inside is equipped with a tube bundle
into which water vapor (or any other heat-transfer medium) with a
temperature of about 130-135 °C is supplied, which heats the MWS to 130
°C, transferring the MWS to a completely vapor phase. In this case, the
salts are removed from the bottom of the evaporator in the form of a
brine solution. The MWS vapor stream leaving the evaporator preheats the
MWS input stream in HEx-1 and the MWS stream before the evaporator in
HEx-2.

Before entering for regeneration, dissolved H\tsb{2}S must be
removed from the MWS, because product methanol should not contain this
extremely undesirable, highly toxic and corrosive component. To remove
H\tsb{2}S, an evaporation column (7) is installed, consisting
of the column itself, as well as:

- top of the condenser column (ACU (6) or water cooler), reflux tank and
reflux return pump;

- the bottom of the reboiler column for heating the bottom liquid (water
vapor can act as a heating agent).

It is also possible to use hot oil as a heating agent, or to heat the
bottom liquid directly in a fire heater: both of these options require
fuel gas.

MWS with a pressure of 2 bara and at a temperature of 45 °C enters the
upper part of the evaporation column. According to calculations, the
temperature of the upper product of the evaporation column after the
reflux tank was 71.24 °C, the pressure was 1.9 bar, and the temperature
of the bottom liquid was 107 °C. H\tsb{2}S with some water
vapor, about 1.9-2 \% wt., after a reflux tank of 59.304
Sm\tsp{3}pd (79.2 kg/day is discharged to the Low-Pressure
Flare). Since the volume of gas after the stripper column is very small,
this should not have a significant impact on greenhouse gas emissions.

The MWS from the bottom of the evaporation column with a temperature of
107 °C enters the middle part of the distillation column (8). Methanol
is regenerated by rectification at a pressure of 1.9 bara and 2 bara and
temperatures of 86 °C and 120 °C at the top and bottom of the column,
respectively. The upper product of the column is methanol with a purity
of at least 95 \% wt., the lower product is water (99.99 \% wt.) with a
methanol concentration of no more than 0.01 \% wt. Water from the bottom
of the column is discharged into the drainage system, and the
regenerated methanol, having passed through the air cooler (9), where it
is cooled to 45 °C, and the reflux tank is further supplied to the
corresponding tank at the chemical reagent storage site (13). Part of
the regenerated methanol after the reflux tank is supplied to the top of
the column for reflux using a reflux pump.

Additional pumps for pumping product methanol into the storage tank and
bottom water into the drainage system are not needed, since the flow
pressure is \textasciitilde1.8 and 1.85 bara, respectively, is
sufficient for pumping.

Since the water from the bottom of the column is already a fairly pure
product, it can be introduced into the recycling cycle.

According to the modeling results, the methanol product meets the
specification requirements (Table 5).

The simulation results were used to estimate capital costs according to
AACE Class 5 {[}18{]}.
\end{multicols}

\tcap{Table 5 - Methanol Product Specifications}
\begin{longtblr}[
  label = none,
  entry = none,
]{
  colspec = {l Q[c,2cm] Q[c,2cm]}, % first column auto, B1 & B2 fixed 2cm wide
  vlines,
  hline{1,3-10} = {-}{},
  hline{2} = {2-3}{},
}
\textbf{Methanol to storage} & \SetCell[c=2]{c} \textbf{Options} & \\
                             & B1   & B2   \\
Temperature, \textsuperscript{°}C & 45    & 45   \\
Pressure, bar                     & 1.55  & 1.55 \\
\% Methanol recovery              & 83.1  & 77.2 \\
Mass flow, kg/h                   & 27.71 & 3.58 \\
Methanol, \% wt                   & 95    & 95   \\
Water, \% wt                      & 5     & 5    \\
H$_2$S, \% wt                     & 0     & 0    \\
\end{longtblr}

\begin{multicols}{2}
At each stage of project implementation, the error of the calculated
budget is specified, according to the AACE methodology, which is
assigned a class:

- from the first class with an error of 10 \% / 15 \%,

- up to grade 5 with an error of 50 \% / +100 \%.

Capital costs were calculated in the ASPEN Process Economic Analyzer
(APEA) software package, based on modeling the operating process of the
new MRU, with the definition of a list of main equipment (Table 6).
\end{multicols}

\tcap{Table 6 - List of main equipment}
\begin{longtblr}[
  label = none,
  entry = none,
]{
  width = \linewidth,
  colspec = {X[c,0.6] X[c,1] X[c,1.2] X[l,3] X[c,0.8]}, % centered except Description left
  rowhead = 1,    % repeat header on page breaks
  hlines, vlines,
  row{even} = {c},
}

% header (repeats on page breaks)
\textbf{No.} & \textbf{Item number} & \textbf{Name} & \textbf{Description} & \textbf{Quantity, units} \\

% section: Heat exchange equipment
\SetCell[c=4]{c}\textbf{Heat exchange equipment} & & & & 9 \\
1  & 1   & HE-1   & Input heater MWS                      & 1 \\
2  & 4   & HE-2   & MWS heater in front of the evaporator & 1 \\
3  & 6   & AC-1   & Air MWS cooler                        & 1 \\
4  & 9   & AC-2   & Air methanol product cooler           & 1 \\
5  & 10  & AC-3   & Air water cooler                      & 1 \\
6  & 7.1 & C-Str  & Stripper condenser                    & 1 \\
7  & 8.1 & C-Dist & Distillation column condenser         & 1 \\
8  & 7.2 & R-Str  & Stripper reboiler                     & 1 \\
9  & 8.2 & R-Dist & Distillation column reboiler          & 1 \\

% section: Capacitive equipment
\SetCell[c=4]{c}\textbf{Capacitive equipment} & & & & 1 \\
1  & 11  & Buffer Vessel & Water buffer capacity & 1 \\

% section: Separation equipment
\SetCell[c=4]{c}\textbf{Separation equipment} & & & & 7 \\
1  & 2   & Inlet Separator   & Input separator MWS                & 1 \\
2  & 5   & Evaporator        & Evaporator MWS                     & 1 \\
3  & 7.3 & RV-1              & Reflux tank of stripper column     & 1 \\
4  & 8.3 & RV-2              & Reflux tank of distillation column & 1 \\
5  & 7   & Stripping column  & Stripping column                   & 1 \\
6  & 8   & Distillation col. & Distillation column                & 1 \\
7  & 3   & Hydrocyclone      & Hydrocyclone removal of TSP        & 1 \\

% section: Pumps
\SetCell[c=4]{c}\textbf{Pumps} & & & & 3 \\
1  & 12  & P-1   & Water pump                      & 1 \\
2  & 7.4 & RP-1  & Stripper Reflux Pump            & 1 \\
3  & 8.4 & RP-2  & Distillation Column Reflux Pump & 1 \\

% section: Storage containers
\SetCell[c=4]{c}\textbf{Storage containers} & & & & 1 \\
1  & 13  & MeOH Storage & Methanol storage tank & 1 \\

% total
\SetCell[c=4]{c}\textbf{Total units of equipment} & & & & \textbf{21} \\

\end{longtblr}

\begin{multicols}{2}
Based on the simulation of the operating process, the capital costs for
the acquisition of a block-modular MRU were determined, considering the
main equipment, materials and costs for the manufacture of modules
(Table 7). When calculating the total cost of the project, considering
the average complexity and level of uncertainty for this stage, based on
the Operator's capital cost calculation standards, the following
assumptions were made to increase the base estimate:

- 40 \% allowance for possible unaccounted indicators for the items
``Equipment'' and ``Construction and installation work'';

- 30 \% contingency for all cost items.
\end{multicols}

\tcap{Table 7- Results of calculating the cost of manufacturing a block-modular MRU}
\begin{longtblr}[
  label = none,
  entry = none,
  note{} = {*\emph{Prices are in US dollars.}}
]{
  cells = {c},
  hlines,
  vlines,
}
\textbf{Name of costs} & \textbf{Cost according to B1*} & \textbf{Cost according to B 2*} \\
Equipment              & 571~739                        & 450~061                         \\
Pipelines              & 777~477                        & 408~296                         \\
Construction Materials & 699~031                        & 493~684                         \\
Steel products         & 68~145                         & 63~348                          \\
Automation             & 1~578~052                      & 1~389~855                       \\
Electrical equipment   & 981~802                        & 979~736                         \\
Insulation             & 236~172                        & 93~602                          \\
Painting               & 103~977                        & 70~116                          \\
\textbf{Total}         & \textbf{5~016~395}             & \textbf{3~948~699}              
\end{longtblr}

General and administrative expenses are assumed to be 15 \% of the total
procurement and construction costs. The results of calculating capital
costs are presented in Table 8.

\tcap{Table 8 - Results of calculating capital costs for project implementation by options}
\begin{longtblr}[
  label = none,
  entry = none,
  note{} = {\emph{*Prices are in US dollars.}}
]{
  cells = {c},
  hlines,
  vlines,
}
\textbf{Engineering}                & \textbf{Cost according to B1*} & \textbf{Cost according to B2*} \\
Basic Design                        & 985~877                        & 626~943                        \\
Detailed design                     & 1~834~647                      & 1~507~897                      \\
Equipment                           & 6~011~160                      & 4~925~825                      \\
Manufacturing costs                 & 2~227~628                      & 1~614~863                      \\
Construction and installation works & 3~293~171                      & 1~919~150                      \\
General and administrative expenses & 1~331~037                      & 975~620                        \\
\textbf{Total for the Project}      & \textbf{15~683~520}            & \textbf{11~570~297}            
\end{longtblr}

\tcap{Table 9 - Economic assessment results}
\begin{longtblr}[
  label = none,
  entry = none,
]{
  width = \linewidth,
  colspec = {Q[719]Q[110]Q[112]},
  cells = {c},
  cell{2}{2} = {c=2}{0.222\linewidth},
  cell{9}{1} = {c=3}{0.941\linewidth},
  hlines,
  vlines,
}
\textbf{Indicators}                                                             & \textbf{В1} & \textbf{В2} \\
Calculation period                                                              & 2023-2037   &             \\
Cumulative volume of methanol production, thousand tons                         & 5.99        & 3.74        \\
Annual methanol production, t                                                   & 570.00      & 356.00      \\
Total cost of recovered methanol for the calculation period, million US dollars & 6.47        & 4.04        \\
OPEX MOD, million US dollars                                                    & -1.84       & -1.36       \\
CAPEX MOD, USD million                                                          & -16.72      & -12.33      \\
NPV @10, million US dollars                                                     & -10.13      & -7.80       \\
\textbf{Break-even analysis}                                                    &             &             \\
CAPEX, USD                                                                      & 2 566 657   & 1~722 828   \\
Methanol price, USD/t                                                           & 5 172       & 6 132       \\
Methanol volume, t/g.                                                           & 3 229       & 3 010       
\end{longtblr}

\begin{multicols}{2}
Based on the results of calculating capital costs, an assessment of
economic efficiency was made with the following indicators (Table 9).

The economic assessment of the project with accepted forecast
operational and current macroeconomic parameters and cost assumptions
shows:

- positive results (NPV≥0) with CAPEX not exceeding USD 2.6 million
(base scenario) and USD 1.7 million (alternative scenario);

- positive results (NPV≥0) with a methanol price of at least 5 172 USD/t
(base scenario) and 6 132 USD/t (alternative scenario);

- positive results (NPV≥0) with an annual methanol production volume of
at least 3.2 thousand tons/year (base scenario) and 3 thousand t/year
(alternative scenario).

{\bfseries Conclusions.} When modeling the existing methanol removal
scheme, we found that the recovery rate was insufficient. To increase
the efficiency of regeneration, it is necessary to supplement the
process diagram with additional equipment for purification from
mechanical impurities, salts, and hydrogen sulfide.

We have determined the optimal technical and process parameters of the
distillation column and hydrogen sulfide removal column. The product of
the regeneration unit is a saturated 95 \% methanol solution.

The study allowed us to determine the following:

1. From an engineering point of view, the regeneration of methanol in
CGTP conditions is quite feasible.

2. Rectification of MWS is by far the most developed and widespread
technology. There are a significant number of companies on the market
that have extensive experience in the design and manufacture of MRU
using the rectification method.

3. The average annual yield of regenerated methanol at the specified
volumes of MWS and methanol concentration will be: option B1 -- 570
t/year, option B2 - 356 t/year.

4. The low concentration of methanol in the MWS is due to the existing
technological process, which allows only a small amount of methanol to
be recovered from the process streams. The main volume of methanol is
carried away with gas (about 15-20 \%) and condensate (70-75 \%).

5. A prerequisite for the regeneration of methanol in CGTP conditions is
at field ``X'' is the need for preliminary preparation of the MWS with
the mandatory removal of hydrocarbons, mechanical impurities and acid
gases.

\emph{{\bfseries Funding:} This research was funded by the Science
Committee of the Ministry of Science and Higher Education of the
Republic of Kazakhstan (Grant No. AP19679527 and BR24992761).}
\end{multicols}

\begin{center}
{\bfseries References}
\end{center}

\begin{refs}
1. Zanne M.,~Grčić M. Challenges of LNG (Liquefied Natural Gas) Carriers
in 21" Century // Promet-Traffic \& Transportation. - 2009. - Vol.21
(1).- P.49-60. \href{https://doi.org/10.7307/ptt.v21i1.912}{DOI
10.7307/ptt.v21i1.912}.

2. Mohammad N., Ishak W. W. M., Mustapa S. I., Ayodele B. V. Natural Gas
as a Key Alternative Energy Source in Sustainable Renewable Energy
Transition: A Mini Review // Front. Energy Res. -2021. - 9:625023. DOI
\href{https://doi.org/10.3389/fenrg.2021.625023}{10.3389/fenrg.2021.625023}.

3. Kang D. W., Lee W., Ahn Y. H., Kim K., \& Lee J. W. Facile and
sustainable methane storage via clathrate hydrate formation with low
dosage promoters in a sponge matrix // Energy. - 2024. -Vol.292:130631.
DOI
\href{https://doi.org/10.1016/j.energy.2024.130631}{10.1016/j.energy.2024.130631}

4. Chakeeva K. S., Mombekova G. R., Kuzenbayeva E. R. The current state
of the oil and gas industry of the Republic of Kazakhstan//State Audit.-
2023. -Vol.2(59). - P.107.-117. DOI
\href{https://doi.org/10.55871/2072-9847-2023-59-2-107-117}{10.55871/2072-9847-2023-59-2-107-117}

5. Imangozhina Z.A. Sovremennoe sostojanie gazovoj otrasli Respubliki
Kazahstan// Vestnik universiteta Turan. -2021.-№ 1 (89) - S.201-208.
DOI
\href{https://doi.org/10.46914/1562-2959-2021-1-1-201-208}{10.46914/1562-2959-2021-1-1-201-208}.
{[}in Russian{]}

6. Kaiser, M. J., \& Pulsipher, A. G. A review of the oil and gas sector
in Kazakhstan // Energy Policy.- 2021.- Vol.35 (2). - P.1300 - 1304. DOI
\href{http://dx.doi.org/10.1016/j.enpol.2006.03.020}{10.1016/j.enpol.2006.03.020}

7. Maldynova A., Bodaukhan G., Aitkhojayeva G., Ilyas A., B.
Murzabekova. Innovative Potential of the Oil and Gas Industry of
Kazakhstan // Eurasian Journal of Economic and Business Studies. - 2023.
- Vol.67 (2). - P.33-34. DOI 10.47703/ejebs.v2i67.253

8. Besley D., Kohli S., Rahardja S., Knight D., Farole T., Collier J.
B., et.al. ``Country Climate and Development Report (CCDR):
KAZAKHSTAN,'' The World Bank Group, 1818 H Street NW, Washington, DC
20433, 2022. {[}Online{]}. Available:
\href{https://documents1.worldbank.org/curated/en/099420411012246024/pdf/P1773690ad92b401b089700f5be8659ecf0.pdf}{https://documents1.worldbank.org}.-
Date of address: 26.05.2025

9. Aregbe A. Gas Hydrate-Properties, Formation and Benefits // Open
Journal of Yangtze Oil and Gas. - 2017.- Vol.2 (1). - P.27. DOI
\href{https://doi.org/10.4236/ojogas.2017.21003}{10.4236/ojogas.2017.21003}

10. You K., Flemings P. B., Malinverno A., Collett T. S., Darnell K.
Mechanisms of Methane Hydrate Formation in Geological Systems // Reviews
of Geophysics. - 2019.- Vol.57(41). - P.1146. DOI\\
\href{https://doi.org/10.1029/2018RG000638}{10.1029/2018RG000638}

11. Satenov K. G., Tkenbayev S. M., Tashenov Z. A., Akhmetov Z. E.,
Kadyrov S. R. Kazakhstan journal for oil \& gas industry. - 2024.- Vol.
6 (1).- P.99. DOI
\href{https://doi.org/10.54859/kjogi108691}{10.54859/kjogi108691} {[}in
Russian{]}

12. Sayed A. E.-R., Ashour I., Gadalla M. Integrated process development
for an optimum gas processing plant // Chemical Engineering Research and
Design. - 2017. Vol.124.-P.114-123. DOI \\10.1016/j.cherd.2017.05.031

13. Mazumder M., Xu Q. Modeling and Optimization for a Comprehensive Gas
Processing Plant with Sensitivity Analysis and Economic Evaluation
//Chemical Engineering \& Technology. - 2020. -Vol.43 (11). - P.
2198-2207. DOI
\href{https://doi.org/1002/ceat.202000216}{1002/ceat.202000216}

14. Teixeira A. M., Arinelli L. de O., de Medeiros J. L., de Q. F.
Araújo O. Recovery of thermodynamic hydrate inhibitors methanol, ethanol
and MEG with supersonic separators in offshore natural gas processing //
Journal of Natural Gas Science and Engineering. - 2018. - Vol.52. -
P.186-186. DOI\\
\href{https://doi:10.1016/j.jngse.2018.01.038}{10.1016/j.jngse.2018.01.038}

15. Yang Xiao, Shengbin Wu, Hantao Xia, Jinyuan Zhang, Lingling Ding,
Xiaolong Bao, Yi Tang, Yongjie Qi. Simulation of and multi-aspect study
of a novel trigeneration process for crude helium, liquefied natural
gas, and methanol production; operation improvement and emission
reduction // Fuel.-2023.-Vol.347: 128402. DOI
\href{https://doi.org/10.1016/j.fuel.2023.128402}{10.1016/j.fuel.2023.128402}

16. Stryjek R., Vera J. H. PRSV: An Improved Peng-Robinson Equation of
State for Pure Compounds and Mixtures // Canadian Journal of Chemical
Engineering.-1986.- Vol.64 (2).- P.323-333. DOI\\
\href{https://doi:1002/cjce.5450640224}{1002/cjce.5450640224}

17. Proust P., Vera J. H. PRSV: The Stryjek-Vera modification of the
Peng-Robinson equation of state. Parameters for other pure compounds of
industrial interest // Canadian Journal of Chemical
Engineering.-1989.-Vol.67 (1). - P.170-173. DOI
\href{https://doi.org/10.1002/cjce.5450670125}{10.1002/cjce.5450670125}

18. AACE International Recommended Practice No.18R-97: Cost Estimate
Classification System-As Applied In Engineering, Procurement, And
Construction For The Process Industries. AACE International -- 2005.
URL: \url{https://library.aacei.org/pgd01/pgd01.shtml}. Date of address:
26.05.2025
\end{refs}

\begin{info}
\emph{{\bfseries Information about the authors}}

Satenov K. G. - candidate of Chemical Sciences, Acting Associate
Professor, Kazakh University of Technology and Business named after K.
Kulazhanov, Astana, Kazakhstan, e-mail: satenoff@mail.ru;

Suleimen Ye.M. - Ph.D., Associate Professor, Kazakh University of
Technology and Business named after. K. Kulazhanov, Astana, Kazakhstan,
e-mail: syerlan75@yandex.kz

\emph{{\bfseries Сведения об авторах}}

Сатенов К. Г. - к.х.н., и.о. ассоциированного профессора, Казахский
университет технологии и бизнеса им. К. Кулажанова, Астана, Казахстан,
e-mail: satenoff@mail.ru;

Сулеймен Е.М. - PhD, ассоциированный профессор, Казахский университет
технологии и бизнеса им. К. Кулажанова, Астана, Казахстан, e-mail:
syerlan75@yandex.kz.
\end{info}
