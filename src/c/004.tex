\id{МРНТИ 31.21.27}{}

\begin{header}
\swa{}{ЭЛЕКТРОННО-МИКРОСКОПИЧЕСКИЕ И ТЕРМОХИМИЧЕСКИЕ СВОЙСТВА
ОЛИГОСАХАРИДНЫХ КЛАТРАТНЫХ ПРОИЗВОДНЫХ ЦИТИЗИНА}

{\bfseries
\tsp{1}С.Д. Фазылов\envelope,
\tsp{2}А.Ж. Сарсенбекова,
\tsp{3}Р.Е. Бакирова,
\tsp{2}Т.С. Жумагалиева,
\tsp{1,4}А.К. Сыздыков,
\tsp{3}Л.М.~Власова,
\tsp{2}М.Т. Нурмаганбетова,
\tsp{3}Б.Ж. Аширбекова
}
\end{header}

\begin{affil}
\tsp{1}Институт органического синтеза и углехимии, Караганда, Казахстан,

\tsp{2}Карагандинский университет им. Е.А. Букетова, Қараганда, Казахстан,

\tsp{3}Карагандинский медицинский университет, Қараганда, Казахстан,

\tsp{4}Карагандинский индустриальный университет, Темиртау, Казахстан

\envelope Корреспондент-автор: iosu8990@mail.ru
\end{affil}

В статье рассматриваются электронно-микроскопические и термохимические
свойства клатратных композиций β-олигосахаридных комплексов включений
алкалоида цитизина и его диметилфосфатного производного,
функционализированные наночастицами серебра. Для визуализации и анализа
эволюции их структуры при различных температурных режимах на сканирующем
электронном микроскопе проведены морфологические исследования полученных
нанокомпозиций частиц. Методами термогравиметрического и
дифференциального термогравиметрического анализов получены данные
кинетики термического разложения изучаемых субстратов, а также их
инкапсулированных комплексов включений - цитизина и его производных с
наносеребром. Результаты исследований позволили выявить соответствующие
фазовые превращения нанокомпозиций, а также возможные процессы
рекристаллизации и другие изменения, происходящие в комплексах включений
в результате отжига. Полученные энергетические профили, воспроизведенные
различными аналитическими и расчетными методами (непараметрической
кинетики и методом Фридмана), показали, что процессы разрушения молекул
клатратов под воздействием тепла начинаются с реакций с более высокими
значениями энергии активации (\emph{E\tsb{a}}) и продолжается
с равномерным уменьшением \emph{E\tsb{a}}, что свидетельствует
о многостадийности процесса. Полученные кинетические данные способствуют
прогнозированию свойств полученных композиций клатратных комплексов
цитизина и его фосфорпроизводного, а также поиску оптимальных путей их
стабилизации в условиях длительного хранения.

{\bfseries Ключевые слова}: клатрат, комплекс включения, термическое
разложение, наночастицы, цитизин, β-олигосхарид.

\begin{header}
{\bfseries ЦИТИЗИННІҢ ОЛИГОҚАНТТЫ КЛАТРАТТЫ ТУЫНДЫЛАРЫНЫҢ
ЭЛЕКТРОНДЫ-МИКРОСКОПИЯЛЫҚ ЖӘНЕ ТЕРМОХИМИЯЛЫҚ ҚАСИЕТТЕРІ}

{\bfseries
\tsp{1}С.Д. Фазылов\envelope,
\tsp{2}А.Ж. Сәрсенбекова,
\tsp{3}Р.Е. Бәкірова,
\tsp{2}Т.С. Жұмағалиева,
\tsp{1,4}А.К. Сыздықов,
\tsp{3}Л.М.~Власова,
\tsp{2}М.Т. Нұрмағанбетова,
\tsp{3}Б.Ж. Әшірбекова
}
\end{header}

\begin{affil}
\tsp{1}Органикалық синтез және көмірхимиясы институты, Қарағанды, Қазақстан,

\tsp{2}Е.А. Букетов атындағы Қарағанды университеті, Қарағанды, Қазақстан,

\tsp{3}Қарағанды медицина университеті, Қарағанды, Қазақстан,

\tsp{4}Қарағанды индустриялық университеті, Теміртау, Қазақстан,

e-mail: iosu8990@mail.ru
\end{affil}

Мақалада цитизин алкалоидының және оның диметилфосфатты туындысының
β-олигосахаридті кешендерінің күміс нанобөлшектерімен түрлендірілген
клатратты композицияларының электрон\-ды-микроскопиялық және термохимиялық
қасиеттері қарастырылады. Әртүрлі температуралық режимдерде алынған
бөлшектердің нанокомпозицияларының құрылымының эволюциясын
визуализациялау және талдау үшін сканерлеуші электронды микроскопта
морфологиялық зерттеулер жүргізілді. Термогравиметриялық және
дифференциалды термогравиметриялық талдау әдістерімен зерттелетін
субстраттардың термиялық ыдырау кинетикасының деректері, сондай-ақ
олардың цитизин мен оның нанокүмісті туындыларының инкапсуляцияланған
кешендері зерттелді. Зерттеу нәтижелері нанокомпозицияның тиісті фазалық
өзгерістерін, сондай-ақ қайта кристалдану процестерін және күйдіру
нәтижесінде инклюзия кешендерінде болатын басқа өзгерістерді анықтады.
Алынған энергетикалық профильдер әртүрлі аналитикалық және есептеу
әдістерімен (параметрлік емес кинетика және Фридман әдісі) тексеріліп,
жылу әсерінен клатрат молекулаларының ыдырау процестері белсендіру
энергиясының (Ea) жоғары мәндері бар реакциялардан басталатынын және Ea
біркелкі төмендеуімен жалғасатынын көрсетті, бұл процесстердің көп
сатылы екенін білдіреді. Алынған кинетикалық деректер цитизин мен оның
фосфор туындыларының клатрат кешендерінің алынған композицияларының
қасиеттерін болжауға және ұзақ мерзімді сақтау жағдайында оларды
тұрақтандырудың оңтайлы жолдарын іздеуге ықпал етеді.

{\bfseries Түйін сөздер}: клатрат, қосу кешені, термиялық ыдырау,
нанобөлшектер, цитизин, β-олигосхарид.

\begin{header}
{\bfseries ELECTRON MICROSCOPIC AND THERMOCHEMICAL PROPERTIES OF CYTISINE
OLIGOSACCHARIDE CLATHRATE DERIVATIVES}

{\bfseries
\tsp{1}S.D. Fazylov\envelope,
\tsp{2}A.Zh. Sarsenbekova,
\tsp{3}R.E. Bakirova,
\tsp{2}T.S. Zhumagaliyeva,
\tsp{1,4}A.K. Syzdykov,
\tsp{3}L.M.~Vlasova,
\tsp{2}М.Т. Nurmaganbetova,
\tsp{3}B.Zh. Ashirbekova
}
\end{header}

\begin{affil}
\tsp{1}Institute of Organic Synthesis and Coal Chemistry, Karaganda, Kazakhstan,

\tsp{2}E.A. Buketov Karaganda University, Karaganda, Kazakhstan,

\tsp{3}Karaganda Medical University, Karaganda, Kazakhstan,

\tsp{4}Karaganda Industrial University, Temirtau, Kazakhstan,

e-mail: iosu8990@mail.ru
\end{affil}

The article discusses the electron microscopic and thermochemical
properties of clathrate compositions of β-oligosaccharide complexes of
cytisine alkaloid inclusions and its dimethylphosphate derivative
func\-tionalized with silver nanoparticles. Morphological studies of the
obtained nanocompositions of particles were performed on a scanning
electron microscope to visualize and analyze the evolution of their
structure under various temperature conditions. Thermogravimetric and
differential thermogravimetric analyses were used to obtain data on the
kinetics of thermal decomposition of the studied substrates, as well as
their encapsulated complexes of cytisine inclusions and its derivatives
with nanosilver. The research results revealed the corresponding phase
transformations of the nanocomposition, as well as possible
recrystallization processes and other changes occurring in the complexes
of inclusions as a result of annealing. The obtained energy profiles,
reproduced by various analytical and computational methods
(nonparametric kinetics and the Friedman method), showed that the
processes of destruction of clathrate molecules under the influence of
heat begin with reactions with higher activation energy values (Ea) and
continue with a uniform decrease in Ea, which indicates a multi-stage
process. The kinetic data obtained contribute to predicting the
properties of the obtained compositions of cytisine clathrate complexes
and its phosphorous derivative and to finding optimal ways to stabilize
them under long-term storage conditions.

{\bfseries Key words}: clathrate, inclusion complex, thermal decomposition,
nanoparticles, cytisine, β-oligoschar\-ide.

\begin{multicols}{2}
{\bfseries Введение}. По данным Всемирной организации здравоохранения
ежегодно около миллиона человек умирают от различных видов
респирательных заболеваний. При многократном применений известных
противовирусных препаратов возникает резистентность штаммов вирусов к
известным лекарственным препаратам. Это заставляет специалистов искать
новые пути преодоления этих проблем путем создания новых лексредств.
Одним из перспективных объектов при поиске и создании новых антивирусных
препаратов является природный алкалоид цитизин (ЦН), содержащийся в
семенах растений \emph{Thermopsis lanceolata} произрастающий в южных
районах Казахстане. ЦН обладает «ганглиозным» действием и, благодаря
стимулирующему воздействию на дыхание, является дыхательным аналептиком
{[}1,2{]}. Поэтому возможные противовоспалительные, спазмолитические,
антиаритмические, антивирусные и нейротропные свойства ЦН и его
модифицированные производных в настоящее время широко изучаются. По
литературным данным, ряд новых синтетических производных ЦН показали
цитотоксичность и вирусингибирующую активность в отношении респираторных
вирусов человека: вирусов гриппа \emph{А} подтипов H1N1, H3N2, H5N2,
вирусов гриппа В линий \emph{B/Yamagata} и \emph{B/Victoria} и вируса
парагриппа человека типа 3. В работах {[}3-7{]} изучено некоторые
особенности влияния ЦН и его производных на различных стадиях
репродукции вируса гриппа. Однако ЦН относится к высокотоксичным
веществам с LD\tsb{50} 8-10 мг/кг. Синтезированное нами
N-цитизинил-О,О-диметилфосфатное производное (ЦНФ) проявило высокое
противовирусное действие по отношению гепатита \emph{В} при низкой
токсичности (LD\tsb{50} 1800 мг/кг) {[}8{]}. Перечисленные
выше особенности физиологического действия ЦН и его производного ЦНФ
свидетельствуют о перспективности изучения их как эффективных
ингибиторов протеаз вируса COVID-19 и рецептора ACE2. В предыдущих
исследованиях нами были описаны получение и особенности инкапсуляции ЦН
и его диметилфосфатного производного с β-олигосахаридом, содержащем
наночастицы серебра {[}8,9{]}. Это исследование демонстрирует
морфологические и термохимические характеристики нового комплекса
включения ЦН и его ЦНФ производного в качестве органических лигандов в
новой нанокомпозиций β-ЦД-Ag. Разработанная нами композиция находится в
защитной оболочке натурального олигосахарида. Новую композицию
β-ЦД-ЦН(ЦНФ), модифицированную с наночастицами Ag, можно считать
многообещающей платформой для улучшения стабильности и расширения их
потенциала в биомедицинском применении {[}9{]}.

ЦД (α-, β- и γ-ЦД) - это натуральные продукты, образующиеся в результате
расщепления крахмала {[}10{]}. ЦД являются специфическими реагентами для
восстановления солей металлов и могут связываться с поверхностью
наночастиц посредством хемосорбции. β-ЦД широко используется в
инкапсуляции биосубстратов для защиты их от окисления, снижения
токсичности, а также в предотвращении агрегации наночастиц металлов, тем
самым способствуя их стабильности в растворе.

Исследования тепловых свойств клатратоподобных соединений представлены в
периодической литературе фрагментарно. В некоторых работах изучены
электрические свойства клатратов {[}11{]}, тепловое расширение {[}12{]}
и теплопроводность {[}13{]} при низких температурах. Изучены
термодинамические свойства некоторых клатратов при высоких температурах
{[}14,15,16{]}. Изучение термодинамических свойств клатратов и
клатратоподобных соединений при низких температурах является актуальным,
так как именно в этом диапазоне температур возможно определение их
физических параметров, необходимых для анализа и прогнозирования свойств
клатратов в широком температурном интервале, в том числе и при
повышенных температурах.

Цель работы заключалась в изучении закономерностей изменений
кинетических и термодинамических данных в зависимости от структуры и
состава полученных композиций, способствующих прогнозированию свойств
клатратных комплексов цитизина и его фосфорпроизводного, а также поиску
оптимальных путей их стабилизации в условиях длительного хранения.

{\bfseries Материалы и методы.}~Объектами исследования служили цитизин (ЦН)
(Sigma, USA), О,О-диметиловый эфир цитизиниламидофосфат (ЦНФ), комплексы
включения ЦН и его диметилфосфатного производного (ЦНФ), а также их
композиций, модифицированные наносеребром (ЦН(ЦНФ)-β-ЦД-Ag). Получение и
физико-химические характеристики некоторых исходных объектов
исследований описаны нами в предыдущих работах {[}8,9{]}. В этом
исследований нами приведены новые научные результаты по морфологическим
и термохимическим характеристикам комплексов включений ЦН, его
фосфатного производного ЦНФ в качестве органических лигандов c
наночастицами серебра (β-ЦД-ЦН(ЦНФ)-Ag). Поверхностную морфологию
наночастиц определяли с помощью сканирующего электронного микроскопа
Tescon Mira 3 LMN (Czech Republic). Образцы прикрепляли к проводящей
адгезивной поверхности и наблюдали при ускоряющем напряжении 15 кВ.
Исследования кинетики термического разложения изучаемых объектов,
полученные методами термогравиметрического (TГ) и дифференциального
термогравиметрического анализа (ДTГ), проведены в атмосфере азота.

{\bfseries Обсуждение и результаты.} Для полноценного понимания изменений
морфологии комплексов цитизина (ЦН, ЦН-β-ЦД, β-ЦД-ЦН-Ag и β-ЦД-ЦНФ-Ag)
(в атмосфере азота) проведены исследования, направленные на анализ
эволюции их структуры под воздействием различных температурных режимов
(рис.1-7).

Данная работа позволила выявить фазовые превращения, возможные процессы
рекристаллизации и прочие изменения, происходящие в комплексах включения
в результате отжига. На СЭМ-снимках ЦН (рис.1-7) отчетливо наблюдается
сложная структура, после термической обработки, что указывает на наличие
пористой микроструктуры. Низкотемпературная обработка ЦН при температуре
до 90°C (рис.1-3(I, II, III) практически не вызвала изменений. На
рисунках 1-5 приведены результаты термообработки ЦН(ЦНФ), и его
клатратных комплексов при различных температурных режимах.
Изменение морфологии пленок образцов при проведении процессов отжига
материалов (90-360\tsp{о}С) было проанализировано при помощи
сканирующей электронной микроскопии (СЭМ). При термообработке молекулы
ЦН и его амидофосфатного производного ЦНФ наблюдается легкое
деформирование структур, что сопровождается потемнением материала
(рис.1-5a-e). Это связано с процессами деградации и карамелизации
сахарных остатков β-ЦД. При проведении процесса отжига молекулы
исходного ЦН при температуре 160°C фиксируется формирование нового
рельефа, характеризующегося мягкими, плавными контурами.
\end{multicols}

\fig[\textwidth]{c/image38}{{\normalfont\emph a - 90°C\hspace{2cm} b - 160°C\hspace{2cm} c - 250°C\hspace{2cm} d - 315°C\hspace{2cm} e - 360°C}\\Рис.1 - Отожженный ЦН}

\fig[\textwidth]{c/image39}{{\normalfont\emph a - 90°C\hspace{3cm} b - 160°C\hspace{3cm} c - 250°C\hspace{3cm} d - 360°C}\\Рис.2 - Отожженный ЦНФ}

\fig[\textwidth]{c/image40}{{\normalfont\emph a - 80°C\hspace{2cm} b - 270°C\hspace{2cm} c - 300°C\hspace{2cm} d - 350°C\hspace{2cm} e - 450°C}\\Рис.3 - Отожженный комплекс включения β-ЦД-ЦН с мольным соотношением 1:1}

\fig[\textwidth]{c/image41}{{\normalfont\emph a - 90°C\hspace{2cm} b - 160°C\hspace{2cm} c - 250°C\hspace{2cm} d - 315°C\hspace{2cm} e - 360°C}\\Рис.4 - Отожженный комплекс включения β-ЦД-ЦН-Ag}

\fig[\textwidth]{c/image42}{{\normalfont\emph a - 90°C\hspace{2cm} b - 160°C\hspace{2cm} c - 250°C\hspace{2cm} d - 315°C\hspace{2cm} e - 360°C}\\Рис.5 - Отожженный комплекс включения β-ЦНФ-ЦД-Ag}

\begin{multicols}{2}
Дальнейшее обугливание ЦН приводит к значительному изменению цвета
продукта, выражающемуся в переходе к темно-бежевому оттенку (рис.1b).
Данные изменения свидетельствуют о наличии интенсивных преобразований во
внутренней структуре вещества, а также в его физико-химических
свойствах, вызванных термическим воздействием. Эти изменения связаны с
фазовыми переходами, химическими реакциями или реорганизацией
молекулярной структуры ЦН. С повышением температуры до 315°C отмечается
усиленное изменение микрорельефа поверхности данного вещества. Это
выражается в образовании локальных выпуклостей и впадин, что придает ему
значительную шероховатость (рис.1d). Отжиг при температуре 315°C
приводит к некоторому сглаживанию рельефа поверхности, наблюдается
появление металлического блеска на поверхности образцов продукта (рис.
1d). В этот период мембраны становятся более хрупкими. При дальнейшем
отжиге при 360°C фиксируется явное разрушение и рассыпание вещества, что
указывает на практически полный распад ЦН (рис.1e).

На рис.2-5 (a-e) показаны микрофотографии отожженных частиц ЦНФ и его
комплексов - β-ЦД-ЦН, β-ЦД-ЦН-Ag и β-ЦД-ЦНФ-Ag. Исследование морфологии
поверхности комплекса включения β-ЦД-ЦН (2:1) (рис.2) показало, что
последний представляет собой достаточно сложную
структурно-морфологическую организацию. Диаметр частиц серебра
колеблется от единиц до нескольких десятков нанометров. На рис.6
представлены СЭМ-фотографии поверхности образцов комплекса включения
β-ЦД с ЦН, ЦНФ и др. Как следует из анализа данных рис.6, термическое
напряжение, возникающее при нагреве, вызывает трещины и деформации во
всех клатратных материалах.
\end{multicols}

\figstart{Рис.6 - Сканированные электронные микрофотографии образцов:\\(I, II, III) − β-ЦД-ЦН (2:1); (IV, V, VI) - β-ЦД-ЦНФ (2:1);\\(VII, VIII, IX) − β-ЦД-ЦН-Ag; (X, XI, XII) − β-ЦД-ЦНФ-Ag}
\subfig[0.49\textwidth]{5cm}{c/image43}{}\hfill
\subfig[0.49\textwidth]{5cm}{c/image44}{}
\figend

\figstart{Рис.7 - Изображение образца ЦН после термической обработки при
90°С (I, II, III), 160°C (IV, V, VI), 250°C (VII, VIII, IX), 315°C (X,
XI, XII), 360°C (XIII, XIV, XV)}
\subfig[0.49\textwidth]{5cm}{c/image45}{}\hfill
\subfig[0.49\textwidth]{5cm}{c/image47}{}
\subfig[0.49\textwidth]{2.5cm}{c/image46}{}
\figend

\begin{multicols}{2}
При температуре 160°C (рис.7 (IV, V, VI) начинает проявляться первичная
рекристаллизация, при этом наблюдаются небольшие трансформации в
морфологии частиц. При повышении температуры до 250°C (рис.7 (VII, VIII,
IX) фиксируются процессы рекристаллизации и начало фазовых превращений.
Кристаллические фазы становятся более явно выраженными, а также
появляются новые структурные формы. При температуре 315°C (рис.7 (X, XI,
XII) кристаллы увеличиваются в размерах, и начинают формироваться новые
фазы. В этой температурной зоне существуют риски появления как
однородных структур, так и неравномерностей в распределении (рис.7 (X,
XI, XII). При температуре 360°C (рис.7 (XIII, XIV, XV) происходит полная
декомпозиция ряда исходных структур и образование новых фаз. Наблюдаются
значительные разрушения кристаллической решетки, что приводит к
изменению внешнего вида образца и делает его менее однородным (рис.7
(XIII, XIV, XV). На СЭМ снимках ЦНP (рис.8 (I, II, III) отчетливо
наблюдается его сложная структура, представленная в виде округлых
образований с плавными контурами. Поверхность этих образований
демонстрирует значительное разнообразие текстур. Это может указывать на
снижение прочности связи между молекулами и способствовать образованию
макроскопических структур (рис.8 (I, II, III). При нагреве ЦНP до 250°C
наблюдается разъединение отдельных частиц, что приводит к изменению их
общего размера и распределения. Это явление связано с термической
активностью и сопровождается разложением на меньшие компоненты (рис.8
(VII, VIII, IX).
\end{multicols}

\figstart{Рис.8 - Изображение образца ЦНФ после термической обработки при: 90°С (I, II, III), 160°C (IV, V, VI), 250°C (VII,
VIII, IX), и 360°C (X, XI, XII)}
\subfig[0.49\textwidth]{5cm}{c/image48}{}\hfill
\subfig[0.49\textwidth]{5cm}{c/image49}{}
\figend

\begin{multicols}{2}
При длительном нагреве до 360°C формируются специфические повреждения,
такие как микротрещины. Данные дефекты существенно влияют на
механические свойства ЦНP и становятся отчетливо видимыми на
СЭМ-изображениях (рис.8 (X, XI, XII).

Эмпирические наблюдения свидетельствуют о присутствии на поверхности
зерен линий скольжения (рис.9A (II, I-III). Как видно из рис.9A (IV-
IX), после термической обработки комплекса включения β-ЦД-ЦН (2:1) до
270°С произошли существенные изменения в морфологии клатрата.
Термическая обработка комплекса включения β-ЦД-ЦН (2:1) приводят к
образованию многочисленных групп линий скольжения с различными
ориентациями на поверхности (рис.9A II, VII) и возникновению вторичных
трещин (рис.9A II, VIII). Излом, образующийся непосредственно от
микротрещин по линиям скольжения, свидетельствует о том, что
образовавшиеся линии скольжения при повышении температуры являются
очагами разрушения (рис.9A II, IX).

Согласно данным СЭМ (рис.9Б (I, II, III), частицы комплекса включения
β-ЦД-ЦН-Ag имеют преимущественно сферическую форму. Это указывает на
наличие наночастиц Ag в структуре комплекса и на их изменения под
воздействием тепла. По данным СЭМ изображений поверхности образцов
комплексов включений β-ЦД-ЦН-Ag, сформированных при температуре 250°С,
присутствуют частицы округлой формы, средний размер которых около 3 нм
(рис.9Б (VII, VIII, IX). При проведении дальнейшей термообработки эти
частицы могут увеличиваться в размере до 50 нм и более (рис.9Б (VII,
VIII, IX). Таким образом, можно предположить, что наблюдаемые агрегаты
сферической формы являются частицами серебра, которые увеличиваются в
размере при дальнейшей термической обработке. После термической
обработки (рис.9Б (X, XI, XII) при температуре 315°C в контексте
репликации наблюдается преимущественное присутствие не агрегированных
частиц округлой формы в комплексе включения β-ЦД-ЦН-Ag, средний размер
которых составляет приблизительно 50 нм. Существенные изменение
морфологии комплекса включения β-ЦД-ЦНФ-Ag наблюдается после их
термообработки при 360°С. Как следует из рис.10 (XIII, XIV) в комплексе
включения присутствуют полидисперсные частицы сферической формы (размер
основной фракции 50-100 нм).
\end{multicols}

\figstart{Рис.9: А - Микрофотография комплекса включения β-ЦД-ЦН (1:1)
после термической обработки при 80°С (I, II, III), 270°C (IV, V, VI),
300°C (VII, VIII, IX), 350°C (X, XI, XII), 450°C (XIII, XIV, XV);\\Б -
Микрофотография комплекса включения β-ЦД-ЦН-Ag после термической
обработки при 90°С (I, II, III), 160°C (IV, V, VI), 250°C (VII, VIII,
IX), 315°C (X, XI, XII), 360°C (XIII, XIV, XV)}
\subfig[0.49\textwidth]{13.5cm}{c/image50}{А}\hfill
\subfig[0.49\textwidth]{13.5cm}{c/image51}{Б}
\figend

\begin{multicols}{2}
Согласно данным СЭМ (рис.10, I, II, III), частицы комплекс включения
β-ЦД-ЦНФ-Ag имеют преимущественно сферическую форму. Это указывает на
наличие наночастиц серебра в структуре комплекса и на их изменения под
воздействием тепла. При температуре 160°C и выше происходит более
интенсивная термическая деградация компонентов комплекса (рис.10 IV, V,
VI), что может повлиять на их взаимодействие и стабильность.
Существенные изменение морфологии комплекса включения β-ЦД-ЦНФ-Ag
наблюдается после их термообработки при 360°С. Как видно из рис.10,
XIII, XIV, в комплексе включения присутствуют полидисперсные частицы
сферической формы (размер основной фракции 50-100 нм). Реакции
термического разложения таких продуктов, как ЦН, ЦНФ и их комплексы
включения β-ЦД-ЦН, β-ЦД-ЦН-Ag, β-ЦД-ЦНФ-Ag, можно классифицировать как
топологические процессы с локализацией реакционной зоны на поверхности
раздела твердого реагента и продукта {[}17,18{]}. В таких условиях
концентрация реагента утрачивает свое первоначальное значение, и более
удобно использовать параметр \emph{α}, который представляет собой долю
прореагировавшего вещества к определенному моменту времени {[}19-22{]}.
Начальное значение параметра \emph{α} равно \emph{0} (в начальный момент
времени), а при завершении термического процесса - \emph{α} достигает
значения \emph{1}.

Математическую модель рассматриваемых реакций можно представить с
использованием дифференциального уравнения с начальным условием,
отражающим значение \emph{α} для реагента \emph{A} в момент начала
реакции (\emph{t=0}). В контексте топологических процессов, участвующих
в термическом разложении твердых веществ, важно учитывать особенности
локализации реакций на поверхности твердых фаз и динамику изменения
степени превращения реагентов в продукты.

Почти все методы расчета кинетических параметров по рассматриваемым
данным термогравиметрии, основаны на применении уравнения:

\begin{equation}
\frac{d\alpha}{dt}-\beta\frac{d\alpha}{dT}-k(T)f(\alpha)-Aexp(-\frac{E_\alpha}{RT})f(\alpha)
\end{equation}
\end{multicols}

\figstart{Рис.10 - Изображения образца комплекса включения β-ЦД-ЦНФ-Ag
после термической обработки при: 90°С (I, II, III), 160°C (IV, V, VI),
250°C (VII, VIII, IX), 315°C (X, XI, XII), 360°C (XIII, XIV, XV)}
\subfig[0.49\textwidth]{7.5cm}{c/image54}{}\hfill
\subfig[0.49\textwidth]{5cm}{c/image55}{}
\figend

\begin{multicols}{2}
Интерес ученых {[}19-25{]} в ходе проведения исследований привлекает
факт, заключающийся в отсутствии прямой связи между рассчитанными на
основе изотермических данных кинетическими характеристиками и выбранной
моделью. Метод Фридмана {[}19{]} является наиболее распространенным и
часто используемым изоконверсионным методом. Этот метод основывается на
следующем уравнении:

\begin{equation}
ln(-\frac{dx}{dt})=lnA+lnf(x)-E/RT
\end{equation}

Многие приближения имеют общую форму линейного уравнения:

\begin{equation}
ln(\frac{\Beta_i}{T^\Beta_{\alpha,i}})=const-C(\frac{E_\alpha}{RT_\alpha})
\end{equation}

где \emph{β} и \emph{ɑ} -- параметры, определяющие тип температурного
интегрального приближения.

Согласно данных авторов {[}19-25{]}, характер протекания реакций с
увеличением скорости процесса эффективно описывает также метод
непараметрической кинетики (МНПК). Одной из ключевых особенностей этого
метода является возможность определения формально-кинетических
параметров на основе заданного набора данных (dα/dt, T и α) при их
обработке. Рассмотрим подробнее математическую сторону МНПК. Метод
непараметрической кинетики {[}20{]} основан на построении матрицы,
содержащий информацию по \emph{k(Т)} и \emph{f(α)}. МНПК основывается на
формировании матрицы, которая содержит информацию о функциях \emph{k(T)}
и \emph{f(α)}. Подход к этой матрице может быть представлен с
использованием алгоритма сингулярного разложения {[}21,21{]}. Ниже
приведена кинетическая матрица метода НПК, в которой скорость реакции
представляется как произведение двух независимых функций
\(k(T) = \bigl[ \Sigma_{1} v_{1}, \; \Sigma_{1} v_{2}, \; \cdots, \; \Sigma_{1} v_{j} \bigr]\)
и \(f(\alpha) = \bigl[ u_{1},\, u_{2},\, \dots,\, u_{i} \bigr]\).
\end{multicols}

Кинетическая матрица:

\begin{equation}
M = \{ m_{ij} \} =
\begin{bmatrix}
r(\alpha_{1}, T_{1}) & r(\alpha_{1}, T_{2}) & \cdots & r(\alpha_{1}, T_{j}) \\
r(\alpha_{2}, T_{1}) & r(\alpha_{2}, T_{2}) & \cdots & r(\alpha_{2}, T_{j}) \\
\vdots & \vdots & \ddots & \vdots \\
r(\alpha_{i}, T_{1}) & r(\alpha_{i}, T_{2}) & \cdots & r(\alpha_{i}, T_{j})
\end{bmatrix}
\end{equation}

\begin{multicols}{2}
Экспериментальные данные скорости реакции были получены путем вычисления
согласно уравнению (4) и могут быть представлены в трехмерной системе
координат {[}21-23{]}. После применения алгоритма сингулярного
разложения (SVD), матрица M представляет собой вектор \emph{S} с двумя
значимыми величинами. В данном случае матрица М может быть выражена как
сумма:

\begin{equation}
M = M_{1} + M_{2} = u_{1}\nu_{1}^{T} + u_{2}\nu_{2}^{T}
\end{equation}

Это предполагает наличие двух базовых процессов на этапе разложения, и
различие между ними можно определить по значениям объяснимой дисперсии
$\lambda_1$ и $\lambda_2$ $(\lambda_{1} + \lambda_{2} \approx 100\%)$.

Векторы $u_1$ и $u_2$ были сопоставлены с уравнением Шестака-Берггрена
{[}17-19{]}

\begin{equation}
f(a) = \alpha^{m} (1 - \alpha)^{n}
\end{equation}

и векторы $v_1$ и $v_2$, с уравнением Аррениуса, соответственно.

При сравнительном анализе показателей энергии активации (\emph{Е}),
определенные при помощи методов непараметрической кинетики, а также
Шестака-Берггрена (Ш-Б) и Фридмана (ФР) {[}19-25{]}, соединений ЦН, ЦНФ,
комплекса β-ЦД-ЦН (в соотношении 1:1), комплексa β-ЦД-ЦН-Ag, комплекс
β-ЦД-ЦНФ-Ag, установлено, что при понижении параметра \emph{α}=0.1
процесс становится более замедленным, в то время как его увеличение
приводит к ускорению реакции термической деструкции. Показатели энергии
активации (\emph{Е}), определенные при помощи методов непараметрической
кинетики, а также Шестака-Берггрена (Ш-Б) и Фридмана (ФР) {[}19-25{]},
приведены в таблице. Такие результаты говорят о важности контроля
данного параметра для эффективного регулирования процессов, связанных с
данной системой веществ. При увеличении значения энергии активации
(α=0.1) процесс разложения комплекса β-ЦД-ЦН становится медленнее, а при
уменьшении данного параметра процесс протекает быстрее (α=0.2 до
(α=0.8). Аналогичная картина наблюдается и при анализе термических
характеристик β-ЦД-ЦН-Ag, ЦНФ и β-ЦД-ЦНФ-Ag, что свидетельствует о
значительном влиянии изменения энергии активации на скорость разложения
данных комплексов.
\end{multicols}

\tcap{Таблица 1 - Кинетические параметры процесса термодеструкции ЦН, ЦНФ,
их клатратов с β-ЦД, а также их нанокомпозиций с β-ЦД-ЦН-Ag,
β-ЦД-ЦНФ-Ag}
\begin{longtblr}[
  label = none,
  entry = none,
]{
  width = \linewidth,
  colspec = {Q[133]Q[110]Q[96]Q[88]Q[88]Q[110]Q[96]Q[110]Q[96]},
  cells = {font = \small},
  cells = {c},
  cell{1}{1} = {r=3}{},
  cell{1}{2} = {r=3}{},
  cell{1}{3} = {r=3}{},
  cell{1}{4} = {c=2}{},
  cell{1}{6} = {r=3}{},
  cell{1}{7} = {r=3}{},
  cell{1}{8} = {r=3}{},
  cell{1}{9} = {r=3}{},
  cell{2}{4} = {c=2}{},
  vlines,
  hline{1,4-10} = {-}{},
  hline{2-3} = {4-5}{},
}
Образец & {$\overline{E}_{\mathrm{NPK}}$,\\кДж моль\textsuperscript{−1}} & $\overline{A}$, с\textsuperscript{−1} & Шестак-Берггрена & & {$\overline{E}_{S-B}$,\\кДж моль\textsuperscript{−1}} & $\overline{A}$, с\textsuperscript{−1} & {$\overline{E}_{\mathit{FR}}$,\\кДж моль\textsuperscript{−1}} & $\overline{A}$, с\textsuperscript{−1} \\
            &                                   &                                               & $\alpha^{m} (1 - \alpha)^{n}$ &      &                                   &                             &                                   &                                               \\
            &                                   &                                               & m                                                                                & n    &                                   &                             &                                   &                                               \\
ЦН          & 89.05                             & 1.46×10\textsuperscript{5}                    & 0.65                                                                             & 0.34 & 89.05                             & 2.02×10\textsuperscript{6}  & 89.05                             & 1.46×10\textsuperscript{5}                    \\
ЦНФ         & 90.12                             & 1.54×10\textsuperscript{18}                   & 0.51                                                                             & 0.75 & 91.09                             & 2.05×10\textsuperscript{19} & 90.01                             & 2.14×10\textsuperscript{18}                   \\
β-ЦД        & 83.94                             & 1.56×10\textsuperscript{1}\textsuperscript{5} & 0.47                                                                             & 0.53 & 84.60                             & 4.01×10\textsuperscript{16} & 83.41                             & 2.72×10\textsuperscript{1}\textsuperscript{5} \\
β-ЦД-ЦН     & 93.73                             & 1.25×10\textsuperscript{18}                   & 0.35                                                                             & 0.64 & 93.59                             & 1.00×10\textsuperscript{19} & 91.23                             & 1.76×10\textsuperscript{11}                   \\
β-ЦД-ЦН-Ag  & 85.62                             & 2.04×10\textsuperscript{10}                   & 0.53                                                                             & 0.47 & 85.09                             & 7.82×10\textsuperscript{10} & 85.62                             & 2.10×10\textsuperscript{10}                   \\
β-ЦД-ЦНФ-Ag & 93.03                             & 2.89×10\textsuperscript{18}                   & 0.66                                                                             & 1.34 & 94.32                             & 1.05×10\textsuperscript{12} & 90.94                             & 1.01×10\textsuperscript{12}                   
\end{longtblr}

\begin{multicols}{2}
{\bfseries Выводы.} Исследованы электронно-микроско\-пические свойства и
кинетика термического разложения комплексов включения β-ЦД-ЦН и
β-ЦД-ЦНФ, а также их композиций с наночастицами серебра. Эти
исследования проведены с целью определения кинетических параметров этих
реакций в изотермических условиях, обеспечивающих сохранение
кинетического триплета и более точное описание процесса. Полученные
энергетические профили, воспроизведенные различными аналитическими и
расчетными методами, показали, что процесс разрушения молекул клатратов
под воздействием тепла начинается с реакций с более высокими значениями
энергии активации (\emph{E\tsb{a}}) и продолжается с
равномерным уменьшением \emph{E\tsb{a}} вдоль пути реакции,
что характерно многостадийным процессам. Показано, что полученные
кинетические данные могут способствовать прогнозированию характера
устойчивости комплексов включения β-ЦД-ЦН и β-ЦД-ЦНФ с наночастицами
серебра в условиях длительного хранения и поиску оптимальных путей их
стабилизации. Понимание механизмов деградации и стабилизации этих
комплексов является ключевым для оптимизации их использования в реальных
условиях. Полученные данные подчеркивают важность разбора
термодинамических и кинетических аспектов термического разложения
комплексов, что открывает новые горизонты для разработки новых
нанокомпозиций. В будущем дальнейшие эксперименты могут быть направлены
на изучение влияния различных внешних факторов, таких как температура и
влажность на стабильность и активность этих комплексов. Также стоит
отметить, что использование наноразмерных частиц серебра может повысить
эффективность антиоксидантной и антимикробной активности созданных
систем, открывая тем самым новые возможности для применения их в
фармацевтической и косметической промышленностей.
\end{multicols}

\begin{center}
{\bfseries Литература}
\end{center}

\begin{refs}
1. Wolker N., Howe C., Glover M., McRobbie H., Barnes J. Cytisine versus
Nikotine for Smoking Cessation // The New England Journal of Medicine. -
2014. - Vol.371(25). - Р.
2353-2362.~\href{https://doi.org/10.1056/nejmoa1407764}{DOI\\
10.1056/nejmoa1407764}.

2. Prochaska J.J., Das S., Benowitz N.L. Cytisine, the world's oldest
smoking cessation aid // BMJ. ­- 2013. - Vol.347. - Р.198. DOI
10.1136/bmj.f5198.

3. Tsypysheva I.P., Koval'skaya A., Petrova P., Lobov A., Borisevich
S.S., Tsypyshev D., Fedorova V.A., Gorbunova E.A., Galochkina A.V.,
Zarubaev V.V. Diels-Alder Adducts of Nsubstituted Derivatives of
(-)-Cytisine as Influenza A/H1N1 Virus Inhibitors; Stereo
differentiation of antiviral Properties and Preliminary Assessment of
Action Mechanism // Tetrahedron.- 2019. -Vol.75(21). - P.2933-2943.
DOI
\href{https://doi.org/10.1016/j.tet.2019.04.021}{10.1016/j.tet.2019.04.021}.

4. Beard E., Shahab L., Cummings D. M., Michie S., West R. New
pharmacological agents to aid smoking cessation and tobacco harm
reduction: what has been investigated, and what is in the pipeline? //
CNS Drugs. -2016. -Vol.30. -Р.1-33. DOI
\href{https://doi.org/10.1007/s40263-016-0362-3}{10.1007/s40263-016-0362-3}.

5. Fedorova V.A., Kadyrova R.A., Slita A.V., Muryleva A.A., Petrova
P.R., Kovalskaya A.V., Lobov A.N., Zileeva Z.R., Tsypyshev D.O.,
Borisevich S.S., Tsypysheva I.P., Vakhitova J.V., Zarubaev V.V.
Antiviral activity of amides and carboxamides of quinolizidine alkaloid
(-)-cytisine against human influenza virus A(H1N1) and parainfluenza
virus type 3 // Natural Product Research. -2021. -- Vol.35.
-Р.4256-4264.~\href{https://doi.org/10.1080/14786419.2019.1696791}{DOI
10.1080/14786419.2019.1696791}.

6. Vakhitova Yu.V., Farabontova E.I., Zainullina L.F., Vakhitov V.A.,
Tsypysheva I., Yunusov M.S. Search of (-)-cytisine derivatives as
potential inhibitors of NF-kB and STAT1 // Russian Journal of Biorganic
chemistry. -2015. -Vol.41(3). - P.297-304.
DOI:~\href{http://dx.doi.org/10.1134/S1068162015030103}{10.1134/S1068162015030103}.

7. ~ Thomas D., Farrel M., Mcrobbie H., Tutka P. The effectiveness,
safety and cost-effectiveness of cytisine versus varenicline for smoking
cessation in an Australian population: A study protocol for a randomised
controlled non-inferiority trial // Society for the Study of Addiction.
-2018. -Vol.114(5). - P.923-933.
\href{https://doi.org/10.1111/add.14541}{DOI 10.1111/add.14541}.

8. Fedorova V. A., Kadyrova R. A., Slita A. V., Muryleva A. A., Petrova
P. R., Kovalskaya A. V., Lobov A. N., Zileeva Z. R., Tsypyshev D. O.,
Borisevich S. S., Tsypysheva I. P., Vakhitova J. V., Zarubaev V. V.
Antiviral activity of amides and carboxamides of quinolizidine alkaloid
(-)-cytisine against human influenza virus A(H1N1) and parainfluenza
virus type 3. // Natural Product Research. -2021. -Vol.35(22). -
P.4256-4264.~\href{https://doi.org/10.1080/14786419.2019.1696791}{DOI
10.1080/14786419.2019.1696791}.

9. Muldakhmetov Z., Fazylov S., Gazaliev A., Nurkenov O., Seilkhanov O.
The synthesis of new inclusion compounds complexes
cytosine-β-cyclodextrin // News of the national Academy of Sciences of
the Repub\-lic of Kazakhstan Series Chemistry and technology. -2022. -Vol.
2(51). - Р.112-120.
\href{https://doi.org/10.32014/2022.2518-1491.107}{DOI\\
10.32014/2022.2518-1491.107}.

10. Larsen K.L. Large cyclodextrins // Journal of Inclusion Phenomena
and Macrocyclic Chemistry. -2002. -Vol.43(1). -P.1-13.
\href{https://doi.org/10.1023/A:1020494503684}{DOI
10.1023/A:1020494503684}.

11. Nolas G.S., Cohn J., Slack GA et al. Semiconducting Ge clathrates:
Promising candidates for thermo\-electric applications // Applied Physics
Letters. -1998. -Vol.73, № 2. -P.178-180. DOI 10.1063/1.121747.

12. Sales B.C., Chakoumakes B.C., Jin R. et al. Structural, Magnetic,
Thermal, and Transport Properties of X8Ga16Ge30 (X=Eu, Sr, Ba) single
crystals // Physical Review B. -2001. -Vol.63(21). -P.245113-245113.
DOI 10.1103/PhysRevB.63.245113.

13. Cohn J.L., Nolas G.S. Glasslike Heat Conduction in High-Mobility
Crystalline Semiconductors // Physical Review Letters. -1999. -Vol.82,
№ 4. -P.779-782.

14. Kuznetsov V.L., Kuznetsova L.A., Kaliazin A.E. et al. Preparation
and thermoelectric properties of AII 8 B III 16B IV 30 clathrate
compounds // Journal of Applied Physics. -2000. -Vol.87(11). -P.
7871-7875.

15. Tang X., Li P., Deng S., Zhang Q. High temperature thermoelectric
transport properties of double-atom-filled clathrate compounds
YbxBa8-xGa16Ge30 // Journal of Applied Physics. -2008. -Vol.104(1). -P.
1-7. DOI 10.1063/1.2951888.

16. Abramchuk N.S., Carillo-Cabrera W. et al. Homo- and Heterovalent
Substitutions in the New Clathrates I Si30P16Te8-xSex and
Si30+xP16-xTe8-xBrx: Synthesis, Crystal Structure, and Thermoelectric
Proper\-ties // Inorganic Chemistry. -2012. -Vol.51(21). -P.11396-11405.
DOI 10.1021/ic3010097.

17. Hong L., Luo S.H., Yu C.H., Xie Y., Xia M.Y., Chen G.Y., Peng Q.
Functional Nanomaterials and Their Potential Applications in
Antibacterial Therapy // Pharm. Nanotechnol. -2019. -Vol.7. -P.129-146.
DOI 10.2174/2211738507666190320160802.~

18. Yao Y, Liao W, Yu R, Du Y, Zhang T, Peng Q. Potentials of combining
nanomaterials and stem cell therapy in myocardial repair //
Nanomedicine. - 2018. -Vol.13(13). - P.1623-38.
DOI 10.2217/nnm-2018-0013.

19. Friedman H.L. New methods for evaluating kinetic parameters from
thermal analysis data // J. Polym. Sci. Part B. -1969. -Vol.7. -Р.41-46.
\href{https://doi.org/10.1002/pol.1969.110070109}{DOI
10.1002/pol.1969.110070109}.

20. Serra R., Nomen R., Sempere J. The non-parametric kinetics a new
method for the kinetic study of thermoanalytical data // J. Therm. Anal.
Calorim. -1998. -Vol.52. -Р.933-943.
\href{https://doi.org/10.1023/A:1010120203389}{DOI\\
10.1023/A:1010120203389}.

21. Vlase T., Vlase G., Doca N., Bolcu C. Processing of non-isothermal
TG data. Comparative kinetic analysis with NPK method
//\href{https://link.springer.com/journal/10973}{Journal of Thermal
Analysis and Calorimetry}. -2005. -Vol.80. -Р.59-64.
\href{https://doi.org/10.1007/s10973-005-0613-x}{DOI
10.1007/s10973-005-0613-x}.

22. Shin S., Im S.I., Nho N.S., Lee K.B. Kinetic analysis using
thermogravimetric analysis for nonisother\-mal pyrolysis of vacuum residue
// \href{https://link.springer.com/journal/10973}{Journal of Thermal
Analysis and Calorimetry}. -2016. -126. - Р.933-941.
\href{https://doi.org/10.1007/s10973-016-5568-6}{DOI
10.1007/s10973-016-5568-6}.

23. Šesták J., Kratochvíl J. Rational approach to thermodynamic
rrocesses and constitutive equations in isothermal and non-isothermal
kinetics // Journal of Thermal Analysis. -1973. -Vol.5. - Р.193-201.
\href{https://doi.org/10.1007/BF01950368}{DOI 10.1007/BF01950368}.

24. Burkeev M.Zh., Fazylov S.D., Bakirova R., Iskineyeva A.
\href{https://www.sciencedirect.com/science/article/pii/S0959943621000237}{Thermal
decomposition of β-cyclodextrin and its inclusion complex with vitamin
E} //
\href{https://www.sciencedirect.com/science/journal/09599436}{Mendeleev
Communications}. -2021. - Vol.31(1). - Р.76-78.
\href{https://doi.org/10.1016/j.mencom.2021.01.023}{DOI
10.1016/j.mencom.2021.01.023}.

25. Šesták J. Errors of kinetic data obtained from thermogravimetric
curves at increasing temperature // Talanta. -1966. -Vol.13(4).
- Р.567-579. \href{https://doi.org/10.1016/0039-9140(66)80267-9}{DOI
10.1016/0039-9140(66)80267-9}.
\end{refs}

\begin{info}
\emph{{\bfseries Информация об авторах}}

Фазылов С.Д. - академик НАН РК, доктор химических наук, главный научный
сотрудник, Институт органического синтеза и углехимии РК, Караганда,
Казахстан, e-mail:
iosu8990@mail.ru

Сарсенбекова А.Ж. - кандидат химических наук, ассоциированный профессор,
Карагандинский университет имени Е.А.Букетова, Караганда, Казахстан,
e-mail:
chem\_akmaral@mail.ru;

Бакирова Р.Е. - доктор медицинских наук, профессор, Карагандинский
медицинский университет, Караганда, Казахстан, e-mail:
bakir15@mail.ru;

Власова Л.М. - кандидат химических наук, ассоциированный профессор,
Карагандинский медицинский университет, Караганда, Казахстан, e-mail:
vlasova@qmu.kz;

Жумагалиева Т.С. - кандидат химических наук, ассоциированный профессор,
Карагандинский университет имени Е.А.Букетова, Караганда, Казахстан,
e-mail: zhumagalieva79@mail.ru;

Нурмаганбетова М.Т. - кандидат химических наук, ассоциированный
профессор, Карагандинский университет имени Е.А.Букетова, Караганда,
Казахстан; e-mail:
ritunur@mail.ru;

Сыздыков А.К. - младший научный сотрудник, Институт органического
синтеза и углехимии РК, Караганда, Казахстан, e-mail:
ardak.syzdykov.96@inbox.ru;

Аширбекова Б.Ж. - младший научный сотрудник, Карагандинский медицинский
университет, Караганда, Казахстан, e-mail:
ashirbekova@qmu.kz;

\emph{{\bfseries Information about authors}}

Fazylov S.D. - аcademician NAS RK, Doctor of Chemical Sciences,
Professor, Institute of Organic Synthesis and Coal Chemistry, Karaganda,
Kazakhstan, е-mail:
iosu8990@mail.ru;

Sarsenbekova A.Zh. - candidate of chemical sciences, Associate
Professor, E.A.Buketov Karaganda University, Karaganda, Kazakhstan,
е-mail:
chem\_akmaral@mail.ru;

Bakirova R.E. -- doctor of medical sciences, Professor, Karaganda
Medical University, Karaganda, Kazakhstan, е-mail:\\
bakir15@mail.ru;

Vlasova L.M. - candidate of chemical sciences, Associate Professor,
Karaganda Medical University, Karaganda, Kazakhstan,
vlasova@qmu.kz;

Zhumagalyieva T.S. - candidate of chemical sciences, Associate
Professor, E.A.Buketov Karaganda University, Karaganda, Kazakhstan,
e-mail: zhumagalieva79@mail.ru;

Nurmaganbetova M.T. - candidate of chemical sciences, Associate
Professor, E.A.Buketov Karaganda University, Karaganda, Kazakhstan,
e-mail: ritunur@mail.ru;

Syzdykov A.K. - junior researcher, Institute of Organic Synthesis and
Coal Chemistry, Karaganda, Kazakhstan, е-mail:\\
ardak.syzdykov.96@inbox.ru;

Ashirbekova B.Zh. - master of Medicine, Karaganda Medical University,
Karaganda, e-mail:
ashirbekova@qmu.kz.
\end{info}
