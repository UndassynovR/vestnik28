\id{IRSTI 61.01.91}{}

\begin{header}
\swa{}{DETERMINATION OF THE TOXICITY OF AQUEOUS EXTRACT FROM WASTE PROCESSING OF PHOSPHATE RAW MATERIALS USING BIOTESTING ON HYDROBIONTS}

{\bfseries
\tsp{1}N.V. Soraya,
\tsp{2}V.V. Litvinov,
\tsp{1}G.K. Daumova\envelope,
\tsp{3,4}M.A. Yelubay,
\tsp{5}E.A. Kulmagambetova,
\tsp{6}M. Woszczyk
}
\end{header}

\begin{affil}
\tsp{1}D.Serikbayev East Kazakhstan Technical University, Ust-Kamenogorsk, Kazakhstan,

\tsp{2}Proektno-ekologicheskoe bjuro LLP, Ust-Kamenogorsk, Kazakhstan,

\tsp{3}Toraighyrov University, Pavlodar, Kazakhstan,

\tsp{4}Humboldt-Innovation GmbH, Berlin, Germany

\tsp{5}RSE at the National Research Institute for Occupational Safety of the Ministry of Labor and Social Protection of the Population of the Republic of Kazakhstan, Astana, Kazakhstan,

\tsp{6}Adam Mickiewicz University, Poznań, Poland

\envelope Correspondent-author: gulzhan.daumova@mail.ru
\end{affil}

The biotesting method was used to assess the individual and combined
effects of waste from the production of phosphate raw materials at the
designed chemical complex of the EuroChem-Karatau company (Republic of
Kazakhstan). The goal of the study was to determine the hazard class of
waste from the production of phosphate raw materials using biotesting
method. Experimental study was carried out on test organisms: Daphnia
magna Straus freshwater planktonic crustaceans. It has been established
that Daphnia magna is sensitive to waste from the processing of
phosphate raw materials, and the samples under study belong to
practically non-hazardous and low-hazard waste with the possibility of
processing for the purpose of their further use. On hydrobionts, water
extracts without dilution from a mixture of waste module CCP + cake
module 1A (1:1), cake module 1A and synthetic gypsum did not have an
inhibitory effect. An ecological and toxicological study confirmed that
the studied samples belong to the fifth class of environmental hazard
(virtually non-hazardous). For the aqueous extract from the waste of the
CCP module without its dilution, inhibition of the viability of aquatic
organisms was revealed. The death of Daphnia may be associated with
waste components that cause blockage of the respiratory tract with
dispersed particles. It was revealed that with increasing dilution, the
mortality rate of Daphnia decreases; the 10-times dilution ratio had no
effect on hydrobionts.

{\bfseries Keywords} bio-testing, toxicity, hazardous industrial waste,
hazard class, Daphnia magna Straus.

\begin{header}
{\bfseries ОПРЕДЕЛЕНИЕ ТОКСИЧНОСТИ ВОДНОЙ ВЫТЯЖКИ ИЗ ОТХОДОВ ПЕРЕРАБОТКИ ФОСФАТНОГО СЫРЬЯ МЕТОДОМ БИОТЕСТИРОВАНИЯ НА ГИДРОБИОНТАХ}

{\bfseries
\tsp{1}Н.В. Серая,
\tsp{2}В.В. Литвинов,
\tsp{1}Г.К. Даумова\envelope,
\tsp{3,4}М.А. Елубай,
\tsp{5}Э.А. Кульмагамбетова,
\tsp{6}M. Woszczyk
}
\end{header}

\begin{affil}
\tsp{1}Восточно-Казахстанский технический университет им. Д.Серикбаева, Усть-Каменогорск, Казахстан

\tsp{2}ТОО «Проектно-экологическое бюро», Усть-Каменогорск, Казахстан,

\tsp{3}Tорайгыров университет, Павлодар, Казахстан,

\tsp{4}Humboldt-Innovation GmbH, Берлин, Германия,

\tsp{5}Республиканский научно-исследовательский институт по охране труда Министерства труда и социальной защиты населения Республики Казахстан, Астана, Казахстан,

\tsp{6}Университет имени Адама Мицкевича, Познань, Польша,

e-mail: gulzhan.daumova@mail.ru
\end{affil}

Метод биотестирования впервые использован при оценке отдельного и
комбинированного воздействия отходов производства фосфатного сырья
проектируемого химического комплекса компании ЕвроХим-Каратау
(Республика Казахстан). Цель работы состояла в определении класса
опасности отходов производства фосфатного сырья с помощью методов
биотестирования. Экспериментальную работу проводили на тест-организмах:
пресноводных планктонных рачках~Daphnia magna Straus. Установлено, что
Daphnia magna чувствительны к отходам переработки фосфатного сырья, а
исследуемые образцы относятся к практически неопасным и малоопасным
отходам с возможностью переработки с целью их дальнейшего применения. На
гидробионты водные вытяжки без ее разведения из смеси отхода модуля ССР
+ кек модуля 1А (1:1), кека модуля 1А и гипса синтетического не
оказывают угнетающего эффекта. Эколого-токсикологическое исследование
подтвердило, что исследованные образцы относятся к пятому классу
опасности для окружающей среды (практически неопасные). Для водной
вытяжки из отхода модуля ССР без ее разведения выявлено угнетение
жизнеспособности гидробионтов. Гибель дафний, возможно, связана с
компонентами отхода, которые вызывают закупорки дыхательных путей
дисперсными частицами. Выявлено, что при увеличении разведения
смертность дафний понижается, установлена кратность разведения водной
вытяжки (в 10 раз), при которой не выявлено воздействие на гидробионты.

{\bfseries Ключевые слова}: биотестирование, токсичность, опасные
промышленные отходы, класс опасности, Daphnia magna Straus.

\begin{header}
{\bfseries ГИДРОБИОНТТАРДА БИОТЕСТІЛЕУ ӘДІСІМЕН ФОСФАТ ШИКІЗАТЫН ҚАЙТА ӨҢДЕУ ҚАЛДЫҚТАРЫНАН СУ СЫҒЫНДЫСЫНЫҢ УЫТТЫЛЫҒЫН АНЫҚТАУ}

{\bfseries
\tsp{1}Н.В. Серая,
\tsp{2}В.В. Литвинов,
\tsp{1}Г.К. Даумова\envelope,
\tsp{3,4}М.А. Елубай,
\tsp{5}Э.А. Кульмагамбетова,
\tsp{6}M. Woszczyk
}
\end{header}

\begin{affil}
\tsp{1}Д. Серікбаев атындағы Шығыс Қазақстан техникалық университеті, Өскемен, Қазақстан

\tsp{2}"Проектно-экологическое бюро" ЖШС, Өскемен, Қазақстан,

\tsp{3}Tорайғыров университеті, Павлодар, Қазақстан,

\tsp{4}Humboldt-Innovation GmbH, Берлин, Германия,

\tsp{5}Қазақстан Республикасы Еңбек және халықты әлеуметтік қорғау министрлігінің Еңбекті қорғау жөніндегі республикалық ғылыми-зерттеу институты ШЖҚ РМК, Астана, Қазақстан,

\tsp{6}Адам Мицкевич университеті, Познань, Польша,

e-mail: gulzhan.daumova@mail.ru
\end{affil}

Биотестілеу әдісі алғаш рет ЕвроХим-Қаратау компаниясының (Қазақстан
Республикасы) жобаланатын химиялық кешенінің фосфат шикізатын өндіру
қалдықтарының жеке және аралас әсерін бағалау кезінде пайдаланылды.
Жұмыстың мақсаты биотестілеу әдістерін қолдана отырып, фосфат шикізатын
өндіру қалдықтарының қауіптілік класын анықтау болды. Эксперименттік
жұмыс тест-ағзаларда жүргізілді: тұщы су планктонды Daphnia magna Straus
шаян тәрізділер. Daphnia magna фосфат шикізатын қайта өңдеу қалдықтарына
сезімтал екендігі анықталды, ал зерттелетін үлгілер оларды әрі қарай
қолдану мақсатында қайта өңдеу мүмкіндігімен іс жүзінде қауіпсіз және
қауіптілігі төмен қалдықтарға жатады. Гидробионттарға ССР+1А (1:1) кек
модулінің, 1А кек модулінің қалдықтарынан және синтетикалық гипстің
қоспасынан сұйылтылмаған су сығындылары әсер етпейді.
Экологиялық-токсикологиялық зерттеу зерттелген үлгілердің қоршаған
ортаға қауіптіліктің бесінші класына жататынын растады (іс жүзінде
қауіпті емес). ССР модулінің қалдықтарынан оны өсірусіз су сығындлары
үшін гидробионттардың өміршеңдігінің тежелуі анықталды. Дафнияның өлімі
тыныс алу жолдарының дисперсті бөлшектермен бітелуіне әкелетін
қалдықтардың құрамдас бөліктерімен байланысты болуы мүмкін. Өсіруді
көбейткен кезде дафнияның өлімі төмендейтіні, гидробионттарға әсері
болмайтын су сығындысының сұйылту жиілігі (10 есе) анықталды.

{\bfseries Түйін сөздер:} биотестілеу, уыттылық, қауіпті өндірістік
қалдықтар, қауіптілік класы, Daphnia magna Straus.

\begin{multicols}{2}
{\bfseries Introduction.} In world practice, biotesting methods are widely
used to assess the toxicity of water pollution {[}1{]}. The use of these
methods has a number of advantages over physicochemical analysis, which
often does not allow the detection of unstable compounds or the
quantification of ultra-low concentrations of toxic substances, as well
as taking into account their combined effects. Biotesting allows to
quickly obtain an integral assessment of toxicity at a certain point in
time {[}2{]}.

Traditional, frequently used test objects for monitoring the toxicity of
various objects are Cladocera - small planktonic crustaceans, one of the
most numerous and diverse orders of the Branchiopoda class. Well-known
representatives of the order, Daphnia freshwater planktonic crustaceans
(Daphnia magna Straus), are among the most sensitive test organisms and
are considered the basic object of biotesting {[}3, 4{]}.

For example, planktonic crustaceans (Daphnia magna Straus) were used as
a test organism for biotesting of quarry waters in {[}5{]}. When
studying quarry waters using D. magna test organisms, the number of
deaths in the tested Daphnia samples did not exceed 10\%, which
indicated the absence of a toxic effect. The tested culture of
Ch.Vulgaris had a higher sensitivity, however, when the tested samples
were diluted 3 times, the toxicity criterion was not exceeded. The
calculated values of the dilution coefficient with toxic substances made
it possible to rank the water bodies under consideration in increasing
order of the possible toxic impact of their waters: Inkerman Quarry
Lake, Gasfortskoye Quarry Lake and Kadykovskoye Quarry Lake. However,
already with a 3-fold dilution, the tested samples from all reservoirs
became completely harmless.

Another work, {[}6{]}, developed methodological approaches to rapid
assessment of the environment, as well as the toxicity of its components
in the process of biological monitoring at hazardous industrial
facilities. The following biotest objects were used: daphnia (Daphnia
magna Straus), ceriodaphnia (Ceriodaphnia affinis Lilljeborg), ciliates
(Paramecium caudatum), a preparation of lyophilized luminescent bacteria
"Ecolum" based on a recombinant Escherichia strain coli M-17, algae
(Chlorella vulgaris Beijer, Scenedesmus quadri cauda (Turp.) Breb.),
mammalian germ cells; environmental objects: natural water, bottom
sediments, soil samples taken in the sanitary protection zone (SPZ) of
the Balakovo region and a nuclear power plant in the Saratov region. The
toxicity of samples of bottom sediments, soil, as well as water samples
from the cooling pond and the adjacent water area of the Saratov
reservoir was revealed.

Bioanalysis has also been successfully used to determine the hazard of
pollution of environmental objects with oil and waste from the oil
refining industry containing petroleum hydrocarbons, represented by
persistent organic pollutants (POPs) and polycyclic aromatic
hydrocarbons (PAHs) {[}7{]}. The ecotoxicological study used the test
culture Daphnia magna Straus - a sensitive test organism that reacts to
the presence of petroleum products of various fractions and mass
concentrations in water. When conducting biotesting, the work took into
account specific methodological nuances: the death of daphnia in the
aqueous extract from waste, established within 48 hours, did not exceed
10\%, but the morphological state of the tested object indicated a
significant negative impact: daphnia individuals are smaller than in the
control test, their trophic activity is minimal, and only longer
exposure (96 hours) led to the death of the test organism. It has been
established that biotesting, in this case, remains a mandatory method
for determining total toxicity.

The authors of {[}8{]} carried out an environmental and toxicological
assessment of the waste generated as a result of protein production by
growing Hermetia illucens fly larvae and determined the hazard class of
zoocompost using biotesting methods. The hazard class was determined by
the dilution coefficient of the aqueous extract, at which no harmful
effect on the test objects was detected. To study the toxicity of
zoocompost, the species of lower crustacean Daphnia magna Straus were
used. As a result of the experiments, the dilution ratio of zoocompost
extracts was established (equal to 10), at which the death of daphnia
was not observed, therefore, the waste under study was classified as
hazard class 4.

An ecotoxicological assessment of six drill cuttings (waste from the oil
industry) using certified methods of contact and eluate biotesting using
test organisms of different trophic levels was carried out in {[}9{]}.
Biotesting carried out on Daphnia magna Straus showed that aqueous
extracts from all samples of drill cuttings did not have an inhibitory
effect on these organisms.

The authors of {[}10{]} studied the soil of a quarry in the Neva
Lowland, reclaimed using loamy soil and municipal solid waste. When
conducting biotesting of aqueous extracts of soil samples, the presence
of acute toxicity was established at one of the studied sites. The
presence of chronic toxicity in water extracts was identified at all
sites.

In {[}11{]}, the soils of agricultural lands and nearby reservoirs of
the Prokhorovsky district of the Belgorod region were studied.
Watercress Lepidium sativum and crustacean Daphnia magna Straus were
used as a bioindicator. According to the results of biotesting, the
water in reservoirs adjacent to agricultural land corresponds to hazard
class II, the ``slightly polluted'' category. It was revealed that the
soils of agricultural lands of the holding have initial manifestations
of toxic effects.

The toxicity of ash and slag waste obtained after burning coal at power
plants was determined experimentally using test organisms {[}4{]}. When
analyzing the toxicity of the ash and slag mixture after burning the
culm of the Donetsk deposit (Russia), an acute toxic effect of aqueous
extract from waste on the test organisms Daphnia magna Straus and the
weed Scenedesmus guadricauda was revealed. A non-hazardous dilution
factor for the water extract from waste was revealed, causing the death
of no more than 10\% of daphnia in 96 hours and a deviation in the
number of weeds by no more than 20\% in 72 hours compared to the control
value. Ash and slag waste was classified as waste of III-IV
environmental hazard class.

The influence of fluoropolymer production wastewater on test objects of
different systematic groups was studied in {[}12{]}. Toxicity assessment
was carried out by tetrazole-topographic, anaphase-metaphase analysis
methods and micronuclear test of micropreparations. It has been
established that the germination of white mustard seeds (Sinapis alba
L.), the survival of Daphnia Magna Straus and cyanobacteria (Nostoc
paludosum Kutz) are reduced under the influence of fluoropolymer
production wastewater with a decrease in the dilution ratio. In Scots
pine (Pinus sylvestris L.), mitotic activity decreases, and the
proportion of cytogenetic disorders increases with a decrease in the
degree of dilution of the test solution. The negative impact of
fluoropolymer production wastewater on biota and the need for disposal,
excluding discharge into the environment, have been determined.

Thus, biotesting methods make it possible to assess the biological
usefulness of the studied water bodies, their suitability for the life
of aquatic organisms that ensure self-purification processes in the
reservoir and biological oxidation during wastewater treatment {[}13{]}.
The results of biotesting for toxicity make it possible to quickly
determine the dangerous effects of chemical pollution on the life of
aquatic organisms, not by individual components, but by their mixtures,
often of unknown nature and not detected by other methods of analyzing
toxic substances.

Toxic effects recorded by biotesting methods include complex,
synergistic, antagonistic and additional effects of all chemical,
physiological and biological components present in the test water,
adversely affecting the physiological, biochemical and genetic functions
of the test organisms. As a result of the laboratory biotesting
procedure using classical methods, the acute or chronic toxicity of the
test water is established in experiments of various durations. Acute
toxicity is expressed when the intensity of the acting agent is so great
that the body's compensatory and adaptive reactions do not have time to
manifest themselves, and it dies. Chronic toxicity is determined by
intense but longer exposure to toxicants; in this case, an imbalance
occurs between the decay and synthesis of substances in the body of
aquatic organisms, the destruction of the genome and the cessation of
reproduction. Based on the data from the experiments, an acutely lethal
concentration of the substance (or the dilution factor of the test
water) is established, at which the death of organisms does not exceed
that in the control.

{\bfseries Materials and methods.} Determination of the toxicity of aqueous
extract by biotesting on hydrobionts was carried out for samples of
waste from the processing of phosphate raw materials in the production
of simple superphosphate (SSP), obtained by processing by hydrochloric
acid decomposition of phosphorites at a semi-industrial Technophos
installation (Technophos, Prayon Technophos EAD, Devnya, Bulgaria) for
the projected production plant of mineral fertilizers at
EuroChem-Karatau (Kazakhstan).

EuroChem-Karatau chemical complex for processing phosphates is designed
for processing phosphate rock to produce dicalcium phosphate (calcium
hydrogen phosphate dihydride). The production of dicalcium phosphate
consists of the following main technological stages:

1) Processing of phosphorite raw materials by a solution of hydrochloric
acid with the transfer of the valuable component into a soluble state in
the form of a product solution and the production of a solid residue,
which is a technology waste (Module 1A);

2) Treatment of the product solution with ground limestone with
precipitation of dicalcium phosphate and drying of the resulting product
DCP (dicalcium phosphate) and obtaining a solution of calcium chloride
(Module 1B);

3) Treatment of the calcium chloride solution with slaked lime with the
precipitation of magnesium, aluminum and iron impurities into the
hydrate cake and obtaining a purified calcium chloride solution (CCP
module), which is removed from the process for further processing;

4) Treatment of part of the calcium chloride solution with sulfuric acid
to obtain a regenerated hydrochloric acid solution and gypsum
precipitate (Module 4).

In this study, the toxicity assessment was carried out based on the
potential danger to aquatic organisms of the following waste from the
processing of phosphate raw materials:

- Sample No.1 -- cake of Module 1A;

- Sample No.2 - synthetic gypsum;

- Sample No.3 - waste of the CCP module - magnesium hydroxide;

- Sample No.4 - cake of the Module 1A and waste of module CCP (mixture
1:1) (Figure 1).
\end{multicols}

\fig{c/image2}{}

{\bfseries Synthetic gypsum}

{\bfseries Cake of Module 1A}

{\bfseries Module ССР}

\begin{quote}
{\bfseries Fig.1 -- Initial samples of the phosphate raw material
processing waste}
\end{quote}

When conducting these studies, an aquatic laboratory culture of Daphnia
magna of 1-day age was used as a test object (Figure 2).

\fig{c/image3}{}

{\bfseries Fig.2 - Daphnia magna}

The studies were carried out in accordance with ST RK 17.1.4.01-95
{[}14{]} standard and the Methodological Guide for Biotesting of Water
{[}15{]}. We used a method for determining the toxicity of water and
aqueous extracts from soils, sewage sludge, waste by mortality and
changes in the fertility of daphnia.

The survival rate of test objects is the average number of daphnia that
survived in the tested extracts and in the control over a certain time.
The criterion for acute toxicity of the test sample
(DL\tsb{50}) is the death of 50 percent or more of the daphnia
in the analyzed water sample compared to the control during the
biotesting period (48 hours). If the result is more than 50\% death of
individuals, the test sample is considered toxic. The calculation of
dead daphnia in the test water sample in comparison with the control is
calculated using the formula {[}13{]}:

\(A = \ \frac{(X_{c} - X_{t})}{X_{c}} \bullet 100,\ \%\) (1)

where \(X_{c}\)is the arithmetic mean number of daphnia that survived in
the control sample;

\(X_{t}\)-- arithmetic mean number of daphnia surviving in the tested
water.

{\bfseries Experimental.} To obtain ready-made samples during the study,
before starting the tests, the initial samples were air-dried, then
crushed and sifted through a 1-mm sieve.

At the next stage, the moisture content of all studied samples was
determined. The moisture content of the samples was: CCP module -- 12\%,
cake of module 1A -- 2\%, synthetic gypsum -- 3\%, CCP module waste +
cake of module 1A mixture (1:1) -- 7\%. A certain moisture
characteristic was used to calculate the mass of an air-dry sample
intended for preparing an aqueous extract.

Water-soluble compounds were extracted from the samples using
dechlorinated tap water taken in a solid-to-liquid ratio of 1:10 (by
weight). Aqueous extracts from phosphate raw material processing waste
were prepared in the following ratio:

- 88 g of air-dried sample (waste of the CCP module) and 880
cm\tsp{3} of cultivation water;

- 98 g of air-dried sample (module 1A cake) and 980
cm\tsp{3} of cultivation water;

- 97 g of air-dried sample (synthetic gypsum) and 970
cm\tsp{3} of cultivation water;

- 93 g of air-dried sample (mixture (1:1) of CCP module waste + cake of
module 1A) and 930 cm\tsp{3} of cultivation water.

Stirring of the prepared suspensions was carried out for 7 hours. Then
the suspensions settled for 15-20 minutes.

The resulting aqueous extracts from samples of phosphate raw material
processing waste were filtered through a ``white ribbon'' filter on a
Buchner funnel. A weak vacuum was used for filtration using an electric
pump. The resulting leach extract was tested for toxicity.

The biotesting procedure (determination of acute toxicity) was carried
out for the test samples under the conditions specified in Table 1.

{\bfseries Table 1 - Conditions for the biotesting procedure}

%% \begin{longtable}[]{@{}
%%   >{\centering\arraybackslash}p{(\linewidth - 8\tabcolsep) * \real{0.1274}}
%%   >{\raggedright\arraybackslash}p{(\linewidth - 8\tabcolsep) * \real{0.2718}}
%%   >{\centering\arraybackslash}p{(\linewidth - 8\tabcolsep) * \real{0.1919}}
%%   >{\centering\arraybackslash}p{(\linewidth - 8\tabcolsep) * \real{0.2241}}
%%   >{\centering\arraybackslash}p{(\linewidth - 8\tabcolsep) * \real{0.1848}}@{}}
%% \toprule\noalign{}
%% \begin{minipage}[b]{\linewidth}\centering
%% Sample No.
%% \end{minipage} & \begin{minipage}[b]{\linewidth}\centering
%% Date: 22.02.2023
%% \end{minipage} & \begin{minipage}[b]{\linewidth}\centering
%% Temperature, °C
%% \end{minipage} & \begin{minipage}[b]{\linewidth}\centering
%% Dissolved
%% 
%% oxygen, mg/dm\tsp{3}
%% \end{minipage} & \begin{minipage}[b]{\linewidth}\centering
%% pH
%% \end{minipage} \\
%% \midrule\noalign{}
%% \endhead
%% \bottomrule\noalign{}
%% \endlastfoot
%% 1 & Control & 24 & 7.5 & 8.21 \\
%% 2 & Waste of the CCP module & 24 & 6.3 & 8.28 \\
%% 3 & Waste of the CCP module + module 1A cake & 24 & 6.1 & 8.19 \\
%% 4 & Cake of module 1A & 24 & 6.5 & 7.16 \\
%% 5 & Synthetic gypsum & 24 & 6.9 & 7.55 \\
%% \end{longtable}

Using a 2-cm\tsp{3} glass pipette with a cut off and melted
end, 10 specimens of daphnia at the age of 24 hours were transplanted
into a glass and transferred into beakers with analyzed and cultivation
water using a plankton net or a Pasteur pipette, the liquid was sucked
out of the glass and the measured volume of test water was added
carefully not to damage the daphnia.

For testing, 100 cm\tsp{3} of analyzed and cultivation water
were taken into chemical cups with a capacity of 250
cm\tsp{3}. Each sample of the test water was analyzed in
triplicate. Repetition for control samples is also threefold.10
one-day-old daphnia were placed in each glass and exposed at a
temperature of 22±2 °C for 48 hours. Before the start of biotesting,
daphnia were fed.

During the entire biotesting period, the crustaceans were not fed.
Surviving daphnia were counted after 1, 6, 12, 24, and 48 hours.
Crustaceans were considered survivors if they moved freely in the water
or surfaced from the bottom no later than 15 s after a slight rocking of
the glass.

Results of testing are presented in Table 2. The biotesting results were
considered correct if the death of daphnia in the control did not exceed
10\% for the entire observation period and the oxygen concentration in
the test water at the end of the experiment was at least 2
mg/dm\tsp{3} (Table 3).

{\bfseries Table 2 - Results of calculating the survival rate of daphnia}

%% \begin{longtable}[]{@{}
%%   >{\raggedright\arraybackslash}p{(\linewidth - 30\tabcolsep) * \real{0.1836}}
%%   >{\centering\arraybackslash}p{(\linewidth - 30\tabcolsep) * \real{0.0453}}
%%   >{\centering\arraybackslash}p{(\linewidth - 30\tabcolsep) * \real{0.0460}}
%%   >{\centering\arraybackslash}p{(\linewidth - 30\tabcolsep) * \real{0.0460}}
%%   >{\centering\arraybackslash}p{(\linewidth - 30\tabcolsep) * \real{0.0461}}
%%   >{\centering\arraybackslash}p{(\linewidth - 30\tabcolsep) * \real{0.0614}}
%%   >{\centering\arraybackslash}p{(\linewidth - 30\tabcolsep) * \real{0.0460}}
%%   >{\centering\arraybackslash}p{(\linewidth - 30\tabcolsep) * \real{0.0614}}
%%   >{\centering\arraybackslash}p{(\linewidth - 30\tabcolsep) * \real{0.0614}}
%%   >{\centering\arraybackslash}p{(\linewidth - 30\tabcolsep) * \real{0.0614}}
%%   >{\centering\arraybackslash}p{(\linewidth - 30\tabcolsep) * \real{0.0614}}
%%   >{\centering\arraybackslash}p{(\linewidth - 30\tabcolsep) * \real{0.0460}}
%%   >{\centering\arraybackslash}p{(\linewidth - 30\tabcolsep) * \real{0.0618}}
%%   >{\centering\arraybackslash}p{(\linewidth - 30\tabcolsep) * \real{0.0500}}
%%   >{\centering\arraybackslash}p{(\linewidth - 30\tabcolsep) * \real{0.0540}}
%%   >{\centering\arraybackslash}p{(\linewidth - 30\tabcolsep) * \real{0.0680}}@{}}
%% \toprule\noalign{}
%% \multirow{4}{=}{\begin{minipage}[b]{\linewidth}\centering
%% Duration of biotesting
%% \end{minipage}} &
%% \multicolumn{15}{>{\centering\arraybackslash}p{(\linewidth - 30\tabcolsep) * \real{0.8164} + 28\tabcolsep}@{}}{%
%% \begin{minipage}[b]{\linewidth}\centering
%% Number of surviving daphnia, specimens
%% \end{minipage}} \\
%% &
%% \multicolumn{3}{>{\centering\arraybackslash}p{(\linewidth - 30\tabcolsep) * \real{0.1373} + 4\tabcolsep}}{%
%% \begin{minipage}[b]{\linewidth}\centering
%% Control
%% \end{minipage}} &
%% \multicolumn{3}{>{\centering\arraybackslash}p{(\linewidth - 30\tabcolsep) * \real{0.1535} + 4\tabcolsep}}{%
%% \begin{minipage}[b]{\linewidth}\centering
%% Waste of the CCP module
%% \end{minipage}} &
%% \multicolumn{3}{>{\centering\arraybackslash}p{(\linewidth - 30\tabcolsep) * \real{0.1842} + 4\tabcolsep}}{%
%% \begin{minipage}[b]{\linewidth}\centering
%% CCP module waste +
%% 
%% cake of module 1A
%% \end{minipage}} &
%% \multicolumn{3}{>{\centering\arraybackslash}p{(\linewidth - 30\tabcolsep) * \real{0.1692} + 4\tabcolsep}}{%
%% \begin{minipage}[b]{\linewidth}\centering
%% Cake of module 1A
%% \end{minipage}} &
%% \multicolumn{3}{>{\centering\arraybackslash}p{(\linewidth - 30\tabcolsep) * \real{0.1721} + 4\tabcolsep}@{}}{%
%% \begin{minipage}[b]{\linewidth}\centering
%% Synthetic gypsum
%% \end{minipage}} \\
%% &
%% \multicolumn{3}{>{\centering\arraybackslash}p{(\linewidth - 30\tabcolsep) * \real{0.1373} + 4\tabcolsep}}{%
%% \begin{minipage}[b]{\linewidth}\centering
%% No. of experiment
%% \end{minipage}} &
%% \multicolumn{3}{>{\centering\arraybackslash}p{(\linewidth - 30\tabcolsep) * \real{0.1535} + 4\tabcolsep}}{%
%% \begin{minipage}[b]{\linewidth}\centering
%% No. of experiment
%% \end{minipage}} &
%% \multicolumn{3}{>{\centering\arraybackslash}p{(\linewidth - 30\tabcolsep) * \real{0.1842} + 4\tabcolsep}}{%
%% \begin{minipage}[b]{\linewidth}\centering
%% No. of experiment
%% \end{minipage}} &
%% \multicolumn{3}{>{\centering\arraybackslash}p{(\linewidth - 30\tabcolsep) * \real{0.1692} + 4\tabcolsep}}{%
%% \begin{minipage}[b]{\linewidth}\centering
%% No. of experiment
%% \end{minipage}} &
%% \multicolumn{3}{>{\centering\arraybackslash}p{(\linewidth - 30\tabcolsep) * \real{0.1721} + 4\tabcolsep}@{}}{%
%% \begin{minipage}[b]{\linewidth}\centering
%% No. of experiment
%% \end{minipage}} \\
%% & \begin{minipage}[b]{\linewidth}\centering
%% 1
%% \end{minipage} & \begin{minipage}[b]{\linewidth}\centering
%% 2
%% \end{minipage} & \begin{minipage}[b]{\linewidth}\centering
%% 3
%% \end{minipage} & \begin{minipage}[b]{\linewidth}\centering
%% 1
%% \end{minipage} & \begin{minipage}[b]{\linewidth}\centering
%% 2
%% \end{minipage} & \begin{minipage}[b]{\linewidth}\centering
%% 3
%% \end{minipage} & \begin{minipage}[b]{\linewidth}\centering
%% 1
%% \end{minipage} & \begin{minipage}[b]{\linewidth}\centering
%% 2
%% \end{minipage} & \begin{minipage}[b]{\linewidth}\centering
%% 3
%% \end{minipage} & \begin{minipage}[b]{\linewidth}\centering
%% 1
%% \end{minipage} & \begin{minipage}[b]{\linewidth}\centering
%% 2
%% \end{minipage} & \begin{minipage}[b]{\linewidth}\centering
%% 3
%% \end{minipage} & \begin{minipage}[b]{\linewidth}\centering
%% 1
%% \end{minipage} & \begin{minipage}[b]{\linewidth}\centering
%% 2
%% \end{minipage} & \begin{minipage}[b]{\linewidth}\centering
%% 3
%% \end{minipage} \\
%% \midrule\noalign{}
%% \endhead
%% \bottomrule\noalign{}
%% \endlastfoot
%% 22.02.2023 & 10 & 10 & 10 & 10 & 10 & 10 & 10 & 10 & 10 & 10 & 10 & 10 &
%% 10 & 10 & 10 \\
%% 23.02.2023 (24 hours) & 10 & 10 & 10 & 2 & 1 & 2 & 10 & 7 & 7 & 10 & 10
%% & 10 & 6 & 6 & 8 \\
%% 24.02.2023 (48 hours) & 10 & 10 & 10 & 2 & 1 & 2 & 10 & 6 & 5 & 10 & 10
%% & 10 & 6 & 6 & 8 \\
%% Average number of surviving daphnia, specimens &
%% \multicolumn{3}{>{\centering\arraybackslash}p{(\linewidth - 30\tabcolsep) * \real{0.1373} + 4\tabcolsep}}{%
%% 10} &
%% \multicolumn{3}{>{\centering\arraybackslash}p{(\linewidth - 30\tabcolsep) * \real{0.1535} + 4\tabcolsep}}{%
%% 1.7} &
%% \multicolumn{3}{>{\centering\arraybackslash}p{(\linewidth - 30\tabcolsep) * \real{0.1842} + 4\tabcolsep}}{%
%% 7} &
%% \multicolumn{3}{>{\centering\arraybackslash}p{(\linewidth - 30\tabcolsep) * \real{0.1692} + 4\tabcolsep}}{%
%% 10} &
%% \multicolumn{3}{>{\centering\arraybackslash}p{(\linewidth - 30\tabcolsep) * \real{0.1721} + 4\tabcolsep}@{}}{%
%% 6.7} \\
%% A, \% &
%% \multicolumn{3}{>{\centering\arraybackslash}p{(\linewidth - 30\tabcolsep) * \real{0.1373} + 4\tabcolsep}}{%
%% 0} &
%% \multicolumn{3}{>{\centering\arraybackslash}p{(\linewidth - 30\tabcolsep) * \real{0.1535} + 4\tabcolsep}}{%
%% 83} &
%% \multicolumn{3}{>{\centering\arraybackslash}p{(\linewidth - 30\tabcolsep) * \real{0.1842} + 4\tabcolsep}}{%
%% 30} &
%% \multicolumn{3}{>{\centering\arraybackslash}p{(\linewidth - 30\tabcolsep) * \real{0.1692} + 4\tabcolsep}}{%
%% 0} &
%% \multicolumn{3}{>{\centering\arraybackslash}p{(\linewidth - 30\tabcolsep) * \real{0.1721} + 4\tabcolsep}@{}}{%
%% 33} \\
%% \end{longtable}

{\bfseries Table 3 - Oxygen concentration in the tested samples at the end
of the experiment}

%% \begin{longtable}[]{@{}
%%   >{\centering\arraybackslash}p{(\linewidth - 4\tabcolsep) * \real{0.1406}}
%%   >{\raggedright\arraybackslash}p{(\linewidth - 4\tabcolsep) * \real{0.4784}}
%%   >{\centering\arraybackslash}p{(\linewidth - 4\tabcolsep) * \real{0.3810}}@{}}
%% \toprule\noalign{}
%% \begin{minipage}[b]{\linewidth}\centering
%% Sample No.
%% \end{minipage} & \begin{minipage}[b]{\linewidth}\centering
%% After 96 hours
%% \end{minipage} & \begin{minipage}[b]{\linewidth}\centering
%% Dissolved oxygen, mg/dm\tsp{3}
%% \end{minipage} \\
%% \midrule\noalign{}
%% \endhead
%% \bottomrule\noalign{}
%% \endlastfoot
%% 1 & Control & 2.6 \\
%% 2 & Waste of the CCP module & 2.4 \\
%% 3 & CCP module waste + cake of module 1A & 2.0 \\
%% 4 & Cake of module 1A & 2.4 \\
%% 5 & Synthetic gypsum & 2.5 \\
%% \end{longtable}

For the waste of the CCP module, a repeated biotesting procedure was
carried out (dilution factor 10 times). The biotesting procedure
(determination of acute toxicity) was carried out for the test sample
under the conditions specified in Table 4. The results of the
calculation after 48 hours are presented in Table 5.

{\bfseries Table 4 - Conditions for the biotesting procedure}

%% \begin{longtable}[]{@{}
%%   >{\raggedright\arraybackslash}p{(\linewidth - 6\tabcolsep) * \real{0.3912}}
%%   >{\centering\arraybackslash}p{(\linewidth - 6\tabcolsep) * \real{0.2146}}
%%   >{\centering\arraybackslash}p{(\linewidth - 6\tabcolsep) * \real{0.1927}}
%%   >{\centering\arraybackslash}p{(\linewidth - 6\tabcolsep) * \real{0.2014}}@{}}
%% \toprule\noalign{}
%% \begin{minipage}[b]{\linewidth}\centering
%% Date: 28.02.2023
%% \end{minipage} & \begin{minipage}[b]{\linewidth}\centering
%% Temperature, °C
%% \end{minipage} & \begin{minipage}[b]{\linewidth}\centering
%% Dissolved
%% 
%% oxygen, mg/dm\tsp{3}
%% \end{minipage} & \begin{minipage}[b]{\linewidth}\centering
%% pH
%% \end{minipage} \\
%% \midrule\noalign{}
%% \endhead
%% \bottomrule\noalign{}
%% \endlastfoot
%% Control & 24 & 7.5 & 8.15 \\
%% Waste of the CCP module & 24 & 6.5 & 8.20 \\
%% \end{longtable}

{\bfseries Table 5 - Results of calculating the survival rate of daphnia}

%% \begin{longtable}[]{@{}
%%   >{\raggedright\arraybackslash}p{(\linewidth - 12\tabcolsep) * \real{0.4882}}
%%   >{\centering\arraybackslash}p{(\linewidth - 12\tabcolsep) * \real{0.0765}}
%%   >{\centering\arraybackslash}p{(\linewidth - 12\tabcolsep) * \real{0.0925}}
%%   >{\centering\arraybackslash}p{(\linewidth - 12\tabcolsep) * \real{0.0770}}
%%   >{\centering\arraybackslash}p{(\linewidth - 12\tabcolsep) * \real{0.1084}}
%%   >{\centering\arraybackslash}p{(\linewidth - 12\tabcolsep) * \real{0.1084}}
%%   >{\centering\arraybackslash}p{(\linewidth - 12\tabcolsep) * \real{0.0488}}@{}}
%% \toprule\noalign{}
%% \multirow{4}{=}{\begin{minipage}[b]{\linewidth}\centering
%% Duration of biotesting
%% \end{minipage}} &
%% \multicolumn{6}{>{\centering\arraybackslash}p{(\linewidth - 12\tabcolsep) * \real{0.5118} + 10\tabcolsep}@{}}{%
%% \begin{minipage}[b]{\linewidth}\centering
%% Number of surviving daphnia, specimens
%% \end{minipage}} \\
%% &
%% \multicolumn{3}{>{\centering\arraybackslash}p{(\linewidth - 12\tabcolsep) * \real{0.2460} + 4\tabcolsep}}{%
%% \begin{minipage}[b]{\linewidth}\centering
%% Control
%% \end{minipage}} &
%% \multicolumn{3}{>{\centering\arraybackslash}p{(\linewidth - 12\tabcolsep) * \real{0.2657} + 4\tabcolsep}@{}}{%
%% \begin{minipage}[b]{\linewidth}\centering
%% CCP module waste
%% \end{minipage}} \\
%% &
%% \multicolumn{3}{>{\centering\arraybackslash}p{(\linewidth - 12\tabcolsep) * \real{0.2460} + 4\tabcolsep}}{%
%% \begin{minipage}[b]{\linewidth}\centering
%% repetition
%% \end{minipage}} &
%% \multicolumn{3}{>{\centering\arraybackslash}p{(\linewidth - 12\tabcolsep) * \real{0.2657} + 4\tabcolsep}@{}}{%
%% \begin{minipage}[b]{\linewidth}\centering
%% repetition
%% \end{minipage}} \\
%% & \begin{minipage}[b]{\linewidth}\centering
%% 1
%% \end{minipage} & \begin{minipage}[b]{\linewidth}\centering
%% 2
%% \end{minipage} & \begin{minipage}[b]{\linewidth}\centering
%% 3
%% \end{minipage} & \begin{minipage}[b]{\linewidth}\centering
%% 1
%% \end{minipage} & \begin{minipage}[b]{\linewidth}\centering
%% 2
%% \end{minipage} & \begin{minipage}[b]{\linewidth}\centering
%% 3
%% \end{minipage} \\
%% \midrule\noalign{}
%% \endhead
%% \bottomrule\noalign{}
%% \endlastfoot
%% 28.02.2023 & 10 & 10 & 10 & 10 & 10 & 10 \\
%% 01.03.2023 (24 hours) & 10 & 10 & 10 & 10 & 10 & 10 \\
%% 02.03.2023 (48 hours) & 10 & 10 & 10 & 10 & 9 & 9 \\
%% Average number of surviving daphnia, specimens &
%% \multicolumn{3}{>{\centering\arraybackslash}p{(\linewidth - 12\tabcolsep) * \real{0.2460} + 4\tabcolsep}}{%
%% 10} &
%% \multicolumn{3}{>{\centering\arraybackslash}p{(\linewidth - 12\tabcolsep) * \real{0.2657} + 4\tabcolsep}@{}}{%
%% 9.3} \\
%% A, \% &
%% \multicolumn{3}{>{\centering\arraybackslash}p{(\linewidth - 12\tabcolsep) * \real{0.2460} + 4\tabcolsep}}{%
%% 0} &
%% \multicolumn{3}{>{\centering\arraybackslash}p{(\linewidth - 12\tabcolsep) * \real{0.2657} + 4\tabcolsep}@{}}{%
%% 7} \\
%% \end{longtable}

{\bfseries Results and discussion.} As a result of the experiment, it was
established that the survival rate of aquatic organisms in experimental
samples (compared to the control sample) is:

- CCP module waste - 17\%;

- CCP module waste + cake of module 1A - 70\%;

- Cake of module 1A - 100\%;

- Synthetic gypsum - 67\%.

Thus, determining the impact of only the aqueous extract of waste
without its dilution allows to classify the waste: waste of CCP module +
cake of module 1A, cake of module 1A and synthetic gypsum to the fifth
class of environmental hazard.

For the waste of the CCP module, a repeated biotesting procedure was
carried out (dilution factor 10 times). According to the results of the
study for the waste of the CCP module (survival rate of aquatic
organisms was 93\%) based on the dilution factor of the aqueous extract
(10 times), at which no impact on aquatic organisms was detected in
accordance with the dilution factor ranges (Table 6), hazard class 4 was
established.

{\bfseries Table 6 - Assignment of hazard class according to the dilution
factor of the aqueous extract}

%% \begin{longtable}[]{@{}
%%   >{\centering\arraybackslash}p{(\linewidth - 2\tabcolsep) * \real{0.2586}}
%%   >{\centering\arraybackslash}p{(\linewidth - 2\tabcolsep) * \real{0.7414}}@{}}
%% \toprule\noalign{}
%% \endhead
%% \bottomrule\noalign{}
%% \endlastfoot
%% \begin{minipage}[t]{\linewidth}\centering
%% \begin{quote}
%% Waste hazard class
%% \end{quote}
%% \end{minipage} & \begin{minipage}[t]{\linewidth}\centering
%% \begin{quote}
%% Dilution factor of the aqueous extract from hazardous waste, at which
%% there is no harmful effect on aquatic organisms
%% \end{quote}
%% \end{minipage} \\
%% \begin{minipage}[t]{\linewidth}\centering
%% \begin{quote}
%% I
%% \end{quote}
%% \end{minipage} & \begin{minipage}[t]{\linewidth}\centering
%% \begin{quote}
%% \textgreater10,000
%% \end{quote}
%% \end{minipage} \\
%% \begin{minipage}[t]{\linewidth}\centering
%% \begin{quote}
%% II
%% \end{quote}
%% \end{minipage} & \begin{minipage}[t]{\linewidth}\centering
%% \begin{quote}
%% From 10,000 to 1,001
%% \end{quote}
%% \end{minipage} \\
%% \begin{minipage}[t]{\linewidth}\centering
%% \begin{quote}
%% III
%% \end{quote}
%% \end{minipage} & \begin{minipage}[t]{\linewidth}\centering
%% \begin{quote}
%% From 1,000 to 101
%% \end{quote}
%% \end{minipage} \\
%% \begin{minipage}[t]{\linewidth}\centering
%% \begin{quote}
%% IV
%% \end{quote}
%% \end{minipage} & \begin{minipage}[t]{\linewidth}\centering
%% \begin{quote}
%% \textless100
%% \end{quote}
%% \end{minipage} \\
%% \begin{minipage}[t]{\linewidth}\centering
%% \begin{quote}
%% V
%% \end{quote}
%% \end{minipage} & \begin{minipage}[t]{\linewidth}\centering
%% \begin{quote}
%% 1
%% \end{quote}
%% \end{minipage} \\
%% \end{longtable}

{\bfseries Conclusion.} Even though there is no universal test system for
determining all existing toxicants, biotesting as an integral assessment
method is successfully used as an environmental monitoring tool.
Biotesting is one of the research techniques in the field of toxicology,
widely used nowadays to assess the degree of toxicity of various
ecosystems. Biotesting does not cancel the analytical control system,
but effectively complements it with qualitatively new indicators, since
from an environmental point of view, the results of determining the
concentration of pollutants themselves have only relative value. It is
important to know not the levels of pollution, but the biological
effects they cause. As a result of biotesting, it was established that
Daphnia magna is sensitive to waste from the processing of phosphate raw
materials. Aqueous extracts (without dilution) from the mixture of CCP
module waste + cake of module 1A (1:1), cake of module 1A and synthetic
gypsum do not have an inhibitory effect on hydrobionts. An ecological
and toxicological study confirmed that the studied samples belong to the
fifth class of environmental hazard (virtually non-hazardous). For the
aqueous extract from the waste of the CCP module without its dilution,
inhibition of the viability of aquatic organisms was revealed. The death
of daphnia may be associated with waste components that cause blockage
of the respiratory tract with dispersed particles. It was revealed that
with increasing dilution, the mortality rate of daphnia decreases; the
dilution ratio of the aqueous extract was established (10 times), having
no effect on hydrobionts. Thus, the waste from the CCP module, in
accordance with the dilution factor ranges, is assigned to the fourth
hazard class (low hazard) for the environment. The results obtained
allowed to conclude that the studied waste is acceptable for safe
storage (under controlled conditions) and further processing.

{\bfseries References}

1. Goncharuk V.V., Pleteneva T.V., Rudenko A.V., Syroeshkin A.V.,
Kovalenko V.F., Uspenskaya E.V., Saprykina M.N., Zlatskiy I.A. Basic
principles of comprehensive biotesting of drinking water and point
system classification of water quality // Journal of Water Chemistry and
Technology. - 2018. - Vol.40(1).- P.35- 39 DOI
10.3103/S1063455X1801006X.

2. Stravinskene E.S., Subbotin M.A., Grigoriev Yu.S., Shashkova T.L.,
Sorokina G.A. The effect of the ratio of the medium volume and the
number of organisms on the results of toxicological experiments // IOP
Conference Series: Earth and Environmental Science.- 2019. - Vol.315:
042033. DOI 10.1088/1755- 1315/315/4/042033.

3. Starostina I.,~Vasilenko T.,~Simonov~M., Pendurin E. Evaluating
Toxicological Properties of Claydite Gravel, Containing Ferrovanadium
Production Sludge, by Method of Biotesting with the of Higher
Plants//E3S Web Conferences.- 2019.- Vol.126:00068.

DOI 10.1051/e3sconf/201912600068

4. Bushumov S., Korotkova T., Ksandopulo S., Solonnikova N., Demin V.
Determination of the Hazard Class of Ash After Coal Combustion by the
Method of Biotesting // Oriental Journal of Chemistry. -- 2018. - Vol.
34(1). - P.276-285.~DOI 10.13005/ojc/340130~

5. Kucherik G.V., Omelchuk Yu.A., Sytnikov D.M. Biotesting of quarry
lakes as an alternative source of drinking water supply // Scientific
Notes of Crimean V. I. Vernadsky Federal University Biology. Chemistry.
2022. - Vol.8 (74). - P.87--92

6. Lushchay Е.А., Ivanov D.E., Tikhomirova E.I. Development and
Efficiency Assessment of New Methods on Rapid Assessment of Toxicity in
Environmental Monitoring // Povolzhskiy Journal of Ecology. - 2019. -
Vol.4. - P.458-469. DOI 10.35885/1684-7318-2019-4-458-469

7. Morachevskaya E.V., Voronina L.P. Bioassay as a method of integral
assessment for remediation of oil-contaminated ecosystems // Theoretical
and Applied Ecology. - 2022. - Vol.1. - P.34-43. DOI
10.25750/1995-4301-2022-1-034-043

8. Goncharova E.N., Kurzenev I.R., Vasilenko M.I., Pendjurin E.A.
Biotestirovanie zookomposta kul' tivirovanie lichinok
Hermetia illucens // Vestnik RUDN. Serija: Jekologija i
bezopasnost'{} zhiznedejatel' nosti. -
2020. - T.28(4). - S.324-335. DOI 10.22363/2313-2310-2020-28-4-324-335
{[}in Russian{]}

9. Bardina T., Podboronova A., Skljarova L. Jekotoksikologicheskaja
ocenka othodov i pochvennogo pokrova antropogenno zagrjaznennyh
territorij s ispol' zovaniem biotest-sistem // Formuly
Farmacii.-2021.- T.3(4).-S.102-107. DOI 10.17816/phf106205. {[}in
Russian{]}

10. Bardina T.V., Chugunova M.V., Kulibaba V.V.
Ispol'' zovanie metodov biotestirovanija
dlja ocenki jekologicheskogo sostojanija pochvogruntov
rekul'' tivirovannogo
kar'' era // Biosfera.- 2020. -T.12(1-2).
- S.1-11. DOI 10.24855/BIOSFERA.V12I1.539. {[}in Russian{]}

11. Kuzubova E., Grigorenko N., Shaidorova G., Ogneva Z., Potapova M.
Biotesting of Soil Contamination of Agricultural Land Prokhorovsky
District of the Belgorod Region // Engineering Proceedings.~- 2023. -
Vol.37(1):44. DOI 10.3390/ECP2023-14657

12. Jurlov A.A., Suncova N.A., Musihina T.A., Zemcova E.A., Koshkina
N.A., Devjaterikova S.V., Kazienkov S.A. Vlijanie stokov proizvodstva
ftorpolimerov na biotu // Voda i jekologija: problemy i reshenija.-
2018. -№ 3(75). - S.76-84. DOI 10.23968/2305--3488.2018.20.3.76--84
{[}in Russian{]}

13. Biologicheskie metody kontrolja. Metodika opredelenija toksichnosti
vody i vodnyh vytjazhek iz pochv, osadkov stochnyh vod, othodov po
smertnosti i izmeneniju plodovitosti dafnij. Federal' nyj
reestr (FR). FR.1.39.2007.03222. Metodika dopushhena dlja celej
gosudarstvennogo jekologicheskogo kontrolja. Moskva.: «AKVAROS», 2007. -
47 s. {[}in Russian{]}

14. ST RK 17.1.4.01-95 Ohrana prirody. Gidrosfera. Metodika opredelenija
ostroj toksichnosti vody na dafnijah. Gosudarstvennyj standart
Respubliki Kazahstan, Komitet po standartizacii, metrologii i
sertifikacii Respubliki Kazahstan. Almaty. - 1996 {[}in Russian{]}

15. RD 118-02-90 Miheev N.N. Rukovodstvo po opredeleniju metodom
biotestirovanija toksichnosti vod, donnyh otlozhenij, zagrjaznjajushhih
veshhestv i burovyh rastvorov. RJeFIA, NIA-Priroda Moskva. - 2002. {[}in
Russian{]}

\emph{{\bfseries Information about the authors}}

Seraya N.V. - candidate of Chemical Sciences, Professor of D. Serikbayev
East Kazakhstan Technical University, Ust-Kamenogorsk\emph{,}
Kazakhstan, e-mail:
nseraya@mail.ru;

Litvinov V.V. - deputy director of Proektno-ekologicheskoe bjuro LLP,
Ust-Kamenogorsk, Kazakhstan, e-mail:
litvinov_vadim@mail.ru;

Daumova G.K. - candidate of Technical Sciences, Professor of
D.Serikbayev East Kazakhstan Technical University,
Ust-Kamenogorsk\emph{,} Kazakhstan, е-mail:
gulzhan.daumova@mail.ru;

Yelubay M.A. - candidate of Chemical Sciences, Professor of Toraighyrov
University, Pavlodar, Kazakhstan; Researcher, Humboldt-Innovation GmbH,
Berlin, Germany. e-mail: m.yelubay@gmail.com;

Kulmagambetova E.A. - candidate of Chemical Sciences, Senior Researcher
at the Department of Biomonitoring and Occupational Hygiene of the
RRIOSH of the Ministry of Health of the Republic of Kazakhstan, Astana,
Kazakhstan, e-mail:
elya_kulmagambet@mail.ru;

Woszczyk M. - Ph.D., associate professor, Biogeochemistry Research Unit,
Adam Mickiewicz University, Poznań, Poland, е-mail:
woszczyk@amu.edu.pl.

\emph{{\bfseries Сведения об авторах}}

Серая Н.В. - кандидат химических наук, профессор Восточно-Казахстанского
технического университета им. Д.Серикбаева, Усть-Каменогорск, Казахстан,
е-mail: nseraya@mail.ru;

Литвинов В.В. - заместитель директора ТОО «Проектно-экологическое бюро»,
Усть-Каменогорск, Казахстан, е-mail:
\href{mailto:litvinov_vadim@mail.ru}{itvinov\_vadim@mail.ru};

Даумова Г.К. - кандидат технических наук, профессор
Восточно-Казахстанского технического университета, Усть-Каменогорск,
Казахстан; е-mail:
gulzhan.daumova@mail.ru;

Елубай М.А. - кандидат химических наук, профессор Торайгыров
Университет, Павлодар, Казахстан, научный сотрудник, Humboldt-Innovation
GmbH, Берлин, Германия, е-mail:m.yelubay@gmail.com;

Кульмагамбетова Э.А. - кандидат химических наук, ведущий научный
сотрудник отдела биомониторинга и гигиены труда, РГП на ПХВ
«Республиканский научно-исследовательский институт по охране труда
Министерства труда и социальной защиты населения Республики Казахстан»,
Астана, Казахстан, e-mail:
elya_kulmagambet@mail.ru;

Woszczyk M. - доцент, исследовательский отдел биогеохимии, Университет
Адама Мицкевича, Познань, Польша, е-mail:
woszczyk@amu.edu.pl.\
