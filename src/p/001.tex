\id{ҒТАМР 65.59.91}{}

\begin{header}
\swa{}{ҚОЙДЫҢ ҚАРЫНЫНДАҒЫ МАССАНЫҢ (КАНЫГА) КЕПТІРУДЕН КЕЙІНГІ
ФИЗИКО-ХИМИЯЛЫҚ КӨРСЕТКІШТЕРІН ЗЕРТТЕУ}

{\bfseries
Г.С  Кененбай,
У.Ч. Чоманов,
М.А. Идаятова\envelope 
}
\end{header}

\begin{affil}
«Қазақ қайта өңдеу~және тағам өнеркәсіптері ғылыми-зерттеу институты» ЖШС Алматы, Қазақстан

\envelope Корреспондент-автор: idayatova\_m@mail.ru
\end{affil}

Көптеген дамушы елдерде жыл сайын едәуір көлемде жануарлардан алынатын
жанама өнімдер пайда болады. Бұл өнімдер қайта өңдеуге жарамды,
дегенмен, техникалық білімнің жеткіліксіздігінен олар жойылады немесе
толық пайдаланылмайды. Осындай жанама өнімдердің бірі - қойдың
қарынындағы масса (каныга). Қазіргі таңда осы шикізатты қайта өңдеу
мақсатында оның сапалық көрсеткіштерін зерттеу маңызды болып отыр.

Зерттеу нысаны: қойдың қарынындағы масса (каныга). Зерттеу жүргізуде
белгіленген мемлекеттік стандарттар қолданылды. Зерттеудің мақсаты -
қойдың қарынындағы массаның (каныга) физика-химиялық көрсеткіштеріне,
дәрумендер мен минералды заттар құрамына зерттеу жүргізу. Осы мақсатта
бастапқы ылғалдылық, гигроскопиялық ылғалдық, жалпы ылғалдылық, құрғақ
заттар мөлшері, күлділік, қант, азотсыз экстрактивті заттар, талшық,
май, ақуыз мөлшері мен дәрумендер, минералды заттар мөлшері зерттелінді.

Зерттеу нәтижесінде бастапқы ылғалдылық - 5,63\%, гигроскопиялық
ылғалдылық -- 2,31\%, жалпы ылғалдылық -- 7,81\%, құрғақ заттар мөлшері
- 92,19\% көрсетті. Ақуыз мөлшері - 9,19 \%, күлділік -- 9,19\%, азотсыз
экстрактивті заттар - 33,67\%, талшық - 37,75\% көрсетті. Қой
қарынындағы масса Е дәруменіне бай (212,0 мг) және В1 (0,01мг/100г), В2
(0,13 мг/100г), В3 (0,057 мг/100г), В5 (0,044мг/100г), В6 (0,009мг/100г)
дәрумендері анықталды. Зерттеу қорытындысына сәйкес, қойдың қарынындағы
массаның физика-химиялық көрсеткіштеріне талдау жасалынды.

Зерттеу нәтижесінде қойдың қарынындағы масса көптеген минералды заттар
мен дәрумендерге бай, қайта өңдеуге жарамды шикізат екенін көрсетті.

{\bfseries Түйін сөздер:} қойдың қарынындағы масса (каныга),
физико-химиялық талдау, қайта өңдеу, ыл\-ғалдылық, минералды заттар,
дәрумендер.

\begin{header}
{\bfseries ИССЛЕДОВАНИЕ ФИЗИКО-ХИМИЧЕСКИХ ПОКАЗАТЕЛЕЙ КАНЫГИ ОВЕЦ ПОСЛЕ СУШКИ}

{\bfseries
Г.С. Кененбай,
У.Ч. Чоманов,
М.А. Идаятова\envelope
}
\end{header}

\begin{affil}
ТОО " Казахский научно-исследовательский институт перерабатывающей и пищевой промышленности» Алматы, Казахстан,

e-mail: idayatova\_m@mail.ru
\end{affil}

Во многих развивающихся странах ежегодно появляется значительное
количество побочных продуктов животного происхождения. Эти продукты
пригодны для вторичной переработки, однако из-за недостатка технических
знаний они либо утилизируются, либо не используются в полной мере. Одним
из таких побочных продуктов является каныга. В настоящее время в целях
переработки данного сырья важно изучить его качественные показатели.

Объект исследования - каныга овец. При проведении исследования
использовались установленные государственные стандарты. Цель
исследования -- проведение исследований физико-химичес\-ких показателей,
содержания витаминов и минеральных веществ в каныге овец. Для этого были
изучены первоначальная влажность, гигроскопическая влажность, общая
влажность, содержание сухих веществ, зольность, сахар, безазотные
экстрактивные вещества, клетчатка, жир, содержание белка и витаминов,
минеральных веществ.

Исследование показало, что первоначальная влажность составила 5,63\%,
гигроскопическая влажность - 2,31\%, общая влажность - 7,81\%,
содержание сухого вещества - 92,19\%. Содержание белка - 9,19\%, золы -
9,19\%, безазотных экстрактивных веществ -33,67\%, клетчатки - 37,75\%.
Каныга овцы богата витамином Е (212,0 мг) и витаминами В₁ (0,01 мг/100
г), В₂ (0,13 мг/100 г), В₃ (0,057 мг/100 г), В₅ (0,044 мг/100 г), В₆
(0,009 мг/100 г). В соответствии с заключением исследования был проведён
анализ физико-химических показателей каныги овец.

Исследование показало, что каныга - это перерабатываемое сырье, богатое
многими минералами и витаминами.

{\bfseries Ключевые слова}: каныга овец, физико-химический анализ,
переработка, влажность, минеральные вещества, витамины.

\begin{header}
{\bfseries STUDY OF PHYSICAL-CHEMICAL PARAMETERS OF SHEEP RUMEN CONTENT AFTER DRYING}

{\bfseries
G.S., Kenenbay,
U.Ch. Chomanov,
M.A Idayatova\envelope
}
\end{header}

\begin{affil}
Kazakh Scientific Research Institute of Processing and Food Industry, Almaty, Kazakhstan,

e-mail: idayatova\_m@mail.ru
\end{affil}

A significant number of animal by-products are produced annually in many
developing countries. These products are recyclable, but due to a lack
of technical knowledge, they are either disposed of or not fully
utilized. One of these by-products is sheep rumen content. Currently, in
order to process this raw material, it is important to study its quality
indicators.

Object of research: sheep rumen content. State standards were used
during the research. The aim of the study was to investigate the
physical-chemical parameters, vitamin, and mineral content in sheep
rumen content. To this end, the initial moisture, hygroscopic moisture,
total moisture, dry matter content, ash content, sugar, nitrogen-free
extractive substances, fiber, fat, protein content, vitamins, and
minerals were studied.

The study showed that the initial moisture content was 5.63\%,
hygroscopic moisture was 2.31\%, total moisture was 7.81\%, and the dry
matter content was 92.19\%. The protein content was 9.19\%, ash content
was 9.19\%, nitrogen-free extractive substances were 33.67\%, and fiber
content was 37.75\%. Sheep rumen content is rich in vitamin E (212.0 mg)
and vitamins B₁ (0.01 mg/100 g), B₂ (0.13 mg/100 g), B₃ (0.057 mg/100
g), B₅ (0.044 mg/100 g), and B₆ (0.009 mg/100 g). According to the study
conclusion, an analysis of the physical-chemical parameters of sheep
rumen content.

The study showed that canyga is rich in many minerals and vitamins, and
showed that it is a recyclable raw material.

{\bfseries Keywords:} sheep rumen content, physical-chemical analysis,
processing, moisture, mineral substances, vitamins.

\begin{multicols}{2}
{\bfseries Кіріспе.} Қой шаруашылығы - Қазақстанның ежелден жалғасып келе
жатырған мал шаруашылығының дәстүрлі саласы. Мемлекет дамуының барлық
кезеңдерінде қой шаруашылығы ет өндірісін ұлғайтудың басым бағыттарының
бірі, халқымыздың әлеуметтік-тұрмыстық жағдайын жақсартумен қатар,
азық-түлік нарығын қолжетімді және сапалы ет өнімдерімен қамтамасыз
етудің маңызы көзі болып келе жатыр{[}1{]}.

Қой еті өзінің ерекше диеталық, тағамдық құндылығымен және
органолептикалық қасиеттерімен ерекше, сондай-ақ минералдар мен
дәрумендердің бай көзі болуымен жоғары бағаланады. Әлемде қой етін
тұтыну шошқа, құс, сиыр етінен кейін төртінші орынды алады{[}2,3{]}.

Соңғы 20 жылда тек елімізде емес, дүние жүзінде де (соның ішінде Африка,
Азия, Еуропа, Америка Құрама Штаттары) ет пен ет өнімдеріне сұраныстың
артуы байқалды, бұл өмір сапасының жоғарылауына байланысты. Осы орайда,
көп мөлшерде жанама өнімдер артылып қалуы байқалады (союдан кейінгі
жануар массасының шамамен 2/3 бөлігі жанама өнімдерге тиесілі) және
оларды жинау және өңдеу көбірек көңіл бөлуді талап етеді{[}4,5,6{]}.

Дүние жүзіндегі сою өнімдері екі үлкен топқа бөлінеді: еdible (жеуге
жарамды, тағамдық) және non-edible (жеуге жарамсыз, тағамдық емес).
Союдың жеуге жарамды өнімдері (бауыр, бүйрек, тіл, жүрек және т.б.),
жеуге жарамсыз (мүйіздер, тұяқтар және қарындағы масса т. б.) жатады
{[}7{]}.

Жеуге жарамсыз жанама өнімдерді пайдаланудан бас тарту пайда алып
келетін кірістің жоғалуына, сондай-ақ оларды жою үшін шығатын
шығындарының артуына әкеледі. Жеуге жарамсыз және қолданылу мүмкіндігі
әлі күнге дейін толық ашылмаған шикізаттың бірі -- осы қарындағы масса
(каныга). Және қойдың қарынындағы массаның елімізде пайдаланылмай,
жойылуы мал сою және қайта өңдеу кезінде қоршаған ортаны ластанудан
қорғауды қамтамасыз ететін тиімді өндірістік әдісті және жоғары
экономикалық тиімділікті қамтамасыз ететін қайталама шикізатты толық
қайта өңдеуді көздейтін ресурс үнемдеуші технологияларды кеңінен
енгізуді қажет етеді {[}8{]}.

Осы себепті оның пайдаланылу мүмкіндігін зерттеу мақсатында
физика-химиялық көрсеткіштеріне талдау жасау маңыздылығы туындап отыр.

Қойлардың қарынындағы масса -- бұл негізгі азықтың (шөп, жем,
концентраттар) және ас қорыту жүйесіндегі микроорганизмдердің
ферментациясы нәтижесінде пайда болған заттардың қоспасы. Қойлардың
қарынындағы масса көлемі жағынан ірі қара малдың асқазанынан әлдеқайда
кіші екенін атап өткен жөн. Ірі қара малдың асқазанының ішіндегі заттар
жануардың салмағының орта есеппен 20\%-ын құрайды, ал қойларда бұл
көрсеткіш 10\%-ға тең. Осылайша, қойлардың асқазанының салыстырмалы
сыйымдылығы ірі қара малмен салыстырғанда екі есе аз.

Қарында бар қолайлы жағдайлар онда микрофлораның көп мөлшерде дамуына
ықпал етеді. Бұған тұрақты және қолайлы температура, қарынның ішіндегі
құрамның бейтарапқа жақын реакциясы, қоректік заттардың мол болуы және
тұрақты түрде келуі жатады. Қарын микрофлорасын құрайтын
микроорганизмдер ферментативті белсенділігінің әртүрлілігімен
ерекшеленеді. Қарында азықтың заттарын микробты ыдырату процестерімен
қатар, белсенді синтез процестері де жүреді {[}9{]}.

Қойдың қарынындағы массаның физико-химиялық сипаттамалары оның
денсаулығы үшін және оны өңдеудің тиімділігін арттыруда өте маңызды.
Қарындағы массаны кептіру үдерісіндегі ылғалдылық көрсеткіштері, құрғақ
заттардың мөлшері мен басқа да қасиеттері өнімнің сапасына тікелей әсер
етеді. Оларды зерттеу арқылы, өнімнің қоректік құндылығы мен пайдалану
қасиеттерін жақсарту үшін арнайы өңдеу технологияларын әзірлеуге болады.

{\bfseries Материалдар мен әдістер.} Зерттеу нысаны: қойдың қарынындағы
масса. Материалды жинау Алматы облысында жүргізілді.

Ылғалдылықты анықтау МемСТ 13496.3 - 92 сүйене отырып жүргізілді.
Ылғалдылықты анықтау үшін ШС-80-01 СПУ (Ресей) кептіру-зарарсыздандыру
түріндегі шкаф пайдаланылды.

Бастапқы ылғалдылық -65-70\tsp{0} С- кептірілген қойдың
қарынындағы массаны тұрақты массаға дейін кептіру арқылы анықталды.

Гигроскопиялық ылғалдылық - 2 гр шикізатты 100-105\tsp{0}С
тұрақты массаға дейін кептіру арқылы анықталды.

Күлділікті анықтау МемСТ 13979.6-69 сүйене отырып жүргізілді. Ол үшін
жоғары температурада күл алуға арналған SNOL (Литва) муфельді электр
пеші пайдаланылды. Күлділікті анықтау әдісі үлгіні
650-700\tsp{0}С тұрақты массаға дейін күлдендіру арқылы
жүзеге асырылады.

Сынаманы ұнтақтау үшін ылғалдылығы 18,0\% - дан аспайтын дәнді,
дәнді-бұршақты дақылдар мен басқа да жемшөп өсімдіктерінің сынамаларын
ұнтақтауға арналған ЛМЗ (Ресей) электронды астық диірмені пайдаланылды.

Сынамаларды өлшеу үшін сынамалар мен үлгілерді, химиялық реагенттерді
және әртүрлі материалдарды өлшеуге арналған "Shimadzu" (Жапония)
зертханалық электронды таразы қолданылды.

Шикі талшықты, майды, ақуыз, қант және крахмал анықтау үшін жем
анализаторы "NIR DS 2500" (Швеция) пайдаланылды. " NIR DS 2500" 850-ден
2500 нанометрге дейінгі толқын ұзындығының мамандандырылған диапазонында
айрықша дәлдікпен жемшөп пен жемшөп үлгілерін тікелей талдау,
ұнтақталған немесе ұнтақтамай, сондай-ақ түйіршіктелген үлгілерге
ИҚ-талдау жүргізуді қамтамасыз етеді.

Дәрумендер МемСТ 31483-2012 сүйете отырып, хромотографиялық әдіс арқылы
анықталды.

Минералды заттарды анықтау МемСТ 32343-2013 сүйене отырып, атомдық
абсорбциялық спектрометрия әдісімен кальций, мыс, темір, магний,
марганец, калий, натрий және мырыш құрамын анықтау негізінде жүргізілді.

{\bfseries Нәтижелер және талқылау.} Зерттеу жүргізу үшін қойдың қарыныдағы
масса 40°С температурада кептіру шкафында ылғалдылық мөлшері 7,81\%
көлемінде кептіріліп алынды (1 сурет).
\end{multicols}

\figstart{1 - сурет. Қойдың қарынындағы масса}
\subfig[0.45\textwidth]{5cm}{p/image2}{кептіруге дейін}
\subfig[0.45\textwidth]{5cm}{p/image3}{кептіргеннен кейін}
\figend

Алынған массаның ылғалдылығы және құрғақ заттар мөлшері анықталды (2
сурет).

{\bfseries 2 - сурет. Қойдың қарынындағы массаның ылғалдылығы мен құрғақ
заттар мөлшері, \%}

\emph{*БЫ- бастапқы ылғалдылық, ГЫ -- гигроскопиялық ылғалдылық, ЖЫ
жалпы ылғалдылық, ҚЗ - құрғақ заттар мөлшері}

Қарынындағы физико-химиялық қасиеттерді дұрыс бағалау және оларды ескере
отырып өңдеу үдерісін дұрыс ұйымдастыру, өндіріс шығындарын азайтуға
және өнімнің сапасын арттыруға мүмкіндік береді. Бұл мал шаруашылығы
өнімдерін нарықта тиімді сатуға және шығындарды оңтайландыруға септігін
тигізеді.

Зерттеу барысында қойдың қарнындағы массаны (каныга) кептіру үдерісінен
кейінгі физико-химиялық көрсеткіштері талданды. Зерттеу барысында
кептіруден кейінгі қой қарынындағы массаның 3 түрлі ылғалдылық мөлшері
анықталды. Кептірілген өнімнің ылғалдылық көрсеткіштері және құрғақ
заттар мөлшері 2- суретте көрсетілген. Бастапқы ылғалдылық - 5,63\%
көрсетті, бұл өнімнің табиғи күйінде бар ылғалдың мөлшерін білдіреді,
Гигроскопиялық ылғалдылық - 2,31\% болды, бұл өнімнің ауадағы ылғалды
сіңіру қабілетін сипаттайтын көрсеткіш. Гигроскопиялық ылғалдылықты
анықтау өнімнің ұзақ уақыт сақталуын бағалауда маңызды, өйткені жоғары
гигроскопиялық ылғалдылық өнімнің ылғалдылығын тез жоғалтып, сапасының
нашарлауына әкелуі мүмкін. Бұл көрсеткіш қойдың қарнындағы ылғалдылықтың
басқа қоршаған орта факторларына (мысалы, температура мен ауа
ылғалдылығына) қалай әсер ететінін анықтауға мүмкіндік береді. Зерттеу
нәтижелері бойынша гигроскопиялық ылғалдылық өнімнің сыртқы ортадан
ылғалды аз сіңіретінін көрсетеді. Жалпы ылғалдылық - өнімнің барлық
құрамындағы ылғал мөлшерін анықтайды, жалпы ылғалдылық - 7,81\% құрады.
Өнімдегі барлық органикалық және бейорганикалық заттардың массасы құрғақ
заттар мөлшерімен анықталады, кептіруден кейін қарындағы массада құрғақ
заттар мөлшері - 92,19\% көрсетті.

{\bfseries 3 - сурет. Қой қарынындағы массаның химиялық көрсеткіштері}

\emph{*АЭЗ - азотсыз экстрактивті заттар}

Және қой қарынындағы массада ақуыз, күлділік, АЭЗ, талшық пен май
мөлшері анықталды (3-сурет). Ақуыз - қойлар рационындағы негізгі
қоректік заттардың бірі, ол организмдегі көптеген процестерге әсер
етеді. Қойдың қарынындағы массада ақуыз мөлшері - 9,19 \% болды.
Күлділік мөлшері - 9,19\% көрсетті, ол мал азығындағы минералдар мен
элементтердің жиынтығы болып табылады, және оның жеткілікті деңгейде
болуы қойдың барлық физиологиялық процестерінің дұрыс жүруін қамтамасыз
етеді. Қой қарынындағы массада көп мөлшерде АЭЗ мен күлділік анықталды.
АЭЗ (азотсыз экстрактивті заттар) - жем-шөптің құрамындағы көмірсулардың
сіңімді және энергияға бай бөлігін қамтиды, оның мөлшері - 33,67\%
болды. Қой қарынындағы массада ең көп мөлшерде - талшық анықталды,
талшық асқазан-ішек жолдарының қалыпты жұмысын қамтамасыз етуде маңызды
рөл атқарады. Ол асқазанның моторикасын жақсартып, ішектің дұрыс жұмыс
істеуіне ықпал етеді. Оның мөлшері 37,75\% құрады. Және 2,21\% май
мөлшері анықталды. Май қойдың ағзасындағы энергия мен жылудың маңызды
көзі.

{\bfseries 4 - сурет. Қойдың қарынындағы массадағы дәрумендер мөлшері}

Қой қарынындағы массаның дәрумендер құрамына да зерттеу
жүргізілді(4-сурет). Қой үшін дәрумендер олардың өсуі, өнімділігі,
репродуктивті денсаулығы, иммунитеті және жалпы физиологиялық жағдайы
үшін өте маңызды. Зерттеу барысында массада көп мөлшерде Е дәрумені
(212,0 мг) және В1 (0,01мг/100г), В2 (0,13 мг/100г), В3 (0,057 мг/100г),
В5 (0,044мг/100г), В6 (0,009мг/100г) дәрумендері анықталды.

Қой қарынындағы массадағы минералды элементтерді анықтау -- зерттеудің
өзекті бағыттарының бірі. Қалыпты метаболизм мен энергияны,
ферменттердің, гормондардың түзілуін қамтамасыз ету үшін минералды
заттар қойдың денесіне жеммен үнемі жеткізіліп отырады.

Ал, минералды заттар жағынан қой қарынындағы масса әр түрлі минералды
элементтермен бай (5-сурет).

{\bfseries 5 - сурет. Қой қарынындағы массадағы минералды заттар мөлшері}

\emph{*Cu - мыс, K - калий, Na - натрий, Mg - магний, Zn - мырыш, Fe -
темір, P - фосфор, Ca - кальций}

\begin{multicols}{2}
Қой қарынындағы массада ең көп мөлшердегі минералды элемент калий болды,
оның мөлшері - 213,16 мг/100г құрады. Калий (K) жүйке және бұлшықет
функциялары үшін маңызды, қарынының дұрыс жұмыс істеуін қамтамасыз
етеді, ферментациялық процестерді реттейді. Екінші орында көп мөлшерде
кездескен минералды элемент - фосфор - 115,63мг/100г., үшінші орында -
натрий (109,74 мг/100г). Фосфор (P) сүйек пен тіс құрылымын
қалыптастыруда маңызды, энергия алмасуына қатысады, ал натрий (Na)
су-электролиттік балансын сақтау үшін қажет, организмдегі сұйықтықтың
тепе-теңдігін қамтамасыз етеді. Одан кейінгі орында - кальций - 96,42
мг/100г және магний - 21,5мг/100 г. Кальций (Ca) сүйектердің беріктігін
арттырады, бұлшықет жиырылуына ықпал етеді және қарынының дұрыс жұмыс
істеуіне әсер етеді. Магний (Mg) сүйек жүйесін нығайтуға, ферменттердің
қызметін реттеуге және энергетикалық алмасуды қолдауға ықпал етеді. Және
аз мөлшерде темір - 5,06мг/100г; мырыш - 3,22 мг/100г; мыс - 1,95
гм/100г. кездесті. Темір (Fe) гемоглобиннің құрамына кіріп, оттегіні
тасымалдауға қатысады, қан айналымын жақсартады. Мырыш (Zn) ақуыз
синтезін және өсу процестерін қолдайды, терінің және жүннің сапасына
әсер етеді және ауруларға төзімділікті қалыптастырады. Мыс (Cu) көптеген
ферменттердің маңызды құрамдас бөлігі, темірдің тасымалын және қоректік
заттардың алмасуын қамтамасыз етеді, сондай-ақ жүйке, жабын, қаңқа,
иммундық және ас қорыту жүйелері сияқты бірнеше мүшелер жүйесінің дұрыс
дамуы мен сақталуы үшін өте маңызды {[}10{]}.

{\bfseries Қорытынды.} Қарын массасының физико-химиялық көрсеткіштерін
зерттеу арқылы өндіріс қалдықтарын (сою алаңдарындағы қой қарынының
массасы) тиімді қолдануға болады. Бұл қалдықтарды қайта өңдеу мақсатында
зерттеу жүргізу қоршаған ортаға жүктемені азайтады.

Зерттеу жүргізуде белгіленген мемлекеттік стандарттар қолданылып,
бастапқы ылғалдылық, гигроскопиялық ылғалдық, жалпы ылғалдылық, құрғақ
заттар мөлшері, күлділік, қант, азотсыз экстрактивті заттар, талшық,
май, ақуыз мөлшері мен дәрумендер, минералды заттар мөлшері зерттелінді.

Зерттеу нәтижесінде бастапқы ылғалдылық - 5,63\%, гигроскопиялық
ылғалдылық - 2,31\%, жалпы ылғалдылық - 7,81\%, құрғақ заттар мөлшері -
92,19\% көрсетті. Қой қарынындағы масса Е дәрумені (212,0 мг) және В1,
В2, В3, В5, В6 дәрумендері, минералды заттар мөлшері бойынша калий
213,16 мг/100г, фосфор - 115,63мг/100г, натрий 109,74 мг/100г, кальций -
96,42 мг/100г және магний - 21,5мг/100г, темір - 5,06мг/100г; мырыш -
3,22 мг/100г; мыс - 1,95 гм/100г анықталды.

Зерттеу көрсеткендей, қойдың қарынындағы минералды және органикалық
құрамының жоғары болуы оның қоректік және экономикалық құндылығын
арттырады. Оны қайта өңдеу ауыл шаруашылығы саласында жаңа өнімдер алу
мүмкіндігін көрсетеді.

\emph{{\bfseries Қаржыландыру.} Материалдар "Жануарлардан алынатын жемшөп
қоспасын алу үшін сойылған жануарлардың қайталама ет шикізатын қайта
өңдеудің ресурс үнемдеуші технологиясын әзірлеу" Қазақстан Республикасы
Ауыл шаруашылығы министрлігінің 2021-2023 жылдарға арналған бюджеттік
бағдарламасының 213BR10764970 "Шикізат бірлігінен дайын өнімнің
түр-түрін кеңейту және шығару, сондай-ақ өнім өндірісіндегі қалдықтар
үлесін азайту мақсатында ауыл шаруашылығы шикізатын терең өңдеудің
ғылымды қажетсінетін технологияларын әзірлеу" ғылыми-техникалық
бағдарламасы шеңберінде дайындалды.}
\end{multicols}

\begin{center}
{\bfseries Әдебиеттер}
\end{center}

\begin{refs}
1. Омбаев А.М., Мусабаев Б.И., Хамзин К.П. Cовременное состояние и
перспективы развития овцеводства Казахстана // Oвцы, козы, шерстяное
дело. - 2013. -№ 2. - C.86-90

2. Milewski S. Health-promoting properties of sheep products // Medycyna
Weterynaryjna. - 2006. -- Vol.62 (5). -Р.516-519

3. Mitra М., Brian R. Population, World Production and Quality of Sheep
and Goat Products // American Journal of Animal and Veterinary Sciences.
-2020. -- Vol.15 (4). - P.291-299. DOI\\
\href{https://doi.org/10.3844/ajavsp.2020.291.299}{10.3844/ajavsp.2020.291.299}

4. Насонова В.В. Перспективные пути использования субпродуктов // Теория
и практика переработки мяса. - 2018. -№ 3. - С.64-73. DOI
10.21323/2414-438Х-2018-3-3-64-73

5. Кузлякина Ю.А., Юрчак З.А. К вопросу экологической безопасности:
побочное сырье и отходы мясной промышленности // Все о мясе. -2017. -№
6. -с.29-31

6. Нуриманшина Г.Р., Кузлякина Ю.А., Юрчак З.А. Сравнительный анализ
подходов к управлению отходами мясной промышленности биологического
происхождения в РФ, странах ЕС и США // Все о мясе -2021. -№ 6. - с.
4-10. ~\href{https://doi.org/10.21323/2071-2499-2021-6-4-10}{DOI
10.21323/2071-2499-2021-6-4-10}

7. Irshad, A., Sharma, B.D. Abattoir By‑Product Utiliza tion for
Sustainable Meat Industry: A Review // Journal of Animal Production
Advances. -2015.-№ 5.-P.681-696.
DOI \href{https://doi.org/10.5455/JAPA.20150626043918}{10.5455/JAPA.20150626043918}.

8. Гиро Т.М., Хвыля С.И., Рогожин А.А., Гиро А.В. Переработка
коллагенсодержащих субпродуктов мелкого рогатого скота // Инновационные
технологии обработки и хранения сельскохозяйственного сырья и пищевых
продуктов. Сборник научных трудов ученых и специалистов к 90-летию
ВНИХИ. Москва. - 2020.- с.63-72

9. Cordeiro A.R., Bezerra T.K., Madruga M.S. Valuation of Goat and Sheep
By-Products: Challenges and Opportunities for Their Use
//Animals.~-2022.-Vol.12(23):3277. DOI 10.3390/ani12233277

10. Mahmoud O.A., Khadiga A. Rumen Content as Animal Feed: A Review //
University of Khatrum. Journal of Veterinary Medicine and Animal
Production. -2016. - № 2. -P.80-88
\end{refs}

\begin{center}
{\bfseries References}
\end{center}

\begin{refs}
1. Ombaev A.M., Musabaev B.I., Hamzin K.P. Covremennoe sostojanie i
perspektivy razvitija ovcevodstva Kazahstana//Ovcy, kozy, sherstjanoe
delo. - 2013. -№ 2.- S.86-90.{[}in Russian{]}

2. Milewski S. Health-promoting properties of sheep products // Medycyna
Weterynaryjna. - 2006. -- Vol.62 (5). -Р.516-519

3. Mitra М., Brian R. Population, World Production and Quality of Sheep
and Goat Products // American Journal of Animal and Veterinary Sciences.
-2020. -- Vol.15 (4). - P.291-299. DOI\\
\href{https://doi.org/10.3844/ajavsp.2020.291.299}{10.3844/ajavsp.2020.291.299}

4. Nasonova V.V. Perspektivnye puti ispol' zovanija
subproduktov // Teorija i praktika pererabotki mjasa. - 2018. -№ 3. -
S.64-73. DOI 10.21323/2414-438H-2018-3-3-64-73. {[}in Russian{]}

5. Kuzljakina Ju.A., Jurchak Z.A. K voprosu jekologicheskoj
bezopasnosti: pobochnoe syr' e i othody mjasnoj
promyshlennosti // Vse o mjase. -2017. -№ 6. - S.29-31.{[}in Russian{]}

6. Nurimanshina G.R., Kuzljakina Ju.A., Jurchak Z.A.
Sravnitel' nyj analiz podhodov k upravleniju othodami
mjasnoj promyshlennosti biologicheskogo proishozhdenija v RF, stranah ES
i SShA // Vse o mjase -2021. -№ 6. - S.4-10. DOI
10.21323/2071-2499-2021-6-4-10.{[}in Russian{]}

7. Irshad, A., Sharma, B.D. Abattoir By‑Product Utiliza tion for
Sustainable Meat Industry: A Review // Journal of Animal Production
Advances. -2015.- № 5.-P.681-696.
DOI \href{https://doi.org/10.5455/JAPA.20150626043918}{10.5455/JAPA.20150626043918}.

8. Giro T.M., Hvylja S.I., Rogozhin A.A., Giro A.V. Pererabotka
kollagensoderzhashhih subproduktov melkogo rogatogo skota //
Innovacionnye tehnologii obrabotki i hranenija
sel' skohozjajstvennogo syr' ja i
pishhevyh produktov. Sbornik nauchnyh trudov uchenyh i specialistov k
90-letiju VNIHI. Moskva. - 2020.- S.63-72.

9. Cordeiro A.R., Bezerra T.K., Madruga M.S. Valuation of Goat and Sheep
By-Products: Challenges and Opportunities for Their Use
//Animals.~-2022.-Vol.12(23):3277. DOI 10.3390/ani12233277

10. Mahmoud O.A., Khadiga A. Rumen Content as Animal Feed: A Review //
University of Khatrum. Journal of Veterinary Medicine and Animal
Production. -2016. - № 2. -P.80-88
\end{refs}

\begin{info}
\emph{{\bfseries Авторлар туралы мәліметтер}}

Кененбай Г.С. - техника ғылымдарының кандидаты, Қазақ қайта өңдеу~және
тағам өнеркәсіптері ғылыми-зерттеу институты, Алматы, Қазақстан, e-mail:
g.kenenbay@rpf.kz;

Чоманов Уришбай - ҚР ҰҒА академигі, техника ғылымдарының докторы,
профессор, Қазақ қайта өңдеу~және тағам өнеркәсіптері ғылыми-зерттеу
институты, Алматы, Қазақстан, e-mail:
chomanov\_u@mail.ru;

Идаятова М.А. - техника ғылымдарының магистрі, Қазақ қайта өңдеу~және
тағам өнеркәсіптері ғылыми-зерттеу институты, Алматы, Қазақстан, e-mail:
idayatova\_m@mail.ru.

\emph{{\bfseries Information about authors}}

Kenenbay G.S. - candidate of technical sciences, Kazakh Research
Institute of Processing and Food Industries, Almaty, Kazakhstan, e-mail:
g.kenenbay@rpf.kz;

Chomanov Urishbay - Academician of the National Academy of Sciences of
the Republic of Kazakhstan, Doctor of Technical Sciences, Professor,
Kazakh Research Institute of Processing and Food Industries, Almaty,
Kazakhstan, e-mail:\\
chomanov\_u@mail.ru;

Idayatova M. A. - master of technical sciences, Kazakh Research
Institute of Processing and Food Industries, Almaty, Kazakh\-stan,
e-mail:idayatova\_m@mail.ru.
\end{info}
