\id{МРНТИ 65.09.05}{}

{\bfseries ОБЗОР О ПОЛЕЗНЫХ СВОЙСТВАХ МОРКОВИ (Аналитический обзор)}

{\bfseries \tsp{1}А.А.
Абдулдаева}\fig{p/image1}{}{\bfseries ,
\tsp{1}С.К.
Тарджибаева}\fig{p/image1}{}{\bfseries ,
\tsp{1}Г.Н.
Досжанова}\fig{p/image1}{}{\bfseries ,
\tsp{1,2}Н.А.Муханбетова}\fig{p/image1}{}{\bfseries \envelope ,}

{\bfseries \tsp{1}О.М.
Кожамкулов}\fig{p/image1}{}

\emph{\tsp{1}Медицинский университет Астана, Астана,
Казахстан,}

\emph{\tsp{2}Казахский агротехнический исследовательский
университет им. Сакена Сейфуллина, Астана, Казахстан}

\envelope  Корреспондент-автор:
mukhanbetova.n@amu.kz

Морковь (Daucus carota L.) - важная сельскохозяйственная культура
семейства Apiaceae, имеющая высокую пищевую и биологическую ценность.
Корнеплоды моркови содержат широкий спектр питательных веществ и
биологически активных соединений, включая каротиноиды и флавоноиды. В
данной статье представлен аналитический обзор современных научных данных
о полезных свойствах моркови, включая ее антиоксидантное,
противовоспалительное и онкопротекторное действия, а также влияния на
иммунную систему. Отмечена высокая профилактическая ценность моркови как
для взрослых, так и для детей. Регулярное потребление моркови в составе
ежедневного рациона, в том числе в виде функциональных продуктов и
натуральных добавок, способствует укреплению здоровья и снижению риска
развития ряда хронических заболеваний.

{\bfseries Ключевые слова:} морковь, каротиноиды, флавоноиды,
антиоксидантная способность, антиканцерогенное свойство, иммунная
система, функциональные продукты питания, натуральные добавки.

{\bfseries СӘБІЗДІҢ ПАЙДАЛЫ ҚАСИЕТТЕРІНЕ ШОЛУ (Аналитикалық шолу)}

{\bfseries \tsp{1}А.А. Абдулдаева, \tsp{1}С.К.
Тарджибаева, \tsp{1}Г.Н. Досжанова,
\tsp{1,2}Н.А.Муханбетова\envelope ,}

{\bfseries \tsp{1}О.М. Кожамкулов}

\emph{\tsp{1}Астана медицина университеті, Астана,
Қазақстан,}

\emph{\tsp{2}Сәкен Сейфуллин атындағы Қазақ агротехникалық
зерттеу университеті, Астана, Қазақстан,}

e-mail:
mukhanbetova.n@amu.kz

Сәбіз (Daucus carota L.) -- жоғары қоректік және биологиялық
құндылығымен бағаланатын, Apiaceae тұқымдасының маңызды ауыл шаруашылығы
дақылы. Сәбіз тамырларында қоректік заттар мен биологиялық белсенді
қосылыстар, соның ішінде каротиноидтар мен флавоноидтар бар. Бұл
мақалада сәбіздің пайдалы қасиеттері, оның ішінде антиоксиданттық,
қабынуға қарсы және онкопротекторлық әсерлері, сондай-ақ иммундық жүйеге
әсері туралы заманауи ғылыми деректерге аналитикалық шолу берілген.
Сәбіздің жоғары профилактикалық құндылығы ересектер үшін де, балалар
үшін де атап өтілді. Сәбізді күнделікті рационның бөлігі ретінде, оның
ішінде функционалды тағамдар мен табиғи қоспалар түрінде үнемі тұтыну
денсаулықты жақсартуға және бірқатар созылмалы аурулардың даму қаупін
азайтуға көмектеседі.

{\bfseries Түйін сөздер:} сәбіз, каротиноидтар, флавоноидтар,
антиоксиданттық қабілет, антиканцерогендік қасиет, иммундық жүйе,
функционалдық тағамдар, табиғи қоспалар.

{\bfseries A REVIEW OF THE HEALTH BENEFITS OF CARROTS (Analytical review)}

{\bfseries \tsp{1}A. Abduldayeva, \tsp{1}S.
Tarjibaeva, \tsp{1}G. Doszhanova, \tsp{1,2}N.
Mukhanbetova\envelope ,}

{\bfseries \tsp{1}O. Kozhamkulov}

\emph{\tsp{1}Astana Medical University, Astana, Kazakhstan,}

\emph{\tsp{2}Saken Seifullin Kazakh Agrotechnical Research
University, Astana, Kazakhstan,}

e-mail:
mukhanbetova.n@amu.kz

Carrot (Daucus carota L.) is an important agricultural crop belonging to
the Apiaceae family, valued for its high nutrional and biologigal
significance. Carrot roots contain a wide range of nutrients and
bioactive compounds, including carotenoids and flavonoids. This article
presents an analytical review of current scientific data on the
beneficial properties of carrots, including their antioxidant,
anti-inflammatory, and anticancer effects, as well as their important on
the immune system. The high preventive value of carrots is noted for
both adults and children. Regular consumption of carrots as part of the
daily diet, including in the form of functional foods and natural
supplements, contributes to improved health and a risk of developing
various chronic diseases.

{\bfseries Keywords:} carrot, carotenoids, flavonoids, antioxidant
activity, anticarcinogenic property, immune system, functional foods,
natural supplements.

{\bfseries Введение.} Природным источником важнейших витаминов и
минеральных веществ в ежедневном рационе, согласно принципам здорового
питания, служат овощи и фрукты. Одним из наиболее часто потребляемых
продуктов в этой группе является морковь (Daucus carota L.). Этот
популярный корнеплод выращивается по всему миру с древнейших времен, и
интерес к нему не угасает до сих пор -- особенно после открытия его
выраженных антиоксидантных свойств. Морковь отличается разнообразием
сортов, что проявялется в широкой цветовой палитре корнеплодов -- от
белого до фиолетового. Наиболее распространенной и популярной остается
оранжевая морковь, богатая β-каротином - провитамином А {[}1{]}.

\fig{p/image4}{}

{\bfseries Рис.1 - Анатомия корнеплода моркови (A) продольный вид; (B)
поперечный разрез}

{\bfseries (перидерма, флоэма и ксилема).
(\href{http://www.carrotmuseum.co.uk}{www.carrotmuseum.co.uk})}

В Казахстане морковь выращивается во всех регионах страны, и ее
потребление среди населения носит устойчивый характер. В 2023 году
посевные площади моркови в хозяйствах составили 19,6 тыс. гектаров.
Основные регионы-производители -- Павлодарская область (34,3\% от общего
объема посевов), Жамбылская область (20,4\%) и Алматинская область
(14,7\%). С развитием технологии капельного орошения, несмотря на
колебания погодных условий, урожайность моркови по стране в 2023 году
достигло 283,8 центнеров с гектара {[}2{]}.

На начало 2022 года в Казахстане насчитывалось 206~065 тонн запасов
моркови, что подчеркивает ее значимость как одного из основных продуктов
продовольственного запаса страны и свидетельствует о высокой доступности
продукта для населения {[}3{]}. Потребление моркови в Казахстане
соответствует рекомендациям Всемирной организации здравоохранения (ВОЗ)
и составляет от 8 до 12 кг в год. Морковь широко используется в
различных формах: в свежем виде (салаты, гарниры), в переработанном виде
(соки, детское питания, морковное пюре, сублимированные морковные
чипсы), в кулинарии (первые и вторые блюда, выпечка), в сушёном и
замороженном виде для хранения и экспорта, для производства
хлебобулочных и кондитерских изделий (хлеб, цукаты, мармелад, пастила).

В Казахстане выращиваются различные сорта моркови зарубежной селекции
(Laguna F1, Dordoni, Redko) и отечественной (Дербес, Ұшқын, Алау,
Арнау-25). Оригинатором этих сортов является ТОО "Казахский
научно-исследовательский институт картофелеводства и овощеводства"
(КазНИИКО), все они внесены в Государственный реестр селекционных
достижений, рекомендуемых к использованию в РК. Данные сорта
характеризуются высокой устойчивостью к заболеваниям, хорошими вкусовыми
качествами и способностью адаптироваться к различным климатическим
условиям регионов страны {[}4{]}. Характеристики казахстанских сортов
моркови, выведенных в Казахстане, представлены в таблице 1.

Морковь предпочитает рыхлые, хорошо дренированные почвы с нейтральной
реакцией рН и высоким содержанием органических веществ -- эти условия
способствует активному накоплению каратиноидов. Для получения
качественного урожая необходимо соблюдение агротехнических приемов:
регулярного полива, особенно в засушливые периоды, а также защиты от
вредителей, в том числе морковной мухи. Соблюдение правильной
агротехнической техники позволяет получать стабильные урожаи с высокими
пищевыми и потребительскими характеристиками.

{\bfseries Таблица 1. Казахстанские сорта моркови}

%% \begin{longtable}[]{@{}
%%   >{\raggedright\arraybackslash}p{(\linewidth - 6\tabcolsep) * \real{0.1614}}
%%   >{\raggedright\arraybackslash}p{(\linewidth - 6\tabcolsep) * \real{0.2795}}
%%   >{\raggedright\arraybackslash}p{(\linewidth - 6\tabcolsep) * \real{0.2943}}
%%   >{\raggedright\arraybackslash}p{(\linewidth - 6\tabcolsep) * \real{0.2648}}@{}}
%% \toprule\noalign{}
%% \begin{minipage}[b]{\linewidth}\centering
%% {\bfseries Сорт моркови}
%% \end{minipage} & \begin{minipage}[b]{\linewidth}\centering
%% {\bfseries Описание сорта}
%% \end{minipage} & \begin{minipage}[b]{\linewidth}\centering
%% {\bfseries Преимущества сорта}
%% \end{minipage} & \begin{minipage}[b]{\linewidth}\centering
%% {\bfseries Качественные показатели сорта}
%% \end{minipage} \\
%% \midrule\noalign{}
%% \endhead
%% \bottomrule\noalign{}
%% \endlastfoot
%% Сорт «Алау» & Это один из самых первых сортов моркови, выведенных
%% КазНИИКО. Он характеризуется ярко-оранжевым цветом корнеплодов и
%% отличными вкусовыми качествами. & В 2022 году в Кызылординской области
%% провели полевые испытания, где сорт
%% 
%% показал высокую урожайность и товарность корнеплодов. Отличается хорошей
%% лёжкостью и способностью сохранять свои вкусовые качества в течение
%% долгого времени.
%% 
%% Устойчивость к болезням и вредителям. & Сорт подходит для использования
%% в свежем виде и переработке, отличается высоким содержанием каротина и
%% витаминов. \\
%% Сорт " Ұшқын" & Сорт выведен методом индивидуального отбора из образца
%% иностранного происхождения с последующим направленным отбором на
%% продуктивность и качество.~ Окраска корнеплода, мякоти и сердцевины
%% оранжевая.~ & Сорт высокоурожайный, среднепоздний. Слабо поражается
%% болезнями.~ Лёжкость корнеплодов при зимнем хранении высокая. Длина
%% корнеплодов - 22-25 см, диаметр - 4,2-5,8 см. С 2015~года допущен к
%% использованию в Алматинской области. & Средняя масса плода 131,4 г.
%% Товарность 92,2\%. Дегустационная оценка 5 баллов. Среднее содержание
%% сухого вещества 12,9\%, сахара 6,3\%, каротиноидов 9,8 мг/\%. Сорт
%% характеризиуется высоким содержанием каротинов. Рекомендуется для
%% потребления в свежем виде, переработки и длительного хранения. \\
%% Сорт "Дербес" & Окраска поверхности и мякоти корнеплода оранжевая и
%% светло-оранжевая, сердцевина -- светло-оранжевая. & Сорт среднеспелый.
%% Болезнями и вредителями поражается не значительно.~Устойчив к мучнистой
%% росе, относительно устойчив к альтернариозу. Длина корнеплодов в среднем
%% 18 см. Рекомендуется для возделывания в Алматинской, Акмолинской
%% областях. & Средняя масса плода 154,3 г. Товарность 81,3\%. Содержание
%% сухого вещества 12,8\%, общего сахара 8\%, каротиноидов 10,6 мг/100 г.
%% Рекомендуется для использования в свежем виде в осенне-зимний и весений
%% периоды. \\
%% Сорт "Арнау-25" & Сорт среднеспелый, гибрид Казахстанской селекции от
%% КазНИИКО. Окраска корнеплода, мякоти и сердцевины оранжевая. & Устойчив
%% к болезням и климатическим условиям, устойчив к растрескиванию.
%% Корнеплод конической формы, длина 11-18 см. Рекомендуется для
%% возделывания в Алматинской, Жамбылской областях. & Средняя масса
%% корнеплода~145 г., товарность 93\%. Дегустационная оценка 4,7 баллов.
%% Содержание сухого вещества 12,7\%, витамина «С» 6,8 мг/\%, каротина 16,2
%% мг/\%.
%% 
%% Предназначен для потребления, переработки и длительного хранения. \\
%% \end{longtable}

Переработка моркови осуществляется как на крупных агропромышленных
комплексах, так и в фермерских хозяйствах. Одним из перспективных
направлений является использование морковного жмыха при производстве
хлебобулочных изделий и молочной продукции функционального значения
{[}5,6{]}.

Морковь отличается не только выраженными вкусовыми качествами, но и
богатым химическим составом, охватывающим все части растения от
корнеплода до семян. Она является источником витаминов (А, В1, В2, В6,
С, Е, РР), минеральных веществ и других биологически активных веществ,
соединений, оказывающих благоприятное влияние на здоровье человека.
Корнеплоды содержат калий (200-282 мг), кальций (35-50 мг), магний
(21мг), натрий (45 мг), фосфор (31-50 мг), марганец (40 мг), железо (0,7
мг) и йод (3,8 мг). По содержанию бора морковь занимает одно из ведущих
мест среди овощей. Характеристика химического состава сырой моркови
представлена в таблице 2.

При ежегодном потреблении свежей моркови в объеме 11 - 15,5 кг в год,
согласно рекомендациям Института питания РК, при общем потреблении
овощей 128 - 164 кг, организм получает необходимое количество
питательных веществ, способствующих поддержанию здоровья {[}7{]}.

{\bfseries Таблица 2 - Химический состав сырой моркови {[}8{]}}

%% \begin{longtable}[]{@{}
%%   >{\centering\arraybackslash}p{(\linewidth - 2\tabcolsep) * \real{0.5528}}
%%   >{\centering\arraybackslash}p{(\linewidth - 2\tabcolsep) * \real{0.4472}}@{}}
%% \toprule\noalign{}
%% \begin{minipage}[b]{\linewidth}\centering
%% \emph{Химический компонент}
%% \end{minipage} & \begin{minipage}[b]{\linewidth}\centering
%% \emph{Содержание}
%% \end{minipage} \\
%% \midrule\noalign{}
%% \endhead
%% \bottomrule\noalign{}
%% \endlastfoot
%% Влажность & 86-89 мг/100 г СМ \\
%% Зольность & 0.97-1.2 г/100 г СМ \\
%% Органические кислоты & 1.07-2.79 г/100 г СМ \\
%% Сахара & 2.73-11.24 г/100 г СМ \\
%% Кальций & 34-80 мг/100 г СМ \\
%% Фосфор & 25-53 мг/100 г СМ \\
%% Калий & 240 мг/100 г СМ \\
%% Магний & 9 мг/100 г СМ \\
%% Марганец & 0.2-0.8 мг/100 г СМ \\
%% Железо & 0.4-2.2 мг/100 г СМ \\
%% Натрий & 40 мг/100 г СМ \\
%% Витамин С (L-аскорбиновая кислота) & 1.05-5.3 мг/100 г СМ \\
%% Фенолы & 7.3-224 мг/100 г СМ \\
%% Тетратерпеноиды (каротиноиды, хлорофиллы) & 0.2-4.1 мг/100 г СМ \\
%% Фалькаринол* & 16-84 мг/ кг СМ \\
%% Фалькариндол* (С17 полиацетилены) & 8-27 мг/ кг СМ \\
%% Фалькариндол-3-ацетат* & 8-40 мг/ кг СМ \\
%% \end{longtable}

\emph{СМ-сырой массы. В зависимости от времени и условий хранения (через
30 дней значение может снизиться до 50\% от первоначального значения). *
Из культивируемой оранжевой моркови (D. carota ssp. sativus)}

Морковь содержит целый ряд ценнейших биологически активных веществ таких
как: каротиноиды, антоцианы, флавоноиды и фенольные соединения.
Основными пигментами, придающими корнеплодам оранжевую, желтую или
красную окраску, являются α- и β-каротины. Желтая окраска обусловлена
присутствием лютеинов, а фиолетовая -- высоким содержанием антоцианов.
Уровень каротиноидов зависит от сорта, условий выращивания, стадии
зрелости, условий хранения, так как свет и кислород могут разрушать
пигменты. Наибольшее количество β-каротина содержит темно-оранжевые
сорта моркови -- его доля может достигать 80-90\% от общего содержания
каротиноидов. Морковь содержит в среднем от 16 до 38 мг/100 г
каротиноидов {[}9{]}. Данные о содержании каротиноидов представлены в
таблице 3.

{\bfseries Таблица 3 - Содержание каротиноидов в моркови}

%% \begin{longtable}[]{@{}
%%   >{\centering\arraybackslash}p{(\linewidth - 4\tabcolsep) * \real{0.2340}}
%%   >{\centering\arraybackslash}p{(\linewidth - 4\tabcolsep) * \real{0.3595}}
%%   >{\raggedright\arraybackslash}p{(\linewidth - 4\tabcolsep) * \real{0.4064}}@{}}
%% \toprule\noalign{}
%% \begin{minipage}[b]{\linewidth}\centering
%% {\bfseries Каротиноид}
%% \end{minipage} & \begin{minipage}[b]{\linewidth}\centering
%% {\bfseries Содержание, (мг/100 г СМ)}
%% \end{minipage} & \begin{minipage}[b]{\linewidth}\centering
%% {\bfseries Функция/особенности}
%% \end{minipage} \\
%% \midrule\noalign{}
%% \endhead
%% \bottomrule\noalign{}
%% \endlastfoot
%% β-каротин & 6--12 & Главный предшественник витамина A (провитамин A) \\
%% α-каротин & 1.0--2.5 & Также является провитамином A \\
%% Лютеин & 0.3--1.0 & Не является провитамином A, но важен для зрения \\
%% Зеаксантин & 0.05--0.3 & Антиоксидант, играет роль в здоровье глаз \\
%% Фитофлуен, фитонен & 0.01--0.1 & Промежуточные соединения биосинтеза
%% каротиноидов \\
%% \end{longtable}

Богатое содержание в моркови флавоновых гликозидов таких как:
даукостерин, кемпферол, апигенин, кверцетин и лютеолин, а также
полифенолов, включая хлорогеновую, кофейную и п-гидроксибензойную
кислоты -- обуславливает ее выраженные антиоксидантные и противораковые
свойства. Это делает морковь перспективным сырьем для создания
эффективных фитопрепаратов и натуральных биологически активных добавок
{[}9{]}.

Семена моркови также представляют интерес благодаря содержанию активных
компонентов с биологическим действием. Так флавоновый гликозид
даукостерин, содержащийся в экстракте из семян моркови, может
способствовать расширению коронарных артерий и улучшению кровообращение
в сердце. Благодаря этим свойствам он применяется в медицине для
снижения артериального давления и улучшения сердечно-сосудистой функции.
Это делает даукостерин особенно полезным для людей с заболеваниями
сердечно-сосудистой системы, а также для тех, кто ограничивает
потребление продуктов, содержащих холестерин. Эффективность даукостерина
подтверждена клиническими исследованиями. Из семян моркови также
получают препарат «Даукарин», обладающим спазмолитическим действием
{[}10{]}.

Другие соединения, содержащиеся в моркови, такие как фалькаринол и
фалькариндиол, сесквитерпеноиды, кумарины и алкалоиды обладают
антиоксидантным, антиканцерогенным действием, а также стимулирует
иммунную систему {[}11{]}.

{\bfseries Цель:} провести аналитический обзор данных, опубликованных
современным международным научным сообществом, по вопросам изучения
полезных свойств моркови.

{\bfseries Результаты и обсуждение.}

{\bfseries 1. Антиоксидантная способность.} Согласно научным данным,
представленным в литературе, одним из важных свойств моркови является ее
антиоксидантная активность. Это связано с высоким содержанием в моркови
пищевых каротиноидов, полифенолов, полиацетиленов и витаминов,
обладающих антиоксидантными свойствами. Эти компоненты защищают клетки
организма от окислительного стресса, предотвращая повреждение ДНК и
других клеточных структур путем нейтрализации свободных радикалов
{[}12{]}.

В исследовании ученых \emph{Sameena Lone, Sumati Narayan и соавт}.
(2025) было проанализировано 297 линий моркови, из которых 52 сорта
использовались для оценки их терапевтического потенциала. Целью данного
исследования было выявление генетического разнообразия и биологически
активных соединений в этих сортах. Результаты показали наличие генотипов
с высокой антиоксидантной и противораковой активностью {[}13{]}, что
подтверждает их перспективность для использования при создании
нутрицевтиков и функциональных пищевых продуктов.

Морковь богата органическими кислотами, среди которых особенно
выделяется \emph{L}-аскорбиновая кислота (витамин С). Она играет
ключевую роль в поддержании здоровья -- помогает регулировать
артериальное давление, предотвращает дефицит железа и укрепляет иммунную
систему. Помимо неё, другие кислоты, такие как бензойная,
гидроксикоринная и галловая, обладают антибактериальными,
противовоспалительными и антимутагенными свойствами соответственно.
Вместе с тем, пантотеновая, фолиевая, уксусная, янтарная, лимонная,
молочная кислоты и их соли также способствуют лучшему усвоению железа
организмом {[}14{]}.

Антиоксидантными свойствами обладает аскорбиновая кислота. При этом
важно отметить, что \emph{L}-аскорбиновая кислота не синтезируется в
организме человека, и основным источником является пища -- в том числе
овощи, такие как морковь. Согласно исследованию, \emph{Alasalvar и
соавт}. (2001), содержание \emph{L}-аскорбиновой кислоты в сырой моркови
варьируется от 1,0 до 5,3 мг на 100 г сырого продукта. Наименьшее
количество зафиксировано в белой моркови (1,3 мг/100г), а наибольшее --
в оранжевой (5,3 мг/100г), что делает последнюю ценным источником этого
жизненно важного витамина {[}15{]}.

Исследования \emph{Zhang D. \& Hamauzu Y.} (2004) показали, что
флавоноиды и производные фенола, содержащиеся в моркови (Daucus carota
L.), обладают также выраженными антиоксидантной активностью. Авторы
пришли к выводу, что регулярное потребление моркови может способствовать
снижению воспалительных процессов в организме {[}16{]}.

Анализ работы \emph{Hartati R, Amalina M. N. \& Fidrianny I.} (2020)
продемонстрировал, что экстракты моркови, особенно из листьев и корней,
обладают выраженной и различной по степени антиоксидантной активностью,
что связано с высоким содержанием фенолов и флавоноидов {[}17{]}. Это
подчеркивает биологическую ценность не только корнеплода, но и его
надземных частей.

{\bfseries 2. Антиканцерогенное свойство.} Следующей важной особенностью
данной культуры является её противоопухолевое действие. В ряде
исследований были установлено, что некоторые соединения, содержащиеся в
моркови, способны подавлять рост клеток различных линий раковых опухолей
путём вмешательства в процессы клеточной пролиферации, апоптоза и
воспаления.

Например, доказано, что среди полиацетиленовых оксилипинов, присутствуют
такие вещества как фалькаринол (FaOH,
(3R,9Z)-1,9-гептадекадиен-4,6-диин-3-ол,)~и фалькариндиол (FaDOH,
(3R,8S,9Z)-1,9-гептадекадиен-4,6-диин-3,8-диол). Эти соединения (рис.2)
обладают как цитотоксической, так и противовоспалительной активностью,
что делает их перспективными в профилактике онкологических заболеваний,
особенно колоректального рака (КРР) {[}18{]}.

\fig{p/image5}{}

{\bfseries Рис.2 -- Химические структуры фалькаринола, фалькариндиола и
фалькариндиол 3-ацетата. Эти соединения частично отвечают за характерный
морковный аромат}

В 2022 году в журнале \emph{Frontiers in Nutrition} была опубликована
статья \emph{Zongze Jiang и соавт.}, в которой исследуется связь между
потреблением моркови и каротиноидов и риском заболеваемости и смертности
от КРР в рамках рандомизированного клинического исследования Prostate,
Lung, Colorectal, and Ovarian Cancer Screening Trial (PLCO). В данном
исследовании приняли участие 101~680 человек, за которыми наблюдали в
среднем 9,4 года. Это крупное когортное исследование показало снижение
риска развития КРР при регулярном употреблении моркови (7,6--15,3
г/день) на 21\%, что подтверждает её профилактический эффект {[}19{]}.
Влияние моркови и входящего в её состав фалькаринола на развитие
пренеопластических поражений толстой кишки у крыс, вызванных
азоксиметаном, рассматривали также \emph{Kobaek-Larsen M.} и его коллеги
(2005), где были получены результаты значительного снижения количества
опухолевого роста у крыс, которые получали диетическое лечение морковью
и фалькаринола {[}20{]}.

Ученые из Дании оценили связь между потреблением моркови и риском
развития КРР в когортном исследовании с участием 57~053 людей при
длительном периоде наблюдения. Участники, потреблявшие более 2- 4 сырых
моркови в неделю (свыше 32г/день), имели на 17\% более низкий риск
развития КРР по сравнению с теми, кто не употреблял сырую морковь, даже
после всесторонней корректировки статистической модели. В то же время
потребление менее 2 -- 4 кг морковей в неделю (\textless32г/день) не
продемонстрировали статистически значимого снижения риска. Эти данные
подтверждают защитный эффект моркови против КРР, ранее выявленный в
доклинических исследованиях на животных {[}18{]}.

Кроме того, согласно метаанализу, в ходе которого было идентифицировано
81 исследование (в итоговый анализ включено 5 работ), было установлено
статистически значимое снижение риска развития рака желудка на 26\% при
потреблении моркови (OR = 0.74; 95\% ДИ: 0,68 -- 0,81; Р
\textless0,0001) {[}21{]}.

Есть исследования, показывающие, что полиацетилены, содержащиеся в
моркови, обладают противолейкозной активностью. Согласно работе
\emph{Zaini R. и соавт.} (2011), экстракты морковного сока индуцируют
апоптоз и тормозят пролиферацию клеток миелоидного и лимфоидного
лейкоза. Это действие авторы связывают с β-каротином и фалькаринолом,
что делает морковь потенциальным источником биоактивных соединений для
терапии лейкемии {[}22{]}.

Употребление моркови также связано со снижением риска развития рака
легких. Мета-анализ \emph{Xu et al.} (2019), включающий 202~969 человек
и 5~517 пациентов с раком легких, показал, что регулярное потребление
моркови снижает риск развития этого заболевания на 42\% (OR = 0,58).
Более глубокий защитный эффект наблюдался при аденокарциноме и смешанных
формах рака лёгких. Ключевыми защитными веществами являются α- и
β-каротины, обладающие антиоксидантными свойствами, нейтрализующими
свободные радикалы и защищающими лёгочную ткань. Хотя в данной работе
влияние моркови на пациентов с уже диагностированным раком напрямую не
рассматривалось, авторы отмечают, что каротиноиды могут тормозить рост
опухолевых клеток, снижать их инвазивность и миграцию, а также
поддерживать антиоксидантный баланс организма. Однако это требует
дополнительного клинического подтверждения {[}23{]}.

Таким образом, включение сырой моркови в рацион питания может
рассматриваться как эффективная стратегия профилактики некоторых видов
рака, особенно колоректального, благодаря содержанию в ней биоактивных
каротиноидов и полиацетиленов.

{\bfseries 3. Витамины и минералы.} Морковь отличается также содержанием
витаминов и минералов, в том числе водорастворимым витамином С и
жирорастворимыми формами витамина Е (α- и γ-токоферолы, токотриенолы).
Витамин Е, обладающий антиоксидантным действием, защищает клеточные
мембраны от окислительного стресса, фотостарения и ряда заболеваний,
включая рак и атеросклероз. α-Токоферол как основной антиоксидант
эпидермиса кожи человека особенно важен для защиты её от внешних
факторов.

Кроме высокого содержания β-каротина, предотвращающего нарушения зрения,
включая дегенерацию жёлтого пятна и катаракту, и других каротиноидов с
высокой антиоксидантной активностью, морковь является источником
витаминов А, В1, В2, В6, В9, К, РР, биотин, которые участвуют в
метаболизме и поддержании роста детей {[}8{]}.

Витамин С способствует усвоению негемового железа и имеет важное
значение для борьбы с инфекциями, а витамин К (филлохинон) влияет на
гемостаз. В1 (тиамин) благоприятно воздействует на нервную систему и
психическое состояние; В2 (рибофлавин) необходим для клеточного дыхания
и образования красных кровяных телец; В6 (пиридоксин) снижает риск
сердечных заболеваний {[}8{]}.

Морковь богата на пищевые волокна и микроэлемент молибден, который
относится к редким, встречающимся в овощах. Молибден важен в усвоении
железа и метаболизме жиров и углеводов. В моркови имеется достаточное
количество магния и марганца. Магний необходим для формирования костной
ткани, синтеза белков, активации витаминов группы В и для расслабления
мышц, свертывания крови, участвует в процессе секреции инсулина.
Марганец способствует метаболизму глюкозы, взаимодействуя с ферментами в
организме {[}12{]}.

{\bfseries 4. Каротиноиды.} Каротиноиды -- это группа изопреноидных
молекул, присутствующих во всех фотосинтезирующих растениях. Каротиноиды
можно найти в зеленых листьях, пыльце цветковых растений, лепестках
цветов, корнях, зернах и плодах растений, водорослях.

В моркови присутствуют два типа каротиноидов -- каротины и ксантофилы.
Основными каротиноидами (рис.3) в корнеплодах моркови являются
β-каротин (75\%); α-каротин (23\%), лютеин (1,9\%), а также
β-криптоксантин, ликопин и зеаксантин {[}24{]}.

\fig{p/image6}{}

{\bfseries Рис.3 -- Химическая структура каротиноидов, наиболее часто
встречающихся в свежей моркови}

Морковь является основным растительным источником провитамина А и
накапливает высокие уровни β- и α-каротина. При общем содержании
каротиноидов 268,64 мг/100г сырой массы, количество β-каротина
составляет 156,91, α-каротина -- 108,53 мг/100г или 58,4 и 40,4\%
соответственно. В зависимости от окраски моркови содержание каротинов
изменяется и может составлять, мг/кг: желтая -- 2,6; темно-оранжевая --
160; красная -- 73; фиолетово-желтая -- 92; фиолетовооранжевая -- 40.
Желтые и красные сорта моркови также богаты лютеином и ликопином
{[}25{]}.

В работе казахстанских ученных по изучение районированных сортов
плодоовощной продукции для разработки технологий получения
биоэкологических продуктов с функциональными свойствами были исследованы
два сорта моркови на общее содержание каротина (мг/кг) и витамина С
(мг\%), которые составили: для сорта моркови ``Алау'' 9,8±0,2 и 5,66±0,1
и сорта ``Шантанэ'' 9,6±0,2 и 5,86±0,1 соответственно {[}26{]}.

Животные и человек не могут синтезировать каротиноиды, что делает
каротиноиды незаменимыми компонентами в питании. В различных странах
среднее потребление β-каротина варьируется в пределах 1,8 -- 5,0 мг в
сутки, а рекомендуемые суточные дозы составляют 5-6 мг. При содержании
β-каротина в 100 граммах моркови от 4 до 20 мг употребление этой порции
моркови в день покрывает полную физиологическую потребность {[}25{]}.
Для максимального усвоения β-каротина и его превращения в витамин А,
морковь следует употреблять в сыром виде с добавлением небольшого
количества жиров -- достаточно одной столовой ложки растительного масла
или сметаны на 100 г продукта.

В 2009 году ученые \emph{Brazionis L.} \emph{и соавт}. представлили
работу о потенциальной пользе моркови при лечении диабета. У пациентов с
диабетом 2 типа, у которых наблюдались более низкие уровни каротиноидов,
таких как ликопен, лютеин и заексантин, имелись более высокие уровни
ретинопатии {[}27{]}. В другой работе показано, что морковный сок,
ферментированный Lactobacillus rhamnosus GG (LGG) и обогащенный фенолами
и короткоцепочечными жирными кислотами, может регулировать уровень
глюкозы в крови и инсулина у крыс с сахарным диабетом, изменяя состав
кишечной микробиоты {[}28{]}.

Полезные свойства моркови при употреблении у детей приносят такую же
пользу организму, как и у взрослых. Каротиноиды моркови играют решающую
роль в здоровье детей, и исследования демонстрируют различные
взаимосвязи. Высокие концентрации α- и β-каротинов обратно коррелируют с
показателями ожирения и уровня триглицеридов у детей, подчеркивая их
защитные эффекты в отношении признаков кардиометаболического риска
{[}29{]}.

По данным исследования \emph{Rosok L. в соавт}. (2022), высокий
каротиноидный статус связан с высоким уровнем когнитивных функций и
широкими математическими навыками у детей школьного возраста {[}30{]}.
Приём добавок с витамином А играет решающую роль в снижении уровня
заболеваемости и смертности среди детей в возрасте от 6 месяцев до 5 лет
{[}31{]}. Исследования \emph{Goodwin S. и соавт}. (2023) показывают, что
ранее введение богатой флавоноидами диеты может потенциально снизить
риск сердечно-сосудистых заболеваний в будущей жизни. Иммуномодулирующее
действие данных биоактивных веществ влияет и на формирование, и на
укрепление иммунной системы у детей {[}32{]}. В качестве источника таких
соединений в детском меню чаще всего предлагается морковь.

Для детей варёная морковь предпочтительнее сырой, так как в ней меньше
грубых растительных волокон. При варке моркови на пару витамины в ней
сохраняются лучше, особенно если её варить в кожуре, а содержание
β-каротина увеличивается от 25\% (при паровой обработке) и до 79\% (при
запекании) {[}33{]}. Отварная морковь также содержит меньше клетчатки и
пектина, что способствует ее лучшему перевариванию. В то же время
установлено, что термическая обработка моркови, особенно обжаривание,
снижает содержание каротиноидов и, соответственно, ее общую
антиоксидантную способность {[}34{]}.

Физиологическая роль каротиноидов весьма разнообразна, однако
современные исследования показывают, что её понимание всё ещё не
завершено. Каждый год появляются новые данные о влиянии как хорошо
изученного β-каротина, так и других представителей этого класса веществ.

Анализ научной литературы свидетельствует о высокой пищевой и
биологической ценности моркови, также о её значительном положительном
влиянии на здоровье и профилактику различных заболеваний.

В НИИ профилактической медицины имени Академика Е.Д.Даленова НАО
«Медицинский университет Астана» в целях создания пищевых натуральных
добавок и функциональных продуктов из моркови начались работы по
изучению химического состава, определению уровня содержания β-каротина и
уровня антиоксидантной активности в казахстанских сортах столовой
моркови Алау, Дербес, Ұшқын, Арнау, а также работы по влиянию различных
методов термической обработки на уровень витаминов и β-каротина в
моркови. Эти исследования в дальнейшем важны для разработки эффективных
и питательно оптимизированных кулинарных подходов, особенно в контексте
улучшения питания населения и профилактике различных заболеваний.

{\bfseries Выводы.} Научные работы, проанализированные в обзоре
подтверждают, что морковь, благодаря высокому содержанию каротиноидов,
антоцианов, флавоноидов, фенолов и других биологически активных
соединений, обладает выраженными антиоксидантными, иммуномодулирующими,
антимутагенными, антиинфекционными и антиканцерогенными свойствами. Эти
качества делают морковь ценным источником сырья для создания здоровых
продуктов питания и натуральных добавок. Особое внимание следует уделить
включению моркови в любом виде в рацион питания детей, так как
регулярное её потребление поможет снизить риск заболеваний в их
дальнейшей жизни и поспособствует укреплению здоровья на протяжении всей
жизни.

{\bfseries Литература}

1. Simon P.W. Domestication, Historical Development and Modern Breeding
of Carrot.//Plant Breeding Reviews. - 2000.- Vol.19. - Р.157-190. DOI
10.1002/9780470650172.ch5.

2. Аналитическая статья «Рынок моркови Казахстана - некоторые
тенденции». {[}Электронный ресурс{]}.
URL:\href{https://ab-centre.ru/news/rynok-morkovi-kazahstana---nekotorye-tendencii/}{https://ab-center.ru/news/rynok-morkovi-kazahstana-\/-\/-nekotorye-tendencii/}-
Дата обращения: 01.04.2025.

3. Аналитическая статья {[}Электронный ресурс{]} URL:
\url{https://east-fruit.com/novosti/na-nachalo-yanvarya-2022-goda-zapasy-morkovi-v-kazakhstane-prevyshali-200-tys-tonn/}.-
Дата обращения: 01.04.2025.

4. Информация о казахстанских сортах моркови. {[}Электронный ресурс{]}
URL:
\url{https://baibolsyn.kz/ru/semena/?search=\%D0\%BC\%D0\%BE\%D1\%80\%D0\%BA\%D0\%BE\%D0\%B2\%D1\%8C&page=1/.-}
Дата обращения: 01.04.2025.

5. Абай~Г.К., Жонысова~М.У., Тултабаева~Т.Ч. Исследование витаминного
состава добавки из моркови с целью обогащения молочной продукции
функционального назначения//Вестник Алматинского технологического
университета. -2019. -№ 3. - С.52-56.

6. Нечушкина А. Д., Альшевская М. Н. Обоснование возможности
использования жмыха моркови и рисовой муки в технологии мучных
кондитерских изделий типа «крекеры»// Научный журнал «Вестник молодежной
науки».- 2021.-Vol.3(30). DOI10.46845/2541-8254-2021-3(30)-14-14.

7. Косанов С. У. Сулейменова Ж. Ж. Влияние сроков посева на
продуктивность моркови сорта «Алау» в условиях Кызылординской
области.//Молодой ученый.- 2017.- № 20(154).- С.14-16.

8. Mandrich L, Esposito A.V, Costa S, Caputo E. Chemical Composition,
Functional and Anticancer Properties of Carrot// Molecules.-2023.-
Vol.28(20):7161.
DOI~\href{https://doi.org/10.3390/molecules28207161}{10.3390/molecules28207161}.

9. Ahmad T., Cawood M., Iqbal Q., Ariño A., Batool A., Tariq R.M.S.,
Azam M., Akhtar S. Phytochemicals in~Daucus carota~and Their Health
Benefits -Review Article.// Foods. -2019.~-- Vol.8(9):424. DOI
\href{https://doi.org/10.3390/foods8090424}{10.3390/foods8090424}.

10. Киселева Т.Л. и др. Лечебные свойства некоторых огородных растений
семейства сельдерейные//Традиционная медицина.- 2009.- № 3. - С.30-36.

11. Fu HW, Zhang L, Yi T, Feng YL, Tian JK. Two new guaiane-type
sesquiterpenoids from the fruits of Daucus carota L// Fitoterapia. -
2010. -Vol.81(5): 443-6.
DOI~\href{https://doi.org/10.1016/j.fitote.2009.12.008}{10.1016/j.fitote.2009.12.008}.

12. Khyati Varshney , Kirti Mishra. An Analysis of Health Benefits of
Carrot //International Journal of Innovative Research in Engineering and
Management (IJIREM).- 2022.- Vol.9(1).-Р.211-214. DOI
\href{http://dx.doi.org/10.55524/ijirem.2022.9.1.40}{10.55524/ijirem.2022.9.1.40}

13. Lone S., Narayan. S, Hussain K., Malik M., Yadav SK., Khan FA, Safa
A., Ahmad A., Masoodi KZ. Investigating the antioxidant and anticancer
potential of Daucus spp. extracts against human prostate cancer cell
line C4-2, and lung cancer cell line A549//Journal of
Ethnopharmacology.- 2025-337(Pt 2):118855.
DOI~\href{https://doi.org/10.1016/j.jep.2024.118855}{10.1016/j.jep.2024.118855}.

14. Yusuf E.,Tkacz K., Turkiewicz I.P., Wojdyło A., Nowicka P. Analysis
of chemical compounds' content in different varieties of carrots,
including qualification and quantification of sugars, organic acids,
minerals, and bioactive compounds by UPLC.// Eur. Food Res.Technol.~-
2021. -Vol.247.- Р.3053-3062. DOI
\href{https://link.springer.com/article/10.1007\%2Fs00217-021-03857-0}{10.1007/s00217-021-03857-0}.

15. Alasalvar C., Grigor JM., Zhang D., Quantick PC., Shahidi F.
Comparison of volatiles, phenolics, sugars, antioxidant vitamins, and
sensory quality of different colored carrot varieties. //Agric Food
Chem.- 2001. - Vol.49(3): 1410-6.
DOI~\href{https://doi.org/10.1021/jf000595h}{10.1021/jf000595h}.

16. Zhang. D., Hamauzu Y. Phenolic Compounds and Their Antioxidant
Properties in Different Tissues of Carrots (Daucus carota L.)//Journal
of Food, Agriculture and Environment (JFAE). -- 2004.- Vol.2(1). - Р.
95-100. DOI~ \href{javascript:void()}{10.1234/4.2004.102}.

17. Hartati R., Amalina M. N., Fidrianny I. Antioxidant activities of
roots, leaves, and stems of carrot (Daucus carota L.) using DPPH and
FRAP methods//International Journal of Research in Pharmaceutical
Sciences.-2020.-Vol.11(SPL4).- Р.2856-2863. DOI
\href{http://dx.doi.org/10.26452/ijrps.v11iSPL4.4570}{10.26452/ijrps.v11iSPL4.4570}.

18. Ulrik Deding, Gunnar Baatrup, Lars Porskjær Christensen, and Morten
Kobaek-Larsen. Carrot Intake and Risk of Colorectal Cancer: A
Prospective Cohort Study of 57,053 Danes.// Nutrients.~- 2020.-
Vol.12(2): 332.
DOI~\href{https://doi.org/10.3390/nu12020332}{10.3390/nu12020332}.

19. Zongze Jiang, Huilin Chen, Ming Li, Wei Wang, Chuanwen Fan, Feiwu
Long. Association of Dietary Carrot/Carotene Intakes With Colorectal
Cancer Incidence and Mortality in the Prostate, Lung, Colorectal, and
Ovarian Cancer Screening Trial.//Frontiers in Nutrition.- 2022. Vol.9.
DOI~\href{https://doi.org/10.3389/fnut.2022.888898}{10.3389/fnut.2022.888898}.

20. Kobaek-Larsen M., Christensen L.P, Vach W., Ritskes-Hoitinga J.,
Brandt K. Inhibitory effects of feeding with carrots or (-)-falcarinol
on development of azoxymethane-induced preneoplastic lesions in the rat
colon // J Agric Food Chem.- 2005.-Vol.53(5):1823-7. DOI
\href{https://doi.org/10.1021/jf048519s}{10.1021/jf048519s}

21. Hossein Fallahzadeh, Ali Jalali, Mahdieh Momayyezi, Soheila Bazm.
Effect of Carrot Intake in the Prevention of Gastric Cancer: A
Meta-Analysis.// Journal of Gastric Cancer. - 2015.-
Vol.15(4).-P.256-261. DOI 10.5230/jgc.2015.15.4.256.

22. Rana Zaini, Malcolm R. Clench, and Christine L. Le Maitre. Bioactive
chemicals from carrot (Daucus carota)~juice extracts for the treatment
of leukemia// Journal of Medicinal Food.-2011.-Vol.14(11).- Р.
1303-1312. DOI 10.1089/jmf.2010.0284

23. Hongbin Xu, Heng Jiang, Wei Yang, Fujian Song, Shijiao Yan.,
\emph{et al}. Is carrot consumption associated with a decreased risk of
lung cancer? A meta-analysis of observational studies// British Journal
of Nutrition.- 2019.-Vol.122(5).-P.488-498. DOI
\href{https://doi.org/10.1017/s0007114519001107}{10.1017/S0007114519001107}.

24. Søltoft M., Bysted A., Madsen K.H., Mark A.B., Bügel S.G., Nielsen
J., Knuthsen P. Effects of organic and conventional growth systems on
the content of carotenoids in carrot roots, and on intake and plasma
status of carotenoids in humans//J. Sci. Food Agric.-2011.-91(4).-
Р.767-775.
DOI~\href{https://doi.org/10.1002/jsfa.4248}{10.1002/jsfa.4248}.

25. Кощаев И.А., Рядинская А.А., Чуев С.А. и др. Технологии производства
и переработки моркови: монография// Екатеринбург: Ridero, 2022. - 236 c.
ISBN~978-5-0059-1675-4.

26. Велямов М.Т., Оспанов А.Б., Попова Н.В. и др. Изучение
районированных сортов плодоовощной продукции для разработки технологий
получения биоэкологических продуктов с функциональными
свойствами//Вестник ЮУрГУ. Серия «Пищевые и биотехнологии». -- 2022. -
Т.10(1). - С.30-38. DOI 10.14529/food220104.

27. Brazionis L., Rowley K., Itsiopoulos C., O'Dea K. Plasma Carotenoids
and Diabetic Retinopathy // The British Journal of Nutrition.- 2009.
-101(2). - Р.270-277.
DOI~\href{https://doi.org/10.1017/s0007114508006545}{10.1017/S0007114508006545}.

28. Rongkang Hu., Feng Zeng, Linxiu Wu, Xuzhi Wan, Yongfang Chen,
Jiachao Zhang., Bin Liu Fermented carrot juice attenuates type 2
diabetes by mediating gut microbiota in rats.//Food \& Function. -2019.-
Vol.10.- P.2935-2946.
DOI~\href{https://doi.org/10.1039/c9fo00475k}{10.1039/c9fo00475k}.

29. Srinivas Mummidi, Vidya S., Farook, Lavanya Reddivari, Joselín
Hernández-Ruiz, Alvaro Diaz-Badillo, et al. Serum carotenoids and
Pediatric Metabolic Index predict insulin sensitivity in Mexican
American children.// Scientific Reports. -2021.- Vol.11(1):871.
DOI~\href{https://doi.org/10.1038/s41598-020-79387-8}{10.1038/s41598-020-79387-8}.

30. Rosok L.M, Cannavale C.N, Keye S.A, Holscher H.D, Renzi-Hammond L.,
Khan N.A. Skin and macular carotenoids and relations to academic
achievement among school-aged children.// Nutr Neurosci. -- 2025.-28(3).
- Р.308-320.
DOI~\href{https://doi.org/10.1080/1028415x.2024.2370175}{10.1080/1028415X.2024.2370175}.

31. Imdad A., Mayo-Wilson E., Haykal M.R., Regan A., Sidhu J., Smith A.,
Bhutta Z.A. Vitamin A. supplementation for preventing morbidity and
mortality in children from six months to five years of age.//Cochrane
Database Syst Rev. - 2022.-Vol.3(3): CD008524.

DOI~\href{https://doi.org/10.1002/14651858.cd008524.pub4}{10.1002/14651858.CD008524.pub4}.

32. Goodwin S., Lyons-Wall P., Lo J., et al. Flavonoid provision to
2--3-year-old children in 30 long day care centres across metropolitan
Perth, Western Australia//Proceedings of the Nutrition Society. -2023.
82(OCE2):E146. DOI
\href{http://dx.doi.org/10.1017/S0029665123001556}{10.1017/S0029665123001556}.

33. Agnieszka Narwojsz, Tomasz Sawicki, Beata Piłat and Małgorzata
Tańska. Effect of Heat Treatment Methods on Color, Bioactive Compound
Content, and Antioxidant Capacity of Carrot Root. // Appl. Sci.- 2025.-
Vol.15(1). -
P.254.~DOI\href{http://dx.doi.org/10.3390/app15010254}{10.3390/app15010254}.

34. Amy Schmiedeskamp, Monika Schreiner, Susanne Baldermann. Impact of
Cultivar Selection and Thermal Processing by Air Drying, Air Frying, and
Deep Frying on the Carotenoid Content and Stability and Antioxidant
Capacity in Carrots (Daucus carota~L.)// Journal of Agricultural and
Food Chemistry. - 2022. - Vol.70(5). - Р.1629-1639.
DOI~\href{https://doi.org/10.1021/acs.jafc.1c05718}{10.1021/acs.jafc.1c05718}.

{\bfseries References}

1. Simon P.W. Domestication, Historical Development and Modern Breeding
of Carrot.//Plant Breeding Reviews. - 2000.- Vol.19. - Р.157-190. DOI
10.1002/9780470650172.ch5.

2. Analiticheskaja stat' ja «Rynok morkovi Kazahstana -
nekotorye tendencii». {[}Jelektronnyj resurs{]}.
URL:https://ab-center.ru/news/rynok-morkovi-kazahstana-\/-\/-nekotorye-tendencii/-
Data obrashhenija: 01.04.2025.{[}in Russian{]}.

3. Analiticheskaja stat' ja {[}Jelektronnyj resurs{]}
URL:
https://east-fruit.com/novosti/na-nachalo-yanvarya-2022-goda-zapasy-morkovi-v-kazakhstane-prevyshali-200-tys-tonn/.-
Data obrashhenija: 01.04.2025. {[}in Russian{]}.

4. Informacija o kazahstanskih sortah morkovi. {[}Jelektronnyj resurs{]}
URL:
https://baibolsyn.kz/ru/semena/?search=\%D0\%BC\%D0\%BE\%D1\%80\%D0\%BA\%D0\%BE\%D0\%B2\%D1\%8C\&page=1/.-
Data obrashhenija: 01.04.2025. {[}in Russian{]}.

5. Abaj G.K., Zhonysova M.U., Tultabaeva T.Ch. Issledovanie vitaminnogo
sostava dobavki iz morkovi s cel' ju obogashhenija
molochnoj produkcii funkcional' nogo
naznachenija//Vestnik Almatinskogo tehnologicheskogo universiteta.
-2019. -№ 3. - S.52-56. {[}in Russian{]}.

6. Nechushkina A. D., Al' shevskaja M. N. Obosnovanie
vozmozhnosti ispol' zovanija zhmyha morkovi i risovoj
muki v tehnologii muchnyh konditerskih izdelij tipa «krekery»// Nauchnyj
zhurnal «Vestnik molodezhnoj nauki».- 2021.-Vol.3(30).
DOI10.46845/2541-8254-2021-3(30)-14-14. {[}in Russian{]}.

7. Kosanov S. U. Sulejmenova Zh. Zh. Vlijanie srokov poseva na
produktivnost'{} morkovi sorta «Alau» v uslovijah
Kyzylordinskoj oblasti.//Molodoj uchenyj.- 2017.- № 20(154).- S.14-16.
{[}in Russian{]}.

8. Mandrich L, Esposito A.V, Costa S, Caputo E. Chemical Composition,
Functional and Anticancer Properties of Carrot// Molecules.-2023.-
Vol.28(20):7161.
DOI~\href{https://doi.org/10.3390/molecules28207161}{10.3390/molecules28207161}.

9. Ahmad T., Cawood M., Iqbal Q., Ariño A., Batool A., Tariq R.M.S.,
Azam M., Akhtar S. Phytochemicals in~Daucus carota~and Their Health
Benefits -Review Article.// Foods. -2019.~-- Vol.8(9):424. DOI
\href{https://doi.org/10.3390/foods8090424}{10.3390/foods8090424}.

10. Kiseleva T.L. i dr. Lechebnye svojstva nekotoryh ogorodnyh rastenij
semejstva sel' derejnye//Tradicionnaja medicina.- 2009.-
№ 3. - S.30-36. {[}in Russian{]}.

11. Fu HW, Zhang L, Yi T, Feng YL, Tian JK. Two new guaiane-type
sesquiterpenoids from the fruits of Daucus carota L// Fitoterapia. -
2010. -Vol.81(5): 443-6.
DOI~\href{https://doi.org/10.1016/j.fitote.2009.12.008}{10.1016/j.fitote.2009.12.008}.

12. Khyati Varshney , Kirti Mishra. An Analysis of Health Benefits of
Carrot //International Journal of Innovative Research in Engineering and
Management (IJIREM).- 2022.- Vol.9(1).-Р.211-214. DOI
\href{http://dx.doi.org/10.55524/ijirem.2022.9.1.40}{10.55524/ijirem.2022.9.1.40}

13. Lone S., Narayan. S, Hussain K., Malik M., Yadav SK., Khan FA, Safa
A., Ahmad A., Masoodi KZ. Investigating the antioxidant and anticancer
potential of Daucus spp. extracts against human prostate cancer cell
line C4-2, and lung cancer cell line A549//Journal of
Ethnopharmacology.- 2025-337(Pt 2):118855.
DOI~\href{https://doi.org/10.1016/j.jep.2024.118855}{10.1016/j.jep.2024.118855}.

14. Yusuf E.,Tkacz K., Turkiewicz I.P., Wojdyło A., Nowicka P. Analysis
of chemical compounds' content in different varieties of carrots,
including qualification and quantification of sugars, organic acids,
minerals, and bioactive compounds by UPLC.// Eur. Food Res.Technol.~-
2021. -Vol.247.- Р.3053-3062. DOI
\href{https://link.springer.com/article/10.1007\%2Fs00217-021-03857-0}{10.1007/s00217-021-03857-0}.

15. Alasalvar C., Grigor JM., Zhang D., Quantick PC., Shahidi F.
Comparison of volatiles, phenolics, sugars, antioxidant vitamins, and
sensory quality of different colored carrot varieties. //Agric Food
Chem.- 2001. - Vol.49(3): 1410-6.
DOI~\href{https://doi.org/10.1021/jf000595h}{10.1021/jf000595h}.

16. Zhang. D., Hamauzu Y. Phenolic Compounds and Their Antioxidant
Properties in Different Tissues of Carrots (Daucus carota L.)//Journal
of Food, Agriculture and Environment (JFAE). -- 2004.- Vol.2(1). - Р.
95-100. DOI~ \href{javascript:void()}{10.1234/4.2004.102}.

17. Hartati R., Amalina M. N., Fidrianny I. Antioxidant activities of
roots, leaves, and stems of carrot (Daucus carota L.) using DPPH and
FRAP methods//International Journal of Research in Pharmaceutical
Sciences.-2020.-Vol.11(SPL4).- Р.2856-2863. DOI
\href{http://dx.doi.org/10.26452/ijrps.v11iSPL4.4570}{10.26452/ijrps.v11iSPL4.4570}.

18. Ulrik Deding, Gunnar Baatrup, Lars Porskjær Christensen, and Morten
Kobaek-Larsen. Carrot Intake and Risk of Colorectal Cancer: A
Prospective Cohort Study of 57,053 Danes.// Nutrients.~- 2020.-
Vol.12(2): 332.
DOI~\href{https://doi.org/10.3390/nu12020332}{10.3390/nu12020332}.

19. Zongze Jiang, Huilin Chen, Ming Li, Wei Wang, Chuanwen Fan, Feiwu
Long. Association of Dietary Carrot/Carotene Intakes With Colorectal
Cancer Incidence and Mortality in the Prostate, Lung, Colorectal, and
Ovarian Cancer Screening Trial.//Frontiers in Nutrition.- 2022. Vol.9.
DOI~\href{https://doi.org/10.3389/fnut.2022.888898}{10.3389/fnut.2022.888898}.

20. Kobaek-Larsen M., Christensen L.P, Vach W., Ritskes-Hoitinga J.,
Brandt K. Inhibitory effects of feeding with carrots or (-)-falcarinol
on development of azoxymethane-induced preneoplastic lesions in the rat
colon // J Agric Food Chem.- 2005.-Vol.53(5):1823-7. DOI
\href{https://doi.org/10.1021/jf048519s}{10.1021/jf048519s}

21. Hossein Fallahzadeh, Ali Jalali, Mahdieh Momayyezi, Soheila Bazm.
Effect of Carrot Intake in the Prevention of Gastric Cancer: A
Meta-Analysis.// Journal of Gastric Cancer. - 2015.-
Vol.15(4).-P.256-261. DOI 10.5230/jgc.2015.15.4.256.

22. Rana Zaini, Malcolm R. Clench, and Christine L. Le Maitre. Bioactive
chemicals from carrot (Daucus carota)~juice extracts for the treatment
of leukemia// Journal of Medicinal Food.-2011.-Vol.14(11).- Р.
1303-1312. DOI 10.1089/jmf.2010.0284

23. Hongbin Xu, Heng Jiang, Wei Yang, Fujian Song, Shijiao Yan.,
\emph{et al}. Is carrot consumption associated with a decreased risk of
lung cancer? A meta-analysis of observational studies// British Journal
of Nutrition.- 2019.-Vol.122(5).-P.488-498. DOI
\href{https://doi.org/10.1017/s0007114519001107}{10.1017/S0007114519001107}.

24. Søltoft M., Bysted A., Madsen K.H., Mark A.B., Bügel S.G., Nielsen
J., Knuthsen P. Effects of organic and conventional growth systems on
the content of carotenoids in carrot roots, and on intake and plasma
status of carotenoids in humans//J. Sci. Food Agric.-2011.-91(4).-
Р.767-775.
DOI~\href{https://doi.org/10.1002/jsfa.4248}{10.1002/jsfa.4248}.

25. Koshhaev I.A., Rjadinskaja A.A., Chuev S.A. i dr. Tehnologii
proizvodstva i pererabotki morkovi: monografija// Ekaterinburg: Ridero,
2022. - 236 c. ISBN 978-5-0059-1675-4. {[}in Russian{]}.

26. Veljamov M.T., Ospanov A.B., Popova N.V. i dr. Izuchenie
rajonirovannyh sortov plodoovoshhnoj produkcii dlja razrabotki
tehnologij poluchenija biojekologicheskih produktov s
funkcional' nymi svojstvami//Vestnik JuUrGU. Serija
«Pishhevye i biotehnologii». -- 2022. - T.10(1). - S.30-38. DOI
10.14529/food220104. {[}in Russian{]}.

27. Brazionis L., Rowley K., Itsiopoulos C., O'Dea K. Plasma Carotenoids
and Diabetic Retinopathy // The British Journal of Nutrition.- 2009.
-101(2). - Р.270-277.
DOI~\href{https://doi.org/10.1017/s0007114508006545}{10.1017/S0007114508006545}.

28. Rongkang Hu., Feng Zeng, Linxiu Wu, Xuzhi Wan, Yongfang Chen,
Jiachao Zhang., Bin Liu Fermented carrot juice attenuates type 2
diabetes by mediating gut microbiota in rats.//Food \& Function. -2019.-
Vol.10.- P.2935-2946.
DOI~\href{https://doi.org/10.1039/c9fo00475k}{10.1039/c9fo00475k}.

29. Srinivas Mummidi, Vidya S., Farook, Lavanya Reddivari, Joselín
Hernández-Ruiz, Alvaro Diaz-Badillo, et al. Serum carotenoids and
Pediatric Metabolic Index predict insulin sensitivity in Mexican
American children.// Scientific Reports. -2021.- Vol.11(1):871.
DOI~\href{https://doi.org/10.1038/s41598-020-79387-8}{10.1038/s41598-020-79387-8}.

30. Rosok L.M, Cannavale C.N, Keye S.A, Holscher H.D, Renzi-Hammond L.,
Khan N.A. Skin and macular carotenoids and relations to academic
achievement among school-aged children.// Nutr Neurosci. -- 2025.-28(3).
- Р.308-320.
DOI~\href{https://doi.org/10.1080/1028415x.2024.2370175}{10.1080/1028415X.2024.2370175}.

31. Imdad A., Mayo-Wilson E., Haykal M.R., Regan A., Sidhu J., Smith A.,
Bhutta Z.A. Vitamin A. supplementation for preventing morbidity and
mortality in children from six months to five years of age.//Cochrane
Database Syst Rev. - 2022.-Vol.3(3): CD008524.

DOI~\href{https://doi.org/10.1002/14651858.cd008524.pub4}{10.1002/14651858.CD008524.pub4}.

32. Goodwin S., Lyons-Wall P., Lo J., et al. Flavonoid provision to
2--3-year-old children in 30 long day care centres across metropolitan
Perth, Western Australia//Proceedings of the Nutrition Society. -2023.
82(OCE2):E146. DOI
\href{http://dx.doi.org/10.1017/S0029665123001556}{10.1017/S0029665123001556}.

33. Agnieszka Narwojsz, Tomasz Sawicki, Beata Piłat and Małgorzata
Tańska. Effect of Heat Treatment Methods on Color, Bioactive Compound
Content, and Antioxidant Capacity of Carrot Root. // Appl. Sci.- 2025.-
Vol.15(1). -
P.254.~DOI\href{http://dx.doi.org/10.3390/app15010254}{10.3390/app15010254}.

34. Amy Schmiedeskamp, Monika Schreiner, Susanne Baldermann. Impact of
Cultivar Selection and Thermal Processing by Air Drying, Air Frying, and
Deep Frying on the Carotenoid Content and Stability and Antioxidant
Capacity in Carrots (Daucus carota~L.)// Journal of Agricultural and
Food Chemistry. - 2022. - Vol.70(5). - Р.1629-1639.
DOI~\href{https://doi.org/10.1021/acs.jafc.1c05718}{10.1021/acs.jafc.1c05718}.

\emph{{\bfseries Сведения об авторах}}

Абдулдаева А. А.- к.м.н., профессор, директор НИИ профилактической
медицины имени академика Е.Д.Даленова, НАО «Медицинский университет
Астана», Астана, Казахстан, e-mail:
abduldayeva.a@amu.kz;

Тарджибаева С. К. - к.м.н., доцент, ведущий научный сотрудник
лаборатории диагностики здоровья НИИ профилактической медицины им.
академика Е.Д.Даленова, НАО «Медицинский университет Астана», Астана,
Казахстан,
e-mail:\href{mailto:tardzhibaeva.s@amu.kz;\%20}{tardzhibaeva.s@amu.kz;}

Досжанова Г. Н.- PhD, главный научный сотрудник НИИ профилактической
медицины имени академика Е.Д.Даленова, НАО «Медицинский университет
Астана», Астана, Казахстан, e-mail:
doszhanova.g@amu.kz;

Муханбетова Н. А. - магистр химии, научный сотрудник НИИ
профилактической медицины им. академика Е.Д.Даленова, НАО «Медицинский
университет Астана», докторант 2 курса специальность НАО «Казахский
агротехнический исследовательский университет им. С.Сейфуллина», Астана,
Казахстан, e-mail:
mukhanbetova.n@amu.kz;

Кожамкулов О. М. -- магистрант НИИ профилактической медицины им.
академика Е.Д.Даленова, НАО «Медицинский университет Астана», Астана,
Казахстан, e-mail:
olzhas1919@mail.ru.

\emph{{\bfseries Information about the authors}}

Abduldayeva A. - Candidate of Medical Sciences, Professor, Director of
the Research Institute of Preventive Medicine named after Academician
E.D.Dalenov, "Astana Medical University",Astans, Kazakhstan, e-mail:
abduldayeva.a@amu.kz;

Tarjibaeva S. - Candidate of Medical Sciences, Associate Professor,
Leading Researcher of the Research Institute of Preventive Medicine
named after Academician E.D.Dalenov, "Astana Medical University",
Astana, Kazakhstan, e-mail:
tardzhibaeva.s@amu.kz;

Doszhanova G.-PhD, Chief Researcher of the Research Institute of
Preventive Medicine named after Academician E.D.Dalenov, "Astana Medical
University",Astana, Kazakhstan, e-mail: doszhanova.g@amu.kz;

Mukhanbetova N.- Master of Chemistry,
\href{https://context.reverso.net/\%D0\%BF\%D0\%B5\%D1\%80\%D0\%B5\%D0\%B2\%D0\%BE\%D0\%B4/\%D0\%B0\%D0\%BD\%D0\%B3\%D0\%BB\%D0\%B8\%D0\%B9\%D1\%81\%D0\%BA\%D0\%B8\%D0\%B9-\%D1\%80\%D1\%83\%D1\%81\%D1\%81\%D0\%BA\%D0\%B8\%D0\%B9/research+assistant}{Research
Assistant} of the Research Institute of Preventive Medicine named after
Academician E.D.Dalenov, "Astana Medical University", 2nd year PhD
student "S.Seifullin Kazakh agrotechnical research university", Astana,
Kazakhstan, e-mail
mukhanbetova.n@amu.kz;

Kozhamkulov O.- Master's student at the Research Institute of Preventive
Medicine named after Academician E.D.Dalenov, "Astana Medical
University",Astana, Kazakhstan, e-mail:
olzhas1919@mail.ru.