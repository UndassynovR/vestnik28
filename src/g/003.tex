\id{МРНТИ 52.47.15}{}

\begin{header}
\swa{}{БИОЛОГИЧЕСКАЯ ДИАГНОСТИКА БУРОВЫХ РАСТВОРОВ, ПРИМЕНЯЕМЫХ ДЛЯ БУРЕНИЯ НЕФТЯНЫХ И ГАЗОВЫХ СКВАЖИН}

{\bfseries
\tsp{1}Ж.К. Надирова\envelope,
\tsp{1}В.П. Бондаренко,
\tsp{1}К.С. Надиров,
\tsp{2}А.Е. Верисокин,
\tsp{1}М.К. Жантасов
}
\end{header}

\begin{affil}
\tsp{1}Южно-Казахстанский университет им. М. Ауэзова, Шымкент, Казахстан,

\tsp{2}Северо-Кавказский федеральный университет, Ставрополь, Россия

\envelope Корреспондент-автор: zhanna.nadirova@inbox.ru
\end{affil}

В статье рассматриваются методы биологической диагностики буровых
растворов на водной основе и их токсикологическая оценка. Авторами
статьи были проведены экспериментальные исследования по определению
фитотоксичности среды, загрязненной компонентами буровых растворов на
водной основе. В работе исследовались буровые растворы на основе гудрона
хлопкового модифицированного, которые предложено использовать при
бурении нефтяных и газовых скважин. Показано, что на уровень токсичности
бурового раствора оказывают влияние комбинированные реагенты: NaOH,
Na\tsb{2}CO\tsb{3}, карбоксиметилцеллюлоза, и другие
микрокомпоненты (измельченные алюминиевый шлак и лом, оксид алюминия,
полимеры полиакриламид, полиакрилонитрил), входящие в его состав. В
результате экспериментов установлено влияние процентного содержания
комбинированных реагентов бурового раствора на морфологические признаки
тест- культуры семян московской фасоли. Выбранная нами тест -- культура
позволяет относительно быстро провести биотест и получить достаточно
точные и воспроизводимые результаты. В ходе эксперимента фиксировалось
всхожесть, энергия прорастания, длина наземной корневой системы.
Доказано, что метод проростков можно эффективно использовать как
экспресс-диагностику для определения токсичности почв на месторождениях,
загрязненных отходами процессов бурения (буровыми шламами и буровыми
растворами).

{\bfseries Ключевые слова:} экологическая безопасность, биотестирование,
буровой раствор, отходы производства, тест-культура, морфологические
признаки, угнетение роста, фитотоксичность.

\begin{header}
{\bfseries МҰНАЙ ЖӘНЕ ГАЗ ҰҢҒЫМАЛАРДЫ БҰРҒЫЛАУ ҮШІН ҚОЛДАНЫЛАТЫН БҰРҒЫЛАУ ЕРІТІНДІЛЕРІНІҢ БИОЛОГИЯЛЫҚ ДИАГНОСТИКАСЫ}

{\bfseries
\tsp{1}Ж.К. Надирова\envelope,
\tsp{1}В.П. Бондаренко,
\tsp{1}К.С. Надиров,
\tsp{2}А.Е. Верисокин,
\tsp{1}М.К. Жантасов
}
\end{header}

\begin{affil}
\tsp{1}М.Әуезов атындағы Оңтүстік Қазақстан университеті, Шымкент, Қазақстан,

\tsp{2}Солтүстік Кавказ федералды университеті, Ставрополь, Ресей,

e-mail: zhanna.nadirova@inbox.ru
\end{affil}

Бұл мақалада су негізіндегі бұрғылау ерітінділердің биологиялық
диагностикалау әдістері және олардың токсикологиялық бағалау
қарастырылады. Авторлары су негізіндегі бұрғылау ерітінділер\-інің
компоненттерімен ластанған ортаның фитоуыттылығын анықтау бойынша
эксперименттік зерттеулер жүргізді. Жұмыста мұнай және газ ұңғымаларын
бұрғылау кезінде қолдануға ұсынылатын мақта модификацияланған гудрон
негізіндегі бұрғылау ерітінділері зерттелді. Бұрғылау ерітіндісінің
уыттылық деңгейіне аралас реагенттер әсер ететіні көрсетілген: NaOH,
Na\tsb{2}CO\tsb{3}, карбоксиметилцеллюлоза және оның
құрамына кіретін басқа микрокомпоненттер (ұсақталған алюминий шлактары
мен сынықтары, алюминий оксиді, полимерлер полиакриламид,
полиакрилонитрил). Эксперименттер нәтижесінде бұрғылау ерітіндісінің
аралас реагенттерінің пайыздық мөлшерінің мәскеу бұршақ тұқымдарының
сынақ дақылының морфологиялық белгілеріне әсері анықталды. Авторлармен
таңдалған сынақ дақылдары биотестті салыстырмалы түрде жылдам жүргізуге
және жеткілікті нәтижелерге қол жеткізуге мүмкіндік береді. Тәжірибе
барысында өнгіштік, өну энергиясы, жердегі тамыр жүйесінің ұзындығы
тіркелді. Көшет әдісін бұрғылау процестерінің қалдықтарымен (бұрғылау
шламдары мен бұрғылау ерітінділері) ластанған кен орындарындағы
топырақтың уыттылығын анықтау үшін жедел диагностика ретінде тиімді
қолдануға болатындығы дәлелденді.

{\bfseries Түйін сөздер}: экологиялық қауіпсіздік, биотестілеу, бұрғылау
ерітіндісі, өндіріс қалдықтары, сынақ-дақылдары, морфологиялық
белгілері, өсудің тежелуі, фитоуыттылық.

\begin{header}
{\bfseries BIOLOGICAL DIAGNOSTICS OF DRILLING FLUIDS USED FOR DRILLING OIL AND GAS WELLS}

{\bfseries
\tsp{1}Zh.K. Nadirova\envelope,
\tsp{1}V.P. Bondarenko,
\tsp{1}K.S. Nadirov,
\tsp{2}A.E. Verisokin,
\tsp{1}M.K. Zhantasov
}
\end{header}

\begin{affil}
\tsp{1}M. Auezov South Kazakhstan University, Shymkent, Kazakhstan,

\tsp{2}The North Caucasus Federal University, Stavropol, Russia,

e-mail: zhanna.nadirova@inbox.ru
\end{affil}

{\bfseries Abstract:} The article discusses methods of biological
diagnostics of water-based drilling fluids and their toxicological
assessment. The authors of the article conducted experimental studies to
determine the phytotoxicity of an environment contaminated with
components of water-based drilling fluids. The paper investigated
drilling fluids based on modified cotton tar (HCM), which are proposed
to be used in drilling oil and gas wells. It is shown that the combined
reagents NaOH, Na2CO3, carboxymethylcellulose (CMC), and other
micro-components (crushed aluminum slag and scrap, aluminum oxide,
polyacrylamide polymers, polyacrylonitrile) that make up its composition
influence the toxicity level of drilling mud. As a result of the
experiments, the influence of the percentage of combined drilling mud
reagents on the morphological characteristics of the test culture of
Moscow bean seeds was established. The test culture we have chosen
allows us to conduct a biotest relatively quickly and obtain fairly
accurate and reproducible results. During the experiment, germination,
germination energy, and the length of the terrest\-rial root system were
recorded. It has been proven that the seedling method can be effectively
used as an express diagnostic for determining the toxicity of soils in
fields contaminated with drilling process waste (drilling mud and
drilling fluids).

{\bfseries Key words:} environmental safety, biotesting, drilling mud,
production waste, test culture, morphologi\-cal signs, growth inhibition,
phytotoxicity.

\begin{multicols}{2}
{\bfseries Введение.} Высокий и все более нарастающий рост добычи и
переработки нефти в мире приводят к тому, что уровень загрязнения
окружающей среды нефтепродуктами и отходами добычи нефти в настоящее
время приобретает глобальный характер. Не исключением являются нефтяные
и газовые месторождения Республики Казахстан.

Анализ, проведенный авторами в работе {[}1{]} по отходам бурения
нефтяных скважин свидетельствует об усилении их загрязненности
органическими соединениями. Это связано с широким применением для
обработки буровых растворов полимерных химических реагентов,
стабилизаторов и понизителей водоотдачи.

Буровые растворы, применяемые для бурения нефтяных и газовых скважин,
представляют собой сложные полидисперсные композиции (суспензии,
эмульсии) переменного состава. Их базовые составы обычно включают
морскую (речную) и пресную воду (около 80\%), барит (3-15\%), бентонит
(2-7\%), лигносульфонат (0,5-1 \%), карбоксиметилцеллюлозу (КМЦ),
полимеры и другие микрокомпоненты (менее 1-2\%) Широкое распространение
в практике бурения скважин получили буровые растворы на углеводородной
основе в связи с уникальными физико-химическими свойствами и высокой
эффективностью. Однако широкому внедрению в практику бурения этих
растворов препятствуют экологические проблемы, вызванные их применением
{[}2{]}.

Жидкой основой буровых растворов (до 75-80\%) служит либо вода, либо
нефтяные углеводороды (сырая нефть, дизельное топливо), либо
синтетические продукты. Именно этим, прежде всего, определяется степень
фитотоксичности буровых растворов при их контакте с живыми организмами.
Наибольшее и повсеместное распространение в современной практике бурения
получили буровые растворы на водной основе.

В настоящее время к буровым растворам предъявляются следующие
требования: они должны выполнять свои функции в различных условиях
бурения скважин (температура, давление), стабильно функционировать при
бурении различных скважин (вертикальных, горизонтальных,
структурно-направленных) и при этом не причинять вреда окружающей среде.

На основании научно-методических принципов, выявленных в результате
проведения комплексных исследований, в лаборатории буровых и тампонажных
растворов ЮКУ им. М. Ауэзова разработана технология получения химических
реагентов с заданными химическими свойствами на основе местного сырья
(каустической и кальцинированной соды, бентопорошка из глин
Дарбазинского месторождения, поверхностно-активных веществ) и отходов
производства (гудрона хлопкового масла и отходов алюминиевой
промышленности) {[}3{]}. В таблице 1 представлены базовые компоненты для
получения буровых растворов, участвующих в тестировании.
\end{multicols}

\tcap{Таблица 1 - Базовые компоненты для получения буровых растворов}
\begin{longtblr}[
  label = none,
  entry = none,
]{
  width = \linewidth,
  colspec = {Q[400]Q[600]},
  cells = {c, font=\small},
  hlines,
  vlines,
}
Компонент                                  & Сертификация                                         \\
Гудрон хлопковый (госсиполовая смола)      & ОСТ-18-114-73                                        \\
Каустическая сода                          & ГОСТ 2263-79                                         \\
Кальцинированная сода                      & ГОСТ5100-85                                          \\
Алюминиевый лом                            & протестирован в лаборатории ИРЛИП ЮКУ им. М. Ауэзова \\
Шлам из отходов алюминиевой промышленности & протестирован в лаборатории ИРЛИП ЮКУ им. М. Ауэзова \\
Оксид алюминия                             & протестирован в лаборатории ИРЛИП ЮКУ им. М. Ауэзова \\
Na-карбоксиметилцеллюлоза                  & ТУ2231-002-05277563-2000                             \\
Полиакрилонитрил                           & протестирован в лаборатории ИРЛИП ЮКУ им. М. Ауэзова \\
Полиакриламид (ПАА)                        & протестирован в лаборатории ИРЛИП ЮКУ им. М. Ауэзова 
\end{longtblr}

\begin{multicols}{2}
Результаты исследований, опубликованные нами ранее в работе {[}4{]}
свидетельствуют о том, что все исследованные типы буровых растворов на
основе гудрона хлопкового модифицированного (ГХМ) практически не
обладают выраженной острой токсичностью. Величины их по результатам
«дафниевого теста» определяют среднюю летальную концентрацию токсиканта
с кратностью разбавления тестируемой среды (LC50), при которой за 96
часов гибнет 50 \% дафний. LC50 (за 96 ч) лежат в пределах 104-106мг/кг
(разбавление 1-100\%).

{\bfseries Материалы и методы.} В модельных экспериментах исследовали
малоглинистые буровые растворы на водной основе, компонентный состав
которых приведен в таблице 2.

Буровые растворы готовили роспуском бентонитовой глины в
дистиллированной воде, после чего в раствор добавляли требуемое
количество реагентов. Образцы растворов на основе ГХМ представлены на
рисунке 1.

Применение ГХМ обосновано тем, что в его состав входит госсипол, который
является антимикробным препаратом, обладает широким спектром действия и
высокой активностью, при низких концентрациях не вызывает коррозии,
является безопасным для человека и окружающей среды, не оказывает
отрицательного влияния на технологические свойства бурового раствора,
имеет доступную сырьевую базу и низкую стоимость {[}4,5{]}. Оценку
токсичности рецептур буровых растворов проводили по методикам {[}6{]}.
\end{multicols}

\tcap{Таблица 2 - Состав исследуемых буровых растворов}
\begin{longtblr}[
  label = none,
  entry = none,
]{
  width = \linewidth,
  colspec = {Q[142]Q[158]Q[96]Q[83]Q[82]Q[252]Q[138]},
  cells = {c, font=\small},
  hlines,
  vlines,
}
Буровые растворы & Госсипо-ловая смола, г & NaOH, г/мл & Na\tsb{2}CO\tsb{3}, г & КМЦ,г & Добавки,г & Бентонит 5\%+Н\tsb{2}О \\
№1               & 65,24                  & 13,04      & 13,04     & 8     & Измельченный алюминиевый шлак 0,64 & Остальное        \\
№2               & 65,24                  & 13,04      & 13,04     & 8     & Измельченный алюминиевым лом 0,56  & Остальное        \\
№3               & 65,24                  & 13,04      & 13,04     & 8     & Окись алюминия 0,64                & Остальное        \\
№4               & 100                    & /300 10\%  & -         & -     & Полиакриламид                      & Остальное        \\
№5               & 65,24                  & /300 10\%  & -         & -     & Полиакрилонитрил                   & Остальное        
\end{longtblr}

\fig[0.6\textwidth]{g2/image2}{Рис.1 - Образцы растворов на основе модифицированного хлопкового гудрона}

\begin{multicols}{2}
Токсичность буровых растворов оценивали с помощью приемов
фитотестирования, непосредственно после залива тест-культуры буровым
раствором {[}7,8{]}. Метод фиотестирования основан на влиянии буровых
растворов на скорость прорастания тест - культур. В качестве тест -
культуры были выбраны семена фасоли сорта "Московская". Семена фасоли
высеивали в вегетативные сосуды, заполненные вермикулитом. В каждый
сосуд высеивалось по пять семян. В качестве земельного покрова
использовали вермикулит. Применение вермикулита обосновано его
способностью обеспечивать равномерную аэрацию семян, а также
противостоять процессам гниения и разложения микроорганизмов. В ходе
эксперимента фиксировались всхожесть, энергия прорастания, длина
наземной корневой системы {[}9-11{]}.

Для обеспечения чистоты эксперимента все горшочки с семенами фасоли
находились в одинаковых условиях (количество света, тепла и т.д.). Опыты
проводили на световых стелажах при поддержании постоянной влажности
воздуха 40-45\%, при температуре 23-25\tsp{о} С. Семена
фасоли заливали буровыми растворами и в течение 14 дней велись
наблюдения за проростками по следующим показателям: длина стебля; ширина
стебля; общая длина; длина листьев; ширина листьев; количество проросших
стеблей; количество листьев; вес общий; вес измеряемого стебля.

Для объективности анализа буровые растворы заливали в четыре емкости и
определяли среднее значения проростков тест-культур. Сравнение
параметров проводили с образцом, залитым отстоянной водой. После
пророста всех семян, была проведена разгерметизация и измерение
проросших семян по параметрам указанным выше.

{\bfseries Результаты и обсуждение.} В результате проведенного
фитотестирования были получены следующие морфометрические данные,
представленные в таблице 3 и на рисунке 2. Разницу показателей до 10\%
по сравнению с контрольным раствором из отстоянной воды не принимали во
внимание и раствор считали экологически чистым, разница в 10--30\%
указывала на слабую токсичность, от 30 до 50\% - на среднюю степень, а
выше 50\% -- на высокую степень фитотоксичности раствора. Растворы №4 и
№5 способствовали проявлению среднего уровня токсичности, это скорее
всего связано с метаболическими процессами под влиянием
поверхностно-активных веществ (полиакриламида и полиакрилонитрила).
Статистическая обработка данных, представленных в таблице 3, показывает
положительное действие растворов, содержащих полиакриламид и
полиакрилонитрил на метрические показатели проростков семян фасоли при
сравнении с контрольным вариантами раствора.

Наибольшее угнетение роста тест-культуры наблюдалось в образцах с
буровыми растворами № 1, 2, 3.
\end{multicols}

\figstart{Рис.2 - Результаты фитототестирования буровых растворов}
\subfig[0.32\textwidth]{4.7cm}{g2/image3}{Образец, проросший на отстоянной воде}
\subfig[0.32\textwidth]{4.7cm}{g2/image4}{Образец, проросший на растворе №1}
\subfig[0.32\textwidth]{4.7cm}{g2/image5}{Образец, проросший на растворе №2}
\subfig[0.32\textwidth]{4.7cm}{g2/image6}{Образец, проросший на растворе №3}
\subfig[0.32\textwidth]{4.7cm}{g2/image7}{Образец, проросший на растворе №4}
\subfig[0.32\textwidth]{4.7cm}{g2/image8}{Образец, проросший на растворе №5}
\figend

\tcap{Таблица 3 - Результаты морфометрического эксперимента}
\begin{longtblr}[
  label = none,
  entry = none,
]{
  width = \linewidth,
  colspec = {Q[230]Q[190]Q[90]Q[119]Q[119]Q[90]Q[90]},
  cells = {c},
  hlines,
  vlines,
}
Параметры              & Контроль на водной основе & {Раствор\\№1} & Раствор №2 & Раствор №3 & {Раствор\\№4} & {Раствор\\№5} \\
Длина стебля, см       & 14,3                      & 8,4           & 8,6        & 5,4        & 11,6          & 12,1          \\
Длина корня, см        & 15,2                      & 14,1          & 12,1       & 11         & 17,8          & 13,3          \\
Общая длина, см        & 27,3                      & 20,1          & 20,7       & 15,9       & 27            & 25,5          \\
Количество листьев, шт & 2                         & 3             & 2          & -          & 3,2           & 2             \\
Длина листьев, см      & 3,72                      & 1,6           & 1,2        & -          & 2,2           & 2,4           \\
Ширина листьев, см     & 3,42                      & 1,5           & 1,6        & -          & 2,6           & 2,6           \\
Количество стеблей, шт & 5                         & 4             & 2          & 3          & 4             & 4             \\
Общий вес, г           & 13,8                      & 9,8           & 6,1        & 8,1        & 10            & 11            
\end{longtblr}

Результаты фитотестирования не выявили острого токсического действия на
рост тест-культуру. Для наглядности представления результатов
исследования была построена гистограмма, приведенная на рисунке 3.

{\bfseries Рис.3 - Влияние буровых раствор на параметры роста тест-культуры}

Далее нами было определено влияние pН буровых растворов на развитие
роста фасоли. Изменение длины проростков в зависимости от рН приведено
на рисунке 4.

{\bfseries Рис.4 - Влияние рН раствора на длину корня фасоли}

\begin{multicols}{2}
Как показывают данные рисунка состояние проростков оказывается лучшим в
варианте с рН=11,34 и 11,95.

{\bfseries Выводы.} Анализируя приведенные результаты исследований можно
заключить, что все исследованные нами буровые растворы оказывают
умеренное токсическое влияние на проростки тест - культур, за
исключением раствора, содержащего окись алюминия, что доказывает
необходимость сокращения его использования в буровых растворах в целях
сохранения здоровой окружающей среды.

На основании анализа морфометрических исследований проростков фасоли мы
получили результаты не всегда совпадающие с широко распространенным
мнением, что нейтральная среда с рН=7,0 оказывает благоприятное влияние
на длину проростков. В наших экспериментах состояние проростков
оказывается лучшим в варианте с рН=11,34 и 11,95 (растворы, содержащие
полиакриламид и полиакрилонитрил).

Предложенная методика фитотестирования буровых растворов является
незаменимым инструментом для решения экологических проблем в бурении
нефтяных и газовых скважин.

\emph{{\bfseries Финансирование:} Данные исследования были выполнены при
поддержке Комитета науки МНВО РК проекта AP26195345}.
\end{multicols}

\begin{center}
{\bfseries Литература}
\end{center}

\begin{refs}
1. Кудайкулова Г.А. Экологизация процесса промывки скважин//Вестник
Национальной академии наук Республики Казахстан. -2010.- № 6. - С.22-32.

2. Нуцкова М.В., Рудяева Е.Ю. Обоснование и разработка
технико-технологических решений для повышения эффективности бурения
скважин в условиях поглощения промывочной жидкости // Недропользование .-
2018. -Т.17(2). - C.104 -114. DOI 10.15593/2224-9923/2018.2.1

3. Бондаренко В.П., Надиров К.С., Бимбетова Г.Ж. Использование
модифицированного гудрона хлопкового масла для приготовления буровых
растворов// Нефть и газ.-2016.- № 5 (96). - С.45-56.

4. A.R. Ismail, N.M. Mohd, N.F. Basir, J.O. Oseh, I. Ismail, S.O.
Blkoor. Improvement of rheological and filtration characteristics of
water-based drilling fluids using naturally derived henna leaf and
hibiscus leaf extracts // J. Pet. Explor. Prod. Technol. - 2020. -
Vol.10. -P.3541-3556. DOI 10.1007/s13202-020-01007-y

5. Bondarenko V.P., Golubev V.G., Zhantasov M.K., Sadyrbayeva A.S.,
Nadirova Zh.K., Ainabekov N.B. Investigation of anti-corrosion
propertiesof environmentally safe additives to drilling solutions based
on tar of cotton oil// Chimica Oggi - Chemistry Today- 2017.-Vol.
35(6).- P.31-35.

6. Другов Ю.С., Родин А.А. Мониторинг органических загрязнений природной
среды: 500 методик: практ. руководство. М.: БИНОМ. Лаборатория знаний,
2009. - 893 с. ISBN 978-5-94774-761- 4.

7. Васильев А.В.1 , Заболотских В.В., Тупицына О.В., Штеренберг А.М.
Экологический мониторинг токсического загрязнения почвы//Электронный
научный журнал «Нефтегазовое дело».- 2012.- № 4 - С.242-249.

8. Биоиндикация и биотестирование в пресноводных экосистемах: учебное
пособие для высших учебных заведений. -- СПб.: РГГМУ, 2019. - 140 с.
ISBN 978-5-86813-491-3

9. Биотестирование. Биологические методы определения токсичности водной
среды: метод. указания / Е.В. Рябухина, С.Л. Зарубин.
Ярославль:ЯРГУ,2006. -- 64 с.
\href{http://www.lib.uniyar.ac.ru/edocs/iuni/20220301.pdf}{}

10. Дмитриев А.И. Биоиндикация. Н. Новгород, 1996. - 33 с. ISBN
5-85152-046-9

11. Заболотских В.В., Васильев А.В., Танких С.Н. Экспресс-диагностика
токсичности почв, загрязнённых нефтепродуктами. Известия Самарского
научного центра Российской академии наук.-2012.- Т.14 №1(3) - С.58-64.
\end{refs}

\begin{center}
{\bfseries Referenses}
\end{center}

\begin{refs}
1. Kudajkulova G.A. Jekologizacija processa promyvki skvazhin//Vestnik
Nacional' noj akademii nauk Respubliki Kazahstan. -2010.-
№ 6. - S.22-32. {[}in Russian{]}

2. Nuckova M.V., Rudjaeva E.Ju. Obosnovanie i razrabotka
tehniko-tehnologicheskih reshenij dlja povy\-shenija jeffektivnosti
burenija skvazhin v uslovijah pogloshhenija promyvochnoj zhidkosti //
Nedropol' zovanie.- 2018.-T.17 (2). - C.104 - 114.
{[}in Russian{]}

3. Bondarenko V.P., Nadirov K.S., Bimbetova G.Zh.
Ispol' zovanie modificirovannogo gudrona hlopkov\-ogo masla
dlja prigotovlenija burovyh rastvorov// Neft'{} i
gaz.-2016.- № 5 (96). - S.45-56. {[}in Russian{]}

4. Nuckova M.V., Rudjaeva E.Ju. Obosnovanie i razrabotka
tehniko-tehnologicheskih reshenij dlja povy\-shenija jeffektivnosti
burenija skvazhin v uslovijah pogloshhenija promyvochnoj zhidkosti //
Nedropol' zovanie. - 2018. -T.17(2). - C.104 -114.
{[}in Russian{]}

5. A.R. Ismail, N.M. Mohd, N.F. Basir, J.O. Oseh, I. Ismail, S.O.
Blkoor. Improvement of rheological and filtration characteristics of
water-based drilling fluids using naturally derived henna leaf and
hibiscus leaf extracts // J. Pet. Explor. Prod. Technol. - 2020. -
Vol.10. -P.3541-3556. DOI 10.1007/s13202-020-01007-y

6. Bondarenko V.P., Golubev V.G., Zhantasov M.K., Sadyrbayeva A.S.,
Nadirova Zh.K., Ainabekov N.B. Investigation of anti-corrosion
propertiesof environmentally safe additives to drilling solutions based
on tar of cotton oil// Chimica Oggi - Chemistry Today- 2017.-Vol.
35(6).- P.31-35.

7. Drugov Ju.S., Rodin A.A. Monitoring organicheskih zagrjaznenij
prirodnoj sredy: 500 metodik: prakt. rukovodstvo. M.: BINOM.
Laboratorija znanij, 2009. - 893 s. ISBN 978-5-94774-761- 4. {[}in
Russian{]}

8. Vasil' ev A.V.1 , Zabolotskih V.V., Tupicyna O.V.,
Shterenberg A.M. Jekologicheskij monitoring toksicheskogo zagrjaznenija
pochvy//Jelektronnyj nauchnyj zhurnal «Neftegazovoe delo».- 2012.- № 4 -
S.242-249. {[}in Russian{]}

9. Bioindikacija i biotestirovanie v presnovodnyh jekosistemah: uchebnoe
posobie dlja vysshih uchebnyh zavedenij. - SPb.: RGGMU, 2019. - 140 s.
ISBN 978-5-86813-491-3. {[}in Russian{]}

10. Biotestirovanie. Biologicheskie metody opredelenija toksichnosti
vodnoj sredy: metod. ukazanija / E.V. Rjabuhina, S.L. Zarubin.
Jaroslavl':JaRGU,2006.- 64 s.
\href{http://www.lib.uniyar.ac.ru/edocs/iuni/20220301.pdf}{http://www.lib.uniyar.ac.ru}. {[}in Russian{]}

11. Dmitriev A.I. Bioindikacija. N. Novgorod, 1996. - 33 s. ISBN
5-85152-046-9. {[}in Russian{]}

12. Zabolotskih V.V., Vasil' ev A.V., Tankih S.N.
Jekspress-diagnostika toksichnosti pochv, zagrjaznjonnyh
nefteproduktami. Izvestija Samarskogo nauchnogo centra Rossijskoj
akademii nauk.-2012.- T.14 №1(3) - S.58-64. {[}in Russian{]}
\end{refs}

\begin{info}
\emph{{\bfseries Сведения об авторах}}

Надирова Ж.К. - кандидат технических наук, ассоциированный
профессор, Южно-Казахстанский университет им. М. Ауэзова, Шымкент,
e-mail:
zhanna.nadirova@inbox.ru;

Бондаренко В.П.-кандидат технических наук, доцент,
Южно-Казахстанский университет им. М. Ауэзова, Шымкент, e-mail:
vbond2011@mail.ru;

Надиров К.С. - доктор технических наук, профессор, Южно-Казахстанский
университет им. М. Ауэзова, Шымкент, e-mail:
nadirovkazim@mail.ru;

Верисокин А.Е. - кандидат технических наук, доцент, Северо-Кавказский
федеральный университет, Россия, Ставрополь, verisokin.aleksandr@mail.ru; 

Жантасов М.К. - кандидат технических наук, профессор, заведующий
кафедрой, Южно-Казахстанский университет им. М.Ауэзова, Шымкент,
Казахстан, e-mail: manapjan\_80@mail.ru.

\emph{{\bfseries Information about the authors}}

Nadirova Zh.K. - Candidate of Technical Sciences, Associate Professor,
M. Auezov South Kazakhstan University,\\
zhanna.nadirova@inbox.ru;

Bondarenko V.P. - Candidate of Technical Sciences, Associate Professor,
M. Auezov South Kazakhstan University, Shymkent, e-mail:
vbond2011@mail.ru;

Nadirov K.S. - Doctor of Technical Sciences, Professor, M. Auezov South
Kazakhstan University, Shymkent, e-mail:\\
nadirovkazim@mail.ru;

Verisokin A.E. - Candidate of Technical Sciences, Associate Professor,
North Caucasus Federal University, Russia, Stavropol,
verisokin.aleksandr@mail.ru;

Zhantasov M.K. - Candidate of Technical Sciences, Professor, Head of the
Department. M.Auezov South Kazakhstan University, Shymkent, e-mail:
manapjan\_80@mail.ru.
\end{info}
