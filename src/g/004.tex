\id{МРНТИ 52.47.33}{}

\begin{header}
\swa{}{АМИНҚЫШҚЫЛДАРЫН ГАЗ ГИДРАТТАРЫНЫҢ КЕН ОРЫНДАРЫ ҮШІН ИНГИБИТОРЛАР РЕТІНДЕ ЗЕРТТЕУ}

{\bfseries
\tsp{1}Д.Т. Бегалиев\envelope,
\tsp{1}Н.М. Жумагалиева,
\tsp{1}Р.И. Джусупкалиева,
\tsp{2}Р.У. Баямирова,
\tsp{1}А.Б. Кыдрашов\envelope,
\tsp{1}Ж.Т. Ержанова,
\tsp{3}Ш.Х. Султанов
}
\end{header}

\begin{affil}
\tsp{1}«Жәңгір хан атындағы Батыс Қазақстан аграрлық-техникалық университеті» КеАҚ Орал, Қазақстан,

\tsp{2}«Ш. Есенов атындағы Каспий мемлекеттік технология және инжиниринг университеті» КеАҚ, Ақтау, Қазақстан,

\tsp{3}Уфа Мемлекеттік Мұнай Техникалық Университеті,Уфа, Ресей

\envelope Корреспондент-автор: dastan.begaliyev@wkau.kz; a.kydrashov@mail.ru
\end{affil}

Газ гидратының түзілуі мен ыдырау процестерін түбегейлі түсіну көптеген
энергетикалық және экологиялық салаларда өте маңызды және мұнай-газ
саласы үшін ағынды қамтамасыз етуде ерекше маңызға ие. Бұл салалар газ
гидраттарын қолданудың негізін құрайды, олар кеңінен зерттелгенімен, әлі
де өсіп келе жатқан зерттеу салалары ретінде дамып келеді. Бұл зерттеу
экологиялық таза аминқышқылдарының, атап айтқанда аланин, гистидин,
фенилаланин және лизиннің метан-пропан газ гидраттарының түзілу
кинетикасына әсерін зерттейді (96\% метан және 4\% пропан). Сонымен
қатар, зерттеу әртүрлі ағын режимдерінің гидраттардың түзілу процесіне
қалай әсер ететінін тексереді. Атап айтқанда, көлбеу жүзді турбиналар
(жоғары ағын) (PBTU) және Раштон турбиналары (радиалды ағын) (RT) екі
түрлі ағынды қамтамасыз етеді. Тәжірибе нәтижелері лизин мен аланиннің
кинетикалық стимуляторлар ретінде әрекет ету арқылы гидраттардың өсуін
тездететінін көрсетеді, ал гистидин мен фенилаланин керісінше
гидраттардың түзілуін тежейтін (ингибиторлық әсер) әсерлерін көрсетеді.
Онымен қоса, радиалды ағын жағдайында индукция уақытының қысқаруына және
аралас ағынмен салыстырғанда гидраттардың газ бен сұйықтықтың өзара
әрекеттесуінің жақсаруына байланысты түзілу жылдамдығының жоғарылауына
ықпал ететіні анықталды.

{\bfseries Түйін сөздер:} газ гидраттары, амин қышқылдары, аланин,
фенилаланин, лизин, гистидин, метан мен пропан қоспасы, ингибитор,
тежелу кинетикасы.

\begin{header}
{\bfseries ИССЛЕДОВАНИЕ АМИНОКИСЛОТ КАК ИНГИБИТОРОВ ДЛЯ МЕСТОРОЖДЕНИЙ ГАЗОВЫХ ГИДРАТОВ}

{\bfseries
\tsp{1}Д.Т. Бегалиев\envelope,
\tsp{1}Н.М. Жумагалиева,
\tsp{1}Р.И. Джусупкалиева,
\tsp{2}Р.У. Баямирова,
\tsp{1}А.Б. Кыдрашов\envelope,
\tsp{1}Ж.Т. Ержанова,
\tsp{3}Ш.Х. Султанов
}
\end{header}

\begin{affil}
\tsp{1}НАО «Западно-Казахстанский аграрно-технический университет имени Жангир хана», Уральск, Казахстан,

\tsp{2}НАО «Каспийский университет технологии и инжиниринга им. Ш.Есенова», Актау, Казахстан,

\tsp{3}Уфимский Государственный Нефтяной Технический Университет, Уфа, Россия,

e-mail: dastan.begaliyev@wkau.kz; a.kydrashov@mail.ru
\end{affil}

Понимание процессов образования и разложения газогидрата имеет важное
значение в различных областях энергетики и экологии. Особую роль оно
играет для нефтегазовой промышленности. Газовые гидраты являются
ключевым элементом в этих областях и продолжают привлекать внимание как
перспективные объекты исследований. В данном исследовании
рассматривается влияние экологически безопасных аминокислот - аланина,
гистидина, фенилаланина и лизина на процессы образования газогидратов
метана и пропана (96\% метана и 4\% пропана). Кроме того исследование
изучает влияние различных режимов потока на процесс образования
гидратов. Особенно турбины с наклонным лезвием (восходящий поток) (PBTU)
и турбины Раштона (радиальный поток) (RT), которые обеспечивают два
различных потока. Результаты эксперимента показывают ускорение роста
гидратов при использовании лизина и аланина как кинетических
стимуляторов. Напротив, гистидин и фенилаланин демонстрируют
ингибирующие эффекты на формирование гидратов. Также отмечено, что
радиальный поток способствует сокращению времени индукции и увеличению
его продолжительности. скорость образования гидратов увеличивается из-за
улучшения взаимодействия между газом и жидкостью по сравнению с потоком,
содержащим оба компонента.

{\bfseries Ключевые слова:} uазовые гидраты, аминокислоты, аланин,
фенилаланин, лизин, гистидин, смесь метана и пропана, ингибитор,
кинетика ингибирования.

\begin{header}
{\bfseries EXAMINATION OF AMINO ACIDS AS INHIBITORS FOR GAS HYDRATE RESERVOIRS}

{\bfseries
\tsp{1}D.T. Begaliyev\envelope,
\tsp{1}N.M. Zhumagaliyeva,
\tsp{1}R.I. Jussupkaliyeva,
\tsp{2}R.U. Bayamirova,
\tsp{1}A.B. Kydrashov\envelope ,
\tsp{1}Zh.T. Yerzhanova,
\tsp{3}Sh.Kh. Sultanov
}
\end{header}

\begin{affil}
\tsp{1}NJSC ``West Kazakhstan Agrarian and Technical University named after Zhangir khan'',Uralsk, Kazakhstan,

\tsp{2}NJSC ``Caspian State University of Technologies and Engineering named after S.Yessenov'', Aktau, Kazakhstan,

\tsp{3}Ufa State Petroleum Technical University,Ufa, Russia,

e-mail: dastan.begaliyev@wkau.kz; a.kydrashov@mail.ru
\end{affil}

Understanding how gas hydrates are formed and break down is crucial for
many fields and especially necessary for flow assurance in oil and gas
transportation. Because of these areas, gas hydrates are now developed
for various uses, even though much research is still underway. This work
explores how alanine, histidine, phenylalanine and lysine impact the
formation of methane--propane (96\% methane and 4\% propane) gas
hydrate. The study looks into how changes in flow regimes play a role in
creating gas hydrates. The two kinds of turbines, pitched blade turbines
(PBTU) and Rushton turbines (RT), offer different types of flow. The
results of experiments demonstrate that lysine and alanine speed up
hydrate development, as they help to promote the process, whereas
histidine and phenylalanine have an opposite effect and retard hydrate
growth. It was also revealed that with radial flow, hydrates are
generated more rapidly since the contact between gases and liquids is
greater compared to mixed flow.

{\bfseries Keywords:} gas hydrates, amino acids, alanine, phenylalanine,
lysine, histidine, methane-propane mixture, inhibitor, inhibition
kinetics.

\begin{multicols}{2}
{\bfseries Кіріспе.} Газ гидраттары-жоғары қысым мен төмен температура
жағдайында газ бен судан түзілетін кристалды қосылыстар. Олардың
құрылымы қонақ молекуласының мөлшеріне байланысты өзгереді, әдетте sI,
sII немесе sH торларын құрайды {[}1-3{]}. Газ гидраттары энергияны
сақтау және басқа да жаңа қосымшалар үшін перспективалы болғанымен, олар
құбырларды бұғаттау қабілетіне байланысты мұнай-газ өнеркәсібі үшін
үлкен проблемалар туғызады. Нәтижесінде гидраттардың пайда болуын
болдырмау маңызды қызмет саласына айналды {[}4-7{]}.

Әдетте жоғары концентрацияда гидраттардың түзілуін тежеу үшін спирттер,
тұздар және гликольдер сияқты термодинамикалық гидратация ингибиторлары
қолданылды. Жақында төмен концентрациядағы тиімділігіне байланысты төмен
дозалы гидратация ингибиторлары (IGND), соның ішінде кинетикалық
гидратация ингибиторлары (КИГ) және анти-агломеранттар (AA) назар
аудартты. КИГ арасында аминқышқылдары биологиялық ыдырау қабілетіне және
қоршаған ортаға әсерінің төмендігіне байланысты көбірек қызығушылық
тудырады {[}8-10{]}.

Алдыңғы зерттеулер метан гидраты немесе көмірқышқыл газы жүйелеріндегі
аминқышқылдарының рөлін зерттегенімен, олардың аралас метан-пропан
жүйелеріндегі мінез-құлқы шектеулі зерттеулерде бағаланды. Бұл зерттеу
төрт аминқышқылдарының - аланин, гистидин, фенилаланин және лизиннің --
метан-пропан қоспасында гидраттардың түзілуіне турбиналардың екі түрін
қолдана отырып кинетикалық әсерін зерттейді (96:4): көлбеу жүзді
турбиналар (жоғары ағын) (PBTU) және Раштон турбиналары (радиалды ағын)
(RT). Зерттеудің мақсаты-олардың ингибиторлық немесе ынталандырушы
әсерін анықтау {[}10-14{]}.

{\bfseries Материалдар мен әдістер.} Ішкі көлемі 1.56 литр болатын үздіксіз
араластырғыш реактор (CSTR) метан мен пропан гидраттарының түзілуін
арттыру әлеуетін зерттеу үшін тапсырыс бойынша жасалған. Реакторды
орнату схемасы 1 суретте көрсетілген.
\end{multicols}

\fig[0.6\textwidth]{g2/image9}{1-сурет. Реакторды орнату схемасы}

\begin{multicols}{2}
Гидратация эксперименттері метан (96\%) және пропан (4\%) газ қоспасын
алдын ала жүктелген тазартылған суы бар реакторға айдау, белгілі бір
жұмыс жағдайларын (t = 2.5 °C, P = 25 бар) сақтау және жұмыс
дөңгелегінің айналуын бастау арқылы жүргізілді. Реактор мен дөңгелектің
өлшемдері 1-кестеде келтірілген.

Газ қоспасы тат баспайтын болаттан жасалған қабылдау түтігі арқылы 0.4
литр тазартылған суы бар реакторға енгізілді. Бұл көлемде сұйықтықтың
биіктігі реактордың ішкі диаметріне сәйкес келді (8 см). Бастапқыда
тоңазытқыштағы температура 10 °C деңгейінде сақталды және газ
берілгеннен кейін біртіндеп -5 °C дейін төмендеді, бұл реактордағы
температураның белгіленген 2.5 °C-қа жетуіне мүмкіндік берді. Содан
кейін араластыру болмаған кезде гидраттардың өздігінен пайда болуын
болдырмау үшін суық бөлмедегі температура -5-тен 0 °C-қа дейін
көтерілді.
\end{multicols}

\tcap{1-кесте. Мөлдір реактордың бір жұмыс дөңгелегі эксперименттері кезінде араластырғыш элементтердің өлшемдері мен орналасуы}
\begin{longtblr}[
  label = none,
  entry = none,
]{
  cells = {c},
  hlines,
  vlines,
}
Таңба & Сипаттама                                & Ұзындығы (см) \\
T     & Реактордың ішкі диаметрі                 & 8.0           \\
D     & Дөңгелектің диаметрі                     & 4.0           \\
H     & Мөлдір реактордың биіктігі               & 30.0          \\
Hg    & Дөңгелектің қалыңдығы                    & 0.2           \\
C     & Дөңгелектің реактордың түбінен қашықтығы & 5.0           
\end{longtblr}

\begin{multicols}{2}
Әрбір эксперимент индукциядан кейін үш сағатқа созылды, бұл гидраттың
бастапқы пайда болуын көрсетеді. Бұл сынақтардың мақсаты таңдалған
аминқышқылдарының қатысуымен дөңгелектердің әртүрлі түрлері мен
бөлімдердің орналасуы гидратация кинетикасына қалай әсер ететінін талдау
болды. Қолданылған турбиналардың арасында Раштон турбинасы (радиалды
ағын) және жоғары конфигурациядағы көлбеу жүзді турбина (аралас ағын)
(PBTU) болды.

Гидратация түзілу кинетикасын бағалау үшін қысым мен температура
деректерін кинетикалық профильдерге түрлендіру үшін деректерді өңдеу
әдісі қолданылды. Нақты газдың күй теңдеуін ( 𝑃 𝑉 = 𝑧 𝑛 𝑅 𝑇 ) қолдана
отырып, уақыт өте келе реактордағы реакцияға түспеген газдың молярлық
мөлшері анықталды. Газдың сығылу коэффициенті (Z) fortran
бағдарламасымен жүзеге асырылған Ли--Кеслер әдісі (1975) арқылы
есептелді.
CH\tsb{4}--C\tsb{3}H\tsb{8}--SI--PBTU--Аланин
жүйесі үшін уақыт бойынша газды тұтынудың тән мысалы 2 суретте
көрсетілген.3 суретте дәл сол жүйе үшін қысым мен температураның
өзгеруін көрсететін жүйе ұсынылған.
\end{multicols}

\fig[0.7\textwidth]{g2/image10}{2-сурет. T = 2.5 °C және P = 25 барда айналуды бастау үшін CH\tsb{4}-C\tsb{3}H\tsb{8}-SI-PBTU-аланин экспериментіндегі бос газ мольдерінің санының өзгеруі}
\fig[0.7\textwidth]{g2/image11}{3-сурет. T = 2.5 °C және P = 25 барда айналуды бастау үшін CH\tsb{4}-C\tsb{3}H\tsb{8}-SI-PBTU-аланин экспериментіндегі қысым мен температураның өзгеруі}

4 суретте тек гидраттың өсу фазасы көрсетілген және бос газ мольдерінің
азаюын сипаттау үшін үшінші ретті көпмүше таңдалды:

\begin{equation}
n = - 7.01 \times 10^{- 16}t^{3} + 1.94 \times 10^{- 11}t^{2} - 1.79 \times 10^{- 7}t + 3.80 \times 10^{- 3}
\end{equation}

мұндағы n-бос газдың моль саны (моль), ал t-уақыт (сек).

Бұл теңдеудің дифференциациясы гидраттың түзілу жылдамдығын береді:

\begin{equation}
\frac{dn}{dt} = - 3 \times 7.01 \times 10^{- 16}t^{2} + 2 \times 1.94 \times 10^{- 11}t - 1.79 \times 10^{- 7}
\end{equation}

мұндағы, dn/dt -- газ шығыны (моль/сек), t-уақыт (сек).

Белгілі бір уақыт аралығында (1, 500, 1000 және 1500 секунд) газ
ағынының жылдамдығы есептеліп,
CH\tsb{4}-C\tsb{3}H\tsb{8}-SI-PBTU-Аланинмен
жасалған тәжірибе нәтижелері 2 кестеде көрсетілді.

\tcap{2 - кесте. CH\tsb{4}-C\tsb{3}H\tsb{8}-SI-PBTU-Аланин үшін газ ағынының жылдамдығы}
\begin{longtblr}[
  label = none,
  entry = none,
]{
  cells = {c},
  hlines,
  vlines,
}
Уақыт (сек)           & 1          & 500        & 1000       & 1500       \\
Газ шығыны (моль/сек) & -1.78×10-7 & -1.59×10-7 & -1.40×10-7 & -1.22×10-7 
\end{longtblr}

\fig[0.7\textwidth]{g2/image12}{4 - сурет. Аланин үшін гидраттың өсу фазасы: T = 2.5 °C және P = 25 бар}

\begin{multicols}{2}
{\bfseries Нәтижелер мен талқылау}. Гидрат алғаш түзіліп басталғаннан кейін
таңдалған төрт уақыт нүктесінде гидраттың түзілу жылдамдығын бағалау
үшін бос газ мольдерінің азаюы бақыланды - 1 сек, 500 сек, 1000 сек және
1500 сек.5, 6-суреттерде турбиналардың екі түрі үшін осы жылдамдықтар
көрсетілген. Осы сыналған жүйелердің ішінде лизиннің қосылуы сәйкесінше
PBTU және RT үшін 13.9 × 10\tsp{-8} моль/сек және 14.4 ×
10\tsp{-8} моль/сек жететін жоғары гидратация жылдамдығына
әкелді. Лизиннің тиімділігін оның ұзағырақ полярлы емес бүйірлік тізбегі
түсіндіруі мүмкін, ол метил топтарын гидрат құрылымына біріктіру арқылы
гидрат торын тұрақтандыруы мүмкін {[}14-16{]}.

Аланин, қарапайым құрылымы бар тағы бір полярлы емес амин қышқылы,
гидраттардың түзілуін ынталандыруда лизиннен кейін тікелей жүреді. Оның
жылдамдығы сәйкесінше PBTU және RT турбиналары үшін 15.6 ×
10\tsp{-8} моль/сек және 16.2 × 10\tsp{-8}
моль/сек болды. Бір қызығы, аланин біздің жүйеде гидраттардың түзілуіне
ынталандырушы әсер көрсеткенімен, алдыңғы зерттеулер оның CO₂ гидраттары
үшін ингибиторлық сипатын көрсетті {[}17{]}.
\end{multicols}

\fig[0.6\textwidth]{g2/image13}{5 - сурет. T = 2.5\,°C және P = 25 бар жағдайында айналуды
бастау үшін 1, 500, 1000 және 1500 секунд өткенде, таза су мен төрт
түрлі аминқышқылдарының қатысуымен PBTU турбинасында жүретін гидратация
жылдамдығы}

\fig[0.6\textwidth]{g2/image14}{6 - сурет. T = 2.5\,°C және P = 25 бар жағдайында айналуды
бастау үшін 1, 500, 1000 және 1500 секунд өткенде, таза су мен төрт
түрлі аминқышқылдарының қатысуымен RT турбинасында жүретін гидратация
жылдамдығы}

\begin{multicols}{2}
Гистидин гидратацияның айтарлықтай төмен жылдамдығында айқын тежегіш
әсерін көрсетті: PBTU және RT үшін сәйкесінше 9.54 ×
10\tsp{-8} моль/сек және 9.74 × 10\tsp{-8}
моль/сек болды. Зарядталған полярлы амин қышқылы ретінде гистидин гидрат
кристалдарының бетімен Вандерваальс немесе электростатикалық күштер
арқылы әрекеттесіп, кристалдардың пайда болуына да, одан кейінгі өсуіне
де кедергі келтіруі мүмкін. Бұл деректер CO₂ гидрат жүйелеріндегі
гистидиннің белгілі тежегіш қасиеттеріне сәйкес келеді {[}18{]}.

Фенилаланин барлық сыналған аминқышқылдарының ең төменгі көрсеткіштерін
көрсетті: PBTU және RT үшін сәйкесінше 9.24 × 10\tsp{-8}
моль/сек және 9.36 × 10\tsp{-8} моль/сек болды. Оның
гидрофобты хош иісті бүйірлік тізбегі газ молекулаларының айналасындағы
судың құрылымына кедергі келтіруі мүмкін, осылайша гидрат жасушаларының
түзілуін бұзады {[}19,20{]}.

Гидродинамикалық тұрғыдан алғанда, 5 және 6-суреттерді талдау радиалды
ағындық жүйелер аралас ағындық жүйелерге қарағанда жоғары гидратация
жылдамдығын қамтамасыз ететінін көрсетеді. Бұл газ бен сұйықтықтың өзара
әрекеттесуінің жақсарғанын және радиалды ағын жағдайында масса
тасымалына төзімділіктің төмендегенін көрсетеді, бұл араластыру
тиімділігінің жоғарылауына әкеледі. Барлық эксперименттер осьтік ағынды
күшейту және орталық құйынның пайда болуын тежеу үшін толығымен
қорғалған реактор конфигурацияларын қолданды. Сонымен қатар, гидраттың
пайда болуының басталуы жүйенің температурасының шамалы жоғарылауымен
қатар жүретіні байқалды, бұл газ гидратының түзілу реакциясының
экзотермиялық сипатымен түсіндіріледі.

{\bfseries Қорытынды.} Осы зерттеу сыналған аминқышқылдарының ішінде лизин
мен аланин тиімді кинетикалық стимуляторлар ретінде әрекет ете отырып,
метан (96\%) және пропан (4\%) гидраттарының түзілу жылдамдығын
айтарлықтай арттыратынын көрсетеді. Ал гистидин мен фенилаланин
керісінше ингибиторлық әсерге ие, соның ішінде фенилаланин барлық ағын
конфигурацияларында гидратацияның ең төмен жылдамдығына әкеледі. Ағынның
түріне байланысты, радиалды ағындық жұмыс дөңгелегі жүйесі гидраттың
түзілу жылдамдығы бойынша да, жеткізу уақытының қысқаруы бойынша да
аралас ағындық жүйелерден асып түседі, бұл газ бен сұйықтықтың тамаша
байланысын және жақсартылған араластыру динамикасын көрсетеді.

Аталмыш аминқышқылдар ішінде аланин ең тиімді стимулятор болып шықты,
екі импеллерлік жүйеде де (PBTU және RT) ең жоғары түзілу
көрсеткіштеріне қол жеткізді, ал фенилаланин бірдей жағдайларда ең
төменгі көрсеткіштерді көрсетті. Бұл бақылаулар аминқышқылдарының
бүйірлік тізбектерінің молекулалық құрылымы мен қасиеттері гидрат
кинетикасын модуляциялауда маңызды рөл атқаратынын растайды.

Аминқышқылдарының пайдалану сипаттамаларынан басқа, әдеттегі
термодинамикалық гидратация ингибиторларына (THIs) қарағанда бірқатар
артықшылықтары бар. Олардың қоршаған ортаға төмен әсері және биологиялық
ыдырау қабілеті оларды экожүйені сақтау маңызды қайраңда қолдануға
қолайлы етеді. Сонымен қатар, ұшпайтын табиғатының арқасында
аминқышқылдары қалпына келтіру процесінде шығындарға аз ұшырайды, бұл
олардың ұзақ мерзімді тиімділігін арттырады.

Сонымен қатар, аминқышқылдарының құрылымдық ұқсастығы механикалық
зерттеулер үшін құнды негіз береді. Функционалды топтарды өзгерту арқылы
молекулалық деңгейде тежелу процесі туралы тереңірек түсінік алуға
болады, бұл ылғалданумен күресудің тиімдірек және экологиялық таза
стратегияларын жасауға мүмкіндік береді.

\emph{{\bfseries Алғыс.} Мен доктор Сотириос Лонгиносқа өзінің құнды
басшылығы, қолдауы және зерттеу барысында пайдалы ұсыныстары үшін шын
жүректен алғысымды білдіргім келеді.}
\end{multicols}

\begin{center}
{\bfseries Әдебиеттер}
\end{center}

\begin{refs}
1. Bhattacharjee G. and Linga P. Amino acids as kinetic promoters for
gas hydrate applications: A mini review // Energy \& Fuels. -- 2021.~--
Vol.35(9). -- P.7553-7571. DOI 10.1021/acs.energyfuels.1c00502.

2. Longinos S.N. and Parlaktuna M. Kinetic study of the effect of amino
acids on methane (95\%)---propane (5\%) hydrate formation //~Reaction
Kinetics, Mechanisms and Catalysis. -- 2021. -- Vol.~133(2). -- P.
753-763. DOI 10.1007/s11144-021-02023-7.

3. Longinos S.N. and Parlaktuna M. Are the amino acids inhibitors or
promoters on methane (95\%)--propane (5\%) hydrate formation?
//~Reaction Kinetics, Mechanisms and Catalysis. -- 2021. -- Vol.~132(2).
-- P.795-809. DOI 10.1007/s11144-021-01959-0.

4. Roosta H., Dashti A., Mazloumi S.H. and Varaminian F. The dual effect
of amino acids on the nucleation and growth rate of gas hydrate in
ethane+ water, methane+ propane+ water and methane+ THF+ water systems
//~Fuel.~-- 2018. -- Vol.212. -- P.151-161. DOI
10.1016/j.fuel.2017.10.027.

5. Jiang Z., Yang C., Jiang W., Liu Z., Zhou L., Li F. and Zhang X. The
Influence of Different Types of Amino Acids on the Formation Kinetics of
Methane Hydrate //~Industrial \& Engineering Chemistry Research. --
2024. -- Vol.~63(33). -- P.14611-14621. DOI 10.1021/acs.iecr.4c01898.

6. Chaovarin C., Yodpetch V., Inkong K., Veluswamy H.P., Kulprathipanja
S., Linga P. and Rangsunvigit P. Improvement of methane hydrate
formation using biofriendly amino acids for natural gas storage
applications: Kinetic and morphology insights // Energy \& Fuels. --
2022. -- Vol.~36(20). -- P.12826-12841. DOI
10.1021/acs.energyfuels.2c02780.

7. Longinos S.N., Longinou D.D., Parlaktuna M. and Toktarbay Z. The
impact of methionine, tryptophan and proline on methane (95\%)--propane
(5\%) hydrate formation //~Reaction Kinetics, Mechanisms and Catalysis.
- 2021.- Vol.134(2). - P.653-664. DOI 10.1007/s11144-021-02089-3.

8. Zhu J., Li X., Liu Z., Sun X., Zhao L., Shi Y., Zhou G., Rui Z. and
Lu G. Effect of biofriendly amino acids on methane hydrate
decomposition: Insights from molecular dynamics simulations //~Fuel. -
2022. - Vol.~325.- P.124919. DOI 10.1016/j.fuel.2022.124919.

9. Englezos P, Kalogerakis N, Dholabhai P.D, Bishnoi P.R. Kinetics of
formation of methane and ethane gas hydrates // Chem Eng Sci. - 1987. -
Vol.42(11).- P.2647-2658. DOI 10.1016/0009-2509(87)87015-X.

10. Linga P, Daraboina N, Ripmeester J.A, Englezos P. Enhanced rate of
gas hydrate formation in a fixed bed column filled with sand compared to
a stirred vessel // Chem Eng Sci. - 2012. - Vol.68(1). - P.617-623.
DOI 10.1016/j.ces.2011.10.030.

11. Longinos S.N. and Parlaktuna M. Examination of asparagine, aspartic
acid and threonine in methane (95\%)-propane (5\%) gas hydrates as
kinetic inhibitors //~Reaction Kinetics, Mechanisms and Catalysis. -
2021. - Vol.~134(1). - P.87-94. DOI 10.1007/s11144-021-02052-2.

12. Liang H, Yang L, Song Y, Zhao J. New approach for determining the
reaction rate constant of hydrate formation via X-ray computed
tomography // J Phys Chem C. -2020. - Vol.125(1). - P.42--48. DOI
10.1021/acs.jpcc.0c07801.

13. Sa J.H, Lee B.R, Park D.H, Han K, Chun H.D, Lee K.H. Amino
acids as natural inhibitors for hydrate formation in CO2
sequestration//Environ Sci Technol. -
2011. -Vol.45(13).-P.5885-5891. DOI 10.1021/es200552c.

14. Longinos S.N, Parlaktuna M. Kinetic analysis of methane---propane
hydrate formation by the use of different impellers // ASC Omega.-2021.
-Vol.6(2). - P.1636--1646. DOI 10.1021/acsomega.0c05615.

15. Kelland M.W, Ke W. Kinetic hydrate inhibitor studies for gas hydrate
systems: a review of experimental equipment and test methods // Energy
Fuels.-2016.-Vol.30(12). - P.10015-10028. DOI\\
10.1021/acs.energyfuels.6b02739.

16. Kelland M.A, Reyes F.T, Trovik K.W. Tris (diakyloamino)
cyclopropenium chlorides: tetrahy drofuran hydrate crystal growth
inhibition and synergism with polyvinylcaprolactam as gas hydrate
kinetic inhibitor // Chem Eng Sci. - 2013. - Vol.93. - P.423-428. DOI
10.1016/j.ces.2013.02.033.

17. Roosta H, Dashti A, Mazloumi H, Varaminian F. Inhibition properties
of new amino acids for prevention of hydrate formation in carbon dioxide
water system: experimental ans modelling inves tigations // J Mol Liq. -
2016. -Vol.215. - P.656-663. DOI 10.1016/j.molliq.2016.01.039.

18. Longinos S.N. and Parlaktuna M. Kinetic analysis of arginine,
glycine and valine on methane (95\%)--propane (5\%) hydrate formation
//~Reaction Kinetics, Mechanisms and Catalysis. -2021. - Vol.~133(2).
-P.741-751. DOI 10.1007/s11144-021-02018-4.

19. Longinos S.N. and Longinou D.D. Examination of different amino acids
as methane-propane gas hydrate kinetic inhibitors in upstream industry.
2022. ~URL~\href{https://www.researchgate.net/profile/Sotirios-\%C2\%A0Longinos/publication/361040342_EXAMINATION_OF_DIFFERENT_AMINO_ACIDS_AS_METHANE-\%C2\%A0PROPANE_GAS_HYDRATE_KINETIC_INHIBITORS_IN_UPSTREAM_INDUSTRY/links/629926fd6886635d5cb871a0/EXAMINATION-OF-DIFFERENT-AMINO-ACIDS-AS-METHANE-PROPANE-GAS-HYDRATE-KINETIC-INHIBITORS-IN-UPSTREAM-INDUSTRY.pdf}{https://www.researchgate.net} Date of address: 04.06.2022

20. Longinos S., Longinou D.D., Tuleugaliyev M. and Parlaktuna M.
Examination of Cysteine, Glutamine and Isoleucine as Methane-Propane Gas
Hydrate Kinetic Inhibitors //~SPE Annual Caspian Technical Conference. -
2022. DOI 10.2118/212055-MS.
\end{refs}

\begin{info}
\emph{{\bfseries Авторлар туралы мәлімет}}

Бегалиев Д.Т.- «Жәңгір хан атындағы Батыс Қазақстан
аграрлық-техникалық университеті» КеАҚ, техника ғылымдарының магистрі,
Орал, Қазақстан, e-mail:dastan.begaliyev@wkau.kz;

Жумагалиева Н.М.- «Жәңгір хан атындағы Батыс Қазақстан
аграрлық-техникалық университеті» КеАҚ, техника ғылымдарының магистрі,
Орал, Қазақстан, e-mail:zhumagaliyeva.nurzhamal@inbox.ru;

Джусупкалиева Р.И.- «Жәңгір хан атындағы Батыс Қазақстан
аграрлық-техникалық университеті» КеАҚ, техника ғылымдарының магистрі,
Орал, Қазақстан, e-mail: rozaid2@mail.ru;

Баямирова Р.У.- «Ш. Есенов атындағы каспий мемлекеттік технология және
инжиниринг университеті» КеАҚ, техника ғылымдарының кандидаты,
қауымдастырылған профессор, Ақтау, Қазақстан, e-mail:
ryskol.bayamirova@yu.edu.k;

Кыдрашов А.Б. «Жәңгір хан атындағы Батыс Қазақстан аграрлық-техникалық
университеті» КеАҚ, PhD докторы, қауымдастырылған профессор, Орал,
Қазақстан, e-mail: a.kydrashov@mail.ru;

Ержанова Ж.Т. «Жәңгір хан атындағы Батыс Қазақстан аграрлық-техникалық
университеті» КеАҚ, техника ғылымдарының магистрі, Орал, Қазақстан,
e-mail:nazim2008@mail.ru;

Султанов Ш. Х.- Уфа Мемлекеттік Мұнай Техникалық Университеті, техника
ғылымдарының докторы, профессор, Уфа, Ресей, e-mail:
ssultanov@mail.ru.

\emph{{\bfseries Information about the authors}}

Begaliyev D. T. - NJSC ``West Kazakhstan Agrarian and Technical
University named after Zhangir khan'', master of technical sciences,
Uralsk, Kazakhstan, e-mail: dastan.begaliyev@wkau.kz;

Zhumagaliyeva N. M.- NJSC ``West Kazakhstan Agrarian and Technical
University named after Zhangir khan'', master of technical sciences,
Uralsk, Kazakhstan, e-mail: zhumagaliyeva.nurzhamal@inbox.ru;

Jussupkaliyeva R.I.- NJSC ``West Kazakhstan Agrarian and Technical
University named after Zhangir khan'', master of technical sciences,
Uralsk, Kazakhstan, e-mail: rozaid2@mail.ru;

Bayamirova R.U.- NJSC ``Caspian State University of Technologies and
Engineering named after S.Yessenov'', candidate of technical sciences,
associate professor, Aktau, Kazakhstan, e-mail: ryskol.bayamirova@yu.edu.kz;

Kydrashov A.B.- NJSC ``West Kazakhstan Agrarian and Technical University
named after Zhangir khan'', PhD doctor, associate professor, Uralsk,
Kazakhstan, e-mail: a.kydrashov@mail.ru;

Yerzhanova Zh. T.- NJSC ``West Kazakhstan Agrarian and Technical
University named after Zhangir khan'', master of technical sciences,
Uralsk, Kazakhstan, e-mail: nazim2008@mail.ru;

Sultanov Sh. Kh.- Ufa State Petroleum Technical University, doctor of
technical sciences, professor, Ufa, Russia, e-mail: ssultanov@mail.ru.
\end{info}
