\newpage
\let\cleardoublepage\clearpage
\chapter{Горное и нефтегазовое дело}
\ID{МРНТИ 52.47.25}{}

\begin{header}
\swa{}{ИННОВАЦИОННЫЙ СПОСОБ ОЧИСТКИ НЕФТЯНЫХ ТРУБ ОТ ПАРАФИНОВЫХ ОТЛОЖЕНИЙ ДЛЯ ПОВЫШЕНИЯ НЕФТЕОТДАЧИ}

{\bfseries
\tsp{1}Ж.Н Алишева\envelope,
\tsp{2}М.А Сарсенбаев,
\tsp{3}Ж.А.Сарсенбаев
}
\end{header}

\begin{affil}
\tsp{1}Казахский Национальный Университет им. Аль-Фараби, Алматы, Казахстан,

\tsp{2}ТОО "Научно-технологический парк" КазНУ им. аль-Фараби, Алматы, Казахстан,

\tsp{3}ТОО «Инновации Плюс», Алматы, Казахстан

\envelope Корреспондент-автор: zhannat\_86.2007@mail.ru
\end{affil}

Проблема парафиновых отложений в нефтяных трубах продолжает оставаться
одной из главных для нефтедобывающей отрасли. При накоплении парафинов
резко снижается пропускная способность трубопроводов, увеличивается
гидравлическое сопротивление и, как следствие, падают объемы добычи
нефти. В таких условиях поиск и внедрение эффективных технологий очистки
труб приобретает особую актуальность.

В представленной работе описан новый механический метод очистки нефтяных
труб с использованием усовершенствованного штангового скребка,
разработанного авторами. Особое внимание уделено специфике нефтедобычи
на казахстанских месторождениях - таких как Тенгиз, Каражанбас и Жетыбай
--- где из-за интенсивного парафинообразования проблема особенно остра.
В этих регионах парафиновые отложения не только снижают дебит скважин,
но и способствуют росту аварийности и увеличению эксплуатационных
затрат.

Разработанный штанговый скребок устанавливается на насосно-компрессорные
трубы и осуществляет очистку в процессе работы насосного оборудования.
За счёт оптимизированной геометрии рабочих элементов устройство
эффективно удаляет как плотные, так и мягкие отложения, предотвращая
образование парафиновых пробок и обеспечивая стабильный транспорт нефти.
Практические испытания, проведенные на ряде казахстанских месторождений,
подтвердили: применение данной технологии способствует увеличению дебита
скважин, снижению расходов на эксплуатацию и уменьшению коррозионного
воздействия на оборудование.

Особо стоит отметить, что механический метод очистки, в отличие от
химических способов, является экологически безопасным. Это
обстоятельство приобретает особое значение в свете ужесточающихся
требований к охране окружающей среды. Проведённый сравнительный анализ
показал, что по таким критериям, как экономическая эффективность,
долговечность и адаптивность к различным условиям эксплуатации,
предложенная технология существенно превосходит традиционные методы
очистки.

Авторы предлагают рекомендации по интеграции штангового скребка в
комплексную систему обслуживания трубопроводов. Среди них - адаптация
конструкции устройства под различные условия добычи, организация
мониторинга состояния труб и сочетание механической очистки с другими
способами повышения нефтеотдачи.

Таким образом, предлагаемый метод представляет собой перспективное
направление повышения эффективности нефтедобычи и надежности
эксплуатации скважин, особенно актуальное для условий казахстанских
месторождений.

{\bfseries Ключевые слова:} очистка нефтяных труб, парафиновые отложения,
механический подход, штанговые скребки, повышение нефтеотдачи,
эффективность очистки, депарафинизация, дебит.

\begin{header}
{\bfseries МҰНАЙ ҚҰБЫРЛАРЫН ПАРАФИН ТҰНБАЛАРЫНАН ТАЗАРТУДЫҢ ИННОВАЦИЯЛЫҚ ӘДІСІ - МҰНАЙ ӨНІМДІЛІГІН АРТТЫРУ}

{\bfseries
\tsp{1}Ж.Н.Алишева\envelope,
\tsp{2}М.А. Сарсенбаев,
\tsp{3}Ж.А. Сарсенбаев
}
\end{header}

\begin{affil}
\tsp{1}әл-Фараби ат. Қазақ Ұлттық Университеті, Алматы, Қазақстан,

\tsp{2}әл-Фараби ат. Қазақ Ұлттық Университетінің "Ғылыми-технологиялық парк" ЖШС, Алматы, Қазақстан,

\tsp{3}«Инновации Плюс» ЖШС Алматы, Қазақстан

e-mail: zhannat\_86.2007@mail.ru
\end{affil}

Мұнай құбырларындағы парафинді тұнбалар мәселесі бүгінгі күнге дейін
мұнай өндіру саласындағы негізгі қиындықтардың бірі болып қала береді.
Парафин шөгінділері жиналған сайын құбырлардың өткізу қабілеті төмендеп,
гидравликалық кедергі артып, нәтижесінде мұнай өндіру көлемі азаяды.
Осындай жағдайда құбырларды тиімді тазарту технологияларын іздеу және
енгізу аса өзекті мәселеге айналады.

Осы зерттеуде авторлар әзірлеген жаңартылған штангалы қырғышты қолдану
арқылы мұнай құбырларын механикалық тазартудың жаңа әдісі ұсынылады.
Ерекше назар парафиннің қарқынды жиналуы байқалатын Қазақстандағы Теңіз,
Қаражанбас және Жетібай сияқты мұнай кен орындарының жұмыс жағдайларына
аударылды. Бұл аймақтарда парафин шөгінділері ұңғымалардың өнімділігін
төмендетіп қана қоймай, авариялық жағдайлардың жиілеуіне және пайдалану
шығындарының өсуіне алып келеді.

Жаңартылған штангалы қырғыш насос-компрессорлық құбырларға орнатылады
және жабдық жұмыс істеп тұрған кезде үздіксіз тазартуды қамтамасыз
етеді. Жұмыс элементтерінің оңтайландырылған геометриясы қатты және
жұмсақ тұнбаларды тиімді жоюға мүмкіндік береді, парафин тығындарының
пайда болуына жол бермейді және мұнайдың тұрақты тасымалдануын
қамтамасыз етеді. Қазақстанның бірнеше кен орындарында жүргізілген
өндірістік сынақтар ұсынылған технологияның ұңғыма өнімділігін
арттыруға, пайдалану шығындарын азайтуға және жабдықтың коррозиялық
тозуын төмендетуге ықпал ететінін көрсетті.

Әсіресе, бұл әдістің химиялық тәсілдерден айырмашылығы --- экологиялық
қауіпсіздігінде екенін атап өткен жөн. Қазіргі уақытта экологиялық
талаптардың күшеюі жағдайында бұл артықшылық ерекше маңызды.
Салыстырмалы талдау нәтижелері ұсынылған технологияның экономикалық
тиімділігі, ұзақ мерзімділігі және әртүрлі пайдалану жағдайларына
бейімделу қабілеті бойынша дәстүрлі тазарту әдістерінен айтарлықтай
басым екенін көрсетті.

Авторлар штангалы қырғышты мұнай құбырларын кешенді техникалық қызмет
көрсету жүйесіне интеграциялау бойынша ұсыныстар береді. Оларға әртүрлі
пайдалану жағдайларына бейімделу үшін құрылымдық өзгерістер енгізу,
құбырлардың жағдайын тұрақты бақылау және механикалық тазарту әдісін
мұнай өндіруді арттырудың басқа технологияларымен біріктіру жатады.

Жалпы алғанда, ұсынылған әдіс Қазақстан мұнай кен орындарының күрделі
жұмыс жағдайларында мұнай өндіру тиімділігін арттыру мен ұңғымаларды
сенімді пайдалану бағытында перспективалық шешім болып табылады.

{\bfseries Түйін сөздер:} мұнай құбырларын тазарту, парафинді тұнбалар,
механикалық әдіс, штангалы қырғыштар, мұнай өндіру тиімділігі, тазарту
өнімділігі, парафинсіздендіру, дебит.

\begin{header}
{\bfseries INNOVATIVE METHOD FOR CLEANING OIL PIPELINES FROM PARAFFIN DEPOSITS TO ENHANCE OIL RECOVERY}

{\bfseries
\tsp{1}Zh.N. Alisheva\envelope,
\tsp{2}M.A. Sarsenbayev,
\tsp{3}Z.A. Sarsenbaev
}
\end{header}

\begin{affil}
\tsp{1}Al-Farabi Kazakh National University, Almaty, Kazakhstan,

\tsp{2}Scientific and Technological Park оf Al-Farabi Kazakh National University" LLP, Almaty,~Kazakhstan,

\tsp{3}Innovation Plus LLP, Almaty, Kazakhstan

e-mail: zhannat\_86.2007@mail.ru
\end{affil}

The issue of paraffin deposits in oil pipelines remains one of the major
challenges for the oil production industry. As paraffin accumulates, it
significantly reduces the pipeline throughput, increases hydraulic
resistance, and ultimately leads to a decline in oil production volumes.
Under such conditions, the search for and implementation of effective
pipeline cleaning technologies becomes increasingly important.

This study presents a novel mechanical method for cleaning oil
pipelines, utilizing an improved rod scraper developed by the authors.
Particular attention is given to the specific conditions of oil
production at Kazakhstani fields, such as Tengiz, Karazhanbas, and
Zhetybai, where intensive paraffin formation exacerbates operational
difficulties. In these areas, paraffin buildup not only decreases well
productivity but also leads to a higher frequency of failures and rising
maintenance costs.

The developed rod scraper is installed on sucker-rod pump tubing and
performs continuous cleaning during the equipment's operation. Thanks to
the optimized geometry of the working elements, the device effectively
removes both hard and soft deposits, preventing the formation of
paraffin plugs and ensuring stable oil transportation. Field trials
conducted at several oilfields in Kazakhstan confirmed that the use of
this technology contributes to increased well productivity, reduced
operational costs, and minimized corrosion of equipment.

It is particularly important to emphasize that the mechanical cleaning
method, unlike chemical approaches, is environmentally safe. This
advantage is especially critical given the tightening of environmental
protection requirements. Comparative analysis has demonstrated that the
proposed technology significantly outperforms traditional cleaning
methods in terms of economic efficiency, durability, and adaptability to
varying operating conditions.

The authors also outline recommendations for integrating the rod scraper
into a comprehensive pipeline maintenance system, including adaptation
of the scraper design to different production environments, regular
pipeline monitoring, and the combination of mechanical cleaning with
other enhanced oil recovery technologies.

Overall, the proposed method offers a promising direction for improving
oil production efficiency and well operation reliability, particularly
under the challenging conditions of Kazakhstani oilfields.

{\bfseries Keywords:} оil pipeline cleaning, paraffin deposits, mechanical
approach, rod scrapers, enhanced oil recovery, cleaning efficiency,
deparaffinization, well productivity.

\begin{multicols}{2}
{\bfseries Введение.} Парафиновые отложения в нефтяных трубах представляют
собой одну из наиболее серьезных технических и экономических проблем в
нефтедобывающей отрасли. Парафины, содержащиеся в нефти, имеют тенденцию
выпадать в осадок при изменении условий температуры и давления. Эти
отложения приводят к ряду негативных последствий, включая уменьшение
пропускной способности трубопроводов, увеличение энергозатрат на
перекачку нефти, снижение дебита скважин и сокращение общей
эффективности эксплуатации нефтяных месторождений.

Мировая проблема парафиновых отложений обусловлена широким
распространением месторождений с высоким содержанием парафинов в нефти.
Например, такие страны, как Россия, Венесуэла и Бразилия, сталкиваются с
этой проблемой в связи с геологическими особенностями и климатическими
условиями. В холодных регионах, таких как Сибирь и Аляска, риск
образования парафиновых пробок особенно велик из-за низких температур.
На решение этой проблемы направлены различные технологии, включая
химические, термические и механические методы очистки, однако они
требуют значительных финансовых затрат и могут быть экологически
небезопасными {[}1-3{]}.

На рисунке 1 показано, как дебит скважины снижается в процессе
эксплуатации из-за накопления парафиновых отложений. Это подтверждает
необходимость внедрения инновационных методов очистки, таких как
механический штанговый скребок.

В Казахстане проблема парафиновых отложений является особенно актуальной
для таких месторождений, как Тенгиз, Каражанбас, Жетыбай и Узен. Нефть
этих месторождений содержит высокий процент парафинов, что приводит к
быстрому накоплению отложений в насосно-компрессорных трубах, колоннах и
трубопроводах. Условия эксплуатации, такие как значительная глубина
скважин, сложная геология и перепады температур, дополнительно
усугубляют ситуацию {[}4{]}. Например, на многих казахстанских
месторождениях температура нефти может значительно снижаться в процессе
эксплуатации и добычи, что способствует интенсивному выпадению
парафинов. Это приводит к увеличению частоты аварийных ситуаций,
простоям оборудования и снижению общей рентабельности добычи {[}5{]}.
\end{multicols}

\fig[0.7\textwidth]{g/image2}{Рис.1 - Изменение дебита скважины в процессе эксплуатации при наличии парафиновых отложений}

\begin{multicols}{2}
Экономические последствия парафиновых отложений в Казахстане оказываются
весьма ощутимыми. Они проявляются в росте расходов на очистку
трубопроводов, увеличении затрат на ремонт оборудования, а также в
снижении объёмов добычи нефти. Дополнительную серьёзность проблеме
придаёт экологический фактор: многие традиционные методы борьбы с
парафином, особенно химические, сопряжены с риском негативного
воздействия на окружающую среду. Это требует особой осторожности при
выборе технологий и подчёркивает необходимость поиска более безопасных
решений {[}6{]}.

{\bfseries Материалы и методы.} Исследование направлено на разработку и
оценку эффективности механического способа очистки нефтяных труб от
парафиновых отложений с использованием нового штангового скребка. В ходе
выполнения работы были проведены лабораторные исследования,
моделирование условий эксплуатации, а также опытно-промышленные
испытания на месторождении Узень.

Объектом исследования стали насосно-компрессорные трубы (НКТ) диаметром
73--89 мм и насосные штанги диаметром 19 и 22 мм, подверженные
образованию асфальто-смоло-парафиновых отложений (АСПО). Исследования
проводились на образцах труб, извлеченных из скважин месторождений
Тенгиз, Каражанбас, Жетыбай. Количественный и качественный анализ
отложений показал, что их состав включает твердые углеводороды (парафины
40-60\%), асфальтены (10-15\%) и смолистые вещества (20-30\%) {[}7-9{]}.

Основной целью исследования является разработка и внедрение
инновационного механического устройства - автоматического штангового
скребка для очистки насосно-глубинных скважин от парафиновых и других
отложений. Устройство предназначено для устранения эксплуатационных
осложнений, вызванных образованием асфальто-смоло-парафиновых отложений
(АСПО), которые существенно снижают производительность скважин,
увеличивают износ оборудования и приводят к росту затрат на обслуживание
и ремонт.

{\bfseries Обсуждение и результаты.} В рамках исследования были решены
следующие задачи:

- Усовершенствована конструкция штангового скребка для более
эффективного удаления парафиновых отложений с внутренних стенок
насосно-компрессорных труб и насосных штанг.

- Проведены опытно-промышленные испытания устройства на скважинах №118
ГУ-90, №2734 ГУ-34, № 6095 ГУ-15 АО ``Озенмунайгаз'' с различными
геолого-техническими условиями.

- Выполнено сравнение предлагаемого метода с существующими технологиями
очистки (тепловыми, химическими и другими механическими) с точки
зрения эффективности, экологической безопасности и экономической
целесообразности.

\emph{Значимость предлагаемой технологии:}

1. Решение проблемы асфальто-смоло-парафиновых отложений (АСПО)

Парафиновые отложения продолжают оставаться одной из главных
эксплуатационных проблем нефтедобывающей отрасли, особенно на
месторождениях Казахстана, таких как Тенгиз и Жетыбай. Они приводят к
снижению дебита скважин, увеличению энергозатрат и частым ремонтам
оборудования. Предложенное устройство эффективно устраняет эту проблему
за счёт механической очистки, которая интегрируется непосредственно в
процесс работы скважины без остановки добычи.

2. Увеличение межочистного периода (МОП)

Применение саморегулирующегося штангового скребка позволяет значительно
увеличить интервал между очистками скважин. По результатам
опытно-промышленных испытаний на казахстанских месторождениях
установлено, что межочистной период возрастает в 3--4 раза. Это, в свою
очередь, существенно снижает частоту вынужденных остановок и расходы на
обслуживание.

3. Экологическая безопасность

В отличие от химических методов, новая технология исключает
использование реагентов, которые могут оказывать негативное влияние на
окружающую среду. Это делает устройство особенно востребованным в
условиях ужесточения экологических норм и требований к безопасной
эксплуатации оборудования.

4. Экономическая эффективность

Благодаря уменьшению потребности в частой очистке и увеличению срока
службы оборудования, применение скребка позволяет сократить расходы на
капитальный ремонт и эксплуатацию до 80\%. Конструкция устройства
отличается высокой износостойкостью и не требует сложного технического
обслуживания, что дополнительно снижает затраты.

5. Гибкость и адаптивность конструкции

Штанговый скребок может применяться на скважинах с различной
архитектурой --- как вертикальных, так и наклонных или горизонтальных.
Конструкция адаптирована под насосно-компрессорные трубы и насосные
штанги различных диаметров, что делает её универсальной для большинства
эксплуатационных условий в нефтяной промышленности.

\emph{Основные элементы конструкции:}

1. Корпус скребка - изготовлен из износостойкого металла и оснащён
реверсивными зубцами и пластинчатыми пружинами, что обеспечивает
эффективное удаление отложений со стенок труб.

2. Дугообразные пластинчатые пружины - обеспечивают фиксацию устройства
на насосной штанге и создают необходимую силу трения для эффективной
очистки внутренней поверхности труб.

3. Реверсивные зубцы - направляют движение скребка и предотвращают его
обратный ход, обеспечивая последовательную циклическую очистку.

4. Реверсивные коммутаторы - верхний и нижний ограничивают рабочую зону
скребка и обеспечивают смену направления его движения.

5. Предохранительное соединение - изготовлено из магнитного материала и
служит для улавливания металлических примесей, предотвращая возможное
заклинивание оборудования.

6. Реверсивный механизм - координирует движение скребка, обеспечивая его
эффективную работу при каждом цикле возвратно-поступательного движения
насосных штанг.

7. Ступенчатая пружина - позволяет адаптировать устройство к различным
диаметрам насосно-компрессорных труб (НКТ), снижая уровень износа и
поддерживая стабильную прижимную силу.

Конструкция устройства позволяет эффективно удалять отложения, снижая
эксплуатационные затраты и повышая надёжность оборудования {[}1{]}.

\emph{Принцип работы конструкции}

Скребок устанавливается на насосную штангу между верхним и нижним
реверсивными коммутаторами, ограничивающими рабочий участок. При
движении насосной штанги вниз дугообразные пружины плотно прижимаются к
внутренней поверхности НКТ, срезая отложения. Реверсивные зубцы
фиксируют устройство, не позволяя ему двигаться в обратном направлении.

При изменении направления движения насосной штанги скребок проходит
через расширительные камеры коммутаторов. В этот момент реверсивные
зубцы изменяют своё положение, что позволяет устройству начать движение
в обратную сторону, продолжая процесс очистки внутренних поверхностей
труб. Основные технические параметры работы устройства представлены в
таблице 1 {[}1{]}.
\end{multicols}

\tcap{Таблица 1 - Основные технические характеристики работы устройства}
\begin{longtblr}[
  label = none,
  entry = none,
]{
  cells = {c},
  hlines,
  vlines,
}
№ & \textbf{Наименования				} & \textbf{Свойства				}\\
1 & Диаметр
				НКТ & от
				73 до 89 мм\\
2 & Диаметр
				насосных штанг & 19
				мм и 22 мм\\
3 & Длина
				рабочего участка & до
				1500 м\\
4 & Интервал
				очистки & от
				3 до 100 суток\\
5 & Максимальная
				кривизна скважины & до
				3° на 100 м
\end{longtblr}

\begin{multicols}{2}
\emph{Преимущества конструкции}

Разработанное устройство позволяет осуществлять очистку трубопроводов в
процессе эксплуатации скважины, без необходимости остановки добычи
нефти. Это существенно снижает риск потерь добычи при техническом
обслуживании.

Кроме того, конструкция скребка отличается высокой универсальностью: он
может быть использован с различными типами насосов и адаптирован к
различным условиям эксплуатации.

Еще одним важным преимуществом является экологическая безопасность
устройства: для его работы не требуется применение химических реагентов,
что снижает негативное воздействие на окружающую среду. Монтаж скребка
не требует сложных технических операций, а обслуживание минимально, что
делает его использование удобным и экономичным.

\emph{Сравнение с существующими методами}

Проблема парафиновых отложений в нефтяной отрасли традиционно решается с
помощью различных технологий: механических, химических, термических и их
комбинаций. Каждая из этих методик имеет свои сильные и слабые
стороны.

В рамках настоящего исследования проведён сравнительный анализ
предлагаемого механического устройства --- штангового скребка --- с
существующими методами удаления парафиновых отложений {[}10-15{]}.
Основные результаты анализа представлены в таблице 2.
\end{multicols}

\tcap{Таблица 2 - Сравнительный анализ предложенной технологии с существующими методами удаления парафиновых отложений}
\begin{longtblr}[
  label = none,
  entry = none,
]{
  width = \linewidth,
  colspec = {X[20]X[100]X[156]X[144]X[150]X[167]X[173]},
  cells = {c, font=\small},
  hlines,
  vlines,
}
№ & Критерий & Механические методы & Химические методы & Тепловые методы & Комбинированные методы & Предложенный штанговый скребок\\
1 & Принцип работы & Механическое удаление парафина жесткими скребками & Растворение парафина химическими реагентами & Разогрев нефти и труб для плавления отложений & {Совмещение химических, механических и тепловых методов} & Очистка труб за счет возвратно-поступательного движения насосных штанг\\
2 & Необходи\-мость остановки добычи & Требуется для замены скребков & Нет, реагенты вводятся в поток & Требуется при обработке паром или горячей неф-тью & Частично, в зависимости от комбина-ции методов & Не требуется, очистка происходит во время эксплуатации\\
3 & Энерго-эффек-тивность & Средняя, требует периодического извлечения скребков & Высокие затраты на химические реагенты & Высокие энергозатраты на нагрев & Высокие затраты на комплексную обработку & Высокая, не требует допол-нительного энергоснаб-жения\\
5 & Экологи\-ческая безопа-сность & Средняя, уда-ленные пара-фины могут потребовать утилизации & Низкая, химикаты могут загрязнять окружающую среду & Низкая, высокая энергоемкость и возможное разрушение пласта & Низкая, совмещение химии и нагрева усиливает негативные последствия & Высокая, не использует химические реагенты и энергоемкие процессы\\
7 & Сложность внедрения & Средняя, требует замены оборудования & Средняя, необходимо оборудование для подачи химикатов & Высокая, требует установки нагревательных систем & Очень высокая, требует сложной координации технологий & Низкая, легко интегрируется в существующую систему без модернизации\\
8 & Необходи\-мость постоянного контроля & Высокая, из-за износа механических частей & Высокая, требуется постоянное дозирование реагентов & Высокая, требуется контроль температуры & Очень высокая, требуется мониторинг всех используемых технологий & Низкая, работает автоматически в процессе эксплуатации\\
9 & Долговеч\-ность и надежность & Средняя, требует регулярного обслуживания & Средняя, возможно накопление побочных продуктов в трубах & Низкая, частый перегрев может повредить пласт & Средняя, высокая сложность контроля процессов & Высокая, минимальный износ при длительном использовании
\end{longtblr}

\begin{multicols}{2}
\emph{Преимущества и перспективы предлагаемой технологии}

В отличие от традиционных механических методов, разработанная технология
с автоматическим штанговым скребком обеспечивает непрерывную очистку
трубопроводов без необходимости останавливать добычу нефти. Конструкция
устройства с реверсивными коммутаторами и зубцами позволяет скребку
перемещаться вдоль насосной штанги вверх и вниз, эффективно удаляя
отложения в ходе эксплуатации. Такой подход позволяет значительно
увеличить межочистной период (МОП) и, как следствие, снизить
эксплуатационные расходы.

Предложенный метод механической очистки хорошо сочетается с
существующими технологиями удаления парафина и в ряде случаев может их
дополнять или заменять. Например, интеграция штангового скребка с
периодической термической обработкой горячей нефтью позволяет уменьшить
частоту термических промывок и сократить потребность в использовании
химических реагентов {[}1{]}.

Основные направления дальнейшего развития и совершенствования данной
технологии представлены в таблице 3.
\end{multicols}

\tcap{Таблица 3 - Возможности дальнейшего применения и усовершенствования технологии}
\begin{longtblr}[
  label = none,
  entry = none,
]{
  width = \linewidth,
  colspec = {Q[20]Q[300]Q[600]},
  cells = {c, font=\small},
  cell{2}{1} = {c=3}{0.938\linewidth},
  cell{6}{1} = {c=3}{0.938\linewidth},
  cell{10}{1} = {c=3}{0.938\linewidth},
  cell{13}{1} = {c=3}{0.938\linewidth},
  cell{16}{1} = {c=3}{0.938\linewidth},
  hlines,
  vlines,
}
№ & \textbf{Направление развития} & \textbf{Описание}\\
\textbf{Расширение областей применения} &  & \\
1 & Горизонтальные и наклонные скважины & Адаптация конструкции скребка для работы в условиях сложной траектории скважин, включая многозабойные системы.\\
2 & Высоковязкие нефти и осложненные условия добычи & Модификация устройства для работы с высоковязкими нефтями, содержащими большое количество асфальто-смоло-парафиновых отложений (АСПО).\\
3 & Применение в штанговых глубинных насосах (ШГН) & Адаптация устройства к работе с разными диаметрами насосных штанг и насосно-компрессорных труб.\\
\textbf{Технологическое усовершенствование} &  & \\
1 & Оптимизация материалов & Разработка более износостойких и антикоррозионных покрытий для продления срока службы устройства в агрессивных средах.\\
2 & Улучшение конструкции реверсивного механизма. & Оптимизация реверсивной системы зубцов с целью обеспечения более плавного хода скребка, повышения эффективности очистки и снижения нагрузки на насосные штанги.\\
3 & Снижение сопротивления трению & Улучшение конструкции пластинчатых пружин для снижения энергопотребления насосного оборудования.\\
\textbf{Применение цифровых решений для повышения эффективности очистки} &  & \\
1 & Разработка интеллектуальных систем контроля работы очистных механизмов & \\
2 & Внедрение систем предиктивной аналитики & Использование алгоритмов искусственного интеллекта для прогнозирования образования парафиновых отложений и оптимизации работы очистного устройства.\\
\textbf{Комплексные методы борьбы с парафиновыми отложениями} &  & \\
1 & Интеграция механической очистки с применением химических ингибиторов парафинообразования & Применение механического скребка в сочетании с дозаторами ингибиторов для повышения эффективности борьбы с парафиновыми отложениями.\\
2 & Интеграция с термическими методами & Оптимизация процесса депарафинизации за счет периодического применения тепловой обработки в сочетании с механической очисткой.\\
\textbf{Масштабируемость и коммерциализация} &  & \\
1 & Широкое внедрение на казахстанских месторождениях & Использование технологии на месторождениях с высокой склонностью к парафинобразованию (Каражанбас, Жетыбай, Узень и др.).\\
2 & Развитие экспортного потенциала & Адаптация технологии для применения на зарубежных нефтяных месторождениях с аналогичными эксплуатационными осложнениями (в России, Канаде, Китае, Бразилии).\\
3 & Разработка передвижных систем очистки & Разработка автономных модулей для быстрого развертывания очистных систем на удалённых скважинах.
\end{longtblr}

\begin{multicols}{2}
Таким образом, дальнейшее развитие и усовершенствование технологии
позволит значительно повысить эффективность добычи нефти, снизить
эксплуатационные расходы и минимизировать негативное воздействие
парафиновых отложений на нефтедобывающую инфраструктуру.

{\bfseries Выводы.} В рамках проведённого исследования было разработано и
успешно протестировано механическое устройство для автоматической
очистки насосно-компрессорных труб (НКТ) и насосных штанг от парафиновых
отложений. Результаты анализа подтвердили, что по сравнению с
традиционными методами очистки - механическими, химическими,
термическими и комбинированными - предлагаемая технология обладает рядом
существенных преимуществ:

- {\bfseries Непрерывная очистка}: устройство функционирует в
автоматическом режиме, обеспечивая очистку труб без остановки добычи
нефти. Это позволяет значительно увеличить межочистной период (МОП) и
сократить число внеплановых остановок скважин.

- {\bfseries Экономическая эффективность}: отказ от дорогостоящих
химических реагентов и тепловой обработки позволяет значительно
снизить эксплуатационные расходы.

- {\bfseries Экологическая безопасность}: отсутствие химического
воздействия минимизирует риск загрязнения окружающей среды и снижает
коррозионную нагрузку на трубопроводы.

- {\bfseries Простота эксплуатации}: скребок легко интегрируется в
существующие насосные системы, не требуя сложной модернизации
оборудования.

- {\bfseries Универсальность}: конструкция устройства адаптирована для
работы с насосно-компрессорными трубами различных диаметров и подходит
для эксплуатации в условиях осложнённых парафиновыми отложениями
скважин Казахстана.

{\bfseries Рекомендации по дальнейшему применению технологии:}

- Расширить внедрение устройства на нефтедобывающих предприятиях
Казахстана, особенно на месторождениях с высокой интенсивностью
парафиновых отложений, таких как Тенгиз, Жетыбай и Каражанбас.

- Проводить регулярный мониторинг эффективности очистки для
своевременной оптимизации рабочих параметров скребка в зависимости от
конкретных условий добычи.

- Применять комплексный подход, сочетая механическую очистку с
периодической термической обработкой в случае необходимости, для
достижения максимальной эффективности.

- Продолжить исследования по повышению износостойкости материалов и
адаптации конструкции скребка для работы в сложных геолого-технических
условиях, включая горизонтальные и высоковязкие нефтяные скважины.

- Разработать автоматизированную систему мониторинга работы устройства,
позволяющую в режиме реального времени отслеживать его состояние и
эффективность очистки.
\end{multicols}

\begin{center}
{\bfseries Литература}
\end{center}

\begin{refs}
1. Патент № 7700. Способ очистки нефтяных насосно-глубинных скважин от
парафина и других отложений / Сарсенбаев А. А., Жан Й. М., Карибай Е.,
Алтыбай Қ. А., Сарсенбаев М. А., Сарсенбаев Ж. А. - № 7700; заявл.
30.12.2022, опубл.30.12.2022.

2. Metaksa G.P., Alisheva Z.N., Metaksa A.S., Fedotenko N.A. Scientific
and technical fundamentals of changing the properties of hydrocarbons
in conditions of optimal subsoil use//Eurasian Mining. - 2023. - Vol.
2. - P.75--79. DOI
\href{https://doi.org/10.17580/em.2023.02.16}{10.17580/em.2023.02.16}.

3. Мастобаев Б.Н., Хасанова К.И., Дмитриев М.Е. Повышение эффективности
применения средств и методов борьбы с асфальтосмолопарафиновыми
отложениями в процессе транспорта нефти по магистральным
трубопроводам//Транспорт и хранение нефтепродуктов и углеводородного
сырья.- 2013. - № 3. - С.7-12.

4. Зарипова Л.М., Габдрахимов М.С., Сулейманов Р.И., Галимуллин М.Л.,
Давыдов А.Ю., Хабибуллина Р.Г., Зарипов А.К. Современные методы очистки
от асфальтосмолопарафиновых отложений // Sciences of europe. -- 2017.
№19 (19). -- С.58-60.

5. Сатыева, Ю.П. Совершенствование очистки нефтесборных и насосно
--компрессорных труб от АСПО /Ю.П. Сатыева, Л.М. Зарипова// В сборнике:
Современные технологии в нефтегазовом деле2015 Сборник трудов
международной научно-технической конференции: в 2 томах. - 2015. С.
133-139.

6. Афанасьев С. В., Волков В. А., Турапин А. Н. Очистка магистральных
трубопроводов сложной конфигурации и переменного диаметра от отложений
// Neftegaz- 2019. - №\,12.-С.56-63.

7. Алшавка Х.Х. Пути решения проблемы очистки нефтепроводов от парафина и
других отложений // Теория и практика современной науки. - 2021.- № 4
(70). -С.36-40.

8. Орлов А.И. Метод оперативного контроля состояния парафиновых
отложений при очистке демонтированных нефтепроводных труб: Дисс. канд.
техн. наук. Казань, Казанский гос. энергетический университет. - 2011.
-129 c.

9. Патент РФ № 2344338. Способ определения толщины отложений на
внутренней поверхности трубопроводов. Опубл.20.01.2009.

10. Подоплелов Е. В., Качан К. П., Тикунова Н. С. Методы депарафинизации
нефтяных скважин // Вестник Ангарского Государственного Технического
Университета. - 2024. - Т.1(18).- C.109-112.

11. Глущенко, В. Н. Предупреждение и устранение
асфальтеносмолопарафиновых отложений // Нефтепромысловая химия. - 2009.
-- 475 с. ISBN 9785902063407

12. Хоффман Р., Амундсен Л. Способ удаления парафина и измерения его
толщины // Патент EA018505B1. - 2013.

13. Хохлов, Н. Г. Удаление асфальто-смолистых веществ и парафина из
нефтепроводов НГДУ «Южарлан - нефть» // Нефтяное хозяйство.- 2006. -№ 1.
- С.110-111.

14. Lebedev A., Cherepovitsyn A. Waste Management during the Production
Drilling Stage in theOil and Gas Sector: A Feasibility Study // Journal
of Petroleum Science and Engineering. - 2024. -- 13 (12). - 26. DOI
10.3390/resources13020026 .

15. Sotnikov G., Balakirev V. A., Tkach Yu. V., Yatsenko T. Yu.
High-Frequency Method of Removal of Paraffin Plugs in the Equipment of
Oil Wells and Oil Pipelines// Journal of Engineering Physics and
Thermophysics. -- 2002.- Vol.75. - No 5. - 2002.

DOI 10.1023/A:1021180027735.
\end{refs}

\begin{center}
{\bfseries References}
\end{center}

\begin{refs}
1. Patent № 7700. Sposob ochistki neftjanyh nasosno-glubinnyh skvazhin ot
parafina i drugih otlozhenij / Sarsenbaev A. A., Zhan J. M., Karibaj E.,
Altybaj Қ. A., Sarsenbaev M. A., Sarsenbaev Zh. A. - № 7700; zajavl.
30.12.2022, opubl.30.12.2022. {[}in Russian{]}

2. Metaksa G.P., Alisheva Z.N., Metaksa A.S., Fedotenko N.A. Scientific
and technical fundamentals of changing the properties of hydrocarbons in
conditions of optimal subsoil use//Eurasian Mining. - 2023. - Vol.2. -
P.75--79. DOI
\href{https://doi.org/10.17580/em.2023.02.16}{10.17580/em.2023.02.16}.

3. Mastobaev B.N., Hasanova K.I., Dmitriev M.E. Povyshenie
jeffektivnosti primenenija sredstv i metodov bor' by s
asfal' tosmoloparafinovymi otlozhenijami v processe
transporta nefti po magistral' nym
trubo provodam // Transport i hranenie nefteproduktov i uglevodorodnogo
syr' ja.- 2013. - № 3. - S.7-12. {[}in Russian{]}

4. Zaripova L.M., Gabdrahimov M.S., Sulejmanov R.I., Galimullin M.L.,
Davydov A.Ju., Habibullina R.G., Zaripov A.K. Sovremennye metody
ochistki ot asfal' tosmoloparafinovyh otlozhenij //
Sciences of europe. -- 2017. №19 (19). -- S.58-60. {[}in Russian{]}

5. Satyeva, Ju.P. Sovershenstvovanie ochistki neftesbornyh i nasosno
--kompressornyh trub ot ASPO /Ju.P. Satyeva, L.M. Zaripova// V sbornike:
Sovremennye tehnologii v neftegazovom dele2015 Sbornik trudov
mezhdunarodnoj nauchno-tehnicheskoj konferencii: v 2 tomah. - 2015. S.
133-139. {[}in Russian{]}

6. Afanas' ev S. V., Volkov V. A., Turapin A. N. Ochistka
magistral' nyh truboprovodov slozhnoj konfigu\-racii i
peremennogo diametra ot otlozhenij // Neftegaz- 2019. - №\,12.-S.56-63.
{[}in Russian{]}

7. Alshavka H.H. Puti reshenija problemy ochistki nefteprovodov ot
parafina i drugih otlozhenij // Teorija i praktika sovremennoj nauki. -
2021. - № 4 (70). -S.36-40. {[}in Russian{]}

8. Orlov A.I. Metod operativnogo kontrolja sostojanija parafinovyh
otlozhenij pri ochistke demontirovan\-nyh nefteprovodnyh trub: Diss. kand.
tehn. nauk. Kazan', Kazanskij gos. jenergeticheskij
universitet. - 2011. -129 c. {[}in Russian{]}

9. Patent RF № 2344338. Sposob opredelenija tolshhiny otlozhenij na
vnutrennej poverhnosti trubo\-provodov. Opubl.20.01.2009. {[}in
Russian{]}

10. Podoplelov E. V., Kachan K. P., Tikunova N. S. Metody deparafinizacii
neftjanyh skvazhin // Vestnik Angarskogo Gosudarstvennogo Tehnicheskogo
Universiteta. - 2024. - T.1(18).- C.109-112. {[}in Russian{]}

11. Glushhenko, V. N. Preduprezhdenie i ustranenie
asfal' tenosmoloparafinovyh otlozhenij //
Nefte\-promyslovaja himija. - 2009. - 475 s. ISBN 9785902063407. {[}in
Russian{]}

12. Hoffman R., Amundsen L. Sposob udalenija parafina i izmerenija ego
tolshhiny // Patent EA018505B1. - 2013. {[}in Russian{]}

13. Hohlov, N. G. Udalenie asfal' to-smolistyh veshhestv i
parafina iz nefteprovodov NGDU «Juzharlan - neft'» //
Neftjanoe hozjajstvo.- 2006. -№ 1. - S.110-111. {[}in Russian{]}

14. Lebedev A., Cherepovitsyn A. Waste Management during the Production
Drilling Stage in theOil and Gas Sector: A Feasibility Study // Journal
of Petroleum Science and Engineering. - 2024. -- 13 (12). - 26. DOI
10.3390/resources13020026 .

15. Sotnikov G., Balakirev V. A., Tkach Yu. V., Yatsenko T. Yu.
High-Frequency Method of Removal of Paraffin Plugs in the Equipment of
Oil Wells and Oil Pipelines// Journal of Engineering Physics and
Thermophysics. -- 2002.- Vol.75. - No 5. - 2002.

DOI 10.1023/A:1021180027735.
\end{refs}

\begin{info}
{\bfseries Сведения об авторах}

Алишева Ж.Н.- PhD, и.о. доцента, Казахский Национальный Университет им.
Аль-Фараби, Алматы, Казахстан, e-mail: zhannat\_86.2007@mail.ru;

Сарсенбаев М.А.- ТОО "Научно-технологический парк" КазНУ имени
аль-Фараби, Директор, Алматы, Казахстан, e-mail:
mukhtar.sarsenbaev@mail.ru;

Сарсенбаев Ж.А., ТОО «Инновации Плюс», специалист, Алматы, Казахстан,
e-mail: zhasstin@mail.ru.

\emph{{\bfseries Information about the authors}}

Alisheva Zh.N.- Doctor PhD, Associate Professor, Al-Farabi Kazakh
National University, Almaty, Kazakhstan, e-mail:\\
zhannat\_86.2007@mail.ru;

Sarsenbayev M.A.- Director of "Scientific and Technological Park оf
Al-Farabi Kazakh National University" LLP, Almaty, Kazakhstan, e-mail:
mukhtar.sarsenbaev@mail.ru;

Sarsenbaev Z.A.- Engineer, Innovation Plus LLP, Almaty, Kazakhstan,
e-mail: zhasstin@mail.ru.
\end{info}
