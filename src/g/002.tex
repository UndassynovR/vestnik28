\id{ҒТАМР 52.47.15}{}

\begin{header}
\swa{}{ТҰТҚЫРЛЫҒЫ ЖОҒАРЫ МҰНАЙДЫ ӨНДІРУДЕ ҰҢҒЫМАНЫҢ ТҮП АЙМАҒЫН ӨҢДЕУ
ӘДІСТЕРІ}

{\bfseries
\tsp{1}М.У. Калменов,
\tsp{2}М.Г. Абдуллаев,
\tsp{1}С.Е. Байботаева\envelope,
\tsp{1}А.С. Садырбаева
}
\end{header}

\begin{affil}
\tsp{1}Әуезов атындағы Оңтүстік Қазақстан университеті, Шымкент, Қазақстан,

\tsp{2}Әзірбайжан Республикасының Мұнай және газ институты, Баку,Әзірбайжан

\envelope Корреспондент-автор: sbaibotaeva@mail.ru
\end{affil}

Бұл мақалада ұңғыманың түп аймағын өңдеу тиімділігін арттырудағы жаңа
әдіс әзірленген, жылутасығышты ұңғыманың түп аймағына енгізу арқылы,
оның құрамының еру қабілетін, өндіруші ұңғымалардың түп аймағы бөлігін
өңдеу кезінде ерімейтін шөгінділердің еру жылдамдығы мен көлемін,
соныменқатар, қабатқа терең енуді арттыру болып табылады.

Бұл композиция қабаттардың мұнаймен қаныққан кеуектеріне енген кезде,
скипидар жақсы еріткіш ретінде мұнай-қышқыл эмульсияларының тұрақтылығын
айтарлықтай төмендетеді және және ұңғыманың түп аймағындағы калий
бихроматының сулы ерітіндісінің скипидармен кейінгі реакциясы күшейеді
және реакцияның қабаттан тыс жерде ығысуымен байланысты құрамның
қабаттан шығуына қарай жалғасады. Қолданылатын композиция ароматты
көмірсутектермен, сондай-ақ парафиндермен әрекеттеседі. Реакциялар
нәтижесінде әр түрлі қышқылды күрделі эфирлер алынады, олар фазааралық
керілуді төмендетеді, ингибиторлық әсерді арттырады, тау жынысы
ұңғымасының кеуектеріндегі ауыр мұнай компоненттерінің шөгінділерін
ерітеді және ұңғы түбінің және қабаттың шалғай аймағының өткізгіштігін
қалпына келтіреді, нәтижесінде қабаттан ұңғымаға мұнай ағыны жақсарады.

{\bfseries Түйін сөздер:}түп қабат аймағы, өткізгіштік, мұнай, су, калий
бихроматы

\begin{header}
{\bfseries МЕТОДЫ ОБРАБОТКИ ПРИЗАБОЙНОЙ ЗОНЫ СКВАЖИНЫ ПРИ ДОБЫЧЕ ВЫСОКОВЯЗКИХ НЕФТЕЙ}

{\bfseries
\tsp{1}М.У. Калменов,
\tsp{2}М.Г. Абдуллаев,
\tsp{1}С.Е. Байботаева\envelope,
\tsp{1}А.С. Садырбаева
}
\end{header}

\begin{affil}
\tsp{1}Южно-Казахстанский университет им.М.Ауэзова, Шымкент, Казахстан,

\tsp{2}Институт нефти и газа Азербайджанской Республики,Баку, Азербайджан,

e-mail: sbaibotaeva@mail.ru
\end{affil}

В данной статье разработан новый способ, задачей которого является
повышение эффективности обработки призабойной зоныскважины, за счет
введения в призабойную зону теплоносителя, которая повышает растворяющую
способность состава, скорость и объем растворения труднорастворимых
отложений при обработке призабойной зоны добывающих скважин, а также
глубоко проникающий в пласт.

При попадании данного состава в нефтенасыщенные поры пласта, скипидар
как хороший растворитель, существенно снижает стойкость нежелательных
нефтекислотных эмульсий и дальнейшая реакция водного раствора
бихроматакалия соскипидаром усиливается в призабойной зонескважины и
продолжения реакции в глубине пласта за счет вытеснение состава с газом
извне. Применяемый состав реагируется с ароматическими углеводородами, а
также с парафинами. В результате реакций получаются различные кислые
эфиры, которые позволяют уменьшить межфазное натяжение, увеличить
ингибирующее действие, хорошо растворяют отложения тяжелые компоненты
нефти в порах породы и восстанавливают проницаемость призабойной и
удаленной зоны пласта, в результате чего улучшается приток нефти из
пласта к скважине.

{\bfseries Ключевые слова:} призабойная зона пласта, проницаемость, нефть,
вода, бихромат калия.

\begin{header}
{\bfseries METHODS OF PROCESSING THE BOTTOMHOLE ZONE OF A WELL DURING THE EXTRACTION OF HIGH-VISCOSITY OILS}

{\bfseries
\tsp{1}M.U. Kalmenov,
\tsp{2}M.Q. Abdullayev,
\tsp{1}S.Ye. Baibotayeva\envelope,
\tsp{1}A.S. Sadyrbayeva
}
\end{header}

\begin{affil}
\tsp{1}M.Auezov South Kazakhstan University, Shymkent, Kazakhstan,

\tsp{2}Institute of oil and gas of the Republic of Azerbaijan, Baku, Azerbaijan,

e-mail:sbaibotaeva@mail.ru
\end{affil}

In this article, a new method has been developed, the task of which is
to increase the efficiency of processing the bottomhole zone of a well
by introducing a coolant into the bottomhole zone, which increases the
solubility of the composition, the rate and volume of dissolution of
difficult-to-dissolve deposits during processing of the bottomhole zone
of producing wells, as well as penetrating deeply into the reservoir.

When this composition enters the oil-saturated pores of the reservoir,
turpentine, as a good solvent, significantly reduces the resistance of
undesirable petroleum acid emulsions and the further reaction of an
aqueous solution of potassium bichromate with turpentine increases in
the bottomhole zone of the well and the continuation of the reaction in
the depth of the reservoir due to the displacement of the composition
with gas from the outside.The applied composition reacts with aromatic
hydrocarbons, as well as with paraffins. As a result of the reactions,
various acid esters are obtained, which reduce the interfacial tension,
increase the inhibitory effect, well dissolve deposits of heavy oil
components in the rock pores and restore the permeability of the
bottomhole and remote zone of the reservoir, resulting in improved oil
flow from the reservoir to the well.

{\bfseries Keywords:} bottom-hole formation zone, permeability, oil, water,
potassium bichromate

\begin{multicols}{2}
{\bfseries Кіріспе.} Қабаттың түп аймағы белгілі болғандай, ерекше аймақ
болып табылады, өйткені, мұнда қысымның, температураның өзгеруі, ауыр
мұнай компоненттерінің шөгуі және т.б. сияқты күрделі процестер жүреді.
Бұл процестердің барлығы ұңғыма өнімділігіне, әсіресе ньютондық емес
мұнайларды пайдаланатын өндіру ұңғымаларының жұмысына кері әсер етеді.

Кен орындарын игеру кезінде қабат қысымын ұстап тұру үшін қабатқа суық
су айдалатыны белгілі. Бұл қабат температурасының айтарлықтай
төмендеуіне әкеліп соғады, бұл ұңғыма түбіндегі мұнайдың құрамындағы
ауыр компоненттердің тұндырылуына әсер етеді және қабаттың кеуектерінің
бітелуіне әкеледі. Капиллярлық арналардың құрылымымен сипатталатын
қабаттың бастапқы негізгі өткізгіштігі асфальтты-шайырлы және парафинді
шөгінділердің әсерінен бұзылады және қабаттың ұңғы аймағының
өткізгіштігінің төмендеуі орын алады.Нәтижесінде көмірсутектерді
бастапқы өндірудің мәнін қалпына келтіру мақсатында ұңғы түбінің түзілу
аймағына әсер ету қажеттілігі туындайды.

Қабаттың өнімділігін арттыру үшін ұңғыма түбіне әсер ету қабаттың түп
аймағының өткізгіштігін қалпына келтіруді қамтиды. Қышқылдық, термиялық,
химиялық, термохимиялық және т.б көптеген ғылыми еңбектер ұңғы түбінің
түзілу аймағының өткізгіштігін қалпына келтіру мәселелеріне арналған.
әдістері. Ұңғыманың түп аймағына қышқылдық әсер ету механизмі тау
жыныстарының қаңқасының белгілі бір бөлігінің және әртүрлі шөгінділердің
қышқылмен әрекеттесуі нәтижесінде мұнай өндіру жылдамдығының қалпына
келуіне негізделген.

Ұңғымаларды қышқылмен өңдеу процесі - бұл ұңғыманы пайдалану кезінде
ұңғыманың түп аймағын тазартуды, яғни ұңғыманың түп аймағын тұздардан,
парафинді-шайырлы шөгінділерден және коррозия өнімдерінен тазартуды,
ұңғыманың түп аймағындағы тау жыныстарының өткізгіштігін арттыруды,
сондай-ақ бұрғылаудан кейін ұңғыманы игеру кезінде, ұңғыманың түп
аймағын қайта бұрғылаудан тазартуды қамтитын операция. Қарапайым
қышқылдандыру, ең алдымен, өткізгіштігін қалпына келтіру немесе арттыру
үшін ұңғыма оқпанының жанындағы қабатқа енуге арналған. Процесс жоғары
қысыммен қабатқа қышқылды айдау арқылы жүзеге асырылады.

Ұңғыманың түп аймағы - бұл қабаттардың ұңғыманың түп аймағына тікелей
іргелес жатқан бөлігі (радиусы 2-3 метр цилиндрлік бөлік). Қабаттың бұл
бөлігінде сұйықтықтың сүзілу жылдамдығы өзгереді, қысым төмендейді, ауыр
мұнай құрамдастары шөгеді, энергия шығыны мен ағынның кедергісі
максималды жоғары болады. Қабаттың төменгі қабатының шамалы ластануы да
ұңғыманың өнімділігін айтарлықтай төмендетеді. Ұңғыманың түп аймағының
өткізгіштігін қалпына келтіру немесе арттыру қабаттардың ұңғыманың түп
аймағына сыртқы әсер ету немесе қабаттарда пайда болған шөгінділердің
еруі негізінде жүзеге асырылады.

Пайдалану процесінде өндіру және айдау ұңғымаларының түп аймағының
өткізгіштігінің төмендеуінің негізгі себептеріне келесі факторлар
жатады:

- температура мен қысым жағдайлары өзгерген кезде мұнайдың
асфальтенді-шайырлы-парафинді құрамдас бөліктері немесе ілеспе қабат
суларынан тұздардың ағуы баяулайды және шөгеді;

- ұңғымаларды ағымдағы және күрделі жөндеу кезінде ұңғымаға бітелткіштер
(ағын су немесе теңіз суы) немесе жуу сұйықтығының түсуі;

- су-мұнай эмульсиясының түзілуі;

- саз бөлшектерін тұщы сумен қанықтыру кезінде терригендік қабаттардың
цементтелуінің ісінуі;

- ұңғымаларды сөндіру немесе шаю кезінде механикалық қоспалар мен металл
коррозиясы өнімдерінің ұңғыманың түп аймағына енуі және т. б.

Ұңғыманың түбінің аймағына әсер ету үшін жұмысшы агентінің құрамын
таңдау кезінде оның келесі критерийлерге сәйкестігін қамтамасыз ету
маңызды:

- құрам ұңғыма түбінің аймағының қажетті тереңдігіне дейін енуіқажет;

- құрам жыныспен, оны қанықтыратын сұйықтықтармен немесе
кольмотанттармен әрекеттескеннен кейін қайта ерімейтін шөгінділерді
тудырмауы керек;

- реагенттер ұңғымаларды жөндеуде, қабат суларында және басқа
технологиялық сұйықтықтарда қолданылатын тығындау ерітінділерімен
үйлесімді болуы және мұнай өндіруге, оны тасымалдауға және оны
тазартудың технологиялық сатыларына кері әсер етпеуі керек;

- құрамның компоненттері мүмкіндігінше аз улы болуы керек.

Ұңғыманың түп аймағын тазалау технологиясының негізгі принциптері:

- химиялық реагенттердің кольматтаушы заттарға әсер етуіне байланысты
перфорациялау арналары мен ұңғының түп аймағының жағдайын шектеумен ұңғы
түбінің қабат аймағының өткізгіштігін қалпына келтіру немесе жоғарылату,
сондай-ақ өнімділікті арттыру немесе қабатқа су айдау мүмкіндігін
арттыру;

- қабаттың түптік ұңғыма аймағына жақын жерде де, шалғай бөліктерінде де
тау жыныстары қаңқасының кеуекті кеңістігінің құрылымына әсер ету
нәтижесінде ұңғымалардың өнімділігін немесе қатаю қабілетін арттыру;

- айдалатын химиялық реагенттермен физикалық және химиялық өзара
әрекеттесу кезінде колматанттың жойылуы және т. б.

{\bfseries Материалдар мен әдістер.} Аталған бағыттар бойынша көптеген
ғылыми мақалалар, технологиялар, композициялар мен әдістер, сондай-ақ
қойылған мақсаттарға жету үшін түрлі химиялық реагенттер пайдаланылған
патенттік жұмыстар жарияланды. Төменде осы жұмыстардың кейбірін
қарастыру ұсынылады.

Әдістердің {[}1{]} бірінде мұнай ұңғымасының ұңғыма түбінің аймағын
өңдеу ұңғыма сұйықтығымен (негізгі су) толтырылған ұңғыманың өңделген
интервалына ұңғыма сұйықтығынан оқшауланған зат түріндегі гидрореактивті
құрамды (ГРҚ) жеткізуді қамтиды. Гидрореактивті құрам ұңғыма
сұйықтығының суымен гетерогенді химиялық реакцияға түсу, нәтижесінде
жылу және газ тәрізді өнімдер бөлініп, гидрореактивті құрамды ұңғыма
сұйықтығымен жанасу арқылы химиялық реакцияны бастау қасиеттеріне ие
болады.Процессте алюминий хлоридінің кем дегенде түйіршіктері және
алюминий хлориді үшін еріткіш болып табылмайтын органикалық мұнай
шөгінділері үшін еріткіш бар гидрореактивті композиция қолданылады. Бұл
әдісті қолданғанда реакция ұңғыманың перфорация аймағының ішінде жүреді.
Бұл экзотермиялық реакция нәтижесінде бөлінетін жылудың жоғалуына ықпал
етеді.

Ұсынылған әдісте {[}2{]}, өнімді қабаттың термохимиялық өңдеуі тазарту
аймағына жанғыш-тотықтырғыш құрамды айдауды және жану инициаторын өңдеу
аймағына жіберуді қамтиды, ол сілтілі металдың боргидридіне және
метанолға немесе диэтил эфиріне негізделген композиция 5-40 масс.\%
мөлшерінде пропилкарборан 5 - 25 масс.\% мөлшерінде сілті, немесе қатты
изопропилкарборанның мөлшері 5 - 40 мас.\% болып табылады.

Жану инициаторы тығыздалған контейнерді мұнайкәсіпшіліктегі бұрғылау
ұршығы арқылы сорапты және компрессорлық құбырлар колоннасына түсіру
арқылы жеткізіледі, содан кейін шнурлы торпедасын жару арқылы контейнер
жойылады. Шнур торпедасы контейнердің бүкіл ұзындығы бойынша орнатылады.

Сонымен қатар, қабаттың түп аймағын ыстық қышқылмен (термоқышқыл) өңдеу
әдісі қолданылды {[}3{]}. Бұл әдіс формальдегид ерітіндісінде
тасымалдаушы сұйықтық ретінде кездесетін түйіршіктелген немесе ұнтақ
магний қабаттың түп аймағынабөлек айдаудан, сондай-ақ онымен
әрекеттесетін химиялық реагент ретінде аммоний хлоридінің сулы
ерітіндісін бөлек инъекциялаудан тұрады.

Бұл жағдайда құбырға енгізілген ерітінділер арасындағы реакциядан
бөлінетін жылуды жоғалтуды болдырмау үшін оларды енгізу арасында да,
соңғы компонентті айдаудан кейін де буферлік сұйықтық айдалады. Бұл
қабаттың түп аймағында термохимиялық әсердің пайда болуын қамтамасыз
етеді.

Бірақ бұл жағдайда реакциядан бөлінетін жылу мұнайдың ауыр
компоненттерінің толық еруі үшін жеткіліксіз және сазды жыныстардағы
саздардың ісінуі мүмкін.

Ерітіндіні жоғары температураға дейін алдын ала қыздыру арқылы ұңғы
түбінің түзілу аймағын әлсіз қышқылдармен өңдеудің тағы бір әдісі
белгілі {[}4{]}. Бұл жағдайда термиялық әрекет тау жынысындағы химиялық
реагенттермен біріктіріледі. Қышқыл алдын ала қыздырылған суға немесе
буға енгізіледі. Қышқылды суға немесе буға енгізгеннен кейін алынған
температура мақсатқа жету үшін жеткіліксіз болып, жұмыс көп еңбекті
қажет етеді.

Құрамында фтор қышқылы, тұз қышқылы және су бар ұңғы түбінің түзілу
аймағын өңдеуге арналған қышқыл ерітіндісі де белгілі {[}5{]}. Бұл
ерітінді іс жүзінде карбонатты жыныстарды ерітпейді, өйткені фтор
қышқылының карбонаттармен әрекеттесуі реакцияға кедергі келтіретін және
ұңғыма жабдығын қатты коррозияға ұшырататын нашар еритін флюорит
шығарады.

Нейтралдау тиімділігін арттыру және қышқыл ерітіндісінің ену тереңдігін
жоғарылату үшін қышқылды айдау алдында гидрофобты эмульсияны айдауымен
ерекшеленетін қышқылдық және гидрофобты эмульсияның ерітінділерін
айдаудан тұратын мұнай қабатының ұңғы аймағын қышқылмен өңдеу әдісі
тәжірибеде де қолданылды {[}6{]}. Бұл шешімнің кемшілігі -- ұңғы түбінің
түзілу аймағын өңдеудің төмен тиімділігі, сонымен қатар қабат
тереңдігіне аз ғана енуі болып табылады.

Тұз қышқылын ұңғыма аймағына айдау арқылы ұңғымаларды өңдеу үшін қолдану
әдісі белгілі {[}7{]}. Дегенмен, тұз қышқылы терригендік жыныстарды өте
нашар ерітеді, сонымен қатар ұңғымалардың түбі аймағын өңдеу кезінде
жабдықты қатты тоттандырады. Бұл композицияның тағы бір кемшілігі -
ұңғыманың түбін қалыптастыру аймағын тазалау сапасының төмендігі.

Жұмыста {[}8{]} ұңғымалардың түбі аймағын тазалауға арналған композиция
ұсынылған, оның құрамдас бөліктері тұз қышқылы мен натрий гидроксидінен
тұрады. Бұл компоненттер бір-бірімен әрекеттескен кезде экзотермиялық
реакция пайда болады, нәтижесінде үлкен мөлшерде жылу бөлінеді. Бірақ
құрамында тұз қышқылының болуы ұңғыма жабдығының коррозиясын тудырады
және реакция кезінде бөлінетін жылу өте жоғары емес.

Өндіруші ұңғымалардың өнімділігін арттыру үшін қабаттың төменгі бөлігін
өңдеу кезінде көптеген жұмыстар белгілі екенін атап өткен жөн. Бұл
жұмыстарға гидродинамикалық, физика-химиялық, термиялық, термохимиялық
және т.б. әдістер кіреді {[}9{]}.

Ұңғыма аймағын өңдеудің келесі белгілі әдісі - түйіршіктелген магний мен
тұз қышқылын айдау {[}10{]}. Бұл әдіс түзуді тереңірек қыздыру және
процесті жеделдету есебінен өңдеу процесінің тиімділігін арттыру үшін
магний мен қышқылды айдау бір мезгілде, ал айдау бөлек
жүргізілетіндігімен ерекшеленеді.Дегенмен, бұл шешім ұңғы түбінің түзілу
аймағын өңдеуде, сондай-ақ қабат тереңдігіне енуде тиімділігі төмен. Бұл
жағдайда ұңғыманың түбі аймағында да шөгінділер пайда болуы мүмкін.

Мұнай-газ ұңғымасының түптік аймағын өңдеу әдісі де қолданылады, ол
қабаттың түптік ұңғыма аймағына газ тәрізді хлорлы сутегін айдаудан
тұрады. Бұл әдіс өңделетін аймақтың өткізгіштігі мен көлемін арттыру
үшін ұңғымада соңғысын хлор мен сутегі қоспасымен толтыру арқылы ұңғыма
сағасындағы жарылыс жағдайынан аспайтын қысыммен хлорсутек алуымен
ерекшеленеді. Бұл әдістің кемшілігі - төмен тиімділік және технологиялық
қиындықтар болып табылады.

Жұмыс {[}11{]} ұңғы түбінің аймағын ыстық су немесе бу айдау арқылы
өңдеудің ерекше әдісін ұсынады. Бастапқыда агенттер 80
\tsp{0}C температураға дейін қызады. Содан кейін сол
ұңғымаға алдын ала 120 \tsp{0}С дейін қыздырылған 8-15\% HCl
ерітіндісі айдалады. Біраз уақыт әрекетсіз болғаннан кейін ұңғыма
пайдалануға беріледі.

Сондай-ақ ұңғыманың түптік қабат аймағын термохимиялық өңдеу арқылы
ұңғыма өнімділігін арттыру әдісі белгілі {[}12{]}. Ұсынылған әдісте
өңдеудің тиімділігі ұңғыма аймағына жылу тасымалдағышты енгізу арқылы
жоғарылайды, бұл композицияның еріту қабілетін арттырады. Қолданылатын
композиция хром қышқылының (хромангидридінің сулы ерітіндісі) және
метанолдың (немесе төменгі спирттердің) қоспасы болып табылады.
Композициялар ұңғыманың түбінің түзілу аймағына бірінен соң бірі
айдалады: бірінші хром қышқылы, содан кейін төменгі спирттер (мысалы,
метанол) немесе керісінше.

Сондай-ақ ұңғыманың түбіне тасымалдаушы сұйықтық пен химиялық
реагенттегі өзара әрекеттесетін түйіршіктелген немесе ұнтақталған
магнийді ұңғыма түбіне бөлек енгізуден тұратын ұңғыманың түбінің түзілу
аймағын термиялық қышқылмен өңдеу әдісі енгізілді, мұнда формальдегидтің
сулы ерітіндісі ұңғыма түбіне тасымалдаушы сұйықтық ретінде енгізіледі,
ал аммонийдің реактивті ерітіндісі болып табылады. Бұл жағдайда
ерітінділерді айдау алдында олардың арасында және ұңғымаға айдаудан
кейін буферлік сұйықтық айдалады. Бұл әдістің кемшілігі - процестің көп
сатылылығы, сонымен қатар тиімділігінің төмендігі.

Жұмысында {[}13{]} жұмыс істеп тұрған ұңғымалардың өнімділігін
арттыратын көптеген әдістер ұсынылған, мысалы, күрделі ішкі қабатты
термоқышқылды өңдеу, сонымен қатар жер асты, ядролық (мысалы: атомдық)
жарылыстар, қабаттың төменгі қабатына химиялық әсер ету әдістері, өнімді
қабаттардың төменгі қабатының гидрожаруы және т.б. бұл әдістерді қышқыл
ретінде қолданған кезде негізінен тұз қышқылы қолданылған.

Жұмыста {[}14{]} жұмыс істеп тұрған ұңғымалардың өнімділігін арттырудың
көптеген әдістері ұсынылған, мысалы, күрделі жерасты термиялық қышқылды
өңдеулер, сондай-ақ жер асты ядролық (мысалы: атомдық) жарылыстар,
ұңғыма түбіне химиялық әсер ету әдістері, өнімді қабаттардың ұңғыма
түбінің аймағын гидравликалық жару және т.б. Бұл әдістерді қолданғанда
қышқыл ретінде негізінен тұз қышқылы қолданылды.

Қабаттың түп аймағын термохимиялық өңдеу әдісі {[}15{]} ұсынылған.
Әдістеме бойынша өнімді қабаттың түбінің ұңғыма аймағына
жанғыш-тотықтырғыш композиция айдалады. Құрамына келесі компоненттер
кіреді: карбамид, азот қышқылы, сірке қышқылы, калий перманганаты,
карборан, аммиякты селитра және массалық \% үлесіне сәйкес мөлшерде су.
Сонымен қатар, жанғыш-тотықтырғыш құрамы орналасқан аймаққа келесі
құрамдас бөлігі бар жану инициаторы енгізіледі: алюминий 10-30, хром
оксиді 70-90. Енгізілген жану инициаторының мөлшері жанғыш-тотықтырғыш
құрамның құрамдас бөліктерінің массасының 10\% аспайды. Карборан ретінде
изопропилметакарборан қолданылады. Бұл әдісті қолдану жаңа ұңғымаларды
игеру процесінің тиімділігін арттыруға және жұмыс істеп тұрған
ұңғымалардан ағынды күшейтуге мүмкіндік береді.

Дегенмен, ұсынылған әдісте қолданылатын композиция көп компонентті және
оны өндірістік жағдайларда, әсіресе теңіз жағдайында дайындау мүмкін
емес.

Өнімді қабаттарды термохимиялық өңдеу әдісі {[}16{]} белгілі. Бұл әдісте
қабаттың өңделу аймағына жанғыш тотықтырғыш қосылыс (ЖТҚ) айдалады,
содан кейін өңделу аймағына жану инициаторы енгізіледі. Жану инициаторы
металл гидридіне немесе металлоидқа негізделген қатты немесе сұйық
қосылыс болып табылады. Атап айтқанда, сілтілі металл борандар сияқты
тұз гидридтерін қолдануға болады. Металоидты гидрид негізіндегі жану
инициаторының сұйық құрамы құрамдастардың тиісті массасында \% диэтил
эфирі немесе метил спирті сияқты органикалық еріткіш негізіндегі бордың
суспензия ерітіндісі болуы мүмкін.

Жану инициаторын өнеркәсіптік көтергішті пайдалана отырып, тығыздалған
контейнерді сорғы мен компрессорлық құбырлар бағанына түсіру, содан
кейін көтеру кабелінің ұштары сорапты- компрессорлы құбырларының
аяқшасындағы қуат көзіне тигеннен кейін контейнердің бүкіл ұзындығы
бойынша орнатылған кері зарядты жару арқылы контейнерді жою арқылы
енгізу ұсынылады.

Көріп отырғаныңыздай, бұл әдісті қолдану өте көп еңбекті қажет етеді
және көптеген жабдықты және қолдануға ерекше көзқарасты қажет етеді.

Мұнай қабатының ұңғыма маңындағы аймағын термохимиялық өңдеу әдісі
белгілі {[}17{]}. Әдіс құрамында оттегі бар органикалық қосылыстар
немесе олардың қоспасы және натрий нитритінің сулы ерітіндісі бар отын
тотықтырғыш құрамын (ОТҚ) түзілу аймағына кезекпен айдаудан тұрады.

Қабатқа реагенттерді айдағаннан кейін ұңғыма маңындағы аймақта
қабатішілік жану жүреді, ұңғыма ішіндегі экзотермиялық реакцияның
индукциялық кезеңі 100 минуттан астам уақытты алады, ал ұңғыманы
өңдеудің жалпы уақыты 400 минутқа жетеді, бұл ұңғыма ішілік өңдеу әдісін
кеңінен қолданудың негізгі шектеуі болып табылады.

Қабатты термохимиялық өңдеудің тағы бір әдісі {[}18{]} белгілі, ол
құрамында аммоний селитрасының (аммоний нитраты), аммоний хлориді немесе
аммоний фосфатының сулы ерітіндісі бар жанғыш-тотықтырғыш құрамды өнімді
түзілу аймағына айдау және жану инициаторын (тотықтырғыш зат, жанғыш зат
орналасқан ұнтақ орналасқан жер) енгізу кіреді. Өте ұзақ өңдеу уақыты
(\textasciitilde130 с), жарылғыш заттарды қолдану және әдісті біршама
күрделі ұйымдастыру оны пайдалану мүмкіндіктерін шектейді.

Көптеген осы бағыттағы орындалған еңбектерді зерттеу нәтижелері бойынша
бұл әдістердің ешқайсысы әмбебап емес екенін және оларды қолдануда әлі
де көптеген қиындықтар бар екенін көрсетті. Кейде бұл әдістерді
қолдануда пайдаланатын композициялар көп компонентті, кейде
композицияның кейбір компоненттері қымбат болады, кейде оларды
дайындауда қиындықтар туындайды, әсіресе оларды өндірістік жағдайда
дайындау мүмкін еместігін көрсетті.

Осы қиындықтардың барлығын ескере отырып, ұңғыманың түп аймағына және
қабаттың тереңірек қабаттарына әсер ету мүмкіндігін, сондай-ақ
кәсіпшілік жағдайында дайындаудың қарапайымдылығын ескеретін жаңа әдіс
әзірленді. Тағы бір артықшылығы -- бұл әдісте қолданылатын композицияның
құрамдас бөліктері Қазақстан Республикасында өндіріледі.

{\bfseries Нәтижелер және талқылау.} Ұсынылып отырған әдістің мақсаты --
ұңғыманың түптік аймағын өңдеу кезінде нашар еритін шөгінділердің
жылдамдығы мен көлемін ұлғайту арқылы ұңғыма түбінің түзілу аймағына
жылу тасымалдағышты енгізу арқылы өңдеудің тиімділігін арттыру,
сондай-ақ ұңғыманың түбіндегі тереңдікте түзілу мүмкіндігін туғызу.

Бұл мақсатқа жету үшін алдымен натрий бихроматының (немесе калий
бихроматының) сулы ерітіндісін қабаттың ұңғыма аймағына тазарту
ерітіндісін тереңірек енгізу үшін тотықтырғыш ретінде ұңғыма оқпанына
айдап, содан кейін скипидардың есептелген мөлшерін ұңғыма оқпанына
айдағанда, экзотермиялық реакция пайда болады. Осыдан кейін, қоспа
жоғары қысымды газбен ұңғыма түбінің аймағынан қабаттың тереңдігіне
ығыстырылады, нәтижесінде бөлінетін жылу және жақсы еритін реакция өнімі
қабат түбінің ұңғы аймағының тереңдігіне түседі.

Скипидар негізінен терпенді көмірсутектердің
C\tsb{10}H\tsb{16} күрделі қоспасы, өзіне тән
қарағай иісі бар мөлдір, түссіз, ұшқыш сұйықтық болып табылады. Скипидар
полярлы емес органикалық еріткіштерде, этанолда, ацетонда жақсы ериді
және суда ерімейді. Т\tsb{кип}=153--180 \tsp{0}С,
d\tsp{20}\tsb{4}=0,855--0,865,
n\tsp{20}\tsb{d}=1,460--1,478.

Құрамындағы терпенді көмірсутектер сияқты скипидар өте белсенді және
ауада, әсіресе жарықта химиялық тотықтырғыштармен (концентрлі HNO3,
хромангидрид және т.б.) оңай тотығады. Скипидар жану арқылы тотығады.

Құрамдас бөліктерді ұңғыма оқпанының аймағына осы ретпен айдаған кезде,
«жоғары қысым мен жылу көзі» ретінде әрекет ететін көп мөлшерде жылу мен
газдардың бөлінуімен термохимиялық реакция жүреді. Осыған байланысты
бөлінетін газ өңдеуге арналған компоненттердің тотығу реакциясының
әсерінен қыздырылған қоспаны қабаттың тереңдіктеріне енгізеді, ал әлсіз
жарылыстардың әсерінен реакция қабаттың түбі аймағының үлкен тереңдігін
қамтиды. Осыдан кейін температура 500-600°С-тан жоғары көтеріледі, бұл
ұңғыманың түп аймағында мұнайдың ауыр компоненттерінің еруіне және сол
арқылы ұңғыманың түп аймағының өткізгіштігінің жоғарылауына әкеледі.

Жұмыс аяқталғаннан кейін ұңғыма 1,0 -- 2,0 сағатқа жабылады. Бұл уақыт
мұнайдың ауыр құрамдас бөліктерін ұңғы аймағының кеуектерінде және қабат
тереңдігінде еріту және өңделген ерітіндіні қабатқа тереңірек енгізу
үшін жеткілікті. Осыдан кейін ұңғыма бірқалыпты пайдалануға беріледі.
Айта кету керек, ұңғымаларды іске қосқаннан кейін саңылаулардағы кез
келген қалдықтарды, соның ішінде сазды-құмды бөлшектер мен бұрғылау
ерітіндісінің қалдықтарын тазалауда термохимиялық реакция нәтижесінде
бөлінген газ және тазарту ерітіндісін қабатқа терең енгізу басты рөл
атқарады. Ұңғымалардың түп аймағындағы кеуектердегі ауыр мұнай
компоненттерінің балқыған қалдықтары кеуекті кеңістікте ілулі күйінде
қалады. Бұл жағдайда қабаттан жоғары қысыммен ұңғымаға қайтып келетін
газ өзінің жоғары жылдамдығының арқасында, өз жолын кез келген
кедергілерден, соның ішінде осы ілмелі қалдықтардан тез тазартады. Бұл
ұңғыдан мұнай өндіруді ұлғайта отырып, ұңғы түбінің қабат аймағының
өткізгіштігінің одан әрі артуына ықпал етеді.

{\bfseries Қорытынды:}

- тұтқырлығы жоғары мұнай кен орындарында жұмыс істейтін өндіру
ұңғымаларының түптік аймағын термохимиялық өңдеу үшін жаңа құрам
әзірленді;

- тұтқырлығы жоғары мұнай кен орындарында өндіру ұңғымаларының түптік
аймағын термохимиялық өңдеу әдісі ұсынылды;

- ұсынылған әдісті қолдана отырып, ұңғыма түбі аймағының өткізгіштігін
толық қалпына келтіруге қол жеткізуге болады;

- әртүрлі кен орындарындағы мұнайлардың физикалық және химиялық
қасиеттерін ескере отырып, композициялық құрамдастардың кең диапазондағы
өзгеруі эксперименталды түрде мүмкін болады;

- ұсынылған әдісте қолданылатын композицияның жоғары тежеу қасиетіне ие;
\end{multicols}

\begin{center}
{\bfseries Әдебиеттер}
\end{center}

\begin{refs}
1. \href{https://patents.google.com/?assignee=\%D0\%90\%D0\%B3\%D0\%BB\%D0\%B8\%D1\%83\%D0\%BB\%D0\%BB\%D0\%B8\%D0\%BD+\%D0\%9C\%D0\%B8\%D0\%BD\%D1\%82\%D0\%B0\%D0\%BB\%D0\%B8\%D0\%BF+\%D0\%9C\%D0\%B8\%D0\%BD\%D0\%B3\%D0\%B0\%D0\%BB\%D0\%B5\%D0\%B5\%D0\%B2\%D0\%B8\%D1\%87&peid=62a579286f398\%3A769\%3A84d87e1b}{Аглиуллин
М.М}. Способ термохимической обработки призабойной зоны нефтяных
скважин. Патент RU 2320862C2, 2017-10-27, 2018-03-27

2. \href{https://patents.google.com/?inventor=\%D0\%95.\%D0\%9D.(RU)+\%D0\%90\%D0\%BB\%D0\%B5\%D0\%BA\%D1\%81\%D0\%B0\%D0\%BD\%D0\%B4\%D1\%80\%D0\%BE\%D0\%B2&peid=62a58a8975e88\%3Ad5\%3A37969d2b}{Александров}
Е.Н.,
\href{https://patents.google.com/?inventor=\%D0\%9A\%D0\%B0\%D1\%80\%D0\%B8\%D0\%BD\%D0\%B0+\%D0\%93\%D1\%80\%D0\%B8\%D0\%B3\%D0\%BE\%D1\%80\%D1\%8C\%D0\%B5\%D0\%B2\%D0\%BD\%D0\%B0+\%D0\%A9\%D0\%B5\%D1\%80\%D0\%B1\%D0\%B8\%D0\%BD\%D0\%B0+(UA)&peid=62a58a8534108\%3Ad3\%3Ab282e17c}{Щербина
К.Г.},
\href{https://patents.google.com/?inventor=\%D0\%95.\%D0\%92.(RU)+\%D0\%94\%D0\%B0\%D1\%80\%D0\%B0\%D0\%B3\%D0\%B0\%D0\%BD&peid=62a58a851ba68\%3Ad2\%3Aa8a96fee}{Дараган}
Е.В.,
\href{https://patents.google.com/?inventor=\%D0\%93.\%D0\%9F.(RU)+\%D0\%94\%D0\%BE\%D0\%BC\%D0\%B0\%D0\%BD\%D0\%BE\%D0\%B2&peid=62a58a84c28a0\%3Ad0\%3Afb85d5a3}{Доманов}
Г.П.,
\href{https://patents.google.com/?inventor=\%D0\%AD.\%D0\%91.(RU)+\%D0\%9C\%D0\%BE\%D0\%B2\%D1\%88\%D0\%BE\%D0\%B2\%D0\%B8\%D1\%87&peid=62a58a7ec2e50\%3Aca\%3A10c64ab3}{Мовшович}
Э.Б. Способ термохимической обработки продуктивного пласта и
горюче-окислительный состав для его осуществления. Патент RU2153065C1,
2009-08-27, 2010-07-20.

3. Edward D. McCabe A way to increase the production of hydrocarbon
wells by treating them with hot acid solutions - US Patent No.3367417,
2018, cl.166-40.

4. John L. Gidley. Method of acid treatment of siliceous formations. -
US patent №.3548945, 2014, cl.166-307.

5. Ибрагимов Г.З., Сорокин В.А. Кислотные обработки с использованием
поверхностно-активных веществ.- А.С. СССР № 828047, 2016, кл. Е21В
43/27.

6. Игнатов А.Н., Кореняко А.В., Здобнова О.Л., Сергеев В.В. Инструкция
по обработке карбонатных пластов кислотными составами с одновременным
ограничением водопритоков ИОС. ЗАО НПФ Бурсинтез-М.2014. -35 с.

7. Хабибуллин Р.А., Краченко О.В., Велигоцкий Д.А. «Комплексное
воздействие на пласт основа перспективных технологий
нефтегазодобычи»//Деловой журнал Neftegaz.RU.- № 3. -2015, с.24-27.

8. Кнунянц И.Л. «Химическая энциклопедия». Из-во Советская энциклопедия,
Москва т.2, 1990.- 673 с. ISBN 5-85270-035-5

9. Когарко СМ.,Маргулов Р.Д., ГарушевА.Р., и др. «Способ обработки
нефтегазовой скважины». - авторское свидетельство СССР № 991034, 2015,
кл. Е21В 43/27.

10. Мирзаджанзаде А.Х., Кузнецов О.Л., Басниев К.С, Алиев З.С. Основы
технологии добычи газа. - М.: ОАО «Издательство «Недра», 2003. - 879 с.
ISBN 5-247-03885-1

11. Салаватов Т.С., Абдуллаев М.Г., Гараев Р.Г., Хамитов Н.М.,
Джaманбаев C.E. Способ повышения продуктивности скважин за счет
термохимической обработки призабойной зоны// Научное обозрение.-2016.- №
9. - С.61-69

12. Мищенко И.Т. Скважинная добыча нефти: учебник, Издательство:Нефть и
газ, 2008.- 826 с. ISBN:5-7246-0404-3

13. Сулейманов А.Б., Мамедов К.К., Нисанова Т.М. и др. «Способ
термокислотной обработки призабойной зоны пласта». - авторское
свидетельство № 1668645, кл.1991, Е21В 43/27.

14. Александров Е.Н. Способ термохимической обработки призабойной зоны
пласта. Патент RU №2126084C1, E21B43/24 E21B43/25, 02.10.1999

15. Дараган Е.В. Способ термохимической обработки продуктивного пласта и
горюче-окислитель\-ная смесь (ГОС) для ее осуществления». Патент №
\href{https://patents.google.com/patent/US6488086B1/en?peid=62a56b340c380\%3Ab7\%3A5c249fc3}{US6488086B1},
E21B43/243, 03.12.2002.

16. Александров Е.Н. Способ термохимической обработки призабойной зоны
пласта. Патент РФ № 2070283, Е 21В43/24, 10.02.1999.

17. Рамазанов Р.Г. Способ разработки залежи высоковязкой и тяжелой нефти
с термическим воздействием. Патент РФ № 2526047, Е21В43/24, 20.08.2014
\end{refs}

\begin{center}
{\bfseries References}
\end{center}

\begin{refs}
1. Agliullin M.M. Sposob termohimicheskoj obrabotki prizabojnoj zony
neftjanyh skvazhin. Patent RU 2320862C2, 2017-10-27, 2018-03-27. {[}in
Russian{]}

2. Aleksandrov E.N., Shherbina K.G., Daragan E.V., Domanov G.P.,
Movshovich Je.B. Sposob termo\-himicheskoj obrabotki produktivnogo plasta
i gorjuche-okislitel' nyj sostav dlja ego
osushhestvlenija. Patent RU2153065C1, 2009-08-27, 2010-07-20. {[}in
Russian{]}

3. Edward D. McCabe A way to increase the production of hydrocarbon
wells by treating them with hot acid solutions - US Patent No.3367417,
2018, cl.166-40.

4. John L. Gidley. Method of acid treatment of siliceous formations. -
US patent №.3548945, 2014, cl.166-307.

5. Ibragimov G.Z., Sorokin V.A. Kislotnye obrabotki s
ispol' zovaniem poverhnostno-aktivnyh veshhestv.- A.S.
SSSR № 828047, 2016, kl. E21V 43/27.

6. Ignatov A.N., Korenjako A.V., Zdobnova O.L., Sergeev V.V. Instrukcija
po obrabotke karbonatnyh plastov kislotnymi sostavami s odnovremennym
ogranicheniem vodopritokov IOS. ZAO NPF Bursintez-M.2014. -35 s. {[}in
Russian{]}

7. Habibullin R.A., Krachenko O.V., Veligockij D.A. «Kompleksnoe
vozdejstvie na plast osnova perspekt\-ivnyh tehnologij
neftegazodobychi»//Delovoj zhurnal Neftegaz.RU.- № 3. -2015, s.24-27.
{[}in Russian{]}

8. Knunjanc I.L. «Himicheskaja jenciklopedija». Iz-vo Sovetskaja
jenciklopedija, Moskva t.2, 1990.- 673 s. ISBN 5-85270-035-5. {[}in
Russian{]}

9. Kogarko SM.,Margulov R.D., GarushevA.R., i dr. «Sposob obrabotki
neftegazovoj skvazhiny». - avtor\-skoe svidetel' stvo SSSR
№ 991034, 2015, kl. E21V 43/27. {[}in Russian{]}

10. Mirzadzhanzade A.H., Kuznecov O.L., Basniev K.S, Aliev Z.S. Osnovy
tehnologii dobychi gaza. - M.: OAO «Izdatel' stvo
«Nedra», 2003. - 879 s. ISBN 5-247-03885-1. {[}in Russian{]}

11. Salavatov T.S., Abdullaev M.G., Garaev R.G., Hamitov N.M.,
Dzhamanbaev C.E. Sposob povyshenija produktivnosti skvazhin za schet
termohimicheskoj obrabotki prizabojnoj zony// Nauchnoe obozrenie.-2016.-
№ 9.- S.61-69. {[}in Russian{]}

12. Mishhenko I.T. Skvazhinnaja dobycha nefti: uchebnik,
Izdatel' stvo:Neft'{} i gaz, 2008.- 826 s.
ISBN:5-7246-0404-3. {[}in Russian{]}

13. Sulejmanov A.B., Mamedov K.K., Nisanova T.M. i dr. «Sposob
termokislotnoj obrabotki prizabojnoj zony plasta». - avtorskoe
svidetel' stvo № 1668645, kl.1991, E21V 43/27. {[}in
Russian{]}

14. Aleksandrov E.N. Sposob termohimicheskoj obrabotki prizabojnoj zony
plasta. Patent RU \\№2126084C1, E21B43/24 E21B43/25, 02.10.1999. {[}in
Russian{]}

15. Daragan E.V. Sposob termohimicheskoj obrabotki produktivnogo plasta
i gorjuche-okislitel' naja smes'{} (GOS)
dlja ee osushhestvlenija». Patent № US6488086B1, E21B43/243, 03.12.
2002. {[}in Russian{]}

16. Aleksandrov E.N. Sposob termohimicheskoj obrabotki prizabojnoj zony
plasta. Patent RF № 2070283, E 21V43/24, 10.02.1999. {[}in Russian{]}

17. Ramazanov R.G. Sposob razrabotki zalezhi vysokovjazkoj i tjazheloj
nefti s termicheskim vozdejstvi\-em. Patent RF № 2526047, E21V43/24,
20.08.2014. {[}in Russian{]}
\end{refs}

\begin{info}
\emph{{\bfseries Авторлар туралы мәліметтер}}

Калменов М.У. - докторант, М.Әуезов атындағы Оңтүстік Қазақстан
университеті, Шымкент, Қазақстан, e-mail:\\mkalmen81@mail.ru;

Абдуллаев М.Г. - техника ғылымдарының кандидаты, қауымдастырылған
профессор, Әзірбайжан Республикасының Мұнай және газ институты,
Әзірбайжан, e-mail: malik.abdullayev.1952@gmail.com;

Байботаева С.Е. -- PhD докторы, доцент, М.Әуезов атындағы Оңтүстік
Қазақстан университеті, Шымкент, Қазақстан, e-mail: sbaibotaeva@mail.ru;

Садырбаева А.С. - техника ғылымдарының кандидаты, қауымдастырылған
профессор, М.Әуезов атындағы Оңтүстік Қазақстан университеті, Шымкент,
Қазақстан, e-mail:
a.sadyrbaeva@mail.ru

\emph{{\bfseries Information about the authors}}

Kalmenov M. U.- doctoral student, M. Auezov South Kazakhstan University,
Shymkent, Kazakhstan, e-mail:\\mkalmen81@mail.ru;

Abdullayev M. G.-Candidate of Technical Sciences, Associate Professor,
Institute of oil and gas of the Republic of Azerbaijan, Azerbaijan,
e-mail: malik.abdullayev.1952@gmail.com;

Baibotaeyva S.Ye. - PhD, associate professor, M.Auezov South Kazakhstan
University, Shymkent, Kazakhstan, e-mail:\\sbaibotaeva@mail.ru;

Sadyrbayeva A.S. - Candidate of Technical Sciences, Associate Professor,
Head of the Department. M.Auezov South Kazakhstan University, Shymkent,
Kazakhstan, e-mail: a.sadyrbaeva@mail.ru
\end{info}
